% The foreword for physics
%\hypertarget{PHYSICS}{}

\quotationbox{
"Is life based on the laws of physics?"\dots "\textit{New laws to be expected in the organism}"\dots 
"From all we have learnt about the structure of living matter, we must be prepared to find it working in a manner that cannot be reduced to the \textbf{\textit{ordinary} laws of physics}.
And \textit{that not on the ground that there is any ‘new force’ or what not}, directing the behavior of the single atoms within a living organism, \textit{but because the construction is different from anything we have yet tested in the physical laboratory}."\dots 
"How can the \textit{events} in \textit{space and time} which take place \textit{within the spatial boundary} of a \textit{living organism} be \textit{accounted for by physics} and chemistry?" \dots
"living matter, while
not eluding the 'laws of physics' as established up to date, is likely to involve 'other laws of
physics' hitherto unknown, which, however, once
they have been revealed, will form just as
integral a part of this science as the former."
\href{https://www.cambridge.org/core/books/abs/what-is-life/is-life-based-on-the-laws-of-physics/5BA8DE837F72FC4EEF70FE8470ADFE7B}{E.~Schrödinger: What is life?}\cite{Schrodinger:1992} @1944
}

In this chapter, we review physical terms, concepts and laws,
from the point of view of biology.
In his very accurately formulated question above, Schrödinger focused on (at least) these significant points
\begin{itemize}
\item '\textit{events}' \index{event} Unlike non-living matter, \textit{living matter is dynamic, changing autonomously by its internal laws}; we must think differently about it, including making hypotheses and testing them in the labs.
Those laws are non-ordinary 'because the construction is different', but its principles must not differ from the ones we already know.
Processes happen inside it, and we can observe some characteristic points.
\item '\textit{space and time}' Those characteristic points are significant changes resulting from processes that have material carriers,
which change their positions with finite speed, so (unlike in classical mechanics) the events also have the characteristics 'time' in addition to their 'position'.
In biology, the \hyperlink{spatiotemporal}{spatiotemporal} behavior is implemented by \textit{slow ion currents}.
\index{spatiotemporal}
The meticulous observations must describe the events by using 
special 'space-time' coordinates (to distinguish it from the one used in
theories of relativity, we call it 'time-space' coordinate).
\index{time-space coordinate}
In other words, instead of 'moments' we must consider 'periods'.
\item '\textit{within the spatial boundary}'
Laws of physics are usually derived for stand-alone systems, in the sense that
the considered system is infinitely far from the rest of the world; also, in the sense that the changes we observe do not significantly change the external world, so its idealized disturbing effect will not change it.
In biology, we must consider changing resources
\item '\textit{accounted for by physics}'[by extraordinary laws]
We are accustomed to abstracting and testing a static attribute, and we derive the 'ordinary' laws of motion for the 'net' interactions.
In the case of physiology, nature prevents us from testing 'net' interactions.
We must understand that some interactions are non-separable, and we must derive 'extraordinary' laws.
The forces are not unknown, but the known 'ordinary' laws of motion of physics are about single-speed interactions. 
\item '\textit{living matter}' To describe its dynamic behavior, we must introduce a dynamic description. 
\item '\textit{yet tested in the physical laboratory}'[including physiological ones] We need to test those 'constructions' in laboratories,
in their actual environment, and in 'working state'. As we did with non-living matter, we need to develop and gradually refine the testing methods as well as the hypotheses.
Moreover, we must not forget that our methods refer to 'states', and this time, we test 'processes'. Not only in measuring them but also in handling them computationally, we need slightly different algorithms.
\end{itemize}


Our \hypertarget{NeuronPhysical}{abstract model} focuses on the
neuron's signal processing function. However, we cannot reach our goal 
without the correct physiological understanding of its mechanisms. In the next chapter, we show that nature and its observable phenomena are complex; it is pointless to dispute
which of the scientific disciplines describes neuronal behavior. The non-disciplinary approach shows that the firm and fast electric interaction sufficiently well describes neuronal operation after introducing the equivalent "thermodynamic electrical potential".


\index{Schrödinger, Erwin}

Science created abstract concepts such as space and time, mass and charge.
It based its concepts and laws on abstractions that were calculated by approximations and contained \textit{only one} of these concepts. 
The recognition of the discreteness of energy led to quantum mechanics. 
The recognition of the non-independence of space and time led to the creation of the theory of relativity.
Both realizations had far-reaching consequences.
In addition, we understood that under certain circumstances we must treat particles as a kind of continuous wave, and continuous waves as particles.
These micro-objects interact with their environment, including our measuring instruments;
with a measurement method suitable for particle detection, we see them as particles, and with an instrument suitable for wave measurement, we see them as waves.
(How one micro-object sees the other remained cryptic.)
This kind of micro-level behavior is inherent in nature; our methods of description and measurement must be adapted to nature and not the other way around.
We have understood that the Newtonian approach has its limitations; finite-velocity interactions can only be described with a different kind of mathematics, the introduction of the concept of space-time.
In another approach, mass is also related to space and time; taking this into account, we can describe nature with curved space-time.
We can also say that according to classical physics, we can describe phenomena with sufficient accuracy using only the aforementioned concepts (the zeroth derivatives).
According to modern physics, however, for a more accurate description, we must take into account the first derivative with respect to time, 
and even the second derivative with respect to time for a general description.

The so-called modern physics was born when experiments began to contradict the idea that all phenomena of nature could be described in such a simple way.
Neuroscience has similarly accumulated a number of experiences that contradict theory, although mainstream neuroscience has either ignored them or explained them with ad-hoc flawed and unfounded hypotheses.
It happens despite Hodgkin's opinion  “\textit{in thinking about the physical basis of the action potential perhaps the most important thing to do at the present moment is to consider whether there are any unexplained observations which have been neglected in an attempt to make the experiments fit into a tidy pattern}”~\cite{HodgkinConduction:1964}, page 70.
In this field (and in the physical foundation of biology in general), a similar modern approach is needed, which first requires a review of the fundamental assumptions of the underlying physics.

\quotationbox{"We make no apologies for making these excursions into other fields, because the separation of fields, as we have emphasised, is merely a human convenience, and an unnatural thing. Nature is not interested in our separations, and \textit{many of the interesting phenomena bridge the gaps between fields}."\\Feyman R. P~\cite{FeynmanThinking:1980}
}
Classical physics classified phenomena into disciplines that largely described nature well.
Modern physics has revealed new relationships, but they have also organically integrated into the fabric of science as a whole.
To understand the living organism, in the spirit of R.P. Feynman, we must make "excursions" between disciplines. 
\index{Feyman R. P}
E.~Schrödinger expressed his conviction that no new force or unknown interaction is emerging;
only a previously unknown regularity concept must be found,
then the non-ordinary laws organically integrate into the fabric of science, together with the already known laws.
In this chapter we make an attempt to find such a concept in the hope that scrutinizing
the features of ions leads to a solution.
 

Regarding ions,  separately, we can handle their charge and mass well, we know the relevant laws.
Their properties are analogous to the particle-wave duality: depending on the interaction (i.e. the measurement method), we see the ion as either a charge or a mass.
\textit{We see the same object, but we measure it based on the concepts of a different discipline.}
Therefore, we must connect charge and mass, just like space and time in the case of relativity.
They do not exist independently of each other and their combined behavior differs from what we expect based on the classical disciplinary approach.
Interestingly, similar to relativity, the velocity establishes the connection between the two quantities here.

Furthermore, we must introduce a connection for the currents represented by ions, such as the one introduced by statistical mechanics when corresponding to the discrete and the continuum approach.
However, in our case, the correspondence is established based on geometric considerations and, due to the connection of the two properties, we cannot apply statistical laws based on one property in an unchanged form.
Moreover, in most biological cases, the quantities involved are so small that statistical interpretation is severely limited.
The fundamental approximation of physics that the force field acts on the particle, but the particle's effect on the field is negligible, is not met.
In some cases, closed spaces give rise to counterforces that fundamentally change the processes taking place.
\textit{Ignoring these properties of living systems has led to the erroneous statement that science cannot describe living matter and its processes. }
It is undoubtedly more complicated and requires a revision of several approximations, but it is possible.
\textit{Physics is much more than a collection of formulas that
can be applied without thinking about is applicability.}




