\section{Electrodiffusion\label{sec:Physiology-Electrodiffusion}}


The Onsager reciprocal relations express the equality of certain ratios between flows and forces in \hypertarget{ThermodynamicalSystems}{thermodynamical systems} out of equilibrium, but \textit{where a notion of local equilibrium exists}. \textit{This is exactly the case for a neuron during producing an action potential.} The closest relative to our derivation is the Poisson-Nernst-Planck (PNP) and its mathematically simplified version (Poisson-Boltzmann-Nernst-Planck)~\cite{Poisson-Boltzmann-Nernst-Planck:2011} model are based on a \textit{mean-field approximation of ion interactions and continuum descriptions of concentration and electrostatic potential}. Given that the Nernst-Planck equation is essentially a flux equation for the 
special case of zero flux, furthermore that Planck essentially included Fick's second law in the PNP model, our approach seems to be self-consistent
and a significant extension to the famous model. 
As we discussed,
it is not reasonable to calculate a mean value for those vastly different
interaction speeds. We derived a realistic approach
to the ion transport problems,  in many areas; in addition to  biological systems, also for semiconductor devices and nanofluidic systems.



Biological cells comprise components such as electrolytes, semipermeable 
membranes, solutions with expremely different concentration.
Surprisingly, they show spontaneous electrical activity.
As Eq.(\ref{eq:PhysicsGradientRatio}) shows, the electrical interaction
speed is million times higher than the chemical one. 


\subsection{Operating regimes\label{sec:Physiology-OperatingRegimes}}

Our equations also call attention to a neglected aspect of evoking
%\gls{AP}s:
APs: the rush-in ions increase the local potential in the proximal
layer to \emph{above the potential of the bulk in the intracellular
segment}, typically even to slightly above the potential in the bulk
of the extracellular segment. Consequently, \emph{the concentration
must also at least approach or even slightly exceed the level of the
extracellular concentration} for a short period and in a very thin
layer near the membrane (the timing relations were discussed above).
The mechanical waves~\cite{MechanicalWaves:2015} provide indirect
evidence for the effect's existence.

We consider three operating regimes for neuronal membranes. Eq.~(\ref{eq:NernstPlanck})
describes the steady state. As we discussed, in the case of the finite
membrane width of biological neurons, a gradient of a particular form
is created in the electrolyte, also comprising a membrane-width-dependent
term. However, otherwise, the state can be described by Eq.~(\ref{eq:NernstPlanck}). 

In single-shot mode, along the axis of the ion channel, at large distances,
the concentration and potential remain essentially unchanged during
the process. Using our time derivatives, we can describe the details,
including the process's time course. Given that the slowest interaction
defines the propagation speed and the proportion of the layer to the
bulk is extremely tiny, no significant change in bulk can be measured.
The interaction speed in the bulk is practically the \emph{drift}
speed (and the gradients are zero).

In the case of high-rate, repetitive measurements, the changes occurring
in the proximal layers can slowly influence the parameters of the
bulk. However, this effect becomes significant only in long-term observations
when a large number of single actions take place in quick succession.
In a continuous high-rate firing mode, the layers have parameters
other than the ones required by Eq.(\ref{eq:NernstPlanck}) for the resting
state for a growing fraction of the time. We can estimate the time
roughly as how long the ions can diffuse to a distance of 0.1\ mm
(in the order of $\frac{10^{-4}m}{10^{-4}m/s}$), and how many times
that distance is greater that the assumed width of the layer proximal
to the membrane's surface (in the order of $\frac{10^{-4}m}{10^{-8}m}$,
that causes a 100\% change). We arrive at that a rate 100\ Hz will
deliver a charge causing a percentage increase of the bulk concentration
is in the order of at least dozens of seconds. 

\subsection{Connecting science to life\label{sec:Physiology-Life}}


The two layers, plus the demon, see section~\ref{sec:Demon-in-the-membrane}, also naturally explain why that difference
comes into existence. As we explained above, when a \emph{finite-width}
membrane separating the two segments appears in the volume (due to
the evolution or the development of the individual biological object),
two thin electrolyte layers will be formed proximal to its surfaces
on the two sides, even if the concentrations are equal. As observed,
``a membrane potential arises when there is a difference in the electrical
charge on the two sides of a membrane, due to a \emph{slight excess}
of positive ions over negative ones on one side and a slight deficit
on the other.''~\cite{NeuroscienceBook:2001} We add that some potential
difference is created by the presence of the membrane alone, as discussed
above. When a demon also appears in the membrane (initially a simple
hole), the random movement of ions with \emph{finite speed }through
the \emph{finite length} of the ion channels may also solve the mystery
of \emph{\hypertarget{CellToLife}{why a cell comes into life} during evolution}.
Maybe we can answer E.~Scrödinger's profound question: "What is life, and how did it emerge from non-life?"


Erwin Schrödinger's famous book "What Is Life? The Physical Aspect of the Living Cell"
provides a nice example of disciplinary thinking.
Schrödinger's lecture focused on one important question:
"how can the events in space and time which take place within the spatial boundary
of a living organism be accounted for by physics and chemistry?"
He discussed the living cell's operation from the point of view of thermodynamics,
forgetting that the ions have also charges, so the electric interaction
must also be considered. The basic difficulty to consider both of them is
that their interaction speed differs by a factor about a million, and
physics has no proper approximation to handle them simultaneously.
Considering the laws of a single discipline, either theory of thermodynamics,
or theory of electricity, is not sufficient. We must elaborate the way
how they cooperate in the nature, even if we need to elaborate new mathematical
methods for that goal, see section~\ref{sec:Physics-Thermodynamics}.


\subsection{Spatiotemporal behavior\label{sec:Physiology-Spatiotemporal}}

as we discussed in section~\ref{sec:Physics-Spatiotemporal},
