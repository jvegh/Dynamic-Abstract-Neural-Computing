\subsection{Membrane\label{sec:Physiology-Membrane}}

Even at writing this text (at the midle  of 202), textbooks comprise the
\href{https://bio.libretexts.org/@api/deki/files/78169/figure-35-01-02.png?revision=1}{old and bad cellular structure}.


Membranes are fundamental in many places, from biological objects
to industrial filters. They operate on the border of microscopic and
macroscopic worlds, separating non-living and living matters, and
combining electrical and thermodynamical interactions. We show that
\hyperlink{DynamicLayer}{an extremely thin skin near to the surface} of biological membranes
is responsible for the biological thermoelectric processes. 


To describe those complex and continuous phenomena at least approximately,
we must separate them to stages. Using omissions, approximations and
abstractions, we can describe the stages approximately, usually considering
only one dominant phenomenon. The described phenomena are interrelated
in a very complex way and depend on different parameters. To some
point, we can describe that thin layer using a static picture and
providing an empirical description of its individual processes, even
we can give some limited validity mathematical descriptions for those
stages. However, we understand that for describing the transition
(contrasting with step-like stage changes) between those well-defined
stages of the athmosphere we need a \emph{dynamic description} and
we need to find out the \emph{laws of motion} governing the processes.

Similar is the case with the neuronal membranes and the neuronal operation.
Now we are at the point where their decades-old static description
is not sufficient. To descibe the neuron's dynamic behaviour, we need
to derive the corresponding laws of motion. We need a meticulous and
unusual analyzis to derive them. 

In a neuron, in the abstraction science uses, we put together only
ionic solution, semipermeable membrane and currents reaching them.
As experienced, at some combination of their parameters, qualitatively
different phenomena happen, which, in the abstraction biology uses,
called signs of life. Given that the approximations, the derived abstractions
and the mathematical formalisms describing them are different for
the two cases, \emph{it looks like we have two different, only loosely
bound worlds}. However, if we realize we arrived at the boundary of
non-living and living matters, we must go back to the first principles
of science. Using our approach, maybe we can defy that "the
emergence of life cannot be predicted by the laws of physics" ~\cite{ConservationOfInformation:2021}.


The layers, for their regular operation, have both source and drain.
In neurons, the source is distributed over the surface of the layers
and the drain is concentrated at the terminating end of the layer.
The two currents are flowing simultaneously, i.e., the source of the
drain current has a time course, so the product of the two currents
can be measured. (actually, it is a differential equation, and in
the elementary cross-section, Kirchoff's Junction Law is valid). Generally,
it takes time until the source current reaches the drain's position.

Initially, biology used the abstraction that the measured resistance and capacitance are distributed
along the membrane's surface. It assumed a \emph{discrete} equipotential
membrane with capacity $C$ and that it leaks through a discrete resistance $R$. 
However, in biology, no discrete elements for storing charge exist.
The notion of storing charge can be used only in the sense that for
the time of passing a finite-size element with finite propagation
speed, the charge carriers spend the corresponding time in the element.
That phenomenon resembles storing the charge, and that imitation enables
us to describe a behavior resemblant to that of the biological circuit.
\emph{Attempting to imitate the effects of biological ``slow'' currents
	using electric parallels hides that generating an 
	\gls{AP} is their native feature};
furthermore, slow currents may also play a role in cognitive functions.

