\documentclass{book}
\def\MyVersion{0.1.2}
\usepackage{xkeyval}
%\tolerance 500%
%\emergencystretch 1em\relax
%\hfuzz .2pt\relax
\usepackage{makeidx}
%\usepackage{eqnarray,amsmath}
\usepackage{url}
\usepackage{hyperref}
\usepackage{glossaries}
\makeglossaries
\makeindex
%\usepackage{wrapfig}
%\usepackage{changepage}
%\usepackage{tcolorbox}
\newglossaryentry{latex}
{
    name=latex,
    description={Is a markup language specially suited
    for scientific documents}
}

\newglossaryentry{maths}
{
    name=mathematics,
    description={Mathematics is what mathematicians do}
}

\newglossaryentry{real number}
{
  name={real number},
  description={include both rational numbers, such as $42$ and
               $\frac{-23}{129}$, and irrational numbers,
               such as $\pi$ and the square root of two; or,
               a real number can be given by an infinite decimal
               representation, such as $2.4871773339\ldots$ where
               the digits continue in some way; or, the real
               numbers may be thought of as points on an infinitely
               long number line},
%  symbol={\ensuremath{\mathbb{R}}
  }
\usepackage{latexml}

   \title{
Dynamic Abstract Neural Computing\\with\ Electronic Simulation\\(DANCES)  Version \emph{\MyVersion}}
%\subtitle{The true physics, physiology, computing \& information science\\behind neuronal operation}

\author{János Végh}
\date{\today}
%============================================================
%============================================================
\begin{document}\label{top}

\printglossary[title=Special Terms, toctitle=List of terms]
\clearpage
\part{First}
\chapter{first/1}
\Gls{maths} \gls{real number}
\section{What's}
\part{second}
\chapter{second's}

in my last~\cite{VeghMembranePotential:2025}
\listoftables
\listoffigures

\backmatter
\printindex%[title=Index, toctitle=Index]

\bibliography{
	~/REPO/LaTeX/CommonBibliography%
	,~/REPO/LaTeX/CommonPrivateBibliography%
	,~/REPO/LaTeX/CommonNeuronalBibliography%
	,~/REPO/LaTeX/CommonAIBibliography%
}
\end{document}

\bibliography{
	CommonBibliography.bib%
}

% \bibliographystyle{spphys}
%\bibliographystyle{plain}
\bibliography{
	~/REPO/LaTeX/CommonBibliography%
	,~/REPO/LaTeX/CommonPrivateBibliography%
	,~/REPO/LaTeX/CommonNeuronalBibliography%
	,~/REPO/LaTeX/CommonAIBibliography%
}

