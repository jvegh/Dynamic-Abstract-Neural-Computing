% This is the Foreword to the 'Abstract Dynamic Neuron' book

\chapter*{Foreword\label{ch:Foreword}}
\addcontentsline{toc}{chapter}{Foreword} 


\quotationbox
{"The Human Brain Project should lay the technical foundation for a new model of ICT-based brain research, driving \textit{integration between data and knowledge from different disciplines}, and catalysing a community effort to achieve a \textit{new understanding of the brain}%, new treatments for brain disease 
\dots and \textit{new brain-like computing technologies}."\\
{the \href{https://www.humanbrainproject.eu/en/}{Human Brain Project},  summarised its goal @2012}}


\paragraph*{Understanding the dynamic brain}
The \textbf{\textit{dynamic}} operation of individual neurons, their connections, higher-level organizations, connections,
the brain with its information processing capability, and finally, the mind with its conscience and behavior,
are still among the big mysteries of science: \textit{at which point
\hyperlink{LivingMatter}
{the non-living matter becomes a living one}}, \textit{at which point \hyperlink{CognitiveMatter}{the living matter becomes intelligent}} and conscious; whether and \hyperlink{PhysicalPicureNeuron}{how science can handle} all this stuff. 
We really need a \hyperlink{NewUnderstanding}
{new understanding}.
%
The hype in the newly launched projects is excessive and seems to lose the hope 
to build brain research on a firm science base.
For example, the newly (at the end of 2025) launched “\href{https://brainminds.jp/en}{Brain/MINDS 2.0}” program in Japan was launched "with the goal of developing a marmoset model for neuroscience", using only mathematical models, \textit{without targeting the understanding of the underlying physical processes}. The case is much similar 
to the 'Theory of Everything' in physics, which is developed since science exists and is a never-ending project. Likely, the length of the time needed for neuroscience will not be much shorter, although the project period is limited to years.

In neuron models, dynamics refers to how the state variables of the neuron (primarily membrane potential and ion channel properties) change and evolve over time in response to internal processes and external inputs. However, unlike in our approach, the usual processes simply use a time parameter in empirical or mathematical functions, see for an example~\cite{NeuralDynamicsGertsner:2014}. We use a real dynamics,
based on physics, in the sense as Newton introduced his laws of motion to describe
response of neurons.


\paragraph*{Abstract discussion}

Nature uses an infinite variety of implementing neurons. Our goal is not to discuss all
infinitely complex molecular, biochemical and physiological details
of neuronal operation; it is not necessary, since in the \gls{CNS} 
they can cooperate
with each other. 'Despite the extraordinary diversity and complexity of neuronal morphology and synaptic connectivity,
\textit{the nervous systems adopt \textbf{a number of
 basic principles}}
for all neurons and synapses'.
\cite{JohnstonWuNeurophysiology:1995}
\index{Johnston Daniel}
\index{Wu Samuel Miao-sin }
We base our holistic discussion on those 
general basic principles
and create an 
\hyperlink{NeuronPhysical}
{'abstract physical neuron'},
skipping the 'implementation details' nature uses.
We need \hyperlink{Abstractions}
{different abstractions and approximations} for describing biological processes. However, an abstraction is usable
in practice only when paired with a generalization: the more abstract the assumption, the more general and widely applicable the concept or conclusion.
\textit{By abstractions, we can reduce the unbelievably detailed world into manageable pieces, and by abstraction, we can learn anything general.}
We must show which approximations are oversimplifications and which phenomena are misunderstood; measured, or interpreted in the wrong approach.
 Abstraction is, in fact, everywhere, including inside each of us. It is a core element of cognition. 



\paragraph*{Zigzag reading}
For the same reasons, we follow von Neumann's method when he described principles of technical computing~\cite{EDVACreport1945}. "The ideal procedure would be, to take up the specific parts in some definite order, to treat
each one of them exhaustively, and go on to the next one only after the predecessor is completely
disposed of. However, this seems hardly feasible. The desirable features of the various parts, and
the decisions based on them, emerge only after a somewhat \hypertarget{zigzagdiscussion}
{zigzagging discussion}. It is, therefore,
necessary to take up one part first, pass after an incomplete discussion to a second part, return after
an equally incomplete discussion of the latter with the combined results to the first part, extend
the discussion of the first part without yet concluding it, then possibly go on to a third part, etc.
Furthermore, these discussions of specific parts will be mixed with discussions of general principles, of the elements to be used, etc."
\index{von Neumann!'zigzag' way}
For example, basic concepts such as the membrane potential and its regulation in resting and transiens states, belong to chapter Physics,
and their detailed scientific, physics-based, discussion is given there, 
but their corresponding aspects must be discussed in chapters 'Abstract neuron' and less abstract 'Physiology' as well.
The contents are inherently interwinned, and the form must follow the contents.
We make this zigzag reading more accessible by using hyperlinks and cross-references throughout the document. 



\paragraph*{Simulator}The site is not exclusively about theory: we also give a programmed implementation of the ideas we describe.
\href{../../docs/index.html}{Our simulator}
%@link SIMULATION_SYSTEMC Our simulator @endlink 
has a direct scientific base instead of ad-hoc mathematical formulas;
and the only one which can reproduce 
the true biological time course
%@link SIMULATION_SYSTEMC the true biological time course@endlink 
of neurons,
from the first science principles, without arbitrary ad-hoc assumptions and limited varlidity formulas.
Our methods enable discussing the major aspects of phenomena of the natural operation of neurons
to analyze the effects of invasive electricity-related investigation methods on neuroscience.
We offer demos, class implementations, performance benchmarks, and test cases
to demonstrate simulating capabilities. We intend to develop full-value
educational, demonstration, and research tools. 




\warningbox
{Please consider that this development is a one-person undertaking.
Moreover, it shall develop theory, evaluate published experiments, implement software, test it, and document it.
Pre-developed code fragments, science publications, and docs exist, so the site develops relatively quickly,
but we need time to put them together consistently.  Please return later and see if something is new (see the date and the version).
}


