% Physics of the components of the abstract model

\section{Neuronal components\label{sec:Physics-Components}}

In this section, we discuss statical and dynamical
components, attribute a new role to the 
\gls{AIS}, and introduce the dynamic layers on the surfaces of the membrane.
Fundamentally, only electrical and thermodynamic forces exert on ions,
but the biological environment decisively affect their movement.
As discussed in section~\ref{sec:Physics-Thermodynamics}, in biological systems, a complex set of forces exerts on ions, see Eq.(\ref{eq:NernstPlanckExtended}). 
We derived the "thermodynamic electrical field" in section~\ref{sec:Physics-ElectrolyteThermalField}, so we use the
concept of "thermoelectric field" to describe the exerting forces.
Considering the forces, 
orients us to discover the phenomena and their reasons. 
As discussed in section~\ref{sec:Calculating-ion's-speed}, 
the ions move with their Stokes-Einstein speed, proportional with
the generalized electrical field, see Eq.(\ref{eq:StokesCurrent}),
so enormously different speeds form the neuronal operation.
Furthermore, because of their cooperation, it is hard to separate the 
operation of the components.

The dynamic components do not fit in the 'old understanding'.
Physiology inherited the static view of anatomy and developed
its own static methods of observation (clamping, see section~\ref{sec:PHYSICS_clamping}), as discussed at several 
points of the site. That concept has no place for dynamically
formed components (a consequence of the finite neuron size and ion speeds).
Recall that, in physics, the \href{https://en.wikipedia.org/wiki/Drift_velocity}{drift speed} and the \href{https://en.wikipedia.org/wiki/Thermal_velocity}{thermal speed} differ by several orders of magnitude; furthermore, that they describe the same object's macroscopic and microscopic movement
However, they are interpreted for a system of neutral particles; the ions
behave differently. We also introduce the electrical \gls{potential-assisted speed}
(scales to between the drift speed and thermal speed)
and the \gls{potential-accelerated speed} (scales to well above the thermal speed) of
ions that differ by  several orders of magnitude.
Of course, these speeds have no well-defined values,
but they orient us in discussing the ions' behavior.
Speed plays a decisive role in biology~\cite{RoleOfInformationTransferSpeed:2022}.


\subsection[Animal electricity]{Animal electricity\label{sec:PHYSICS_ANIMALELECTRICITY}}

\begin{advanced}
“All animals that move have electricity in their bodies.
Electricity is the only thing that’s fast enough to carry the messages
that make us who we are. Our thoughts, our ability to move,
see, dream, all of that is fundamentally driven and organized by
electrical pulses. It’s almost like what happens in a computer but
far more beautiful and complicated.”
--(neuroscientist Rodolfo Llinas quoted in “Electricity’s spark
of life” by Emily Sohn in Science News for Students, 29
September 2003).
\end{advanced}

(We note that the communication, of course, can be seen as electrical pulses
when studied by discipline 'electricity' and can be seen as
mechanical pulses (solitons) when sudied by another discipline.
The speed of biological electricity is million times slower
that the speed of electromagnetic pulses, and matches speed of
the sound in biological fluid. 
The waves of the fluid show electrical behavior because 
the waved medium contains also ions.)

The phenomenon that the body operates with electric signals was discovered about one and a half century ago~\cite{HelmoltzHistory:1851},
and the idea that "In simple cases of ionized substances both the amount of
substance and the force acting may be expressed in electrical terms"~\cite{COLE_CURTIS_IMPEDANCE:1939} shined up nearly a century ago. 
The basic idea of \hypertarget{BiologicalCurrent}{biological current} was correctly defined at the beginning:
"The permeability of a membrane to a penetrating substance is
given quantitatively by the amount of the substance which crosses
a unit area of the membrane in unit time under the action of a unit
force. In simple cases of ionized substances both the amount of
substance and the force acting may be expressed in electrical terms.
Then the permeability may be ultimately converted into coulombs
per second for a square centimeter and a potential difference of 1
volt, which is the conductance, in reciprocal ohms, for a square
centimeter"~\cite{COLE_CURTIS_IMPEDANCE:1939}.
However, at that time physiologically defined fine details were not known.
\index{conductance}
\index{current!biological}
One must never forget that "\textit{movement of
ions} across the plasma membrane \textit{results in changes of electrical potential}
across the membrane, and these potential changes are the primary signals
that convey biological messages"~\cite{JohnstonWuNeurophysiology:1995}
(\textit{using "\indexit{equivalent electrical circuit}s" hides that the potential changes}).
In this section, we proceed along the line pointed by those latter authors in Chapter 2 of their book. 

We must recall that \textit{this charge is delivered chemically,
i.e., it is the result of an thermoelectric process}, so the concentration and the potential change simultanously.
Also, \textit{the charge carriers are ions instead of electrons and the transfer mechanisms are entirely different}.
From purely simplistic considerations concerning the  many orders of magnitude in the ratio of their size and mass, we see that the charge carriers have speeds differing by orders of magnitude, that requires revisiting the classic laws such as instant interaction (behind the ideas such as Kirchoff's laws, Ohm's law, 'retarded potential').
Furthermore, the ions are not fixed to isolating surfaces, and some biological "constructions" may be active elements from the point of view of electronics: the membrane may store charge  (disappears) for some period and ion channels may "produce" charge in the measured system.
Also, the molecules may dissociate and polarize under the effects of local potentials, enabling to create macroscopically different parameters such as potential and concentration, see also \hyperlink{Onsager_relations}
{Onsager's reciprocal relations}~\cite{OnsagerExperimental:1959}. 
In the case of living matter, we must handle thermodynamical and electrical forces simultanously. The are no 'net abstract' cases
where only electrical forces act on the ions. This feature, accompanied by the slowness and large size of the ions, requires attention and rethinking the basic notions of electricity.

We must also call attention to the dynamic behavior of charges
inside living matter. Basically, they are balanced, but, and it is vital for describing the life, they can get unbalanced in a local region for shorter or longer periods. This balanced state is the basis for having a resting potential (longer), and the perturbed (unbalanced) state is the basis for creating and transferring action potential, among others. \textit{The change of those states are described by the laws of motion, and their continuous change is the life itself.}

Although it is fundamentally correct that "because dissociated ions carry electric charges, their movement is influenced not only by concentration gradients but also by electric fields", these gradients are competing with each other and their complex interaction controls the biological processes. "Based on thermodynamic principles,
ions tend to flow from regions of high concentrations to regions of low
concentrations" (see~\cite{JohnstonWuNeurophysiology:1995}, page 9), furthermore, the electric gradient gets more role than presumed; see \hyperlink{Onsager_relations}
{Onsager's reciprocal relations}~\cite{OnsagerExperimental:1959}. 
We must never forget that \textit{ions represent both charge and mass,
so a current means delivering mass and vice versa}.
%
In some cases, when precisely measuring the time course of current
compared to the time course of voltage, one can experience a 'phase delay'
between them. This may inspire further research, such as inventing
'inductive', 'capacitive' and 'resistive' currents in electricity; or one may believe
that the system shows 'non-ohmic' behavior in biology; that is, it cannot be described
by laws of physics.

An often forgotten thought is that
"The things that neurophysiologists typically want to measure
are electrical signals such as action potentials and synaptic potentials,
or the membrane currents responsible for these potentials. \textit{Under ideal
circumstances, the physical act of measuring a neurophysiological event
would have no effect on the electrical signal of interest. Unfortunately,
this is seldom the case in neurophysiology.}"~\cite{JohnstonWuNeurophysiology:1995}, Appendix A. We do not want to repeat the content of that appendix, except some
points where we add notes, corrections or pinpointings.
We add, however, that neurophysiologists make measurements in
hybrid circuits, where the charge carriers are ions in one half of the circuit and electrons in the other. The conversion introduces several issues, among others delays (resulting 
in measuring non-matching value  pairs); introducing negatively
charged high-speed particles into systems having positively charged low-speed particles; intermixing inhomogenous, non-isotropic, structured matter into the systems and applying laws
derived for homogeneous, isotropic, unstructured matter; applying laws derived for the "free electron cloud" to the case of slowly moving dipole molecules.

It is worth to recall: "\textbf{Accuracy}: The degree to which a measurement indicates the true magnitude of a measurable quantity. \textbf{Precision}: The resolution and reproducibility of a measurement; implies nothing about accuracy. A measurement can be precise without being accurate. The reverse, however, is usually not true."~\cite{JohnstonWuNeurophysiology:1995} We add: sometimes, we measure a quantity which differs from the one we wanted to measure. Especially, if we interpret erronously what we wanted to measure and how do we measure it. That is exacly the case when the model is wrong.



\subsection{Ion channels and pumps\label{sec:Physics-IonChannels}}

"The function of \hypertarget{Ion_Channel}{ion channels} is to allow specific inorganic ions
to diffuse rapidly down their electrochemical gradients across the lipid bilayer...
\index{lipid!bilayer}
Nerve cells (neurons), in particular, have made a specialty of using ion channels,
and ... use a diversity of such channels for receiving, conducting, and transmitting signals...
\textit{Ion channels cannot be coupled to an energy source to perform active transport,
so the transport that they mediate is always passive ('downhill')}" \cite{MolecularBiology:2002}.
(We note that the 'downhill' transport requires moving
the ions in a viscous fluid against friction, so it requires energy.
Due to the finite resources, discussed in section~\ref{sec:Physics-Resources}, the moved charge changes its potential energy to kinetic
energy plus dissipation. That potential energy (the membrane's potential) is the source to perform a 'passive' transport, given that the ion affects the membrane's field; for details see section~\ref{sec:Physics-BackgroundLogistics}.
The 'passive transport' would defy energy conservation.) 

The role of ion pumps must be revisited, too. According to
biology,
the ion channels and ion pumps have different charge transmission mechanism.
"These pumps differ from ion channels in two
important details. First, whereas open ion channels
have a continuous water-filled pathway through which
ions flow unimpeded from one side of the membrane
to the other, each time a pump moves an ion, or a group
of a few ions, across the membrane, it must undergo a
series of conformational changes. As a result, the rate of
ion flow through pumps is 100 to 100,000 times slower
than through channels. Second, pumps that maintain
ion gradients use energy, often in the form of adenosine triphosphate (ATP), to transport ions against their
electrical and chemical gradients. Such ion movements
are termed active transport."~\cite{PrinciplesNeuralScience:2013}, page 101.
\index{ion transport}
The so called "Na-K pump" \textit{is supposed} to actively transport (using energy from \gls{ATP})
sodium ions out of the cell and potassium ions into the cell.
However, in the old model no mechanism is provided how the ions gain energy from \gls{ATP} and how the ions are moved (what are the \textit{laws of motion} of ions in those ion channels and what a force (see Eq.(\ref{eq:IonicForces0}) 
can "transport ions against their electrical and chemical gradients").
Instead, some magic
mechanism by protein conformations \textit{is supposed}
without explaining what is the driving force to do so,
how it can deliver the ions in
sufficiently short times needed for the operation,
ans what is the source of the energy needed to move 
the huge amounts of molecular masses,
needed for implementing .





\subsubsection{Ungated channels\label{sec:Physics-UngatedChannels}}


By their function, there are resting (located in the wall of the 
plasma membrane, with low density) and transient (located at the beginning of the axon, with high density) ion channels.
It is believed that the task of the resting ion channels is maintaining ion gradients
in the resting state and establishing the negative resting membrane potential.
Actually, as discussed in section~\ref{sec:Physics-ElectrolyteThermalField}, in the resting state, instead of some magic protein mechanism (ion pumps), the resultant potential
\index{protein!mechanism}
(the sum of the electrical and thermodynamic potentials) moves the
the ions through the channels in the membrane.
The resulting conductance (sumed above the membrane's surface) of those channels
is about 50 times smaller~\cite{ActionPotentialGenerationNatrium:2008,AIS_Updated_Viewpoint:2018} than that of the \gls{AIS} (summed around the beginning of the axon).
In a typical mammalian cell, see Table~\ref{Tab:SummaryTable},
the sum of the electric and thermodynamical potentials are 
\SI{-32}{\milli\volt} %$-32\ [mV]\
for $K^+$ ions and \SI{+4}{\milli\volt} %$+4\ [mV]\
for $Na^+$ ions.
The same values for squids are 
\SI{-33}{\milli\volt} 
%$-33\ [mV]\ K^+$
and \SI{+13}{\milli\volt}.
%$+13\ [mV]\ Na^+$.
Correspondingly, given that both ions have positive charge, the resulting force moves sodium ions out of the cell and potassium ions into the cell.
In our "new understanding"
it is clear physics: the resulting force moves the ions in and out,
simultanously.
The "ion pumps" are ordinary ion channels, with two-way ion traffic,
depending on the local gradients around the two ends of the 
ion channel. Their operation is the result of the complex interplay
of the ion layers (discussed in section~\ref{sec:Physics-IonLayers})
and the local field (see Eq.(\ref{eq:StokesSpeed})).

The different speeds play a significant role in the correct operation and
the cooperation of different neuronal objects, including ion channels
in the walls of membranes and axons. We discuss 
their effect also in section~\ref{sec:Physics-ResourcesIonChannels}: they affect also the resource-availability of ion charges in the electrolyte segments.
Given that the "transport efficiency of ion channels
is  $10^5$ times greater than the fastest rate of transport mediated by any known carrier protein"~\cite{MolecularBiology:2002}, we can consider that speed as 'infinitely fast' compared to the speeds of neuronal ion currents.
Biology observes the effect of speed of ion transport:
"\href{https://www.ncbi.nlm.nih.gov/books/NBK26910}
{transport efficiency of ion channels}
is $10^5$ times greater than the fastest rate of transport mediated by any known carrier protein"~\cite{MolecularBiology:2002}.
It sees that there is an enormous accelerating voltage
(a \SI{10}{\mega\volt\per\meter} electric field)
across the plasma membrane, but does not connect that voltage
to the ions' charge and the high transport speed.
The ions in biological electrolytes do not obey laws of electricity.

The above numbers are valid for the resting state.
In the resting state (a dynamically balanced state), due to the local charge transfer, the locally potential distribution migh significantly differ from the global potential, so the effective driving force 
in average, is close to zero; resulting in a very low transport speed, as observed. 

For the transient state, the case is different. The membrane potential
can be below and above the resting potential, so the
resultant force can be either positive or negative,
changing the direction of both currents.
As Figure 6 in \cite{NeuralEnergyConsumption:2017} displays,
the ratio of $K+$ and $Na+$ changes sharply during the transient state.
Theoretically, they assume that the neuron “pumps 3 $Na^+$ ions out of the cell and two potassium ions in”; experimentally, they show in their Fig. 6 that the ratio changes between 0.01 and 7.5.
The classical theory cannot explain this behavior and leads
to exciting conclusions:
"Furthermore, we analyzed energy properties of each ion channel and found that, under the two circumstances, power synchronization of ion channels and energy utilization ratio have significant differences. This is particularly true of the energy utilization ratio, which can rise to above 100\% during subthreshold activity."~\cite{NeuralEnergyConsumption:2017}

Our model says that both the concentration (due to $Na+$ rush-in) and local membrane potential drastically changes
during the transient state. As long as the resultant potential is above
\SI{\approx 33}{\milli\volt},
the driving force acting on potassium ions is
positive, i.e., the pump will move both ions out of the cell.
The potential gradient near the membrane explain also the energy delivery: \gls{ATP},
by hydrolysis, generates ions which are moved
to the "plates" of the condenser. (as we discuss in section~\ref{sec:Single-NeuronControlCircle}, in transient state,
the "setpoint" of the control circuit is also changed, and to restore the condenser's voltage, those ions must be collected.)



The rapid influx of ions
causes a sudden increase in the potential on the intracellular side.
\index{layer!diffusion}
Conversely, the ions' removal from the layer on the extracellular
side near the membrane 'empties' the layer (and so: suddenly decreases its potential), and the after-diffusion
(despite the large concentration difference) with the low drift speed
(even if it is assisted by the repulsion of the fellow ions)
takes time. Because of the slow after-diffusion, the transfer stops well before
the ion channels get inactivated.
See also the operation of \hyperlink{voltage-clamping}{clamped axons}: removing the surface ion layer enables the membrane
to prolong its 'open' state (again, in statistical sense).
\index{layer!diffusion}
Basically, the diffusion speed in those layers
(in a statistical sense) and the lack or presence of ions in the proximal layer,
defines the 'open' and 'closed' states of the channel population.
The ion channels have three states, but their population has only two.
One can hardly interpret a third state of ion channels without considering
the effect of the membrane's charged layers as we discuss below.


It is hard to separate the operation of the individual channels
from the operation of their population in the walls of membranes \hyperlink{membrane_layers}{(layers)}.
\index{layer!on membrane}
When ions pass through the channel, they face two effects on the two sides of the membrane.
On the side of departure with high concentration (where they of course have high electric potential), they suddenly "empty"
the thin layer in the immediate vicinity of the membrane (for a detailed discussion see section~\ref{sec:Physics-ResourcesIonChannels} about the effect of finite resources). On the side of arrival with low concentration,
again, the arrived ions suddenly form a 'filled' thin layer.
The ions in both segments can move only with their corresponding diffusion speed
(in the order of $10^{-4}\ m/s$) but in the presence of a voltage gradient they experience each other's electric repulsion
that can speed up their speed to the range $1\ m/s$.
(BTW: this effect can be interpreted
as a sudden ion adsorption \cite{Hodgkin-HuxleyAdsorption:2021}
\index{ion adsorption}
on the surface of the membrane.) The final effect resembles an electric condenser:
\index{electric condenser}
for a short time, layers with opposite charges are formed on the two sides
of the semipermeable isolator membrane, which are canceled
in the frame of issuing an \gls{AP}.
\index{Action Potential}
The two layers
attract each other, so the ions in the layers can diffuse toward their respective neighboring layers 
only moderately. 




\subsubsection{Gated channels\label{sec:Physics-GatedChannels}}
For cardiac \gls{AP}s,
where only a few ion channels participate,
"the slow currents appear to
have been caused by repeated openings of one or more channels"~\cite{CardiacAPS:1980}.
For neuronal 
\gls{AP}s,
where many ion channels participate,
"the durations of channel opening and closing vary greatly";
furthermore,
"the rate at which current flows through an open channel is practically constant"~\cite{MolecularBiology:2002}.

%, reproduced here as Fig.~\ref{fig:Single_VoltageGatedChannel}.
The ion channels are either closed or open without
a noticeable transition state, but as discussed in~\cite{ThreeStateUnidirectional:2004, MarkovianIonChannel:2005}, for their adequate description three states are needed:
they can also be in inactivated state.
We can consider the channel operation as "infinitely fast" compared
to the speed of processes in front and behind of the channel: the massive difference in speeds
explains why ion channel opening and closing resembles a 'digital
operating mode'.
%@link PHYSICS_SPEEDS The different speeds@endlink
The different speeds play a significant role in the correct operation and
the cooperation of different neuronal objects, including ion channels
in the walls of membranes and axons.
%As we discuss
%in section~\ref{sec:Demon-in-the-membrane}, ion channels behave in a mystic way: from some point of view, they behave as 
%\hyperlink{Maxwell-demon}{'demons'}.


Experimental evidence shows that although
\href{https://www.ncbi.nlm.nih.gov/books/NBK26910/bin/ch11f32.gif}
{"the durations of channel opening and closing vary greatly,
	the rate at which current flows through an open channel is practically constant} \cite{MolecularBiology:2002}.
The presence of two layers on the opposite sides of the membrane
actually implements the control square-ware signal on the figure.
Those layers also explain why the ion channels (in a statistical sense)
behave as digital, despite that the individual ion channels are not digital.


\begin{figure*}
\includegraphics[width=.65\textwidth]{fig/ch11f32.jpg}
		\caption{Patch-clamp measurements for a single voltage-gated $Na^+$ channel 
			(Fig.~11.32 in~\cite{MolecularBiology:2002}). Notice that due to the low current intensity, the measurement method comprises integration that "smears" line shape. Also notice that the shape of the aggregate current comprises
different time contributions due to the different distances
of channels from the measurement tip.
			 \label{fig:Single_VoltageGatedChannel}}
\end{figure*}

As Fig.~\ref{fig:Single_VoltageGatedChannel} depicts,
an ion channel is open roughly for~$10^{-2}\ s$~and the 
peak amplitude is~$1.6*10^{-12}\ A$, so the maximum charge
that can be transferred in a single shot is~$1.6*10^{-14}\ Cb$,
which assumes~$10^5$~ions per shot per channel. 
(If we assume that the ions
pass the ion channel one by one, without pausing, the passage time of an ion is
\SI{0.1}{\milli\second}.
%is $10^{-7}\ s$.
Given that the electrolyte electrodes contribute 
a considerable delay, the value might be not accurate.)



The entered ions cannot leave sufficiently quickly 
the proximity of the channel's exit, so their potential prevents
the rest of ions from entering the membrane: the ion layer closes the channel.
Recall that the 
number of the uncompensated ions is about~$5*10^7$, the several
(typically several hundreds) simultanously working ion channels overcompensate the ion balance:
despite the forceful external potential, no more ions can enter
the membrane (the 'downhill' gradient cancels; if we use the approximation that the ions traverse the channel length instantly).
In the rest of the shown period, despite that the external voltage
(the square wave) is still present, the potential of the stalled ions
(the internal voltage) keeps the channel closed: no new rising edge (voltage gradient) arrives.
In the figure, the aggregate current enables us to estimate the 
total number of channels to be~$100-120$. 
Clearly, the aggregate current shows two effects: the individual ion channels' contributions, which are measured within the channels, have steep rising and falling edges. As the figure depicts, after~$20\ ms$~no more new openings happen.
Following that, the currents flow toward the point where
the aggregated current measured:
the charge on the membrane quietly discharges. 
(The reason of the decay is the same as in the case of
the \gls{AP}
a slow current flows on the surface; so their time constant is the same, althought the fluctuation due to the arrival time of the contributions
of the individual channels is more observable.)    Again, the external voltage
is stable, the internal voltage decreases, so the aggregate current decreases. The membrane's conductance does not change. However,
the resulting potential that directs the current is the sum of
the external and internal potentials. As the charge diffuses toward
the bulk region, the internal potential decays, and so does the measurable
aggregate current. 


Although the individual
ion channels open and close 'randomly', the repulsion force
on the two surfaces of the membrane acts as an additional valve; its discussion see in section~\ref{sec:Physics-ResourcesIonChannels}.
As \cite{MolecularBiology:2002} discusses,
'this potential difference ... exists across a plasma membrane only about $5\ nm$ thick,
so that the resulting voltage gradient is about $100,000\ V/cm$'.
\index{voltage gradient}
In a statistical sense, part of the ion channels can be open
after the population members received the 'open' signal, part of the population can be closed
or inactivated,
but when the layer enables, the ions in the proximal layer
can escape to the other side of the membrane.



It is also known that  for their adequate operation, the ion channels need to implement
three states: in addition to the 'on' and 'off' states, they can also be in an inactivated
state~\cite{ThreeStateUnidirectional:2004, MarkovianIonChannel:2005}.
However, the population of the ion channels has only 'on' and 'off' states;
furthermore, for some reason the population get "fatigued":
"the probability, that any individual channel
will be in the open state, decreases with time"~\cite{MolecularBiology:2002}. 
It is due to the finite resources, as we quantitatively discuss it in section~\ref{sec:Physics-ResourcesIonChannels}.

%\cite{HilleIonChannels:1999}

As Fig.~\ref{fig:Single_VoltageGatedChannel} depicts, it is the gradient (the rising edge), 
instead of the membrane potential, which 
starts the individual patch currents (and, of course, 
the aggregate current).
Depending on the environment of the channel's exit (the fluctuation
of the charge density), the channel has an a maximum 'let-in' time.
The representative patch currents show that the channel can definitely 
and quickly open, 


\subsubsection{Ion layers\label{sec:Physics-IonLayers}}

Semipermeable membranes, with ion channels in their walls, separating electrolyte segments with ion concentrations differing by orders of magnitude,  play a unique role in neuronal electric operation.
It is at least problematic to interpret the operation of the individual channels without understanding their dynamic
interaction with the other channels, the electrolyte, and the semipermeable membrane. 

We consider the external concentration constant: the extracellular
space is infinitely large, and the amount of ions remains by orders
of magnitude higher than the internal one. Our assumption is valid for the \textit{global static} concentration (we call it 'bulk'), but not for the \textit{local dynamic} one.
The voltage-controlled ion channels open when
on the lower concentration side, the local voltage exceeds some threshold value.

In the \textit{resting state} (without a voltage offset around the ion channels), the channels keep balance between the separated segments. 
However, when an ion channel gets open (meaning that ions from the high-concentration side can
pass through it to the low-concentration side), for a short period,
the ions change the \textit{local} concentration and potential of the electrolyte in the
proximity of the entrance and exit of the channels,
forming \hypertarget{membrane_layers}{two proximal layers}. 
The case drastically changes if an additional potential gradient appears.
In that case, (part of) the ion layer, formed on the
membrane's surface due to the charge arriving through the ion
channels, is  continuously removed by the macroscopic ion current from the immediate
proximity of the ion channels. The layer gets saturated later, and
the conditions of transferring ions
through the channels persist for longer, so they remain open, enabling a continuous
ion inflow (a macroscopic current; see the
discussion about  \hyperlink{voltage-clamping}{clamping dynamic operation} using  
\gls{AIS}).

The ion channels have three states, but their population has only two.
\index{ion channel!state}
Fundamentally, the lack or presence of \textit{unbalanced} ions in the proximal layers defines the 'open' and 'closed' states of the channel population. The individual
ion channels open and close in a stochastic way. In a statistical sense, part of the ion channels can be open, and another part can be closed or inactivated.
However, only when the layer's potential enables, can the ions in the proximal layer
escape to the other side of the membrane, even if the channel is open. 
The ion channels have no reason to re-open because of the lack of offset voltage (and that layer).
That is, primarily, the presence of the layers on the two sides of the membrane defines the ion inflow,
and the individual ion channels  can freely (re)open, close, or inactivate
until the layer provides a sufficiently large potential offset.
This transient state is the key to understanding the dynamic operation of neurons.

There is a strong electric field on the boundary of the segments. 
As \cite{MolecularBiology:2002} discusses,
'an electrical potential difference about 50–100 mV ... exists across a plasma membrane only about $5\ nm$ thick,
 so the resulting voltage gradient is about $100,000\ V/cm$'.
\index{voltage gradient}
In their 'off' state, the voltage-controlled ion channels are mechanically closed, so the ions cannot follow that gradient.  
However, when (due to the collected synaptic charge or the significant slope of the arriving spike~\cite{LosonczyIntegrative:2006} or clamping) a voltage offset appears at the  ion channel, so it opens. Due to the enormous gradient, ions rush in
from the extracellular segment into the intracellular one. This means a high speed, that is, a 'fast' current, see Eq.~(\ref{eq:StokesCurrent}).

However, upon arriving at the other side of the  membrane, they experience the
electric field disappearing, so the stream of ions stalls. The stalled ions 
increase the local potential (see section~\ref{Physics-OscillatorDifferentiator}) around the  channel's exit,
and the ions will move along the parallel potential gradient toward neighboring 
channel exits. 
'The description just given of an action potential concerns only a small patch of plasma membrane.
However, the self-amplifying depolarization of the patch, 
is sufficient to depolarize neighboring regions of the membrane,
which then go through the same cycle. In this way,
the action potential spreads as a traveling wave from the initial site of depolarization
to involve the entire plasma membrane' \cite{MolecularBiology:2002}.
The depolarization happens in an avalanche-like way \cite{NeuronalAvalanches:2003} over the entire membrane surface.  
This process creates ion-rich layers in the proximity of the membrane
on both sides. At the end of the process, the potential in the layer on the intracellular side temporarily reaches
the  potential inside the bulk of the extracellular side.
\index{layer!potential}
The ions in the layer experience two forces: in the direction 
parallel to the membrane's surface, the electric repulsion due to the fellow ions in the same layer; furthermore, in the perpendicular direction, the attraction of the ions
in the opposite layer.

The first force acts in distributing the potential uniformly  over the surface, and \textit{in this way (per definitionem), an ion 
current flows in parallel with the surface}. This ion current is slow: the ions are moving 
in a viscous solution under the effect of a potential gradient (see Eq.~(\ref{eq:StokesCurrent})), if any. In the lack of external potential, it is of type relaxation. The presence of a current drain (such as \gls{AIS}
on the membrane or the axonal arbor on the axon) also means a potential difference,  and an exponential discharge function of type $exp(-\beta *t)$ 
describes that current, $\beta$ is a time constant.

The second one acts against diffusion and prevents the ions from leaving the layer.
\index{layer!diffusion}
Until that current stops (due to the saturation of the layer),
\textit{an ion current will flow in the direction perpendicular to the surface}.
\index{layer!current}
That current is "fast" only within the ion channel until the driving force disappears,
and becomes "slow" in the electrolyte layer, where the received charge saturates the layer. A current of form $(1-exp(-\alpha *t))$ can describe the saturation, where $\alpha$ is a time constant.
 Recall that the current's speed
depends on the voltage gradient, so the intensity and the temporal behavior of the 
currents are different, even between the "parallel" and "perpendicular" current directions, given that two different mechanisms control the process,
\index{voltage gradient}
despite that we consider the motion of the same charged particles. As a result of the two processes, a function of type $(1-exp(-\alpha *t))*exp(-\beta *t)$ describe the local charge distribution in the function of time. Although the timing constants change
as the potential changes, we use the approximation that
the layer is thin; furthermore, its concentration and potential have zero gradients in a direction perpendicular to the membrane. However,
a steep potential gradient exists between the layer
and the rest of the segment.
\index{layer!thickness}



The 'caps' on the top of the ion channels act as
individual regulators, and the ion channels continuously and randomly open, close, and inactivate. Their statistical population enables a
macroscopic ion inflow throughout the surface and the electric repulsion distributes
the charge over the surface, tending to make the local potential uniform over the surface.
The repulsion and attraction forces
on the two surfaces of the membrane around the channel's exit act as an additional valve on the ion transport:
the population 
of ion channels must cooperate with them, given that the ions move 'downhill'.


This behavior explains why ion currents across the membrane
start up with a sharp exponential rise~\cite{SodiumCurrentDelay:2006}
(one of the big mistakes was fitting polynomial lines~\cite{HodgkinHuxley:1952}
to those critical regions, comprising both exponential and no-current regions: it hides the sudden change of membrane's current~\cite{SodiumCurrentDelay:2006}
caused by the state change of the ion channels); why initiating an
\gls{AP} has precise timings (both the charge-up signal and pressing
ions through the \gls{AIS});
why axonal arbors can provide a precise
``Begin Computing'' signal. 
Measuring the conductance of ion channels, requires special care.
As discussed in section~\ref{sec:PHYSICS_MEASURINGCONDUCTANCE}, it is easy to
make a systematic error, given that the measurement method can affect the result. 


We posit explicitly that our parameters can be directly concluded from the measurable parameters such as membrane surface size, its ion channel density, specific membrane capacitance and absolute resistance of the \gls{AIS}.
Having those parameters of components of the non-living matter, plus the time course of the input currents,
we can describe how and why a the living matter shows the behavior we can observe.
This exact discussion provides an excellent base for understanding neuronal assemblies' operation,
furthermore revealing details of neuronal information storage and transfer.


This behavior explains why ion currents
start up with a sharp exponential rise \cite{SodiumCurrentDelay:2006}
(fitting polynomial lines \cite{HodgkinHuxley:1952}
to those critical regions was a big mistake: it hides the sudden change
caused by the state change (opening) of the ion channels, and that
the 'rising edge' is actually described by an exponential increase); why initiating an \gls{AP} 
has precise timings (both the charge-up signal and pressing
ions through the \gls{AIS}); why axonal arbors can pr
ovide a precise
`ComputingBegin' signal. For the details, see the following subsections.
\hyperlink{electric_conductance}
{Measuring the conductance}
of ion channels, requires special care.
\index{conductance}
It is easy to make a systematic error, given that the measurement device can affect the result.

Notice that the charged layers mean that a population
of ion channels must cooperate. Although the individual
\index{layer!ion channels}
\index{ion channel!operation}
ion channels open and close 'randomly', the repulsion force
on the two surfaces of the membrane acts as an additional valve. In a statistical sense, some ion channels are open
after the population members received the 'open' signal,
but when they are open, only the ions in the proximal layer
can escape to the other side of the membrane.




\subsection{Neuronal membrane\label{sec:Physics-NeuronalMembrane}}

At the dawn of finding methods for describing neuronal operation,
\gls{HH}
published high-precision measurements~\cite{HodgkinHuxley:1952}
enabling detailed testing of theories explaining the seen physiological
behavior. \emph{Their good physical model that }``movement
of any charged particle in the \hypertarget{Neuronal_membrane}{membrane} should contribute to the total
current'' \emph{only lacked considering the finite speed} at which the objects in
their measured system react to the observer's invasion (in addition to assuming the wrong oscillator type); furthermore,
they have started from the commonly used wrong assumption that conductance
is a primary electric entity. This wrong physical basis forced them
to make unphysical assumptions to explain their findings. Although
they attempted to give a physical background, they felt that ``\emph{the
	interpretation given is unlikely to provide a correct picture of the
	membrane}.''~\cite{HodgkinHuxley:1952} Using the Newtonian notion
of interaction speeds is misleading and blocks understanding electrophysiological
phenomena. 

\subsubsection{The 'delayed' membrane current\label{sec:Physics-DelayedMembraneCurrent}}


They could ``find equations which describe the conductances with
reasonable accuracy and are sufficiently simple for theoretical calculation
of the \gls{AP} and refractory period''. \emph{However, their equations
	cannot explain the delay experienced by a sudden change}; furthermore,
they explained that
% \gls{AP}
AP is created because of, for some secret
reason, the membrane's conductance changes in time (although \emph{they noticed the presence of a ``slow'' current
	that behaves differently from the ``fast'' currents that their equations
	describe}). The primary issue with their model is that
it concludes, as they admitted, a wrong description (irrealistic delay) of
sudden changes, such as the arrival of a spike, of making \hyperlink{voltage-clamping}{clamping
measurements}, or of interpreting the mechanism of neuronal information
transfer. 

Their followers modified both the form of their mathematical description
(without assuming any physical model, using ad-hoc equations) to achieve minor improvement in the temporal behavior of the
description. For a review of ideas, see~\cite{Hodgkin-HuxleyAdsorption:2021}.
This latter work attempted to introduce ``a physiologically, physically
and chemically viable model'' that had to assume a physically not
plausible ion-adsorption buildup mechanism to be able to explain the
mentioned delay, see their Eq. (45). Those attempts, however, did
not change what \gls{HH} noticed~\cite{HodgkinHuxley:1952}:
``there is the difficulty that \emph{both sodium and
	potassium conductances increase with a delay when the axon is depolarized
	but fall with no appreciable inflexion when it is repolarized}''.
Without admitting a ``slow'' current exists, we must presume
that sodium and potassium concerted their actions, and conductance
is indeed misinterpreted in both cases. HH concluded~\cite{HodgkinHuxley:1952}
(presumably after many unsuccessful attempts) that "there is little
hope of calculating the time course of the sodium and potassium conductances
from first principles". 
It is correct: the existence of such a time course itself is against the first principles of science.
However, if we make correct (physically
plausible, instead of ad-hoc) assumptions, \textit{we can derive a "time course"} (well, not
of the conductance because it is a misinterpretation of the physical
phenomena, see section~\ref{sec:PHYSICS_MEASURINGCONDUCTANCE}; instead) \textit{of the ionic current from first principles} although
we must mix microscopic and macroscopic parameters. 

It is a long-standing enigmatic phenomenon
that "the emergence of life cannot be predicted by the laws of physics"~\cite{ConservationOfInformation:2021}
(unlike the creation of technical systems). Still, we can provide
a complete description of the biological phenomena from first principles
if we consider the finite interaction speed instead of using the idea
of ``prompt interaction'' taken from classic physics, which is a
fake abstraction for that goal. Models in neuroscience (as reviewed
in~\cite{BrainNetworkModels:2018}) almost entirely leave the mentioned
aspects out of scope. We introduce a finite interaction speed  without introducing
either twisted mathematical handling or obscure physical (for example,
adsorption) mechanisms. In our straightforward physical model, we see
the measurable membrane potential and current change in the function of
the speed of ions $v$. 


The commonly used physical picture behind the process is that the
membrane, as if it were metal, is equipotential, %(although when introducing
%a multi-compartment model, one admits that that assumption was wrong),
and the ``fast'' axonal current flows directly to the membrane.
This assumption is why we expect an instant appearance of the axon's current
in the membrane's current (instead, we experience a ``time-dependent
conductance'').



\subsubsection{The 'true' membrane current \label{sec:Physics-TrueMembraneCurrent}}

This axonal charge-up current, a phenomenon we are exploring from
an abstract perspective, flows into the membrane.
It causes transient changes~\cite{TransientResponses:2008,KochElectricalPropertiesSpike:1983} in its voltage, providing \textit{direct evidence that the membrane is not always equipotential. The ions on the membrane’s surface can propagate at a finite speed}.
The membrane attempts
to remain isopotential, the ions move freely on its surface. 

After the membrane reaches its threshold potential, the voltage-controlled ion channels open, and many ions from the extracellular space rush into the intracellular space, as we explained in section ~\ref{sec:Physics-IonChannels}.
The ion channels open and close themself autonomously and quickly.
\emph{There is no way or no need to simultaneously open other ion channels in the opposite
direction. As we discussed above, the charged ions immediately in front of the membrane generate an electric gradient in the order of $100,000\ V/cm$.} 

The sudden membrane potential change in the charge-up
period acts as a valve. Given that the ions in the axonal arbor need to enter
	the membrane against the actual membrane potential, the potential stops the ion inflow to the membrane for the period while the membrane's voltage is above the threshold: it effectively inhibits further inflow through
all axons. This behavior naturally explains the absolute refractory
period. After the membrane's voltage drops below the threshold value,
the ions can enter the membrane again (see Figures~\ref{fig:MasonFig4} and~\ref{fig:ArtificialCurrent_AP}),
but they need time to reach the \gls{AIS}
later (see Figure~\ref{fig:AP_Conceptual}) when in the meantime
the membrane' voltage proceeded toward its hyperpolarized state;
so they seem to appear dozens of microseconds later at the \gls{AIS},
explaining the relative refractory period (measuren in~\cite{APTemperatureDependenceRefractory:2001}.


The inflow charge generates a "potential wave" (a solid current outflow) through the \gls{AIS};
see the discussion in section~\ref{sec:Physiology-DerivingActionPotential}.
The decreasing charge causes the membrane's potential to decrease
toward its resting potential, so it falls below the threshold voltage
of the axonal gate at some point. If ions are still waiting on the
other side, stopped when the membrane's charge-up process started
(recall that they cannot exit the axon of the presynaptic neuron, and previously they could
not enter the membrane), or newly arrived while the gate was closed,
they can enter the membrane again.
The ions travel a finite distance on the surface of the membrane with
a finite speed, so there must be a delay between their entry and exit times. Furthermore,
the inflow current must equal the outflow current. As discussed in
section~\ref{sec:Physics-Electricity}, charge conservation
is not necessarily valid in \emph{all contexts} of biological operation.
If we measure the input and output currents, they may differ (see
Fig.~1. in~\cite{ActionPotentialGenerationNatrium:2008}); see section~\ref{sec:Physiology-DerivingActionPotential}.

Notice that, to some measure, the case of \hyperlink{voltage-clamping}{switching a clamping voltage
on} is analogous to the arrival of a spike. Initially, the axon contains
no ions. The front evoked by a step function is linear because of
the slow current. In the classic picture, the axonal current flows into the membrane with
capacity $C_{m}$ and increases the membrane's voltage $V_{m}$
with a time constant discussed after Eq.(\ref{eq:I_Membrane_Off})
\begin{equation}
	\frac{dV_{m}}{dt}=-\frac{1}{C_{m}}I_{axon};\ V_{m}(t)=\frac{I_{wall}*(1-e^{-\alpha*t})}{C_{m}}\label{eq:MembraneCurrent}
\end{equation}

\noindent that generates a change in the membrane current
\begin{equation}
	\frac{dI_{m}}{dt}=\frac{1}{R_{m}}\frac{dV_{m}}{dt};\ I_{m}^{on}(t)=g_{m}(V)V_{m}(t)\label{eq:TimeDependentConductance}
\end{equation}
\noindent where $g_{m}=\frac{1}{R_{m}}$ is the conductance of the
membrane. That is, the measurable current equals the product of the conductance
and the clamping voltage. Equs.(\emph{3)-(5) in~\cite{HodgkinHuxley:1952}}
express this relation\emph{. If we assume that the axonal current
	is ``fast'', we arrive at the wrong conclusion that the conductance
	is voltage- or time-dependent.}
In contrast, if we assume that the axonal current is ``slow'',
we naturally conclude that Ohm's Law is correct and valid also for
biology: the conductance/resistance is constant.

\emph{There is no voltage-dependent conductance}~\cite{KochVoltageDependentConductance:1999}.
Instead, the finite speed of ions and the wrong assumption that conductance
is a primary entity misleads physiological research.\emph{ With wording
	that "conductance changes", one states that charge carriers
	appear/disappear/reappear; that is, the charge conservation is not
	fulfilled} (with nonphysical consequences listed in connection with
the model in~\cite{HodgkinHuxley:1952}). The physics background
of the phenomenon is that the number of charge carriers changes (ions
are ``created'' in the axon, and they appear on the membrane, as
we detailed above).

In contrast, when \hyperlink{voltage-clamping}{the clamping voltage is switched off}, the axon is still
filled with charge carriers (but not filled after); the resting potential reaches the end
at the membrane ``instantly''. The driving force disappears, the
ion stream stops, and no more ions enter the membrane. The lack and
the presence of ions in the axon when switching clamping on and off, respectively, produce the difference that ``\emph{conductances
	increase with a delay when the axon is depolarized but fall with no
	appreciable inflexion when it is repolarized}''~\cite{HodgkinHuxley:1952}.
The potential is equalized by the \gls{AIS} current, producing a net
exponential decay:

\begin{equation}
	I_{m}^{off}=I_{Wall}*e^{(-\frac{\alpha}{R_{m}C_{m}}*t)}\label{eq:I_Membrane_Off}
\end{equation}

\noindent 

During the regular operation of a neuronal membrane, after opening
the ion channels, a vast amount of ions flow into the intracellular
space from the extracellular space, imitating \hyperlink{voltage-clamping}{the effect of switching
a clamping voltage}. The essential difference is that
the ions arrive through the axon to the joining point in clamping. In contrast, through the membrane's ion channels, they directly contribute oon the membrane’s entire surface.  The membrane's size is finite, so with a finite
current speed, it takes time until the charges on the membrane's surface
arrive at the \gls{AIS},
the same way as we discussed for the
axonal current. These findings have significant implications for our
understanding of the operation of neurons, including their signal processing and memory.

From a computational point of view~\cite{VeghComputingModel:2021},
a persisting significant deviation from the resting potential (the
arrival of the first spike from one of the upstream neurons) provides
the signal 'Begin Computing', opening the ion channels in the membrane
provides 'End Computing'. After that, we will be in the 'Signal Delivery'
phase until the end of the charge-up process. After that, 'Signal
Transmission' follows. Our simple neuronal condenser can only perform
one operation, to integrate the current it receives. Its
result is the integration time itself. \emph{It cannot distinguish
	its operands} (which synaptic inputs provided the current it integrates).
Furthermore, \emph{not all operands must be present at the beginning of the computation process}.
\emph{The membrane potential slowly returns to its resting value;
furthermore, the current arriving during the 	'relative refractory period', represent a (time-dependent)
memory}, see section~\ref{sec:Single-OperationConceptual}.
Notice that the content of that memory may depend on the
neuronal environment.


\subsection{Axon and axonal arbor\label{sec:Physics-AxonalDelivery}}


We model the axons as electrolyte-filled semipermeable membrane tubes with ion channels in their walls. The axons do not passively follow the potential's time course, but they
mediate the changes in their internal volume by using an ion pool
available in their extracellular volume. The applied potential (including that of the mediated ions) opens the
ion channels in the axon's wall. 

In their native mode of operation, the three modes of ion channels define the 'direction of the time'~\cite{ThreeStateUnidirectional:2004, MarkovianIonChannel:2005,RoleOfInformationTransferSpeed:2022} (the direction of 
the current that transmits the spike). The layer that the 
front of the spike creates on the surface (on both sides
of the tube) propagates in both directions, 
but it cannot open the ion channels on the side where
the spike arrived from, and the ion channels are still
inactivated.


Clamping
sets up an artificial working regime for the ion channels: the permanent
electric field on the outer surface enables ions to enter the inner volume where
formerly no ions (and no potential) existed. The rushed-in ions will
flow away from the place of their entrance (recall that the current removes part of the ion layer on the surface), and a slow current toward the membrane can start. 
Under clamping conditions, the experimenter sets the voltage instead of the transmitted signal and in a static way instead of an  autonomous dynamic one.

Initially, the membrane, the clamping point on the
axon, and the intracellular and extracellular fluid maintain
their resting potential. When an external potential %(such as clamping) 
is applied  suddenly to some point of the axon, an electric field $\frac{dV}{dx}\propto(V_{membrane}-V_{clamp})$
appears on the \emph{outside surface} of the axon. The extracellular
space with its high ion concentration $C_{k}^{ext}$ represents an "ion
cloud".% (see also section~\ref{sec:Charge-transfer-vs-current}).
When the clamping voltage is switched on, a ``fast'' current
instantly delivers the potential along the \emph{outer} surface of
the axon. However, this is not the case (at least not in the initial
moment) on the \emph{inner} surface. \textit{There is no charge present that could
change the potential}: 'the intracellular concentration at rest is around
five orders of magnitude less than that in the extracellular space'~\cite{KochBiophysics:1999}. The physical picture that the clamping potential
instantly appears at the end of the axon at the membrane (i.e., if (apparently) they
have an infinitely large propagation speed) is valid
only if charge carriers exist in the axon.



As described above, the charge gradually increases the potential along
the axon (starting from the position of the clamping electrode) until
the \hyperlink{voltage-clamping}{clamping potential} reaches the axon's end at the membrane. (We
could see the effect when measuring voltage instead of conductance
on the axonal tube instead of the membrane, shown in our Fig.~\ref{fig:The-time-course_Clamping}.)
At that point, the driving force gradually disappears: the potential
at the end of the axon and that on the membrane becomes the same.
The macroscopic streaming of ions inside the tube only slightly complicates
the process: the local internal concentration can saturate only later,
given that part of the inflowing ions is delivered to another place
within the axon. Notice that the current (and the voltage) on the
axon increases in the function of the time exponentially instead of
linearly or step-wise, which would be expected when assuming instant
interaction or no ``slow'' macroscopic current.


In this model, we assume that during the time $dt$, in the volume
$dx$, we have a constant ion inflow $I_{wall}$ through the axon's
wall, which increases the charge and concentration already in the
volume. The charges in the tube experience the field $\frac{dV}{dx}$,
and they move with speed $v$ inside the tube (see Eq.~(\ref{eq:StokesCurrent})).
The ionic fluid with velocity $v$ (proportional
to $\frac{dV}{dx}$) transfers the ionic charge in
the volume to the neighboring element at a distance $v*dt$, and delivers
the charge and concentration from the neighboring element at a distance $-v*dt$ into
this element. At the time $t$, the concentration at $x$ will result
from the inflow at the place $x-v*t$ (see also the general discussion
around Eq.~(\ref{eq:CoulombTimeDependent})). 
%The higher the speed 
%$v$, the more significant the difference between the "inflow" and the "present"
%concentration. The stream inside the axon, a la Minkowski
%(although in this simple case, a Galilei-transform is sufficient), transforms
%the distance to time and vice versa.
%	Under the effect of clamping, the current is decreased by the stream
%proportionally:
%
%\begin{equation}
%	\frac{dI_{axon}}{dt}=-\alpha*I_{axon};\ I_{axon}(t)\approx I_{wall}*(1-\exp(-\alpha*t))\label{eq:AxonalCurrentNoConc}
%\end{equation}
%
%
%\noindent ($\alpha$ is a timing constant
%of dimension $(1/time)$).
%
%
%\begin{figure*}
%	\includegraphics[scale=0.6]{fig/HodgkinHuxleyUI_On}\includegraphics[scale=0.6]{fig/HodgkinHuxleyUI_Off}
%	\caption{The time course of voltage and current a clamping
%		experiment, calculated numerically for
%		switching the clamping on and off.\label{fig:The-time-course_Clamping}}
%\end{figure*}
%

As described above, the charge gradually increases the potential along
the axon (starting from the position of the clamping electrode) until
the clamping potential reaches the axon's end at the membrane. (We
could see the effect when measuring voltage instead of conductance
on the axonal tube instead of the membrane, shown in our Fig.~\ref{fig:The-time-course_Clamping}.)
At that point, the driving force gradually disappears: the potential
at the end of the axon and that on the membrane becomes the same.
The macroscopic streaming of ions inside the tube only slightly complicates
the process: the local internal concentration can saturate only later,
given that part of the inflowing ions is delivered to another place
within the axon. Notice that the current (and the voltage) on the
axon increases in the function of the time exponentially instead of
linearly or step-wise, which would be expected when assuming instant
interaction or no ``slow'' macroscopic current.


The unusual physical situation in making electric measurements in biological systems is that, in the metallic half
of the circuit, the electrode at the membrane (and, if being equipotential,
the membrane itself, too) takes "instantly" the external voltage.
However, in the biological half of the circuit, the voltage $V$ at
the end of the axonal tube initially remains the same: inside the
tube, there is no charge around to produce a potential (actually,
without charge inflow, it is a piece of insulator). 


\subsection{Axon Initial Segment\label{sec:Physics-Axonal-InitialSegment}}


At the time when \gls{HH}
published \cite{HodgkinHuxley:1952} their electrical model for the neuron, the structure of the neuron, the \gls{AIS}
and its role in the electric operation was not yet known.
\gls{HH} introduced the idea explicitly that the electrically equivalent circuit
of a neuron is an $RC$ oscillator.
They did not see any structural elements on the membrane,
so logically, they assumed it was a distributed resistor and capacitor,
which really has resemblance with a \textit{parallelly switched $RC$ oscillator.}
However, they made a wrong choice of the circuit type, and their choice (probably due to inertia)
was repeated in good textbooks such as  (\cite{ JohnstonWuNeurophysiology:1995}
Figure 3.1 or \cite{KochBiophysics:1999} Figure 1.1), and
it is a commonly accepted fallacy even today~\cite{NeuralDynamicsGertsner:2014}.
This wrong choice led to the need to assume a false (rectifying) ionic current and blocks
understanding, among others, \textit{why} \gls{AP} is initiated.

From the discussion and figure above, it is clear that the right choice is a
\url{"https://www.electronics-tutorials.ws/rc/rc-differentiator.html"} \textit{differentiator}  where
'the input signal is applied to one side of the capacitor with the output taken across the resistor'.
The currents
are directly created on the membrane (condenser) and the output voltage
 (\gls{AP})
is taken across the resistor (\gls{AIS}).
In other words: \textit{the neuronal membrane is a serial instead of a parallel circuit},
with far-reaching consequences.

For electrical modeling, we can use the approximation  that a \textit{distributed} condenser
(the neuronal membrane) and a \textit{discrete} resistor (the 
%\gls{AIS})
AIS form an $RC$ circuit,
see also the discussion in section \ref{sec:PHYSICS_MEASURINGOSCILLATOR}.
It is clear that all currents (including the synaptic currents, the membrane's rush-in current,
and the artificial currents either patching them directly to the membrane or clamping them to its axons)
flow into the condenser (and cause potential increases calculated using the membrane's capacitance).
Furthermore, the potential drops only due to the current flowing through the 
%\gls{AIS}.
AIS.
It is the exact equivalent of
 a passive $RC$ \textit{differentiator} circuit:
"the input is connected to a capacitor while the output voltage is taken from across a resistance"
and not to be mismatched with
a passive $RC$ \textit{integrator} circuit
where "the input is connected to a resistance while the output voltage is taken from across a capacitor".

\subsection{Background logistics\label{sec:Physics-BackgroundLogistics}}
The neurons use a quasi-hidden background energy production 
and state restoration process which works in parallel with 
the neuronal functional operation with a quasi-permanent speed.
The \gls{ATP} is produced and delivered to the spot by 
independent subsystems. The \gls{ATP}'s delivery, similarly to that of the ions,
obeys its component gradient. When the neuron's membrane (through the \gls{AIS}, resting ion channels, and pumps) 'loses' ions,
their local gradients in the vicinity of the membrane decrease.
The interesting processes (ionization, ion absorption, and so on)
occur near to the membrane.
The processes enable producing 'fresh' ions from the available neutral molecules by hydrolyzis to restore their concentration; then
new \gls{ATP} diffuses in the place of the 'used' ones. 
Those ions build up onto the membrane, increase its electrical potential. That energy is a kind of potential energy and enables to forward ions to the other segment (their presence slightly increases
the potential there), while part of the energy dissipates when
the potential moves the ions in the viscouos fluid. 

As discussed in section~\ref{sec:Physics-TwoSegments}, outside
the lipid bilayer, electric and concentration gradiens exist.
They enable the ions to move, but since their speed is proportional
to the resultant gradient, at different locations the speed of 
the macroscopic move of ions as well as the individual speed
of ions along their path can change several orders of magnitude,
from the \href{https://en.wikipedia.org/wiki/Drift_velocity}{drift speed} to the \gls{potential-accelerated speed}, while they travel
a nanometer-long path under nanosecundum periods.
For example, a $Na^+$ ion is in rest (has the \href{https://en.wikipedia.org/wiki/Drift_velocity}{drift speed}) on the high-concentration side
when the ion channel gets open, accelerates to a \gls{potential-accelerated speed}, than brakes down to the \href{https://en.wikipedia.org/wiki/Drift_velocity}{drift speed} again on the low-concentration side. 
Even, its velocity components in directions parallel and perpendicular
to the membrane's surface can differ by orders of magnitude when
the electrical field exert on it in the mentioned directions.
The ion-gradient-related movement needs special care,
since the driving force is different for the different chemical elements, and the movement of ions in close vicinity to each other
(such as in ion channels) strongly affect each other.