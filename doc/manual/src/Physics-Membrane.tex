% Physics of membrane and related stuff

\section[Neuronal operation]{Neuronal operation\label{sec:Physics-MembranesChannels}}


\subsection{Membranes \& layers\label{sec:Physics-Membranes}}

As described, separating electrolytes by a finite-width membrane leads to changes in the distribution of ions, 
in their macroscopic parameters 'concentration' and 'potential'. The actual distribution of course depends
on the thickness of the membrane and the electrolyte concentration in the segments. Experience shows that
although the material of the membrane is a good isolator,
the membrane is permeable in some sense (see section~\ref{sec:Demon-in-the-membrane}), sometimes selectively and
sometimes in one direction only; sometimes in a controlled way.
The presence of layers on the membrane is experimentally confirmed. It is perfectly seen that the dissociated ions
are mainly in the proximal $<1~nm$ layer of the membrane,
\href{https://www.ncbi.nlm.nih.gov/books/NBK26910/figure/A2034/?report=objectonly}{see the caption of figure (Fig.~11.22 in~\cite{MolecularBiology:2002})}. The 
ionic concentrations are largely different in its two segments and the ion channels have their cap on the side of segment with lower concentration.
However, it is experimentally hard to measure anything
in this layer; furthermore, the statical interpretation 
of cellular operation prevents understanding their
dynamic operation, including their role in creating and transferring electric signals.

Membranes, and especially the semipermeable ones, are fundamental pieces in many places, from biological objects
to industrial filters. They operate on the border of microscopic and
macroscopic worlds, combine movements having speeds differing by several orders of magnitude, separate non-living and living matters and
combine electrical and thermodynamical interactions. We show that a
fragile skin near the surface of biological membranes
is responsible for the biological thermoelectric processes.
\index{layer!skin}


We might imagine \hypertarget{DynamicLayer}{this layer's} importance and operation in
line with the Earth's atmosphere. Its features drastically deviate
from the features of the bulks on its two sides. It is separated by
a sharp contour on one side and an ill-defined border on the other;
its volume is far from being homogeneous. Gravity keeps it in place, and it is at rest. However, sometimes,
for some periods, also other (thermodynamical and electrical) forces evoke
inside it and lead to transient changes, moving huge masses with  high-speeds
inside it. Its thickness and mass are negligible compared to those
of the bulks on their two sides, and we can describe the bulks without
considering their density, mass, size, etc. Still, this thin layer is
responsible for the weather; its transient processes define the visibility
from both sides (define propagation of electromagnetic fields), and
it can protect us from 
\gls{EM}
radiations. It can temporarily absorb
products of slow processes (water evaporation) and deliver masses of
high density (much above its density, such as water, sand, etc.) to
continental distances, creating the illusion that it stores that matter.
Minor changes (natural ones, such as a slight difference in air temperature, and artificial ones, such as injecting condensation nuclei in clouds)
can result in enormous changes. Even we can imagine volcanic eruptions
as semipermeable gates for material with apparently random operation
and distribution of the injected material.
Physics sees correctly its importance and role:
"the crucial system in biology isn't a molecule or a molecular class whatsoever,
but the interface created by biomolecules in water"~\cite{LivingSystemPhysics:2021}, although we add that mainly
physical processes and speed gradients control the processes.
To describe those complex and continuous phenomena at least approximately,
we must separate them into stages. We can describe the stages approximately using omissions, approximations, and
abstractions, usually considering
only one dominant phenomenon.  The described phenomena are interrelated
in a very complex way and depend on different parameters. To some
point, we can describe that thin layer using a static picture and
provide an empirical description of its processes, even though 
we can give some limited-validity mathematical descriptions for those
stages. However, we understand that for describing the time course of the transition
(contrasting with step-like stage changes) between those well-defined
stages of the atmosphere, we need a \emph{dynamic description} to
discover the \emph{laws of motion} governing the processes.

Similar is the case with the neuronal membrane and the neuronal operation.
Now, we are at the point where their decades-old static description
is insufficient. We need
to derive the corresponding laws of motion to describe the neuron's dynamic behavior. We need a meticulous and
unusual analysis to derive them. 
%
In a neuron, in the abstraction science uses, we put together only
an ionic solution, a semipermeable membrane, and currents that reach and leave
them. All these belong to non-living matter.
As experienced, at some combination of their parameters and gradients, qualitatively
different phenomena happen, which, in the abstraction biology uses,
are called \hypertarget{SignsOfLife}{signs of life};
our system starts to belong to living matter.
Given that the approximations, the derived abstractions,
and the mathematical formalisms describing them are different for
the two cases, \emph{it looks like we have two different, only loosely
bound worlds}. We realize we have arrived at the boundary of non-living
and living matters, and we must go back to the \emph{first principles
of science} to clarify where their boundary is. However, by using our approach, we may defy that "the
emergence of life cannot be predicted by the laws of physics"~\cite{ConservationOfInformation:2021}.
Our artificial duck looks, quacks and swims like a duck. 
\index{duck model}




\subsection{Demon in the membrane\label{sec:Demon-in-the-membrane}}


We intend to build an artificial neuron using materials and principles of non-living science.
We build '\hypertarget{Maxwell-demon}{Maxwell-demon}'-like objects into the separating membrane:
gated ion channels, see Fig.~(\ref{fig:The-membrane's-extra-gradient}). 
The ion channels operate as demons (from the point of view of the segments
and the observer).
Some power opens them, they autonomously transfer ions
in a \emph{potential-accelerated} operating mode, and then that power
puts the cap back on the top of the channel.
Let us separate our volume with a semipermeable membrane  having capped ion channels inserted into its material. 
As long as the caps of the ion channels are closed 
and the ion concentration on the two sides of the membrane are the same, we do not see any change in the state of the solutions. 
However, our construction \href{https://en.wikipedia.org/wiki/Duck_test}{'looks like a duck'}.
\index{duck model}

Although the channels can stochastically open, close, and re-open,
they transmit more or less well defined charge quanta.
Even the channels can
recognize the ions' chemical nature and transmit only a selected ion
type. The channels are passive during those processes, although the enormous
voltage gradient can rearrange their structure and change their behavior through that.
\index{voltage gradient}
The demons also concert their actions using the layer containing charges
\index{layer!demon}
as a communication medium; their population maintains a well defined macroscopic current across the membrane.
Our construction swims like a 'duck'.
\index{duck model}

\subsubsection{Voltage sensing\label{sec:Electrodiffusion_VoltageSensing}}
"Voltage sensing by ion channels is the key event enabling the generation
and propagation of electrical activity in excitable cells."\cite{VoltageSensingIonChannel:2019}
How voltage gating of channels works is still a mystery; one of the worst consequences that Hodgkin and Huxley separated the potential from ions and their current. It is not
easy to investigate it experimentally: "the structural basis of
voltage gating is uncertain because the resting state exists only
at deeply negative membrane potentials"~\cite{VoltageGatedChannelStructure:2019}.
Usually, a "sliding helix" (structural)
model is assumed.


Under certain conditions, an ion channel can be opened in only one direction and
only for a limited period, and this way the membrane becomes semipermeable.
\index{membrane!semipermeable}
We imagine an ion channel as a simple hole (a cylinder) between the
high and low-concentration segments with a cap on its top (on the
side of the low-concentration segment). Until the cap is removed/lifted
(the channel gets open), practically nothing changes. At the points
where the ion channels are located
the ions cannot penetrate the
membrane. Unlike the original Maxwell demon, our demon does not have
information in advance about which particle should be transmitted:
\index{Maxwell-demon}
it is passive in selecting the particle. 
(Passive here means that no biologically produced energy is used:
the electric potential energy from the voltage difference across the membrane moves the ions to the other side of the membrane.)
It only keeps one way closed
for part of the time, and the voltage performs selecting the ions.


We can easily interpret why our
voltage-controlled ion channel model gets opened and closed  due to purely electrostatic
reasons. It works as the two-plate simple
nano-scale electrometer (of type quadrant, Lindermann, Hoffman, and
Wulf) similar to the ones used to measure the small electrical potential
between charged elements (e.g., plates or fine quartz fibers).
\index{electrometer}
\index{ion channel}
Given that the membrane and the cap in their resting state are isolators,
no electrical repulsion is evoked between them and the adhesion sticks them firmly to each other, representing a permanent force. The van der Waals force is inversely
\index{van der Waals force}
proportional to the squared distance between the dipoles in the cap and the membrane, respectively, and is linearly proportional to the perimeter of the channel.

However, when a slow ion current
flows into the surface layer in the proximity of the cap,
charges appear in the layer proximal to the membrane;
the membrane and the cap get covered by a very thin electrical skin.
\index{layer!voltage gradient sensing}
The charge on the cap is proportional to the surface of the cap and similarly inversely proportional to the squared distance between the cap and the membrane. 
\index{voltage gradient}
A local voltage gradient is generated by the local gradient of the slow ion current (see below), and the force acting on the cap is proportional to the
product of the voltage gradient and the area of the cap.
Given that the cap is slightly elevated, the repulsion
force may have a component in the direction of lifting the cap.
Since the van der Waals force is of fixed size, the electrical repulsion exceeds it at a critical voltage gradient value and the channel opens. 
The gate remains open as long as the local charge distribution enables it.
The cap is connected to the membrane only
at one point, so it cannot fly away and also cannot close again until the charge 
on the surface is present. 
In the absense of charge, the cap makes a random movement and the short-distance van der Waals force may eventually fix the cap again to the membrane, this way closing the channel.
The voltage sensing electrometer opens the channel and the lack of 
charge on the surface enables to close it, but the closing is not immediate
(the mass of the cap is by orders of magnitude larger than the mass of an ion that can pass the channel).
The fluctuation of the voltage gradient due to the gradient of the slow
current in the layer in the proximity of the membrane near the ion
channel's exit opens, closes, and re-opens the channel in an apparently
stochastic way (actually, as the repulsion of charges due to the fluctuating current on the cap and
the membrane regulates it), as observed.
Having gates (caps) is needed only for the synchronized operation of the channels: opening and closing work
also in the absence of caps, as discussed.

When one cap is removed, the rushed-in
ions in the proximity of the channel's exit suddenly increase the
local potential (produce fast transient changes~\cite{KochElectricalPropertiesSpike:1983})
proximal to the spot centered at the exit in the layer on the membrane's surface.
The surface outside the spot remains at a lower potential, so the
ions in the layer start moving toward other channel exits, delivering
potential to those channel exits. Given that they are voltage-controlled,
they get open, and the process continues in an avalanche-like way~\cite{NeuronalAvalanches:2003}. The avalanche, as explained, needs a 
sufficiently large voltage gradient; which can be triggered by several
synaptic inputs if they sum up appropriately. Alternatively, a single spike with sufficiently steep front slope~\cite{LosonczyIntegrative:2006}
can be sufficient; providing a simple way of synchronizing neuronal assemblies
(and proving that not the voltage, but the voltage gradient, single of summed, controls the operation).


\subsubsection[Ions' passing]{Passing through the ion channel\label{sec:Electrodiffusion_PassingIonChannel}}

The operation of the ion channel, alone, cannot explain that
the channel closes after a given number of ions passed the channel;
that number is not (entirely) random. Actually, the local behavior of the membrane's
surface layers regulate the number of ions. 

The segments are no longer mechanically separated when the cap is
removed. The charged ions are enabled to rush into the lower concentration
segment. They experience an enormous accelerating gradient: "an
electrical potential difference about $50-100\ mV$ ... exists across
a plasma membrane only about $5\ nm$ thick, so that the resulting
voltage gradient is about $100,000\ V/cm$"~\cite{MolecularBiology:2002}.
\index{voltage gradient}
That enormous gradient, comparable to that of electrostratic particle
accelerators, "snorts" the ions from the high-concentration side
into the low-concentration side and causes a process "like
a flee hopping in a breeze". 
Consequently, "transport efficiency of ion channels
is $10^5$ times greater than the fastest rate of transport mediated
by any known carrier protein"~\cite{MolecularBiology:2002}. 
Recall that, in physics,
the \emph{drift speed}, the \emph{electrical repulsion-assisted speed,
}and the \emph{electrical potential-accelerated speed} of ions differ
by several orders of magnitude (for visibility, the ratio of the gradients
in Fig.~\ref{fig:The-membrane's-extra-gradient} is not proportional).

The snorted ions "hop" into the layer from the another layer.
In the beginning, with
their \emph{voltage-accelerated} speed, it could take less than $\frac{5*10^{-9}m}{10^{3}m/s}\ s$
to pass the channel (simulation~\cite{IonChannelSimulation:2016}
uses a $psec$ representative time interval), in the end, they may
slow down to the \emph{voltage-assisted} level as the potential gradually
decreases (which is still $\frac{5*10^{-9}m}{10^{-1}m/s}\ s$), so
we can omit that time when calculating the charged layer formation.
Due to the enormous speed difference between the \emph{accelerated}
and \emph{assisted} speeds, the passage is practically instant. The
accelerating field through the hole across the layers persists, although
it decreases; see Fig.~\ref{fig:IonChannelRK}. On the high-concentration segment, only the ions in the
layer in the immediate proximity of the entrance can feel the accelerating
potential and move with the potential-accelerated speed. The after-diffusion
with the \emph{potential-assisted} speed from the next neighboring
layer in the high potential segment is by orders of magnitude slower than the passage through the
hole with the \emph{potential-accelerated} speed. Depending on the
process parameters, the local potential can rise above the high-concentration
segment's potential for a short period due to the accelerated current's 'ram pressure' (or
'impact pressure'). Due to their electrical repulsion, the ions induce
a similar change on the opposite segment.

The accelerating potential around the channel's exit gradually (but quickly) disappears when
the particle exits the ion channel (see the green ion in the figure),
and the ion arrives at the bulk potential. It practically stops: it
can continue only with its \emph{potential-assisted }(later with\emph{
drift})\emph{ speed, which is several orders of magnitude lower.}
However, the rest of the ions are still accelerated through the channel,
and somewhat later, they also land in the formerly low-concentration
layer, further increasing its potential and concentration. The passed-through
ions increase the local potential in the layer in the low-concentration
segment and decrease the local potential in the layer in the high-concentration
segment. Given that the after-diffusion speeds in the layers are limited,
"as ion concentrations are increased, the flux of ions through a
channel increases proportionally but then levels off (saturates) at
a maximum rate"~\cite{MolecularBiology:2002}.

Here the efect of the \hyperlink{ChangingResources}
{finite resources}, see section~\ref{sec:Physics-Resources}, explicitly appears.
As we discuss in sections~\ref{sec:Physics-RushinCharge}
\index{resource!finite}
and~\ref{sec:Physics-ChargeOnMembrane}, about $10^3$ ions are transferred 
per channel. These ions are snorted from one layer in the 
high-concentration segment into another layer in the low-concentration layer.
The driving force gradually decreases, see Fig.~\ref{sec:Physics-ResourcesIonChannels}, because ions leave the first layer
and they appear in the second mentioned layer.
The potential-assisted speed
to replace the leaving ions into the first segment from the bulk
as well as diffusing out from the second segment without appropriate driving forces is by orders of magnitude slower,
so we can approximate the process that a gradually 
decreasing accelerating force drives the ions.
The process leads to a special reversal of concentrations and potentials.
In a very short period, in the layer on the formerly low-concentration side
a very thin high-potential layer is formed that prevents further ions
from entering the formerly high-concentration layer: the process
of transferring ions through the channel closes the door behind the needed
amount of ions. Now the gradient diminished and the van der Waals force
can close the channel again.
\index{van der Waals force}

The commonly used picture about the operation of ion channels~\cite{HilleIonChannels:1999} is definitely wrong.
\begin{itemize}
\item the potential generated across the membrane is entirely neglected
\item the ions have no driving force and the hypothesized protein carriers are too slow~\cite{MolecularBiology:2002}
\item the considered van der Waaals force is too weak to be noticed by the ions (the 'cation-attractive negative ends' of the Alpha helices are too far)
\item the assumed force by the 'cation-attractive negative ends' destabilize the ion path: as the deviation from the central path increases, so increases the deviating driving force
\item even if the weak van der Waaals force would work for a single ion,
the next ion would be rejected by the strong Coulomb-force due to the first ion
\index{Coulomb interaction}
\end{itemize}



\subsubsection[Delivering current]{Delivering current across the membrane\label{Electrodiffusion_DeliveringCurrentLayers}}
The passage is too quick to affect the bulk (see also the discussion
in section~\ref{sec:Physiology-OperatingRegimes}), given that the ions
can only use a \emph{potential-assisted} speed to reach distant places
in both segments. Again, the charge and mass conservation works: the
ions pass suddenly from the high-concentration side to the low-concentration
side, only from one layer to another.
The mentioned \emph{layers on the two sides will actively
initiate and terminate the ion transfer through the ion channels,
but the ions can only pass through an open channel.}
One layer saturates, and the
other empties. After a while, \emph{the source of ions will be exhausted}.
\emph{Those layers' existence suggests revisiting the idea of describing
neuronal operation by two single potentials of the bulks on the two
sides of the membrane}.

Following their arrival, the driving force perpendicular to the membrane's voltage disappears, and the ions form a thin "hot spot" in the layer. 
The electric repulsion acts in parallel with the membrane's surface and
leads to distributing the ions (decreasing the gradient by distributing the charge locally) around
the channel's exit.
The ions saturate the layer on the membrane's surface with a time constant
between ($\frac{10^{-8}m}{10^{-1}m/s}\ s$) at the beginning
and $(\frac{10^{-8}m}{10^{-4}m/s}\ s)$
at the end of their arrival period (we assumed $10\ nm$ average distance between
ion channel exits on the membrane). We shall take the longer time,
so that we can expect a time constant for the saturation current around
the ion channel's exit in the order of $0.1\ ms$. When charging up
the membrane in an avalanche-like way, the ions must pass on average
a distance of about $0.05\ mm$ from its center to its farthest point,
so we expect a $0.5\ ms$ ($\frac{5*10^{-5}m}{10^{-1}m/s}\ s$) time
until the membrane's slow current charges up the membrane to its maximum
potential. The created charge must flow out from the farthest point
in the neuron membrane of size $0.1\ mm$ in time of order at or below
$1\ ms$ ($\frac{10^{-4}m}{10^{-1}m/s}\ s$); see the length of the
$\frac{dV}{dt}$ pulse measured at the beginning of the 
\gls{AIS}~\cite{BeanActionPotential:2007},
see Fig.~\ref{fig:VoltageTimeDerivative}, which time is prolonged
up to $10\ ms$ by the neuronal $RC$ circuit; the ions are slow when
the voltage on the \gls{AIS}
is low, see Eq.~(\ref{eq:StokesSpeed}).
Assuming those distances and speeds, including the \emph{potential-assisted}
speed of the slow current, we are on a time scale matching the available
observations.
\index{current!across layers}



\subsubsection{Ion selectivity\label{Electrodiffusion_IonSelectivity}}
Maybe the mechanism of channel passing can also contribute to explaining
ion selectivity. "The normal selectivity cannot be explained by
pore size, because $Na^{+}$ is smaller than $K^{+}$~\cite{MolecularBiology:2002}".
The two ions have the same charge, but $K^{+}$ is nearly 70\% heavier
than $Na^{+}$, a definite disadvantage when accelerated by a vast
electrical gradient. When the layer on the arrival side gets saturated,
its potential reaches the potential of the bulk on the high concentration
side (this is necessary to decelerate the accelerated ions), and so
the channel gets closed (the accelerating potential disappears for
a short period until the ions from the layer flow away toward the
drain or they diffuse toward the bulk). We assume that the ions continuously
accelerate, then decelerate, due to the potential gradient (which
we assume to be constant for a moment). When $Na^{+}$ ions stopped
after passing the channel and built up a repulsive layer proximal
to the channel's exit, the $K^{+}$ ions passed only about 60\% of the
channel's length. The $Na^{+}$ ions, which started from the departure
layer with a handicap of $2\ to$ $3\ nm$, will arrive earlier than
the $K^{+}$ ions from the $0.1\ nm$ thick charged layer proximal
to the channel's entrance. That is, this handicap results in a strong
enrichment of $Na^{+}$ ions. For the detailed calculations see section~\ref{sec:Physics-ElectrodiffusionDynamics}.

Given that the potential reverses, the late ions are decelerated and then
accelerated in the reverse direction (recall that the layer they started
from is still empty and attractive), they simply go back to
the departure side. The ions also repulse each other while being accelerated
(the accelerating gradient acts on a distance of $5\ nm$ while the
ions may approach each other to a distance of $0.1\ nm$, so the mutual
repulsion can be significant). In this way, the heavier ions help
their competitors and vice versa. (The different ions can also connect
to different, heavy-weight components of the solution, drastically
changing the picture.) The result is that only the lighter ions can
pass the channel from an ion mixture when the cup is suddenly removed.
The passage is super-fast; it is in the $psec$ region (with a \emph{voltage-accelerated}
speed compared to the \emph{voltage-assisted} speed of after-loading
ions from the next layer), and the created potential quickly decays by diffusion.

\index{layer!ion selection}
The commonly used picture about the operation of selectivity filters
is surely wrong. The assumed mechanical operation of the pores
is simply too slow: the assument structural change needs $10^{-8}\ s$
and the ions passage time is about $10^{-11}\ s$ (furthermore, it must be repeated about $10^{3}$ times per passage). If a wrong ion is catched,
it must be transported back to its departure side, through the right ions
(against their repulsion), and the right ions must retry.
Neither for moving forward nor backward an appropriate driving force is present in that picture.



\subsection[Equivalent circuits]{Equivalent circuits\label{sec:PHYSICS_EquivalentCircuit}}

One wrong consequence of forgetting that the charge transfer mechanism is entirely different,
the charge carriers are large and heavy (and as a consequence: slow) ions instead of electron (cloud),
furthermore they are not necessary present in the 
volume under test and the 'construction' of biological matter 
enables the tested medium to produce charge carriers,
using 'equivalent circuits' for
neuronal operation. This fallacy
entirely falsifies the conclusions from the measurements.

Another wrong consequence of using 'equivalent circuits' to describe
the electrical operation of neurons is believing that the currents in
the biological circuit do not change the concentration, and through
the concentration, also the potential; see also section~\ref{sec:Physics-Resources}. The 'equivalent circuits',
of course, use a constant potential (they follow the abstraction used
in the theory of electricity, although the 'ideal batteries' also
may produce their voltage using chemical processes), and unreasonably, some mystic process changes the resistance/conductance of components. This wrong abstraction
results in numerous misunderstandings, among others, introducing ideas
such as parallel oscillator equivalent of neuron, input resistance,
delayed rectifier current, resting current, and time- or voltage-dependent
conductance. Furthermore, we cannot 
interpret, among others, \emph{how neuronal electricity
works in lack of external potential}; \emph{how slow currents operate
neuron's infrastructure}, \emph{how and why action potential is generated}.
Deriving the time course of the Nernst-Planck potential opens the
way to a quantitative understanding of neurophysical electrical processes,
including their time course.

Again another wrong consequence is that the two secondary abstractions 'potential',
and 'current', became independent from the primary abstraction 'charge'
and each other (significantly contributing to the fallacy that 'physics cannot describe life'). Our equations and the underlying discussion point
to the fact that \emph{the potential and the current cannot be separated
from the charge}. No 'delayed rectifying current' and 'voltage- (or
time-) dependent conductance' exist. Those notions originate from
the wrong interpretation of measured data derived from mismatching
measured electrical data pairs and the misconception that biological
structures and materials must behave like metals.


As we discussed in section~\ref{sec:Physics-TwoSegments}, 
the unbalanced charge in neurons and the charge injected
suddenly into the intracellular segment during a rush-in action
are in the same order of magnitude. We also discussed that 
the charge injection (the appearance and flow of charge in the proximal layer on the membrane's surface) causes well measureable mechanical, optical, etc. changes~\cite{MechanicalWaves:2015} on the membrane.
Those changes are well observable on the low-concentration side, in the form of slow currents, and large-scale
concentration and potential changes.

Those changes also mean that a large amount of charge
passes from the high-concentration side
to the low concentration side of the membrane, and causes a sudden drop
on the high-concentration side of the membrane. As discussed,
that change means simultaneously a change in the corresponding 
potentials across the membrane that solves the mystery why the
\gls{AP}
AP stops. The potential across the membrane is defined by the difference of the
potential of the two thin layers (see section~\ref{sec:Physics-Membranes}) on the membrane's surface.
As we discussed, a huge electric field exists across the
plates of the neuronal condenser, while a moderate one 
in the proximal layers. That means that the ions can quickly
leave the high concentration when the rush-in injection begins.
They produce an 'empty' layer (and a potential gradient)
whithin that layer and the ions in the neighboring layer
start to move using the 'downhill' potential. However, they do have a much slower speed, so they 'stop' at some distance. 


\subsection[States]{Operating states\label{sec:PHYSICS_STATES}}

The two electric fields show essentially different dependencies on the neuron's parameters.
Recall that although the parameters we used are biologically plausible,
they are only estimations from non-dedicated measurements.
The qualitative conclusions, however, remain valid.
As Figure~\ref{fig:NernstPlanckThermalWidth} shows, in the function of the membrane's thickness,
the electric field across the membrane due to electric charges
(it does not depend on the thickness of the membrane), for different physically plausible assumptions;
furthermore, the thermal electric field.

The balanced state is set to where the thermodynamic and electric "electric field" diagram lines cross each other; 
around $5\ nm$; the value we assumed in evaluating our equations.  
On the other hand, we can check the electric field's dependence on the summed-up concentration of the ions in the segment (see Fig.~\ref{fig:NernstPlanckThermalConcentration}).
Here, the thermodynamic electric field is constant but different at different width parameters. The classic condenser does not provide a reasonably balanced state (a matching line). The dielectric diagram line seems to provide a realistic estimation 
when using biologically reasonable parameters. 
One can describe the neuron as a balanced system at the intersection of the sloping and the horizontal lines. In the resting state, only minor deviations from this state exist
in both the electric field and the chemical concentration.


\begin{figure}
\iflatexml
	\includegraphics[width=.8\columnwidth]{fig/NernstPlanckConcentration.svg}
	\else
	\includegraphics[width=.8\columnwidth]{fig/NernstPlanckConcentration.pdf}
	\fi
	\caption{The electric and the thermal "electric fields" at different concentrations, see Eq.(\ref{eq:ElectricGradient}).The thermal electric field varies with the thickness parameter and depends on the concentration; see Eq.(\ref{eq:NernstPlanckThermal2}). Balanced states are where 
	the sloping lines cross the horizontal ones. The dot marks where
	% \gls{HH}
	Hodgkin and Huxley provided measured data.
	\label{fig:NernstPlanckThermalConcentration}
	}
\end{figure}


In the transient state, if we assume that a charge transfer causes a significant potential change around $100\ mV$ in the  layer of width 
$100\ nm$, it means a $0.1*10^6 \bigl[\frac{V}{m}\bigr]$ jump in the electric field that corresponds to a $100 \bigl[mM\bigr]$ jump in the concentration (for only a short time and only in that mentioned layer not in bulk; see also Fig.~\ref{fig:RestingPotential4}). During issuing an \gls{AP},
the system finds its way back to the crossing point (the path is not simple: both concentrations on both sides of the membrane and the electric field in the mentioned layer change; the process is relaxation, as it is seen in the shape of an 
\gls{AP}).

Even we can imagine how the electric process happens in this 
controlled system (for visibility, the positions of the marks are not proportional to the mentioned numbers). The system is in a balanced state at a crossing point. The two circles connected by bent arrows represent that when an excitation begins at the point with lower
concentration and higher potential (the cytoplasm side and negative resting potential), ions rush into the surface layer, so the ion concentration increases, that increases the electric
field; which is seen that the membrane potential suddenly increases.
The increased potential starts a current, but the current is slow,
and the intensity of the current through the channels in the
membrane's wall is higher than the current of the drain 
(\gls{AIS}) so the system issues an \gls{AP} while returning to its starting point.
The system will move along the line, with the restriction that it is a movement in the phase space
(the change in concentration is not proportional with the time),
and the different points on the membrane's surface may have the potential
at different times (the spatiotemporal behavior).
Due to the capacitive current, the local potential may temporarily drop
below the set-point (as observed in physiology, this is known as hyperpolarization).
The time course of the processes are described by the laws of motion of biology~\cite{VeghNon-ordinaryLaws:2025}
(the time derivatives of the Nernst-Planck equation),
and the simplified model works as described in section 4 of~\cite{VeghTechnomorphBiology:2025}
and its computational details in~\cite{VeghNeuronAlgorithms:2025}.

