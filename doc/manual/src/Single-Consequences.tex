% Physics: consequences of the correct physical mechanisms


\section[Consequences]{Consequences of the resting potential\label{sec:Single-RestingConsequences}}

Our single abstract neuron differs from the classic neurons
in several points.

\subsection{Resting current\label{sec:Single-RestingCurrent}}
Although through the \hyperlink{PhysicalIonLayerThickness}{thickness of the ion layer $\Delta z$}, the ion's water-related behavior can influence the surface charge density,
\index{charge density!surface}
one can draw the conclusion that when only one type of ion is solved in the liquid, approximately the same
concentration difference between the segments produces approximately 
the same "thermodynamic electric field" that can counterbalance the 
\index{electric field}
electric field of the biological condenser. One can notice that in the classic measurement by Hodgkin-Huxley, the sum of the concentrations of $K^+$ and $Na^+$ in the cytoplasm and extracellular fluid  (in mM) are 380 and 390, respectively, while the sum concentrations of $Cl^-$ and $Ca^{2+}$
are 450 and 460; practically all equal, as our analysis derived it must be. The deviations are well within the uncertainty of the 
measured values.  


\gls{HH} did a complete measurement, and derived \textit{precise} 
measured data. However, their measurement was not \textit{accurate} because 
their model was wrong.
In a balanced state, no 
current flows, so no voltage is generated.
For this reason, they assumed that a permanent leaking current $I_L$
was flowing on the membrane, and on its resistance, it generated
a $42.5\ mV$ voltage~\cite{CompanionGuideHodgkinHuxley:2022}.
In \gls{HH}'s picture,
the question \textit{what ion constitutes the current} remains open. 
It was claimed that the only permeability pathways open at rest are $g_{leak}$ and $g_K$. It follows that at the resting membrane potential, the leak current equals the potassium current. That is, at rest, two currents flow (without driving force in a balanced state), one consisting of $K^+$ ions and another of one $leak^{+/-}$ ions (??) (depending on the direction, it could be positive or negative),
and neither of them changes the concentration neither on the departure nor on the arrival segment, neither the concentration of the other ions; defying also Nernst's law.

The idea of leakage current was wrong: the voltage 
\gls{HH} measured~\cite{CompanionGuideHodgkinHuxley:2022}  is correct, but
-- as our model correctly explains -- it was generated by charge separation instead of a current through a distributed resistor.
It is permanently present while ionic segments are on its sides, even  when no current flows. Measuring the energy consumption of neuronal operation~\cite{EnergyNeuralCommunication:2021} confirms that no resting current,
in the sense as \gls{HH} used the notion, exists.



\subsection[Heat absorption]{Heat emission/absorption\label{sec:Physics-HeatAbsorption}}
As explained in section~\ref{sec:PHYSICS_VOLTAGE}, there are
alternative methods of producing a voltage difference across
the membrane's surface; with and without current. Hodgkin and Huxley took the wrong choice.
The lack of resting current also solves the long-standing mystery of "reversible \indexit{heat release} during the action potential", which is
\index{heat absorption}
 inconsistent with the Hodgkin–Huxley model. In the framework of that
theory, Ohmic currents flow through resistors that dissipate heat due to
friction, no matter in which direction ($W=I^2*R$) the ion currents flow. As Hodgkin wrote~\cite{HodgkinConduction:1964}: "Hill and his colleagues found~\cite{HeatProductionNeuron:1958} that an initial phase of \indexit{heat liberation} [of the action potential] was followed by one of
heat absorption. [...] a net cooling on open-circuit was totally unexpected
and has so far received no satisfactory explanation."


\begin{figure}
	\includegraphics[width=\textwidth]
	{fig/ElasticityGradientHeimburg.png}
	\caption{The red diagram lines show the membrane voltage
	and its gradient during producing an action potential~\cite{BeanActionPotential:2007}, Fig. 4.The black diagram lines show the size-related and heat-related changes during producing an action potential~\cite{HeimburgPhysikOfNerves:2009}, Fig. 1.
		and~\cite{ThermalBiophysics:2007}, Fig. 18.5. The voltage gradient has, by convention, opposite polarity.}
	\label{fig:NeuronElectricHeatGradients}
\end{figure}


The issue here is that the dissipation was attributed to the "leakage current."
\index{leakage current}
(Although, due to the different conduction mechanisms, it is not dissipated in the way electron energy is dissipated in solids.)
However, as our explanation shows, the voltage generated across the membrane is due to charging instead of a permanently flowing current, so no dissipation was present given that no resting current was present. Although the energy to build a potential on the membrane and to maintain it needs biological energy consumption, the process of generating an action potential is, in principle, reversible: ions flow into the layer on the membrane and flow out through the 
\gls{AIS}.

According to that, the "temperature of individual particles" with the mean kinetic energy $E$ is $T=\frac{2*E}{3*k_B}$ (where $k_B$ is the Boltzmann constant and $T$ is the thermodynamic temperature of the
bulk quantity of the substance); that is the temperature is directly proportional to the kinetic energy. The ions on the two sides of the membrane are in the state with temperature $T$ before and after the
% \gls{AP}.
AP. When the $Na^+$ ions rush into the intracellular space, they gain energy through electrostatic acceleration by the membrane's electric potential, so the same ions appear on the intracellular side as having more energy, i.e., slightly higher temperature. In the second phase of the 
%\gls{AP},
AP, those more energetic ions that provide excess voltage above the resting potential leave the membrane through the 
%\gls{AIS},
AIS,
 so the temperature decreases in the second phase. 
The heat liberation and absorption is a direct experimental evidence of the "slow current" and the mechanisms described above.



\subsection{Reversal potential\label{sec:Physics-RestingReversal}}
From our picture, it is clear that the reversal potential is reached 
when the potential on the inner segment is lowered or lifted in such a way 
that the orange sloping potential in Fig.~\ref{fig:RestingPotential4} turns horizontal. To reach it (provided that
the potential on the other side is kept fixed), the concentrations on the
invasion site must be  changed. The invasions resulting in the exact (but opposite) change of potentials from the two sides leads to different concentrations.
The invasion acts as changing the potential at the corresponding surface side of the membrane. Given that the difference between the two potentials on the two surfaces defines the driving forces, the concentration on the other side is also affected.
That is, the reversal potential exists, but its value depends on the context, see the discussion about invasions.

\subsection{Pressure wave\label{sec:Physics-PressureWave}}
By using the value of the force acting on a unit charge
in the field across the membrane is
 \begin{equation}
 	F_{Na^+}= 10^7 *1.60217663* 10^{-19} = 1.6*10^{-12}\ [N] \label{eq:UnitForce}
 \end{equation}
we can estimate how the pressure of the neural cell increases due to the rush-in charges at the beginning of the
\gls{AP}. 
As evidence shows, the local potential at the internal surface of the membrane increases by $\Delta U = 100\ [mV]$. This increase means a change in the force acting on an ion (see Eq.~(\ref{eq:UnitForce})) by  $1.6*10^{-12}\ [N]$.
 When we assume $10^7$ rush-in ions and unchanged cell size, the total force acting on the membrane increases by $1.6*10^{-5}\ [N]$.
 This change in force means on the neuron's $8*10^{-9}\ [m^2]$ surface (see~\cite{JohnstonWuNeurophysiology:1995}, page~12) a pressure change 
 \begin{equation}
 	\Delta P = \frac{1.6*10^{-5}\ [N]}{8*10^{-9}\ [m^2]}
 	= 2*10^{3}\quad \biggl[\frac{N}{m^2}\biggr]
 	\label{eq:CellPressureChange}
 \end{equation}
 
 This change is large enough to explain why 
 pressure wave and other mechanical changes~\cite{MechanicalWaves:2015} also start at the beginning of the
 \gls{AP},
and other (such as optical, density) changes 
 are accompanied with it. Nature invests energy also 
 in the conventional way, a term $P\Delta V$, into the thermodynamics of neural operation; not only in the form of storing energy in the 
 changed electric field. Thermodynamics and electricity must not be 
 separated also from the point of view of the appearance of energy investment. 
 
 Given that~\cite{PressureChangeActionPotentialMeasured:1980}measured $5\ \frac{dyn}{cm^2} = 0.5\ \frac{N}{m^2}$, we have good reason to presume that this changed pressure increases the neuron's size and the neuron's elasticity.  
The peak of swelling was shown to coincide fairly accurately with the peak of the action potential (i.e. when the concentration reaches its 
peak value)~\cite{RapidMechanicalBiphaseActionPotentialMeasured:1981}.

 Notice an important difference. The acceleration of an ion is unbelievably large. It is sufficiently large to keep the potential at the same value,
 provided that the ions must follow a small change, such as due to the
 "leaking" current through the
 \gls{AIS}.
 Similarly, the ions can 'instantly' follow quick changes such as
 a square wave gradient.
 However, if many similar ions are ahead, their repulsion decreases
 the acceleration, and the ion travels only at a few $m/s$ speed.
 The huge forces and accelerations means that a potential change
 acts immediately. However, the force decays quickly. The ions start to move
 'instantly', but the charge carriers can move only with a limited speed,
 much below the interaction speed of
 \gls{EM}
 interactions. The effect can propagate only with that lower speed.

 See the case of axon: there exist a mechanical constraint that
 the ions cannot spread through the wall (they must keep the direction), and
 the ion package propagates as Equs.\ref{eq:Nernst-dVdt} and~\ref{eq:Nernst-dCdt} describe it.

 


\subsubsection{Membrane's charge\label{sec:Physics-ChargeOnMembrane}}

\cite{KochBiophysics:1999}, page 7 provides a numeric estimation that "a spherical cell of $5-\mu m$ radius with a resting potential
of $-70\ mV$ stores about $0.22 * 10^{-12}$ coulomb of charge just below the membrane and
an equal but opposite amount of charge outside".
As we estimated above, a charge in the same order of magnitude (or above it)
is involved in forming and action potential, so the assumption that
the membrane is a kind of electric circuit with a fixed potential
and the moving ion current does not change membrane's potential,
is far from reality. That amount of charge represents $10^7$ ions.
From this we can estimate that the ions are at a $0.5\ nm$ distance from each other.

As the caption of Fig.~11.22 in~\cite{MolecularBiology:2002} formulated:
"A small flow of ions carries sufficient charge
to cause a large change in the membrane potential."


\subsection{Role of proteins\label{sec:Single-RoleOfProteins}}

"When measuring currents through small membrane segments, the conductance events appear in a quantized steps,
i.e. one observes a switching between current-on and currentoff events. In biological membranes one normally assumes
that this phenomenon is caused by the opening and closing
of ion channel proteins (Fig. 3a). The two conduction states
are called open and closed channels. As already described
above, these channel proteins play an important role in the
Hodgkin-Huxley model of nerve pulse conduction. Interestingly, one finds very similar events in synthetic membranes
that are free of proteins (e.g., [10, 11]), in particular if one
is close to the melting transition (Fig. 3b). Both current
amplitude and the typical opening time scales are practically indistinguishable from those of protein-containing
membranes. Obviously, the finding of quantized current
events is not a proof for the activity of proteins even though
this is what is generally assumed since membranes close
to transition display the same characteristics. Whether the
events in protein-containing membranes and those in pure
lipid membranes have the same origin is yet an open question."~\cite{HeimburgPhysikOfNerves:2009}

\section[Implications]{Implications for computing, communication, information\label{Single-Implications}}
\subsection[Information]{Delivering information\label{sec:Single-OperationInformation}}

As follows from our interpretation, the appearance of a huge voltage gradient
(evoked by the sudden rush-in current) represents the output
information the neuron delivers, and also the input information
it receives from its upstream neurons through its synapses.
\index{voltage gradient}
Given that the input currents delivered by the spikes
are gated  by the neuron, the information that can be
accounted in the computation must be in the front side of spikes. (The back side is mainly needed for restoring the resting potential.)

The front side (in the form of a sudden step in the value of the voltage gradient) delivers an extremely precise timing information about at what time the rush-in event in the sender happened, explaining the half-understood experience~\cite{SubmillisecondResolution:2008} why
"the timing of spikes is important with a
precision roughly two orders of magnitude greater than the temporal
dynamics of the stimulus". If exceeding the threshold
is the consequence of the charge arriving from a single
upstream neuron, the neuron simply transmits the timing information it received. If several smaller gradients
are summed (and recall that the component gradients decay with
time after their arrival) for reaching the gradient,
then their information content is weighted.


\subsection[Computing]{Implications for computing\label{sec:Single-ImplicationsComputing}}
 
Fig.~\ref{fig:AP_Conceptual} also reveals some secrets of the effective
biological computing.
\begin{itemize}
\item biology makes the "weighted summing" of neuron's synaptic inputs 
simultaneously, in one single operation making multiplication and integration, furthermore selecting the time window and its effective synaptic inputs
\item the heavily used neuronal information "is stored, as it should be, in every circuit"~\cite{SterlingPrinciples:2017}
\item "information stored
directly at a synapse can be retrieved directly"~\cite{SterlingPrinciples:2017}
\item part of the information (in 'volatile' memory) is stored only
for the period when it might be needed (a real temporary cache)
\item "Computing" is much shorter than "Delivering"
\item asynchronous; the operation time varies (no pipelining)
\item "Send only information that is needed, and send it as slowly as possible"~\cite{SterlingPrinciples:2017}
\item using voltage temporal gradients enables transferring more information and functionality;
for example, synchronization~\cite{LosonczyIntegrative:2006}
\item for simulation: there is no need to send and integrate spike shapes,
only \textit{the time of arrival/sending} 

\end{itemize} 


\subsection[Communication]{Implications for communication\label{sec:Single-ImplicationsCommunication}}
