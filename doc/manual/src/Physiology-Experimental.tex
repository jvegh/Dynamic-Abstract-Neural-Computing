% Interpreting published measurement data with our theory

\section{Experimental evidence\label{sec:Physiology-Experimental}}

Using published data, we can derive direct experimental evidence for
our statements.

Researchers, including Hodgkin and Huxley~\cite{HodgkinHuxley:1952},
are thinking in the Newtonian way. That is, they expect
the current to appear promptly after switching on the clamping voltage.
By using so, 
they assume that the transport speed of electrodiffusion in an axon and the electromagnetic field propagation aere the same.
That fallacy is why  Hodgkin and Huxley were convinced that \emph{the slowly moving
charged object that they observed could not be current}.


\subsection{Stokes-Einstein relation\label{sec:Physiology-StokesLaw}}

As we 
derived theoretically in section~\ref{sec:Calculating-ion's-speed}, in general case, the ion current generated by a potential gradient is proportional
with the  
{electric gradient}
(see Eq.~(\ref{eq:StokesCurrent}))
and so the macroscopic current speed {(see Eq.~(\ref{eq:StokesEinsteinSpeeddV})}. By using the
%\hyperlink{StokesCurrent}
{Stokes-Eintein relation} relation we can also express the speed in
the function of the potential gradient.
\index{Stokes-Einstein relation}
According to that prediction, when we interpret 
that the \hyperlink{voltage-clamping}{clamping voltage} is uniformly distributed along the axon, we have a $\frac{dV}{dx}$ (proportional to the clamping voltage),
and the current proportional to ion speed, we can expect a linear dependence between them.
\index{speed!potential-assisted}

The mostly known and influencing axon current measurement
has been published in 1952~\cite{HodgkinHuxley:1952}. 
They used a single-axon input and measured at different \hyperlink{voltage-clamping}{clamping voltages} the
neuronal membrane's current (although they called it as conductance), which in this way was identical to the axon current. 
Their result is reproduced in Fig.~\ref{fig:FittingExperimentalData} as the black bulbs and diagram lines. 


As we discussed, ions diffuse across the axon's wall, producing a saturation-type current in the axon and later on the membrane.
As expected from our theoretical consideration in section~\ref{sec:Axonal-charge-delivery},
the experimental data are fitted with the theoretical function~\ref{eq:AxonalCurrentNoConc} to their experimental data (just reading back
their graphically published measurement results).
The parameters of the
theoretical function are displayed in Fig.~\ref{fig:SpeedDependence}
in the function of the \hyperlink{voltage-clamping}{clamping voltage}.
Our simple
model assumes that the $\alpha$ time constant (through $v$
and $\frac{dV}{dx}$) depends on the clamping voltage). 
As expected, the potassium current, proportional to ion speed, changes linearly in function of
the voltage gradient (proportional to the clamping voltage), see the blue dots.
\index{voltage gradient}
The similar
dependence of the time constant on the clamping voltage (see the red circles) underpins that 
our theoretical discussion leading to Eq.~(\ref{eq:AxonalCurrentNoConc})
is correct. Even, the diffusion coefficient or viscosity can be derived.
The diffused-in ions were transported towards
the membrane as a "slow" macroscopic ionic current (the speed
of current HH
%\gls{HH}
~\cite{HodgkinHuxley:1952} measured
and also theoretically derived to be about $20~m/s$); it is in the
order of magnitude we mentioned for the speed of macroscopic currents
in metals and electrolytes. 


\begin{figure*}
\iflatexml
\includegraphics[width=.65\textwidth]{fig/HodgkinHuxleyFig3.svg}
\else
\includegraphics[width=.65\textwidth]{fig/HodgkinHuxleyFig3.pdf}
\fi
		\caption{Finding time constants and membrane current by fitting data measured
			by HH
%			\gls{HH} 
			(Fig.~3 in~\cite{HodgkinHuxley:1952}) with
			our theoretically derived function (see Eq.(\ref{eq:AxonalCurrentNoConc}))("Copyright {[}1991{]} Society for Neuroscience") \label{fig:FittingExperimentalData}}
\end{figure*}

	
A systematic discrepancy exists at the low time values
of the time course function between the one fitted originally by~\cite{HodgkinHuxley:1952} and the one fitted by us.
The former one is a simple polynomial that is simply a wrong quasi-model; our fitting uses the correct model function. The dependence we use (a sudden and delayed exponential increase in membrane's current) has been experimentally 
measured by~\cite{SodiumCurrentDelay:2006}.
The figure suggests that the 
saturation current depends linearly on the speed of ions (i.e., on the \hyperlink{voltage-clamping}{clamping voltage}, see Fig.~\ref{fig:FittingExperimentalData} and Fig.~\ref{fig:SpeedDependence}) in the tube. 
	
	\begin{figure*}
	\iflatexml
		\includegraphics[width=.65\textwidth]{fig/Fig3_eval.svg}
	\else
		\includegraphics[width=.65\textwidth]{fig/Fig3_eval.pdf}
	\fi	
		\caption{The experimental proof of the validity of the Stokes-Einstein relation to neurons. The proportionality of membrane's current (i.e., ions' speed) and time constant, respectively,
			with the clamping voltage (i.e., the voltage gradients). Data taken from Fig.~\ref{fig:FittingExperimentalData}
			and Table~I of~\cite{HodgkinHuxley:1952}, respectively.\label{fig:SpeedDependence}}
                        \index{Stokes-Einstein relation}
	\end{figure*}
	
We can spot two issues in connection with their measuring
and one with the evaluation method of their excellent measurement.
A fundamental problem to solve when measuring chemical electrolytes
using electronic devices is their interfacing. At some point, the
ionic charge must be converted to electrons (there and back), which
usually happens in electrolyte electrodes. Interfacing the analyzed
electrolytic wire and metallic wire in the measurement circuit introduces
problems, not only the contact potentials but also the time delay
due to the using electrolyte electrodes. These electrodes need to
carry the ions to some distance, and that process is outside of the
time scale of the primary measured process. The effect is noticed
but not explained~\cite{HodgkinHuxley:1952}: ``the steady state
relation between sodium current and voltage could be calculated for
this system and was found to agree reasonably with the observed curve
at 0.2 msec after the onset of a sudden depolarization.'' Moreover,
given that \textit{the speed of ions depends on the depolarizing voltage
(see Eq.~(\ref{eq:StokesCurrent})), this time gap depends on the depolarizing
voltage}: the higher the voltage, the shorter the time gap, demonstrated
in their Fig.~3. 
Actually, their fitted polynomial chooses a wrong time scale
and adds the delaying effect of the electrolyte electrodes
to the measured time. 
Since this delay depends on the clamping voltage,
the measured time constant comprises a systematic voltage-dependent contribution, so it distorts the fitting and delivers wrong time constants.

The second issue is that they measured conductance, which measurement procedure
(as we discuss it in section~\ref{sec:PHYSICS_MEASURINGCONDUCTANCE})
means introducing a small voltage into the measured system and generating a small current in it, which -- by biological mechanisms -- may produce further
charge carriers inside it. The generated small current contributes to the
true current in the system, and the device measures their sum,
that is, the true current plus that offset. As long as the true current 
is large (notice that the clamping voltage spans nearly two orders of magnitude),
the contribution of the device causes only a negligible distortion.
At small \hyperlink{voltage-clamping}{clamping voltages}, however, that contribution is comparable
to the measured effect, so it significantly distorts the measurement 
and shows much higher current than the true one.


The third issue is, that they fitted their data with a polynomial function,
which draws a "smooth" diagram line, at low time values contracting
an initial "no current" period with a period where the current grows exponentially. Given that the "no current" period decreases as the
clamping voltage increases, the polynomial fits a variable composition
of current in those two periods. 

We fitted 
our theoretical function (see Eq.(\ref{eq:I_Source})) to their measured data published in~\cite{HodgkinHuxley:1952}, omitting the 
delay period due to the electrolyte electrodes.  
This way we eliminated the first and third issues. We derived the timing constant and saturation current values using the clamping voltage as parameter. 
In Fig.~\ref{fig:SpeedDependence}, we compare our fitted data  values 
with those derived by HH (displayed in their Table I).


In the case of using the right function for fitting the measured current value,
we receive the theoretically expected conclusion, that the time constant depends linearly
on the \hyperlink{voltage-clamping}{clamping voltage} (that is, on the voltage gradient),
while fitting the data with the wrong (polynomial) function,
the time contants show an opposite dependence.
The saturation current shows in both cases a linear dependence.

The wrong evaluation method led Hodgkin and Huxley to conclusions opposite to the real ones, from the correct measured data. Their measurement is precise but not accurate. 
It has very sever consequences: covers the presence of "slow current" and
disables understanding the physical process happening inside the neuron.
Given that in the meantime the measurement technology developed (smaller 
electrolyte electrodes with much shorter delays furthermore conductance meters 
with higher internal impedance have been developed), and they coded those parameters (without the voltage-dependence we pointed out) into their polynomial coefficients (which are used in their differential equations), those issues significantly contributed to the 
difficulties their followers experienced when applying their equations.


 
\subsection{Axon \label{sec:Axonal-charge-delivery}}

	We have a constant voltage at the end of the tube; the "slow"
	current "flowing out" from the tube increases and appears on the
	membrane, \emph{establishing the illusion that the conductance of
		the tube increases}. \emph{Resistance/conductance cannot be interpreted
		when charge carriers flow into the resistor} (and this is the case
	with the axon) from the environment: the "slow" current produces
	a different behavior (increases the number of charge carriers $n$), see Eq.~(\ref{eq:StokesCurrent}).




%
%\subsubsection{Artificial mode}
%
%
%\section{Experimental evidence}
%
%
%\subsection{Membrane}
%


\begin{figure*}
\iflatexml
\includegraphics[width=.65\textwidth]{fig/MasonPSPFig4.svg}
\else
\includegraphics[width=.65\textwidth]{fig/MasonPSPFig4.pdf}
\fi
	\caption{A) 
	%\gls{PSP}
	PSP decay (curve 1) and the decay after an injected depolarizing
		current pulse (curve 2) recorded in the same cell. B) Voltage traces
		(upper and middle) from which the curves in A were derived, together
		with the current record (lower) for the pulse. The colored marks and
		diagram lines are calculated using the model's "slow"
		current. Measurement data (with black) are reproduced from Fig.~4
		of~\cite{SynapticTransmissionMason:1991} ("\copyright
		[1991] Society for Neuroscience" ). \label{fig:MasonFig4}}
\end{figure*}

\begin{figure*}
\iflatexml
\includegraphics[width=.65\textwidth]{fig/HH_AP.svg}
\else
\includegraphics[width=.65\textwidth]{fig/HH_AP.pdf}
\fi
\caption{The "ghost image" formed by the delayed membrane current: the
		origin of the 
		%\gls{AP}. 
		AP. The finite-speed ions transferred on the
		finite-size surface of the membrane: Kirchoff's Law in biology.
		The assumed delay time between input and output currents is $0.49\ ms$,
		the function form and its parameters are displayed in Fig.~\ref{fig:Post-synaptic_potential_lin}. \label{fig:The-ghost-image_AP}}
\end{figure*}


\begin{figure*}
\iflatexml
	\includegraphics[width=.45\textwidth]{fig/HodgkinHuxleyUI_On.svg}\includegraphics[width=0.45\textwidth]{fig/HodgkinHuxleyUI_Off.svg}
\else
	\includegraphics[width=0.45\textwidth]{fig/HodgkinHuxleyUI_On.pdf}\includegraphics[width=0.45\textwidth]{fig/HodgkinHuxleyUI_Off.pdf}
\fi
	\caption{The time course of voltage and current a clamping
		experiment, calculated numerically for
		switching the clamping on and off.\label{fig:The-time-course_Clamping}}
\end{figure*}


\begin{figure}
\iflatexml
	\includegraphics[width=0.45\textwidth]{fig/MasonPSPlin.svg}
\else
	\includegraphics[width=0.45\textwidth]{fig/MasonPSPlin.pdf}
\fi
	
	\caption{Time course of the post-synaptic potential evoked by a single 
	%\gls{AP}.
	AP.
		The colored marks and diagram lines are calculated using the model's
		"slow" current. Measurement data
		reproduced from Fig.~2 of~\cite{SynapticTransmissionMason:1991}
		("\copyright [1991] Society for Neuroscience").
		\label{fig:Post-synaptic_potential_lin}}
\end{figure}



\subsection{Invasions\label{sec:Physology-ElectrodiffusionInvasion}}

Applying a step-like invasion (either concentration of voltage square-wave
shaped change) 
\subsubsection{Concentration square vawe\label{sec:Physiology-ConcentrationDerivative}}

It has been experimentally investigated the behavior of membrane potential in response to sudden changes of the extracellular concentration of the two permeable ions of the cell.
The extracellular concentrations were abruptly changed, as depicted in the left side of Fig.~\ref{fig:ConcentrationSquareWawe}.
According to Eq.~(\ref{eq:NernstPlanck}), a negative square wave of the concentration change must provoke a drastic positive potential change,
with a time course described by Eq.~(\ref{eq:Nernst-dVdt}).
The step-like change results in a step in the voltage, and according
to the Stokes formula (see Eq.~(\ref{eq:StokesSpeed})) the ions feel a huge voltage gradient, so they will move with high speed, and an intense current will start.
\index{voltage gradient}
Given that the current means also delivering chemical ions, the membrane tends to find another equilibrium state, corresponding to the newly set concentration,
as explained in~\cite{JohnstonWuNeurophysiology:1995}, page 22.

%% Demonstrating the behavior of voltage/concentration derivative
\begin{figure*}
\iflatexml
\includegraphics[width=.53\textwidth]{fig/ConcentrationReduction.svg}
\else
\includegraphics[width=.53\textwidth]{fig/ConcentrationReduction.pdf}
\fi
\includegraphics[width=.47\textwidth]{fig/DifferentiatorWaveForm.png}
\caption{\label{fig:ConcentrationSquareWawe} Left: The effect of applying 
\textit{a square-wave-like concentration change} to membrane; Figure 2.2 from \cite{JohnstonWuNeurophysiology:1995} Right: the effect of applying \textit{a 
square-wave like voltage change} to an 
\href{href="https://www.electronics-tutorials.ws/rc/rc-integrator.html} {electric differentiator-type $RC$ oscillator}.}	
\end{figure*}

From an electric circuit point of view, the abruptly appearing and disappearing new thermodynamical driving force starts to remove charges and recover charges Essentially, it indirectly discharges and charges the solution. In the right side of the figure, the effect on the output voltage of a differentiator-type electric equivalent circuit is shown under  applying a sufficiently long square-ware input voltage change to its input. The \textit{negative} edge of the concentration change causes indirectly a \textit{positive} edge in the voltage, and vice versa, so the  generated membrane potential should be compared to the reply of the electric circuit to a square-wave input voltage. We can deduce that \textit{the serially switched differentiator-type $RC$ circuit faithfully reproduces the membrane's electric behavior}. 
From the figure we can conclude that the corresponding time constant $\tau$ can be around $0.3~RC$.

From the figure we can estimate that the "half width" of the \textit{concentration-provoked} potential change is about $200~sec$. 
(The experiment is not dedicated, we just read back data from a textbook figure.)
In Fig.~\ref{fig:VoltageTimeDerivative}, we see that the 
\textit{electrically-provoked} potential gradient's half width is about $0.1~ms$ (the lower figure, a relatively short non.rectangle excitation). The  experimental value of the ratio of these widths is $2*10^6$ (it provides the ratio of the corresponding interaction speeds). For the theoretical value see Eq.~(\ref{eq:PhysicsGradientRatio}).0 If we assume that the
%\gls{EM}
propagation speed in the electrolite is $2*10^8~m/s$, we can conclude $10^2~m/s$ \textit{potential-assisted} speed for the change of concentration (actually, also the  potential-assisted speed of current), in line with our other estimations and the published measurements.


\subsubsection{Voltage square wave\label{sec:Physiology-VoltageSquareVawe}}

Similarly, one can apply a voltage square wave to a biological cell, see Fig.~\ref{fig:VoltageSquareWawe}. Compare it to the bottom row of 
the electrical simulation. From the figure we can conclude that the corresponding time constant $\tau$ can be around $0.2~RC$.

Notice that the arrival of a square wave evokes
\begin{equation}
  n = \frac{300*10^{-12}*5*10^{-3}}{1.602*10^{-19}} = 10^7
\end{equation}
ions, in the order of we assumed in section~\ref{sec:Physics-RushinCharge}.
\begin{figure*}
\includegraphics[width=.73\textwidth]{fig/KoleSquareResponse.png}
\includegraphics[width=.37\textwidth]{fig/DifferentiatorWaveForm.png}
\caption{\label{fig:VoltageSquareWawe} Left:  The effect of applying 
\textit{a square-wave-like voltage change} to membrane; Figure 2.a from \cite{ActionPotentialGenerationNatrium:2008}. Schematic
diagram of the recording configuration in $Na^+$-free ACSF showing the location
of the two ACSF application pipettes for applying $Na^+$-rich ACSF to the apical
dendrite (blue) and AIS (red), and the somatic whole-cell recording pipette.
Right, examples of whole-cell $Na^+$ current evoked by voltage steps (middle),
which were preceded by brief (5 ms) applications (bottom) of $Na^+$-rich ACSF
to the AIS (‘axon’, red) or the proximal apical dendrite (‘apical’, blue). Right: the effect of applying \textit{a 
square-wave like voltage change} to an 
\href{href="https://www.electronics-tutorials.ws/rc/rc-integrator.html} {electric differentiator-type $RC$ oscillator}.}	
\end{figure*}

\subsubsection{Action potential\label{sec:Physiology-InvasionAP}}

As we discussed, after the membrane's potential exceeds its threshold,
a membrane-evoked external invasion happens.
The suddenly appearing potential change (although it is distributed 
over its surface) on the membrane is in resemblance with those square-wave 
excitation functions. Not surprisingly, the amount of charge
which evokes an
%\gls{AP}
AP
is also similar, as we discuss in section~\ref{sec:Physics-RushinCharge}.