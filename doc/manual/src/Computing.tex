\iflatexml
\else
\minitoc
\fi
\chapter[Computing]{Neural computing\label{ch:Computing}}
	% The foreword for computing
\quotationbox{
\textit{each neuron is a compact, efficient, nonlinear, analog summing device}, \dots \textit{the rules of which are not quite understood as of yet}.~\cite{SomjenBook:1972}~@1972\\
%
The ultimate aim of computational neuroscience is
to explain \textbf{how electrical and chemical signals are used} in the brain
to \textbf{represent and process information}~\cite{SejnowskiComputationalNeuroscience:1988}~@1988
%
The brain computes!
This is accepted as a truism by the majority of neuroscientists.~\cite{ KochBiophysics:1999}~@1999\\
We so far \textit{lack principles to understand rigorously how computation is done}
in living, or active, matter"~\cite{NaturalComputation:2018}~@2018\\
Yet for the most part, \textit{we still do not understand the brain’s underlying computational logic}~\cite{BrainInitiative10:2024}
}

%\begin{advanced}
%"\textit{the dynamic interaction of inputs in dendrites containing voltage-sensitive ion channels is capable of realizing logical operations, nonlinear interactions, and local domains of computation. This raises the possibility that a neuron is itself a network.}"~\cite{SingleNeuronComputation:2014}~@2014\\
%Leads to the question: "Is realistic neuronal modeling realistic?"~\cite{RealisticNeuronalModeling:2016}~@2016\\
%\end{advanced}

%
Despite the impressive results of grandiose projects~\cite{BrainInitiative10:2024,HumanBrainProject:2018}, and the sometimes triumphal communications, there has been no significant advance in understanding neuronal computing in four decades. 
Unfortunately, the they attempt to merge the decades-old, outdated theory~\cite{HBP_HodgkinHuxleyNeuronBuilder:2023} with the modern technical hardware and software facilities.
Furthermore, even the true advances are miscommunicated~\cite{HBP_ScientificAdvances:2023}.
The true advances are mismatched with urrealist designs and plans.
The Human Brain Project believes that it is sufficient to build
a computing system with theoretically sufficient resources for simulating 1 billion neurons, and do not want to admit that it can be used ~\cite{NeuralNetworkPerformance:2018} to simulate only by orders of magnitude less, below 100 thousand, simply because of the presently available serial systems do not enable to reach such a performance~\cite{VeghBrainAmdahl:2019}.
Despite that even the system with 1-million-core proved to be non-usable, building a 10-million-core system recived the green light.
Despite that lack of knowledge, the half-understood
\href{https://www.nature.com/articles/s42254-020-0208-2.pdf}{principles of neuromorphic computing}
are extensively used
\cite{NeuromorphicComputing:2015,PhysicsForNeuromorhicComputing:2020},
although  it is seen that \href{https://www.nature.com/articles/s41586-021-04362-w}{brain-inspired computing needs a master plan}~\cite{BrainMasterPlan:2022}.
Maybe, really, \textit{"a new understanding of the brain" [and the cooperation of scientific disciplines] is needed}.

	
\begin{advanced}	
"Indeed, the operation of our brain differs
vastly from that of human- made computing
systems, both in terms of topology and in
the way it processes information, which
explains its different aptitudes"~\cite{PhysicsForNeuromorhicComputing:2020}

Our "present-day digital computers are optimized for high-precision
calculations but consume an inordinate
amount of energy when they run the type of
cognitive tasks that the brain excels at"
~\cite{PhysicsForNeuromorhicComputing:2020}.
\end{advanced}




Today we have the "golden age" of neuromorphic (brain-inspired, artificial intelligence)  architectures and computing. However, the meaning of the word has changed considerably since Carver Mead~\cite{NeuromorphicSystems:1990}  coined the wording.
Today practically every single solution that borrows at least one single operating principle from the biology, and mimics some of its functionality in a more or less successful way, deserves this name. As always, to grasp out some single aspect and implement it in an environment and from components based on entirely different principles, is dangerous.
Historically, 'neuromorphic' architectures were suggested to be based on different principles and components, such as mechanics, pneumatics, telephones, analog and digital electronics, computing. Some initial resemblance surely exists, and even some straightforward systems can demonstrate more or less successfully functionality in some aspects similar to that of the nervous system. There is a noteworthy analogy between the deep learning of neuronal nodes and the long-term potentiation found in synapses.

However, when scrutinizing the scalability (i.e., how those systems shall work when used under real-life conditions in which a vast number of similar subsystems shall work and cooperate), the picture is not favorable at all.
"\textit{Successfully addressing these challenges [of neuromorphic computing] will lead to a new class of computers and systems architectures}"~\cite{NeuromorphicComputing:2015} has been targeted. However, as noticed by the judges of the Gordon Bell Prize, "\textit{surprisingly, [among the winners,] there have been no brain-inspired massively parallel specialized computers}"~\cite{GordonBellPrize:2017}. Despite the vast need and investments, furthermore the concentrated and coordinated efforts, just because of mimicking the biological systems with computing inadequately.

Given "\textit{that the quest to build an electronic computer based on the operational principles of biological brains has attracted attention over many years}"~\cite{FurberNeuralEngineering:2007}, modeling the neuronal operation became
a well-known field in both electronics and computing. At the same time, more and more details come to light about the 
computational operations of the brain. However, it would appear, that the 'wet' neuroscience
is miles ahead of the 'silicon' neuroscience. 
There are projects and exaggerated claims about extremely large computing systems,
even about targeting the simulation of the
brain of some animals and eventually even the human brain. 
Often these claims are followed by a long silence, or some rather slim or no results.
As that the operating principles of the large computer systems tend to deviate from the operating principles of a single processor,
it is worth reopening the discussion on a decade-old question
"\textit{Do computer engineers have something to contribute. . . to the understanding of brain and mind?}"
~\cite{FurberNeuralEngineering:2007}.
Maybe, and they surely have something to contribute to the understanding of computing itself.
\textit{There is no doubt that the brain does computing, the key question is how?}

	% The introduction to computing

\section{Introduction\label{sec:Computing-Introduction}}

Here \textit{computing} is handled in a broader sense: information processing \textit{in any implementation}.
It covers conventional computing, biomorphic computing,  biological (neural) computing, and computing relating, among others, (the technology of) artificial intelligence.
The computing objects use both their inputs and their internal state to calculate
their output. The time-aware computing means to consider that \textit{computing means both processing the available data
and delivering data to and from the computing object}.
Furthermore, that those operations must be synchronized
(and in this way they block each other); and that not only that those processes need time,
but \textit{the inputs, the output and the internal states all have their temporal behavior}.
We show that taking into account that temporal dependence explicitly,
leads to considerable differences in their behavior as opposed with the behavior
expected based on the time-unaware description. Please take care when reading.
The text is, of course, computing-oriented, so it uses words processor, core, thread,
hardware thread, memory, etc. However, it uses them in a slightly different way,
in a different meaning. So, please read the corresponding manual, or skim it at least,
before going into details. The approach we take seems to be overly complicated,
but it is needed to build a more effective and capable computing. It majorly simplifies
modern many-thread computing, but its real advantage manifest in large-scale computing.

Technical sciences (mainly electronics and computing science)
have developed to the level where elementary electronic components in number comparable
to the elementary components of the 
CNS
%\gls{CNS}
can be assembled.
Those large systems attempt to resemble each other. On the one side,
biology inspires huge electronical systems (from 
%\gls{HPC}
HPC to  %\gls{ANN}
ANN
).
On the other side, electronic systems (mainly large-scale computers,
but also special-purpose electronic simulators)
attemp to imitate brain-like biological systems, with goals ranging from simulating the dynamics
of molecular processes to creating artificial intelligence.
Furthermore, there are attemps to combine and interface them.

	

	The false parallels with electric circuits (i.e., neglecting the fundamental differences between the \textit{digital} and \textit{neuro-logical} operating modes),
moreover the preconception that nature-made \hypertarget{BiologicalComputing}{biological computing} must follow notions and conceptions of manufactured computing systems hinders understanding genuine biological computing. 



Pinpointing the interpretation of computing
Coming soon!

For now, see \cite{VeghRevisingClassicComputing:2021, VeghScalingANN:2021,VeghHowMany:2020}

The theory of \hypertarget{GeneralizedComputing}{generalized computing}, technical and biological


\section{\label{COMPUTING_COMMUNICATION} Computing and communication}

\section{\label{COMPUTING_INFORMATION} Computing and information}

\section{\label{COMPUTING_BIOLOGY} Computing and biology}
As discussed in \cite{ThreeStateUnidirectional:2004,
MarkovianIonChannel:2005}, for the adequate description of the operation
of ion channels, the major components of neuronal computing, three-state systems must be used.
In the present two-state digital electronic logic systems, discharging the internal capacitances can be considered a "refractory" period, i.e., a third state, which defines time's direction. However, it is not known if such a third state can be available among the quantum states at all.
Because of these reasons,  in the foreseeable future, quantum computers will not
represent an alternative general-purpose architecture.
 "Building such machines are decades away" \cite{ScienceQuantumComputers:2018}.
 However, %@link MODELING_SINGLE_AP_CONCEPTUAL 
 biological neurons are three-state systems @endlink.


\section{ Computational modeling of neuronal membrane\label{MODELING_NEURONAL_MEMBRANE_COMPUTATIONAL}}

%see also @link MODELING_ACTION_POTENTIAL modeling of Action Potential@endlink.



	% The conceptual discusion of abstract Action Potential

\section{Timing relations\label{sec:Computing-Timing}}


  
\begin{figure*}
\iflatexml
\includegraphics[width=\textwidth]{fig/Neumann_Timing_Extended.svg}
\else
\includegraphics[width=\textwidth]{fig/Neumann_Timing_Extended.pdf}
\fi
		\caption{Timing relations of von Neumann's complete timing model, with data transfer time in chained operations; synchronization becomes an issue as the physical size of the computing system grows. Notice that synchrony signals must be bypassed or neglected.
		Timing relations of von Neumann's timing model: the data \textit{transfer} time neglected apart from data \textit{processing} time; synchronization can have small dispersion. Notice that the central synchronization signal appears at different places in the processor at different times. \label{fig:Computing_Timing}}
\end{figure*}      


	% The conceptual discusion of abstract Action Potential

\section{Action Potential\label{sec:Computing-ActionPotential}}

In this section we discuss the  \textit{concept} of
%\hypertarget{SINGLE_AP_CONCEPTUAL}
{action potential}
in an abstract sense that enables to define its notions, features and stages.
Our approach here is hybrid: we know that the events are connected 
to ion movements and that the components' cooperation forms the action potential.
The %\hyperlink{PHYSIOLOGY_AP}
{physiological}
and 
%\hyperlink {PHYSICS_AP}
{physical} details
are discussed in the respective chapters \ref{ch:Physiology} and \ref{ch:Physics}, where we provide citations describing the physiological details.
We consider the neuron as an \textit{abstract computing element}
and show how a neuron implements the generalized computing we discuss in chapter\ref{ch:Computing}.

We assume that the neuron is in state "Relaxing", so the membrane's voltage is at the resting value. The membrane and the 
\gls{AIS}
are at the same potential, so no current is present (no "leaking current" exists).
When input charge (through the synapses or directly through the membrane)
 arrives to the membrane, its potential increases. The increased membrane potential means a potential difference between the membrane and the axon, so it drives a current through the AIS.
% \gls{AIS}
The current
 (not identical with the leaking current) 
 \index{leaking current}
 \index{current!leaking}
 decreases the membrane' potential between adjacent synaptic inputs.
   For simplicity, we assume that the axonal inputs cause a step-like change in the membrane's voltage. Between the inputs the current through the
   \gls{AIS}
decreases the membrane's potential. As we discuss, the neuronal computation actually measures the time between the arrival of the first synaptic input and exceeding the threshold; it is in the order of tenths of a millissecond.

 When the resulting potential exceeds a threshold value, for a very short time
 (in the order of tenth of picoseconds) the ion channels in the membrane's wall open
 and a large amount of ions suddenly (in a step-like way) increase the membrane's potential
 in its \hyperlink{atomic_layer}
 {surface layer}; see section~\ref{sec:Physics-TwoSegments}.
The current due to the rushed-in ions creates a local potential gradient and the ions
(with a potential-assisted speed) saturate the layer
(the mechanism is described in section~\ref{sec:Physiology-Electrodiffusion}),
open all ion channels. The rushed-in ions feel the potential gradient toward the \gls{AIS},
but they can move in the layer on the surface
with a finite speed, so the current from the different points of the membrane need different times to reach the \gls{AIS}.
Correspondingly, the current (due to the current "created" by the nearby ion channels) reaches the \gls{AIS}
instantly, while the current from the farthest point needs tenths of a millisecond to get to the \gls{AIS}.
%
After that, given that the current cannot flow out "instantly",
the current produces a kind of "damped oscillation": drops below 
the resting potential and then asymptotically approaches it, without exceeding it.
It is the effect of the neuronal oscillator, marked as "RC-effect".
In some sense, the ionic current "disappears": it get stored in the neuronal condenser
while it travels on the surface of the membrane.

During this process, the membrane's voltage controls the synaptic inputs.
Given that the ions can reach the membrane using a "downhill" method,
the current stops  when the membrane's potential rises
above that of the axonal arbor,
and will not flow until the membrane's potential drops again below the threshold:
the synapses will be disabled and re-enabled.

Biology observed the "absolute refractory" period, which is interpreted 
that the synapses are disabled for a period, and it is a correct observation.
Different is the case for the "relative refractory" period.
Actually, the synaptic inputs arrive at the junction of the axon, and the current must travel to the \gls{AIS}.
that needs time (in the order of tenth of a millisecond). Given that 
the \gls{AP} is measured at the \gls{AIS}
the main contributor' current in the meantime proceeds toward the "hyperpolarized" state,
and so the synaptic inputs apparently contribute
outside of the "absolute refractory" period,
so this extension is called "relative refractory".
Actually, the origin of both periods is the same,
only the effect's time scale is shifted by the ionic current's travel time.

As discussed, after that the membrane's voltage drops below the threshold potential,
the neuron can start a new computing.
The signal "ComputingBegin" is defined as the signal arriving first after then
the neuron membrane's potential crossed the threshold value from the higher voltage direction.
At that point, the membrane's voltage can be above or below the resting potential,
and, correspondingly, the charge integration starts from a value higher or lower than the resting potential.
Effectively, the value of the potential (more precisely, \textit{when} the first
synaptic input arrives at the \gls{AIS}) represents a memory with initially a negative, later positive, time-dependent content.
%

Our stages slightly deviate from the ones commonly used in physiology.
We define the stage "Computing" as the period between the arrival from the first synaptic input
to exceeding the threshold.
The stage "Delivering" is defined as the period while the membrane's voltage stays above the threshold.
The stage "Relaxing" begins when the "Delivering" ends, and may be interrupted by a synaptic input.
Notice that because of the spatiotemporal nature, the time values have a definite meaning only
if the place of the measurent is also provided: it is one coordinate of a space-time point.







The computer representation (actually a state machine) is shown in Figure~\ref{fig:NeuronStateMachine}.                          


Will be based on \cite{VeghComputingModel:2021} \cite{VeghRevisingClassicComputing:2021}

