\iflatexml
\else
\minitoc
\fi
\chapter[Information]{Neural information\label{ch:Information}}
	% The foreword for information

Applying \hypertarget{InformationForBiology}{Shannon's information theory}~\cite{ShannonOriginal:1948} to neuroscience
started immediately after the significance of Shannon's seminal paper was recognized, and different research directions began to use it (for a review, see~\cite{InformationNeuroscienceDimitrov:2011}).
Although Shannon warned~\cite{ShannonBandwagon:1956} against the indiscriminate use of the theory and called attention to its valid scope:
"The hard core of information theory is, essentially, a branch of mathematics", and it "is not a trivial matter of translating words to a new domain". The improper application of
the information theory to neural communication is going on~\cite{InformationTheoryIntersection:2008,InformationTheoryAbused:2019}.

%	\hypertarget{information-biology}{}

"In the terminology of communication theory and information
theory, [a neuron] is a \hypertarget{NeuronInformationBroadcast}{multiaccess, partially degraded broadcast channel}
that performs computations on data received at thousands of
input terminals and transmits information to thousands of output
terminals by means of a time-continuous version of pulse position.
Moreover, [a neuron] engages in an extreme form of
network coding; \textit{it does not store or forward the information it
receives but rather fastidiously computes a certain functional of
the union of all its input spike trains} which it then conveys to a
multiplicity of select recipients"~\cite{EnergyEfficientNeuralComputing:2010}.
\index{Shannon, Claude}
Will be based on \cite{VeghNeuralShannon:2022}
\cite{RoleOfInformationTransferSpeed:2022}
\cite{VeghChannelCapacity:2023}

%
The theoretical model~\cite{VeghNeuralShannon:2022} described how
the slow operation of biological objects explains biological phenomena,
but  due to the lack of dedicated measurements
it could only indirectly underpin the theory's correctness. Now, we
give an exact quantitative explanation of the precise measurements~\cite{HodgkinHuxley:1952,SynapticTransmissionMason:1991},
which have not been correctly understood in the past decades due to
the lack of understanding of the role of the finite interaction speed
(conduction velocity) in neuronal operations.

\hypertarget{SingleSpikeEntropy}{Issuing the same signal carries no information (represents zero entropy)}