% The physics behind the neural model

\section{The physical model\label{sec:Physiology-Physical}}



As Figure~\ref{fig:NeuronElectricOperationSummary}
summarizes: "Electrical signals travel from the cell body of a neuron (left) to its axon terminal
in the form of action potentials. Action potentials trigger the secretion of neurotransmitters
from synaptic terminals (upper insert). Neurotransmitters bind to postsynaptic receptors
and cause electric signals (synaptic potential) in the postsynaptic neuron (right).
Synaptic potentials trigger action potentials, which propagate to the axon terminal
and trigger secretion of neurotransmitters to the next neuron." 
Furthermore, the "\textbf{neurons convey
neural information} by virtue of electrical and chemical signals"\cite{JohnstonWuNeurophysiology:1995}.
These sentences should read that ions carry the observed potential changes
and that the signal propagation is low. Notice that at that time
it was not yet recognized that the electric signals propagate
with a finite speed also in the dendrites, not only on the axons.


\begin{figure}
	\includegraphics[width=\textwidth]
	{fig/JohnstonSummaryFigure1_2.png}
	\caption{Summary of conveying information by electric and chemical signals. (Fig. 1.2 from \cite{JohnstonWuNeurophysiology:1995}}
	\label{fig:NeuronElectricOperationSummary}
\end{figure}

Basically, we follow the Johnston\&Wu's "general principles.
As in general in science, we must introduce different abstractions and
approximations for describing nature, see section~\ref{sec:Physics-Abstraction}.
Making a model for neuron represents a special challenge. 
As one of the simplest biological entities, the neurons interface the non-living
and living science, furthermore, the microscopic and macroscopic world.

We must be very cautious with modeling the electrical features. "an
understanding of the electrical properties of dendrites is critical for evaluating the errors associated with the electrophysiological measurements
of synaptic function".


???The two conditions imply that we must use slow currents.
\index{slow current}
As we discuss in sections~\ref{sec:Physics-Current} and~\ref{sec:Calculating-ion's-speed}

\index{electrotonic model}
"Electrotonic is a rather arcane term
that is used to describe passive electrical signals, that is, signals (current
or voltage) that are not influenced by the voltage-dependent properties
of the membrane. It will become obvious that this theory is too simple
to explain the complexities of dendrites, but at least it is a good starting
point."~\cite{JohnstonWuNeurophysiology:1995} In other words, that model is
usable only for a static description, "it is too simple", but there is no 
commonly used other idea.
\index{Onsager-relations}
Unfortunately, in physiology, the interrelations expressed by the Onsager-relations remain entirely out out scope; presumably primarily because of the lack of 
laws of motions of thermodynamics. The chemical concentration and 
the electrical potential is only loosely connected, and only statically and for the bulks of cellular segments.


\subsection{Size of neuronal components\label{sec:Physiology-Sizes}}
The
'dendritic trees can be quite large, containing up to 98 \% of the
entire neuronal surface area' \cite{BiologicalConservationLaw:2017}.
'Because the cell body is small compared with the dendritic tree,
its membrane potential is roughly uniform' \cite{ MolecularBiology:2002}; 
we assume that \textit{the neuron's membrane itself is equipotential}.
However, \textit{the dendrites are not equipotential while delivering signals}.
Their potential 'is a composite of the effects of all the signals impinging on the cell,
weighted according to the distances of the synapses from the cell body'.
'Temporal and spatial summation together provide the means  by which ...
many presynaptic neurons jointly control the membrane potential.'
'Each incoming signal is reflected in a local \gls{PSP} of graded magnitude,
which decreases with distance from the site of the synapse'  \cite{MolecularBiology:2002}; see our mathematical discussion of this physiological evidence in section~\ref{sec:Physiology-NeuralCurrents}.
This latter sentence should read that \textit{its measurable effect (their local potential)
decreases}, compared to the
one at the presynaptic terminals. As the surface, over which it propagates
during its journey through the dendrites, extends, the charge density decreases,
but the total charge is conserved until the cell body is reached.
These statements mean (assuming that those signals travel with
the same speed in the dendrites) that \textit{ experiments underpin the presence of
a 'slow current' of ions in neurons}, although the notion is not introduced.

Experimental evidence shows that \hyperlink{slow_current}{the electric signals
have a finite speed} in axons, dendrites, and cell bodies and that within the cell,
the overwhelming majority of propagation time is spent in the dendrites.
The mathematical handling of finite speeds is not simple, especially
within a biological cell, so we separate the cell into two regions
and make the approximation that within the cell body, the interaction is instant
(that is, the Laws of electricity are valid). However, outside the cell body,
the finite interaction speed in the dendrites leads to observable effects that significantly
influence the cell's operation.
\textit{We set up a hybrid model: the cell body is equipotential
(aka: can be described by a 'fast current'), but the dendrites
(and they contribute the overwhelming majority of the signal path within the cell)
are non-equipotential. They must be described by approximations
based on a 'slow current'}.
With that model, we explain the up to now not understood features of neuronal
charge processing; furthermore, why is that 'the interplay
between the synaptic and neuronal dynamics, when the network is near
a critical point, both recurrent spontaneous and stimulated phase
transitions enable the phase-dependent processes to replace each other'
\cite{BiologicalConservationLaw:2017}.

The \hypertarget{finite_size}{size of presynaptic terminals} \href={https://www.ncbi.nlm.nih.gov/books/NBK26910/bin/ch11f38.jpg}{(reproduced here as Fig. \ref{fig_SizeOfPresynapticTerminals})}
 is about two orders of magnitude smaller
than the cell body and its dendrites \cite{MolecularBiology:2002}, chapter 11.
In other words, the
"dendritic trees can be quite large, containing up to e98\% of the
entire neuronal surface area" \cite{BiologicalConservationLaw:2017}.
"Because the cell body is small compared with the dendritic tree,
its membrane potential is roughly uniform" \cite{MolecularBiology:2002}. We pinpoint that 
since the ionic currents spend most of their travel time in the dendritic tree, we assume that the
overwhelming portion of the travel time derives from the dendrites;
so the contribution from the travel on the body is omitted.
In this sense it is unimportant whether \textit{the membrane itself is equipotential}.
However, it is crucial that  \textit{the dendrites are not equipotential while delivering signals}.
Their potential "is a composite of the effects of all the signals impinging on the cell,
weighted according to the distances of the synapses from the cell body".
"Temporal and spatial summation together provide the means  by which ...
many presynaptic neurons jointly control the membrane potential." \cite{MolecularBiology:2002}.
This former sentence should read that \textit{its measurable effect (their potential)
decreases}, compared to the
one at the presynaptic terminals. As the surface, over which it propagates
during its journey through the dendrites, extends, the charge density decreases,
but the total charge conserves until the cell body reached.
The latter sentence should read that the presynaptic \textit{terminals}
and the \textit{membrane potential} mutually control each other. Given that the ions can reach the presynaptic terminal passively by using a "downhill" potential between the axonal arbor and the membrane, once, they cannot enter the mebrane until the membrane's potential is higher than that in the axonal arbor and twice, whwn they can enter, the current depends on the potential difference between the membrane and the arbor. 
\index{axon!arbor}

\begin{figure}
	\includegraphics[width=\textwidth]{fig/ch11f38.jpg}
	
	\caption{The size of presynaptic terminals. \copyright \href{https://www.ncbi.nlm.nih.gov/books/NBK26910/bin/ch11f38.jpg}{Original}
	}
\label{fig_SizeOfPresynapticTerminals}
\end{figure}



\subsection{Neuron's potential\label{sec:Single-AbstractPotentialInside}}
Experimental evidence shows that the electric signals
have a finite speed in axons, dendrites and cell body; furthermore, that \textit{within the cell,
the overwhelming majority of propagation time is spent in the dendrites}.
The mathematical handling of finite speeds is not simple, especially
within a biological cell, so we separate the cell into two regions
and make the approximation that within the cell body the interaction is instant
(that is, the Laws of electricity are valid), but outside the cell, in the dendrites
the finite interaction speed leads to observable effects that significantly
influence cell's operation (we need different approximation; we must not apply
automatically the equations borrowed from electricity).
\textit{We set up a hybrid model: the cell body is equipotential
(aka: can be described by a 'fast current'), but the dendrites
(and they contribute the overwhelming majority of the signal path within the cell)
are non-equipotential and they must be described by approximations
based on the notion of a 'slow current'.}
With that model, we explain the up to now not understood features of neuronal
charge processing, furthermore, why is that 'the interplay
between the synaptic and neuronal dynamics, when the network is near
a critical point, both recurrent spontaneous and stimulated phase
transitions enable the phase-dependent processes to replace each other'
\cite{BiologicalConservationLaw:2017}.


The commonly used physical picture (see, for example, \cite{KochBiophysics:1999}, page 9)
is only half of the truth:
"there is never any actual movement of charge across the insulating membrane ...
the charge merely redistributes itself across the two sides by the way of the rest of the circuit."
On the one side, redistribution of charge \textit{per definitionem} means a current, on the other,
that picture contradicts also the notion of 'specific conductance':
the rest of the circuit cannot participate in a 'leakage current' through a distributed resistor.
\index{leakage current}
The cell has a resistance (see the \gls{AIS}
and an area, but still, no specific resistance
can be interpreted.
The charge moves in  the proximal layer
 of the electrolytes
(in the form of a '{slow current}'
 near to dendrites),
then the circuit closes though the \gls{AIS} and  the extracellular segment.
\textit{We explicitly introduce the notion of 'slow current', and show that we need to divide the membrane's ionic currents roughly into two categories,
whether they flow directly between the intracellular and the extracellular
space or within the layer on the surface of the membrane}.

The physical difference is whether the movement of ions is assisted
by the enormous potential gradient between the extra- and intracellular regions
when passing the ion channels ('fast' current)
or they move in the electrolyte layer proximal to isolating membrane
assisted by the electrostatic repulsion of ions in the same layer
('slow' speed of a macroscopic current).
\index{slow current!cardiatic}
Cardiatic slow currents have been discovered \cite{SlowCurrentChannels:1985}
(actually, current pulses of duration in several msecs range).
It was correctly observed that "the slow currents appear to
have been caused by repeated openings of one or more channels"
and their speed \cite{CardiacAPS:1980} was found in the range of $0.02-5\ m/sec$.
In neurophysiology, ion current speeds
ranging from a few $mm/s$ to dozens of $m/s$ has been observed.

These statements mean (assuming that those signals travel with
the same speed in the dendrites) that the \textit{presence of
a 
{'slow current'} of ions
in neurons is
experimentally underpinned}, although the notion is not introduced
(mainly because its mathematical handling is not solved).
Assuming that the dendrites' size is about $0.1\ mm$ and the synaptic signals appear
at the \gls{AIS} $0.2\ ms$ after their arrival to their presynaptic terminals,
we can estimate the speed of 'slow current' as  $0.5\ m/s$
(see our discussion on the signals appearing in the 'relative refractory'
 period and~\cite{APTemperatureDependenceRefractory:2001}).
\index{refractory!relative}
This result is in line with our result derived
%from the 'ghost image' (see Fig. \ref fig:The-ghost-image_AP) $2\ m/s$,
the speed value $1\ cm/s$ \cite{ActionPotentialGenerationNatrium:2008}
measured  within a cell body and the axonal speed $20\ m/s$ \cite{HodgkinHuxley:1952}.



\subsection{Axon Initial Segment\label{sec:Physiology-AIS}}
"Neurons ensure the directional propagation of signals throughout the nervous system. The functional asymmetry of neurons is supported by cellular compartmentation: the cell body and dendrites (somatodendritic compartment) receive synaptic inputs, and the axon propagates the action potentials that trigger synaptic release toward target cells. Between the cell body and the axon sits a unique compartment called the axon initial segment (AIS)"\cite{AIS_Updated_Viewpoint:2018}.
In the light of the new experimental and theoretical results,
we need to add new components, roles and operating modes to
the one assumed by the present physiology.


\begin{figure}
	\includegraphics[width=\textwidth]{fig/AIS.jpg}
	\caption{The structure of the Axon Initial Segment. %(Cartoon of a multipolar neuron and the molecular composition of the AIS.)
		\cite{ActionPotentialGenerationNatrium:2008}
		Ann. N.Y. Acad. Sci. 1420 (2018) 46--61, Figure 1 \copyright 2018 New York Academy of Sciences. }
	\label{StructureOfAIS}
\end{figure}

%
Also we must add the invention from about three decades later: "Neurons ensure the directional propagation of signals throughout the nervous system.
The functional asymmetry of neurons is supported by cellular compartmentation: the cell body and dendrites (somatodendritic compartment) receive synaptic inputs,
and the axon propagates the action potentials that trigger synaptic release toward target cells.
\textit{Between the cell body and the axon sits a unique compartment called the axon initial segment} (AIS).
%\gls{AIS}
The AIS was first described 50 years ago [i.e., nearly two decades after HH published their study], and its molecular composition and organization have been progressively elucidated during the following decades. \dots. Recent years have also brought crucial insights into the functions of the AIS: how ion channels at its surface generate and shape the action potential."~\cite{AIS_Updated_Viewpoint:2018} We provide the physics and mathematics of how AIS shapes
the action potential (or more precisely, we show what an important role it plays in
forming AP).



In our model, the 
%\gls{AIS}
AIS gets independent from the membrane,
and this separation leads to crucial changes. (BTW, the name is misleading:
the 
%\gls{AIS}
AIS is part of the neuronal oscillator, and it forwards a traveling
potential wave to the axon instead of belonging to it.)
'Although by definition a neuron must have an
axon to assemble an AIS, the relationship between AIS
assembly and axon specification in vivo has not been
determined yet'~\cite{AIS_NeuronalPolarity:2010}.

"The axon initial segment (AIS) is located at the proximal axon and is the site of action potential initiation. This
reflects the high density of ion channels found at the AIS.
... The summation of
synaptic inputs gives rise to action potentials at the
axon initial segment (AIS), a 20--60 $\mu m$ long domain
located at the proximal axon/soma interface that has
a high density of voltage-gated ion channels."	
As discussed in \cite{AISStructureReview:2018}, see also their Figure 1, reproduced here as Figure \ref{StructureOfAIS},
the structure of the Axon Initial Segment
is known to the smallest details.
As the illuminating investigations in 2008
\cite{ActionPotentialGenerationNatrium:2008}
revealed, the 
%\gls{AIS}
AIS has very dense ion channels. That is, from an electrical point
of view, those parallelized channels can be abstracted as a  \textit{discrete  conductance}
(or resistance) between the membrane and the axon.
The membrane itself can be abstracted as a \textit{distributed condenser} with no resistance
(in contrast with the viewpoint of biophysics, that the membrane  plus
%\gls{AIS}
AIS is considered a distributed element, where the capacitor and condenser cannot be separated).
Notice the important point:
	"Neurons are also anatomically polarized, as they can be
subdivided into a somatodendritic input domain and an axonal output domain"~\cite{AIS_NeuronalPolarity:2010};
providing a direct evidence that (unlike in 
%\gls{HH}'s 
HH's model) the input and output currents (and voltage time derivatives) are independent, see also Fig.~\ref{NeuronPolarity}.
More precisely, they form the input and output of a neuronal oscillator, as our model suggests. Notice how the 
%\gls{AP}
AP changes its shape during its propagation in the adjacent segments, as our model explains: the broadening by axonal arbor, the voltage-gradient generated shape on the 
%\gls{AIS},
AIS, the appearance of 
%\gls{iAPTD}
iAPTD at the distant junction. Notice the lack of hyperpolarization at the beginning and end of the pipeline; a clear effect of of the neuronal oscillator.
Inventing 
%\gls{AIS}
AIS changed the viewpoint of neuroscience~\cite{AIS_Updated_Viewpoint:2018}.

\begin{figure}
	\includegraphics[width=\textwidth]{fig/NeuronPolarity.png}
	\caption{Neurons are highly polarized cells
		~\cite{AIS_NeuronalPolarity:2010}, Figure 1 \copyright2010 Macmillan Publishers Limited.}
	\label{NeuronPolarity}
\end{figure}




\subsection{Axons\label{sec:Physiology-Axon}}

We model the axons as electrolyte-filled semipermeable membrane tubes with ion channels in their walls. The axons not passively follow the potential's time course, but they
mediate the changes in their internal volume by using an ion pool
available in their extracellular volume. The applied potential (including that of the mediated ions) opens the
ion channels in the axon's wall. 

In their native mode of operation, the three modes of ion channels define the 'direction of the time'~\cite{ThreeStateUnidirectional:2004, MarkovianIonChannel:2005,RoleOfInformationTransferSpeed:2022} (the direction of 
the current that transmits the spike). The layer that the 
front of the spike creates on the surface (on both sides
of the tube) propagates in both directions, 
but it cannot open the ion channels on the side where
the spike arrived from, and the ion channels are still
inactivated.

\hyperlink{voltage-clamping}{Clamping}
sets up an artificial working regime for the ion channels: the permanent
electric field on the outer surface enables ions to enter the inner volume where
formerly no ions (and no potential) existed. The rushed-in ions will
flow away from the place of their entrance (recall that the current removes part of the ion layer on the surface), and a slow current toward the membrane can start. 
Under clamping conditions, the experimenter sets the voltage instead of the transmitted signal and in a static way instead of an  autonomous dynamic one.

Initially, the membrane, the clamping point on the
axon, and the intracellular and extracellular fluid maintain
their resting potential. When an external potential 
is applied  suddenly to some point of the axon, an electric field $\frac{dV}{dx}\propto(V_{membrane}-V_{clamp})$
appears on the \emph{outside surface} of the axon. The extracellular
space with its high ion concentration $C_{k}^{ext}$ represents an "ion
cloud" (see also section~\ref{sec:Physics-Current}). When the \hyperlink{voltage-clamping}{clamping voltage} is switched on, a ``fast'' current
instantly delivers the potential along the \emph{outer} surface of
the axon. However, this is not the case (at least not in the initial
moment) on the \emph{inner} surface. \textit{There is no charge present that could
change the potential}: 'the intracellular concentration at rest is around
five orders of magnitude less than that in the extracellular space'~\cite{KochBiophysics:1999}. The physical picture that the clamping potential
instantly appears at the end of the axon at the membrane (i.e., if (apparently) they
have an infinitely large propagation speed) is valid
only if charge carriers exist in the axon.


The persisting \hyperlink{voltage-clamping}{clamping voltage} gradually triggers the opening of ion channels in its wall along the axon,
leading to a continuous inflow through the axon's
wall from the extracellular space into the intracellular space as a "fast current"; see section~\ref{sec:Physics-IonChannels}.
The ions entering the intracellular space remain inside the axon: the cylindrical surface enables only a one-way (inward) traffic for the ions. As
discussed in section 11.4 of~\cite{KochBiophysics:1999}, "once
calcium enters the intracellular cytoplasm it is not free to diffuse". The ions start to create an ion-rich layer on the internal surface.
However, a gradient parallel to the wall exists. The ions experience the electric field (which is present initially
only at the clamping point but extends with the passing time) along the axis, speed up, and (after a short while) the ion's
speed becomes constant in time but its value depends on the actual electric
field, see Equ.~(\ref{eq:DriftCurrent}). The \emph{ions will slowly
	move }along the axon \emph{with a field-dependent constant velocity
	in the electric space} in a viscous solution.
The moving ions deliver charge, so
the potential gradually extends along the electrolyte tube (the axon).
 ``In axon fibers, the
effective diffusion constant was estimated to be about one-tenth of
the diffusion coefficient in aqueous solution''~\cite{KochBiophysics:1999}; however, under the effect of the potential gradient, and the mutual repulsion,
they form a ``slow current'' (and that macroscopic current may have a much higher propagation speed). 
\emph{The current and potential are not instant, as we consider in
	the classic theory of electricity}: they propagate with the speed
of the ion current.



In this model, we assume that during the time $dt$, in the volume
$dx$, we have a constant ion inflow $I_{wall}$ through the axon's
wall, which increases the charge and concentration already in the
volume. The charges in the tube experience the field $\frac{dV}{dx}$,
and they move with speed $v$ inside the tube (see Eq.~(\ref{eq:StokesCurrent})).
The ionic fluid with velocity $v$ (proportional
to $\frac{dV}{dx}$) transfers the ionic charge in
the volume to the neighboring element at a distance $v*dt$, and delivers
the charge and concentration from the neighboring element at a distance $-v*dt$ into
this element. At the time $t$, the concentration at $x$ will result
from the inflow at the place $x-v*t$ (see also the general discussion
around Eq.~(\ref{eq:CoulombTimeDependent})). The higher the speed 
$v$, the more significant the difference between the "inflow" and the "present"
concentration. The stream inside the axon, a la Minkowski
(although in this simple case, a Galilei-transform is sufficient), transforms
the distance to time and vice versa.
	Under the effect of \hyperlink{voltage-clamping}{clamping}, the current is decreased by the stream
proportionally:

\begin{equation}
	\frac{dI_{axon}}{dt}=-\alpha*I_{axon};\ I_{axon}(t)\approx I_{wall}*(1-\exp(-\alpha*t))\label{eq:AxonalCurrentNoConc}
\end{equation}


\noindent ($\alpha$ is a timing constant
of dimension $(1/time)$).


When the stimulation happens inside the axon, and the axon
forwards the charge package in the reverse direction~\cite{BackpropagationAP:2012}, towards the
\gls{AIS} .
The \gls{AIS}
uses a "downhill" method of charge forwarding (it is a barrier in both directions), so it can forward
the imitated
"\gls{AP}" towards the soma.
It is a propapation in the reverse direction, since the stimulation
arrives from the "wrong" direction, and the but not a backpropagation.
The capacitance of the dendrids explains the shape of the signal.


\begin{figure}
	\includegraphics[width=0.65\textwidth]{fig/AISBackPropagation.png}
	
	\caption{Action Potential Initiation in the Axon Initial Segment~\cite{BackpropagationAP:2012}\label{fig:APBackPropagation}.}
%	\gls{AIS}.
\end{figure}
The AP's
%\gls{AP}'s
first front already passed to the axon and is forwarded there.
It is hard to imagine that the commonly used "transversal current",
the ion channels, not only synchronize themselves to the length and speed
of the spike, but they also sense the direction of the potential gradient.




	\subsection{Membrane\label{sec:Physiology-Membrane}}

Even at writing this text (at the midle  of 202), textbooks comprise the
\href{https://bio.libretexts.org/@api/deki/files/78169/figure-35-01-02.png?revision=1}{old and bad cellular structure}.


Membranes are fundamental in many places, from biological objects
to industrial filters. They operate on the border of microscopic and
macroscopic worlds, separating non-living and living matters, and
combining electrical and thermodynamical interactions. We show that
\hyperlink{DynamicLayer}{an extremely thin skin near to the surface} of biological membranes
is responsible for the biological thermoelectric processes. 


To describe those complex and continuous phenomena at least approximately,
we must separate them to stages. Using omissions, approximations and
abstractions, we can describe the stages approximately, usually considering
only one dominant phenomenon. The described phenomena are interrelated
in a very complex way and depend on different parameters. To some
point, we can describe that thin layer using a static picture and
providing an empirical description of its individual processes, even
we can give some limited validity mathematical descriptions for those
stages. However, we understand that for describing the transition
(contrasting with step-like stage changes) between those well-defined
stages of the athmosphere we need a \emph{dynamic description} and
we need to find out the \emph{laws of motion} governing the processes.

Similar is the case with the neuronal membranes and the neuronal operation.
Now we are at the point where their decades-old static description
is not sufficient. To descibe the neuron's dynamic behaviour, we need
to derive the corresponding laws of motion. We need a meticulous and
unusual analyzis to derive them. 

In a neuron, in the abstraction science uses, we put together only
ionic solution, semipermeable membrane and currents reaching them.
As experienced, at some combination of their parameters, qualitatively
different phenomena happen, which, in the abstraction biology uses,
called signs of life. Given that the approximations, the derived abstractions
and the mathematical formalisms describing them are different for
the two cases, \emph{it looks like we have two different, only loosely
bound worlds}. However, if we realize we arrived at the boundary of
non-living and living matters, we must go back to the first principles
of science. Using our approach, maybe we can defy that "the
emergence of life cannot be predicted by the laws of physics" ~\cite{ConservationOfInformation:2021}.


The layers, for their regular operation, have both source and drain.
In neurons, the source is distributed over the surface of the layers
and the drain is concentrated at the terminating end of the layer.
The two currents are flowing simultaneously, i.e., the source of the
drain current has a time course, so the product of the two currents
can be measured. (actually, it is a differential equation, and in
the elementary cross-section, Kirchoff's Junction Law is valid). Generally,
it takes time until the source current reaches the drain's position.

Initially, biology used the abstraction that the measured resistance and capacitance are distributed
along the membrane's surface. It assumed a \emph{discrete} equipotential
membrane with capacity $C$ and that it leaks through a discrete resistance $R$. 
However, in biology, no discrete elements for storing charge exist.
The notion of storing charge can be used only in the sense that for
the time of passing a finite-size element with finite propagation
speed, the charge carriers spend the corresponding time in the element.
That phenomenon resembles storing the charge, and that imitation enables
us to describe a behavior resemblant to that of the biological circuit.
\emph{Attempting to imitate the effects of biological ``slow'' currents
	using electric parallels hides that generating an 
	\gls{AP} is their native feature};
furthermore, slow currents may also play a role in cognitive functions.



