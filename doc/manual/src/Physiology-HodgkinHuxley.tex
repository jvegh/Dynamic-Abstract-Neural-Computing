% Evaluating the Hodgkin-Huxley model

\section{Hodgkin\&Huxley's empirical description\label{sec:Physiology-HodgkinHuxley}}

As we discuss in section~\ref{sec:Physics-Electricity}, the body's electric signals
were discovered early, the principles, notions and technical equipments have been elaborated.
Even, some meticulous measurements correctly interpreted some of its signals. 
The development of electronical technology enabled their systematic study 
in the beginning of the 50's. However, experimenters 
often forgot that "\textit{Under ideal
circumstances, the physical act of measuring a neurophysiological event
would have no effect on the electrical signal of interest. Unfortunately,
this is seldom the case in neurophysiology.}"~\cite{JohnstonWuNeurophysiology:1995}.
	

The first systematic attempt to describe the results of observations in terms of well-known laws of electricity was published around 1952~\cite{HodgkinHuxley:1952}.
A collection of their (modernized) papers is available~\cite{CompanionGuideHodgkinHuxley:2022}.
They made a huge amount of \hypertarget{HHmeasurements}{meticulous measurements} and wanted to help the science community with providing equations for practical applicability. 
To speed up reaching that goal, they introduced \textit{empirical functions} 
and derived equations, which, not surprisingly, described the \textit{empirical observations} quite accurately. 
The importance of their work is best highlighted by that it inspired 
different disciplines for discussion. 

\begin{advanced}
"although highly successful
in predicting and explaining many of the electric characteristics of the action
potential, the HH model, nevertheless cannot accommodate the various non-
electrical physical manifestations (mechanical, thermal and optical changes)
that accompany action potential propagation, and for which there is ample
experimental evidence.~\cite{NerveSignalAsWindow:2023}
\end{advanced}

A technical reason is described in~\cite{HodgkinMemoryBook:1992}: "The fact that the whole process for calculation of a $4-5\ ms$ interval, showing the initiation of and recovery following an action potential, could be accomplished in 8 hours is astonishing." The must-be oversimplification does not belong to the core of the theory. When using the recent computing systems, not necessarily the old oversimplified models should be coded. One could also use modern, more compute intensive, and physically correct models.

\subsubsection{Self-evaluation\label{sec:Physiology-HodgkinHuxleySelfEvaluation}}
In their brilliant publication~\cite{HodgkinHuxley:1952}, 
Hodgkin and Huxley evaluated their results "that our equations [must not be taken as] anything more than an empirical
description" and "the [partial] success of the
equations is no evidence in favour of the mechanism". 
When validating their observations,
they have found serious question marks: "a number of points were noted on which the calculated behaviour of our model did not agree with the experimental results. We
shall now discuss the extent to which these discrepancies can be attributed to
known shortcomings in our equations."
"One was that the membrane capacity was assumed to
behave as a \textit{'perfect' condenser}, \dots the other was
that the equations governing the potassium conductance do not give as much
\textit{delay in the conductance rise} on depolarization as was observed in voltage clamps".
They did have the intuition that something was wrong and they correctly guessed its reason: "it seems difficult to escape the conclusion that the changes
in ionic permeability depend on \textit{the movement of some component} of the
membrane \textit{which behaves as though it had a large charge}.\dots it is necessary to suppose
that \textit{there are more carriers and that they react or move more slowly} \dots \textit{there is no evidence from our
experiments of any current associated} with the change in sodium permeability, apart from the contribution of the sodium ion itself".

They emphasized that their work 
"must not be taken as evidence that their equations are anything more than an empirical description". They made the first step of "a great journey into the unknown" and were very cautious by saying that "the success of the equations is no evidence in
favour of the mechanism that they tentatively had in
mind when formulating them".

In his late work~\cite{HodgkinMemoryBook:1992}, Hodgkin evaluated the filtered experiences:
"We soon realized that \textit{the carrier model could not be made to fit certain results},
for example the nearly linear instantaneous current voltage relationship, and
that it had to be replaced by some kind of voltage-dependent gate. As soon as
we began to think about molecular mechanisms it became clear that \textit{the electrical data would by themselves yield only very general information} about the class
of system likely to be involved. So we settled for the more pedestrian aim of
finding a simple set of mathematical equations which might plausibly represent
the movement of electrically charged gating particles."


\subsubsection{Nobel-laudation\label{sec:Physiology-HodgkinHuxleyNobel}}

Unfortunately, they also attempted to understand which physical processes happen in the membrane, but they concluded with the feeling that "\textit{the interpretation given is unlikely to provide a correct picture of the membrane}."
Despite their explicite warning, that "the success of the equations is no evidence in
favour of the mechanism that we tentatively had in
mind when formulating them". 
Despite, they received the Nobel-prise "for their discoveries concerning the \textbf{ionic mechanisms} involved in excitation and inhibition in the peripheral and central portions of the nerve".
As the philosphical approach to their work discusses, "One could dismiss this curious passage as scientific modesty if it were not for the fact
that Hodgkin and Huxley argue for their conclusions."~\cite{MechanisticModelHodgkinHuxleyCraver:2006}
The science community rushed to apply the equations, instead of validating them.
To compensate for the disagreements with the experimental data, further ad-hoc
assumptions have been introduced, making their admittedly wrong picture even worse.

In the sense of philosophy, "there is a widely accepted distinction between \textit{merely modeling a mechanism’s behavior} and \textit{explaining} it.
The equations
must be supplemented by a causal interpretation: one might, for example, agree by
convention that the effect variable is represented on the left, and the cause variables
are represented on the right, or one might add “these are not mere mathematical
relationships among variables but descriptions of causal relationships in which this
variable is a cause and this other is an effect,” and not vice versa", for more details see~\cite{MechanisticModelHodgkinHuxleyCraver:2006,Mechanistic_HodgkinHuxley_2008}. The lack of causality is one reasons why 
HH have had the feeling they missed the correct picture of the membrane. The other reason is that some fine details of their oversimplified picture was not accurate, althought the additional (and arbitrary) ad-hoc assumptions have hidden the disagreements. The followers have "fitted elephants"~\cite{FittingElephants:2021} by adding many more ad-hoc effects with too many parameters.
%

\subsection{Measurement science\label{sec:Physiology-HHPhilosophy}}

As highlighted in ~\cite{MechanisticModelHodgkinHuxleyCraver:2006}, the performed a very meticulous analysis and characterized the time-course of the action potential phe
nomenally terms of different features (the concise listing taken from~\cite{MechanisticModelHodgkinHuxleyCraver:2006})

\begin{itemize}
\item the form, amplitude, and threshold of an action potential;
\item the form, amplitude, and velocity of a propagated action potential;
\item the form and amplitude of the resistance changes during an action potential;
\item the total movement of ions during an action potential;
\item the threshold and response during the refractory period;
\item the existence and form of subthreshold responses;
\item the production of action potentials after sustained current injection (that is,
anodal break);
\item the subthreshold oscillations seen in the axons of cephalopods
\end{itemize}


"A measurement can be precise without being accurate."\cite{JohnstonWuNeurophysiology:1995} \textit{Accuracy means characterizing 
the true measurable quantity.} In this sense, their measurement is precise but not accurate: they measured precisely wrong quantities, which, unfortunately, can be made and thought sufficiently resemblant to the real ones. 

\subsection{Mathematics\label{Physiology-HHMathematics}}
"These equations and the methods that arose from this combination of modeling and
experiments have since formed the basis for every subsequent model for active cells.
The Hodgkin-Huxley model and a host of simplified equations  derived from
them have inspired the development of \textit{new and beautiful mathematics}." \cite{MathNeuroscience:2010}.


\subsection{Physics\label{Physiology-HHPhysics}}

"A. L. Hodgkin and A. F. Huxley published what was to become known as the
model of the action potential. This model would subsequently be considered a cornerstone of
electrophysiology and neuroscience, since it concerned the ionic mechanisms involved in the
operation of the nerve cell membrane."~\cite{PhysicsForHodgkinHuxley:2024}




\subsubsection{Neglecting Coulomb's Law\label{Physiology-HHNeglectingCoulomb}}

\index{Coulomb's law}

\subsection{Driving force\label{sec:Physiology-HHDrivingForce}}


As we discuss in section~\ref{Physics-OscillatorIntegrator}, HH introduced 
by their Eq.~(1) (reproduced by us as Eq.~(\ref{eq:OutputHH}))
the basic decription of \textit{the membrane current \textbf{during a voltage clamp}} with that 
"the justification for this equation is that it is the simplest which can be used
and that it gives values for the membrane capacity which are independent of
the magnitude or sign of $V$ and are \textit{little affected by the time course of $V$}".
(It is again a reversed approach: the capacity, per definitionem, means the ability to store charge
and is one of the attributes of the medium instead of the electric characteristic
of the tested circuit. For the case of \hyperlink{voltage-clamping}{clamping}, it is an approximation
that the temporal course of the clamping voltage
is kept constant, the current remains constant.  
As we discuss in section~\ref{sec:Physiology-NeuralCurrentsPATCHING},
in the case of a constant current where $I=\frac{dQ}{dt}$,  the voltage increase $dV$
on the capacity $C$ of the membrane is $\frac{dQ}{C}=\frac{I*dt}{C}$,
so we can derive the "driving force" (compare it to Eq.~(\ref{eq:Nernst-dVdt}))
as they interpreted it
%
\[\frac{d}{dt} V=\frac{I}{C}\label{eq:Physiology-HHDrivingForce}\]
%
The direct \textit{constant} current input $\frac{d}{dt} V_{PATCH}$
to the neuron cell body is a simple constant current that causes a constant membrane's voltage derivative contribution.
That is, if one checks the dependence of the output voltage on $\frac{d}{dt} V$ and $I$, in the presence of a constant $C$,
one observes the same dependence.
However, the currents are not necessarily constant.


\subsection{Electricity\label{Physiology-HHElectricity}}

\subsubsection{Equivalent circuits\label{Physiology-HHEquivalentCircuits}}



\begin{figure}
\centering
\includegraphics[width=.65\textwidth]{fig/HHcirdiagm.png} 
\caption{The equivalent circuit for the Hodgkin-Huxley model\label{fig:HH_EquivalentCircuit}}
\end{figure}



%\subsection{Circuits\label{sec:Circuits}}
The biological 'equivalent circuit' models assume that the circuits comprise point-like
ideal \textit{discrete elements}
such as condensers and resistors, and some hidden power changes their parameters according to some
mathematical formulas,
furthermore ideal batteries with voltage that may again be changed  by that power.
All they are connected by conducting  (metallic) ideal wires
and their interaction speed is infinitely high (the Newtonian 'instant interaction').
Using that abstract model enables them to use the well-known classic equations, named after
Ohm, Kirchoff, Coulomb, Maxwell and others. However, those abstractions have severe limitations. Biology applies Ohm's Law to its objects while claims that its objects are non-ohmic; understands that neural currents comprise ions while claims that the ions do not feel the Coulomb repulsion; measures ions' propagation speed but in its equations it claims that their effects are instant; hypotesizes non-existing physical mechanisms to make their nature. In in general, biophysics abstracts from the world of the physics-inspired mathematical formulas a fictitious nature where biology lives and hypothesises complementary mechanisms to  achieve some resemblance with the true biological word. The examples include that ions in the ions channels do not repulse each other; that charge, potential and current are independent of each other; that ion currents do not repulse each other neither when they travel from the presynaptic terminal to the
\index{Coulomb's law}
\gls{AIS};
that some hidden power opens the ion channels to let ions in into the intracellular space, that ions with the same charge move in opposite directions when ions rush-in into the intracellular segment of the neuron; that the ions pass ion channels without being accelerated by the potential difference accross the membrane; the telegraph equations are applied to the case of axons although neither external potential nor current loss exists; and so on.


Even that unfortunate idea of equivalent circuits leads to analyzing "electrotonic (electronic circuit equivalent) modeling of realistic neurons and the interaction of dendritic morphology 
and voltage-dependent membrane properties on the processing of neuronal synaptic input"~\cite{SingleNeuronComputation:2014}; that is, to study a simulated neuron built from discrete electronic components. However, the idea needs to put together
many compone "raises the possibility that the neuron is itself a network". On the one side, such an idea misguides the neurophysical research (since actually a fake neural system is scrutinized, the validity of the approach is questionalized~\cite{RealisticNeuronalModeling:2016}), and on the other, since electroengineers understand
the neuronal operation from the wrong model, it also misguides building neuromorphic architectures. 

The equivalent circuits are a source of misinterpretations. As~\cite{JohnstonWuNeurophysiology:1995} formulates, "In other words, the ionic concentration gradients act like DC batteries for cross-membrane currents."
We call the attention again, that \textit{ions} represent the current,
that is \textit{that current changes changes the concentration, that changes
the voltage of the DC battery, unlike in the case of the equivalent circuits}.
This difference is significant in understand how the basic neuronal circuit works.
Introducing equivalent circuits prevents explaining the fundamental 
electric phenomena.

\begin{advanced}
This was how — Richard P. Feynman approached all knowledge: What can I know for sure, and how can I come to know it?
It resulted in his famous quote, “\textit{You must not fool yourself, and you are the easiest person to fool.}”
Feynman believed it and practiced it in all of his intellectual work.
\end{advanced}

\subsubsection{Oscillator type\label{sec:Physiology-HHOscillator}}

If we apply a continuous square wave voltage waveform to an integrator-type electric RC circuit, we receive an output wave form shown in Fig.~\ref{fig:RCLongWaveform}. After switching a voltage to the circuit, a charge-up process starts, then the output voltage saturates.
After switching the voltage off, the condenser discharges.  \textit{The time constant $\tau$ for the charge and discharge processes are identical.} If the time period of the input square wave waveform is made longer
(in the figure with a half-period “$8*\tau$”), the capacitor would then stay fully charged longer and also stay fully discharged longer.


\begin{figure*}
\includegraphics[width=.65\textwidth]{fig/RCLongWaveform.jpg}
		\caption{The output waveform of an integrator-type electric $RC$ circuit at a long ($8*\tau$) pulse width square wave input (see Eq.(https://www.electronics-tutorials.ws/) \label{fig:RCLongWaveform}}
\end{figure*}

HH measured~\cite{HodgkinHuxley:1952} the time course of the neuronal membrane (i.e., a neuronal $RC$ circuit)
when switched \hyperlink{voltage-clamping}{clamping axonal voltage} on and off.
Their measurement result is shown in Fig.~\ref{fig:ClampingOnOff_HH}.
They experienced a formal similarity with switching a voltage of an integrator-type $RC$ circuit on and off, compare to Fig.~\ref{fig:RCLongWaveform}, so \textit{they concluded that 
the response of the neuronal circuit is identical 
to that of the integrator-type electric $RC$ circuit} (see our discussion in sections~\ref{Physics-OscillatorIntegrator} and~\ref{sec:PHYSICS_MEASURINGOSCILLATOR}). 
They (mistakenly) concluded that the electric equivalent circuit
of a neuron is a parallelly switched electric $RC$ oscillator.
One more evidence that "the success of the equations is no evidence in
favour of the mechanism that we tentatively had in
mind when formulating them". 


\begin{figure}
\iflatexml
\includegraphics[width=.65\textwidth]{fig/HodgkinHuxleyFig2Simple.svg}
\else
\includegraphics[width=.65\textwidth]{fig/HodgkinHuxleyFig2Simple.pdf}
\fi
\caption{\label{fig:ClampingOnOff_HH}The green and red diagram lines are calculated for the clamping-on and clamping-off case. The dashed blue line models the case that the neuron is a purely electric system, as opposed to the case of clamping the axonal tube
with ion channels in its wall.  The black bulbs are for measured points. Moreover, the fitted polynomial line is reproduced from~\cite{HodgkinHuxley:1952}.	
}	
\end{figure}

Figure~\ref{fig:ClampingOnOff_HH} shows \textit{two} switch-on diagram lines, with two different $\tau$~time constants. 
The blue line ("electric") is drawn with the measured time constant of the swith-off discharge. 
Acconding to the theory of electricity, the time constants of the falling and rising edge must be the same in the case of an electric integrator.
The green diagram line (the one fitted to the measured data) correspond to a  different $\tau$~time constant. The effect was observed by ~\cite{HodgkinHuxley:1952}, 
but they did not explain and also did not interpret it. 
Although their fitted polynomial nearly hides the effect, a little "hump" at around $3\ ms$ can be observed.
The reason is that the charge-up current is not constant, it also has a time course; despite that they stabilized the voltage. 

%measure is the membrane receives a current 
%
%
% time course of the charge-up current is also a saturation-type current, and its time constant differs from that of the condenser charge-up.  The diagram line must have the shape corresponding to the product of shape of the charge-up current and the saturation curve of the condenser.
  The different time constants should have been a warning sign for HH that \textit{the measured effect differs from the one they had in mind}.   
%
They wanted to believe and demonstrate that they have measured the output signal
of a serial $RC$ circuit.
With the evaluation of their measurement, they suggested some wrong hypotheses 
\begin{itemize}
\item They misidentified the current as \textit{the 
change of conductance}. No conductance change happens, only a condenser 
changes and discharges. They worked with a condenser, not with a resistor.
\item They meticulously observed and measured that the time contants $\tau$
of the exponential charge-up and discharge processes are different;
but did not care that they should be indentical and also did not care of the "hump".
\item They measured the resulting charge-up composite process. As we explain in section~\ref{sec:Physics-Current},
before switching the clamping voltage, there is no charge carrier 
inside the axonal tube; first, the ions must diffuse into the axon.
\item They fitted the rising time course with a polynomial, hiding 
that it comprises a saturating voltage of the condenser, that 
the current through the axon also saturated with a different time constant, 
and at the begining the line had a zero current contribution.
\end{itemize} 
 

On the one side, they underpinned that our equations \ref{eq:I_Source} and \ref{eq:I_Drain} describe correctly the axonal chargeup current and discharge currents, respectively. On the other, as they noticed that the time constants of the two processes are significantly different (see the green and the dashed blue diagram lines), underpinned that
the axonal charge production mechanism significantly changes
the axonal source current. The green line has significantly slower rise
since the axonal current only gradually increases after switching the clamping on. Without that current change, the charge-up current would follow the blue line.  
Unfortunately, there are three physical processes, with similar time constants. One, that they wanted to demonstrate is the steady state:
the current leads to a saturated current. 
Two, that the \hyperlink{voltage-clamping}{clamping "creates" charge carriers} in the axon,
so the charging current changes.
Three, the initial switch-on creates a voltage gradient on the membrane,
so an action potential is also started.
\index{voltage gradient}
 

Their equations more or less precisely describe
the features of the wrong oscillator type and those of the
non-existing $K^+$ current introduced for compensating for the wrong oscillator selection.


\subsubsection{Cable equation\label{Physiology-HHCableEquation}}

Selecting the wrong oscillator type (actually, assuming distributed parallel
oscillator circuit) and the wrong electrotonic model
leads to somewhat surprising consequences,
such as using the telegrapher equations in a wrong way for
describing neuronal transfer.
By using the cable equation, as Hodgkin and Huxley attempted~\cite{HodgkinHuxley:1952}, led to
\index{cable equation}
numerical difficulties, and they faced the principal problem: their
equations assumed infinitely fast electric interaction, and they attempted
to combine them with the (unknown) finite macroscopic speed of current
in neuronal telegraph cables. The validity of using cable equations
for biological objects is at least doubtful: deriving a telegrapher
equation assumes applying an external potential to the cable filled
\index{telegrapher's equations}
with charge carriers, and in the case of biological membranes neither
external potential nor permanently present charge carriers exist.
Furthermore, the cable equation assumes continuous current outflow
(a distributed resistance), which is not true for the neuronal membrane
(current flows only toward the \gls{AIS}).
\index{cable equation}
\index{telegrapher's equations}


