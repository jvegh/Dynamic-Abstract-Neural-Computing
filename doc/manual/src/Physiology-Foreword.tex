% The foreword for physiology


\quotationbox{
In honor and spirit of the Nobel-laureates:\\
%	 
- “For the sake of illustration we
	shall try to provide a physical basis for the equations, but must emphasize
	that the interpretation given is unlikely to provide a correct picture of the
	membrane.”\\	
- "the success of the equations is no evidence in
	favour of the mechanism of permeability change that we tentatively had in
	mind when formulating them."\\	
— Hodgkin and Huxley, 1952~\cite{HodgkinHuxley:1952}\\
They were brave enough to admit that at some point they must stop and to publish their excellent observations "as is"; despite their feeling that they could not grasp adequately the processes. Their intention was to help their fellow researchers in 
using their observations in practical research.
They did their best with attempting to provide a correct picture of the membrane.
We interpret their much inspiring model in light of the later interpretation, theoretical and experimental results, critics and speculations.
We attempt to put those constituents together into a consistent model,
with a science background, with calculable details,
including physical and mathematical handling of thermoelectricity,
handling mixing interaction speeds and slow ion currents.
}

\begin{advanced}
- "From all we have learnt about the structure of living matter, we must be prepared to find it working in a manner that cannot be reduced to the ordinary laws of physics. And \textit{that not on the ground that there is any ‘new force’ or what not}, directing the behaviour of the single atoms within a living organism, \textit{but because the construction is different from anything we have yet tested in the physical laboratory}."
\\
\href{https://www.cambridge.org/core/books/abs/what-is-life/is-life-based-on-the-laws-of-physics/5BA8DE837F72FC4EEF70FE8470ADFE7B}{E.~Schrödinger: What is life?}\cite{Schrodinger:1992} @1992

This is valid also for delayed rectifying current, non-ohmic behavior, ions lacking their repulsion in ionic currents, moving ions with the speed of
\gls{EM}
interaction in electrodiffusion, separating current and voltage, voltage dependent conductance, and so on.
\end{advanced}

\begin{advanced}
"And therefore when we go to investigate we shouldn’t pre-decide what it is we are trying to do except to find out more about it."
"The first principle is that you must not fool yourself, and you are the easiest person to fool."\\
"Scientific knowledge is a body of statements of varying degrees of certainty — some most unsure, some nearly sure, but none absolutely certain."\\
-- Richard P. Feynman
\end{advanced}


\hypertarget{AbstractPhysiology}
{Biological objects}, with their semipermeable membranes, separating
ionic solutions into sections with concentrations differing by orders
of magnitude, furthermore containing voltage-controlled ion channels
that react actively to electric fields, are more complex cases for
understanding their detailed operation. In addition, they implement a complex operational (dynamic) functionality: \href{https://www.pnas.org/doi/10.1073/pnas.1705704114}{"stimulated phase transitions enable the phase-dependent processes to replace each other ... one process to build and the other
to correct"}~\cite{BiologicalConservationLaw:2017}. Classical theory cannot explain
some details of neurophysiological phenomena, including neurons' charge
processing, especially their temporal behavior, that implements its
information processing capability because physiology incorrectly interprets
the fundamental electric terms. Extrapolating notions derived from
metals to electrolytes, especially to biological neurons with electrically
active internal structures, may be misleading.

%how a biological neural network can react at a speed about two orders of magnitude higher than it could be expected based on its static behavior (the time difference between adjacent spikes) and, in general,

