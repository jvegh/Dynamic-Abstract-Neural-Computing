% The action potential in an abstract approach

\section{Resting potential\label{sec:Physiology-DerivingRestingPotential}}



Table~\ref{Tab:SummaryTable} shows our results for three different
famous published cases. It looks like the calibration of the concentration 
of negative and positive ions needs scaling and the measurement of $Ca^{2+}$
comprises issues.


\renewcommand{\arraystretch}{1.3}
\begin{table}
	\caption{The summary data table for equilibrium (Data from Table 2.1 in~\cite{JohnstonWuNeurophysiology:1995}. For the graphic representation 
	of potential balance see Figure~\ref{fig:RestingPotential3} ) \label{Tab:SummaryTable}}
\iflatexml	
	\begin{tabular}{ |p{0.6cm}||p{0.9cm}|p{1.05cm}|p{3.9cm}| p{1cm}| p{1cm}| p{1.1cm}|}
	\else
	\begin{tabular}{ |p{0.55cm}||p{0.65cm}|p{0.65cm}|p{3.38cm}| p{1.1cm}| p{1.1cm}| p{1.1cm}|}
	\fi
	\hline
	\hline
 Ion& [In] &[Out] &  $U_{therm}\frac{RT}{zF} log\bigl( \frac{[C]_{o}}{[C]_{i}}\bigr)$ &$U_{electr}^{theor}$&$U_{electr}^{exper}$&$U_{membr}$\\
	& [mM] & [mM] &  [mV] &  [mV] &[mV]&[mV]\\
	\hline
&\multicolumn{6}{|c|}{Squid \cite{HodgkinHuxley:1952} \  T=293\ $K^o$ \  Width= \textbf{3}\ [nm] \  Field= \textbf{14.0}\ [MV/m] 
}\\
	\hline
\em	$K^+$ & 400 & 20 & $-75.5= 58 log\bigl( \frac{20}{400}\bigr)$& \textbf{41.9}&\textbf{41.4}&-62.0\\
	$Na^+$ & 50 & 440 & $+54.8= 58 log\bigl( \frac{440}{50}\bigr)$&&&\\
	$Cl^-$ & 40 & 560 & $-66.5= 58 log\bigl( \frac{560}{40}\bigr)$&\textbf{41.9}&\textbf{42.0}&63.0\\
$Ca^{2+}$ & $0.27^*$ & 10 & $+45.5= 29 log\bigl( \frac{10}{0.27}\bigr)$&&&\\
&\multicolumn{6}{|l|}{$^*$ used for the published value 0.4} \\

	\hline
&\multicolumn{6}{|c|}{Frog muscle (Conway) \  T=293\ $K^o$ \  W = \textbf{8}\ [nm] 
\  E= \textbf{9.6}\ [MV/m] 
}\\
	\hline
$K^+$ & 124 & 2.25 & $-101= 58 log\bigl( \frac{2.25}{124}\bigr)$&\textbf{88.9}&\textbf{83.6}&-125.4\\
$Na^+$ & 10.4 & 109 & $+59.2= 58 log\bigl( \frac{109}{10.4}\bigr)$&&&\\
$Cl^-$ & 1.5 & 77.5 & $-99.4= -58 log\bigl( \frac{77.5}{1.5}\bigr)$&\textbf{88.9}&\textbf{82.7}&124.1\\
$Ca^{2+}$ & $0.021^*$ & 2.1 & $+58.0= 29 log\bigl( \frac{2.5}{0.25}\bigr)$&&&\\
&\multicolumn{6}{|l|}{$^*$ used for the published value 4.9} \\

	\hline
&\multicolumn{6}{|c|}{Mammalian cell \  T=310\ $K^o$ \  W= \textbf{6}\ [nm] 
\  E= \textbf{11.1}\ [MV/m] 
}\\
	\hline
$K^+$ & 140 & 5 & $-89.7= 62 log\bigl( \frac{5}{140}\bigr)$&\textbf{57.7}&\textbf{57.3}&-85.8\\
$Na^+$ & 15 & 145 & $+61.1= 62 log\bigl( \frac{145}{15}\bigr)$&&&\\
$Cl^-$ & 4 & 110 & $-89.2= -62 log\bigl( \frac{110}{4}\bigr)$&\textbf{57.7}&\textbf{56.9}&85.9\\
$Ca^{2+}$ & $0.04^*$ & 5 & $+60.8= 31 log\bigl( \frac{5}{0.04}\bigr)$&&&\\
&\multicolumn{6}{|l|}{$^*$ used for the published value 0.0001} \\

	\hline
&\multicolumn{6}{|c|}{$Na^+$ ions only~\cite{OriginMembranePotential:2018}\ T=310\ $K^o$ \  W= \textbf{6}\ [nm]
\ E = \textbf{10.0}\ [MV/m] 
}\\
	\hline
$Na^+$ & 12 &145 &$+66.6= 62 log\bigl( \frac{145}{12}\bigr)$&60.7&66.6&\\
	\hline
	\hline
\end{tabular}
\end{table}


When calculating the values in Table~\ref{Tab:SummaryTable}, we used $\Delta z=1.05$ and employed the membrane's thickness data from various publications to demonstrate that our theoretical approach is correct and that using the correct data can provide even the absolute values of the membrane potential.   
The higher-than-expected value of $\Delta z$ may suggest that the ions and membrane's double layers in the electrolyte may also play a role. 
Dedicated, complete investigations can reveal the details.


\begin{figure}
\includegraphics[width=.7\textwidth]{fig/HypotheticRestingPotential.png}

	
	\caption{ (Figure 3.1.1 in~\cite{OriginMembranePotential:2018}) Generation of the Na equilibrium potential across a hypothetical membrane that is permeable only to $Na^+$.
	The $[Na]_o$ in this case is high, about $145\ mM$, whereas $[Na]_i$ is $12\ mM$. Thus the diffusion gradient favors $Na^+$ flux from the outside to the inside of the cell.
	Since $Na^+$ is electrically charged, flux of only $Na^+$ makes a current across the membrane that separates charges and thus produces an electrical potential.
	The electric field exerts forces on Na+ that retards its movement. When the diffusive force exactly balances the electrical force, flux is zero and the potential is $E_{Na}$, the sodium equilibrium potential.
	 \label{fig:HypotheticRestingPotential}}
\end{figure}




