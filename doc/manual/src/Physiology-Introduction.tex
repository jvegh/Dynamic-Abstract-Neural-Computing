% The introduction for physiology
\section{\label{sec:Physiology-Introduction}}


Without having charge, the particles in a solution follow the laws
of thermodynamics. Having charge affects their mass distribution and
leads to measurable effects in the solution's local electric distribution:
the change of one gradient generates the change of the other. However,
the corresponding ``laws of motion'' are still missing. We must
introduce (at least) two interaction speeds (at least 'fast' and 'slow'
currents) and derive physics-based mathematical approximation to describe
the experiences. We generally show a way of handling interactions
with speeds differing by orders of magnitude. It is a pre-requisite
of attempting to describe life by laws of science: life is based on
thermoelectrical processes in biological matters. We derive a method
and the mathematical form of calculating the Nernst-Planck equation's
\index{Nernst-Planck equation}
\index{equation!Nernst-Planck}
time derivatives, showing how that time course can describe processes
with the participation of the (neuronal) membrane. \emph{The derived
equations are the Maxwell equations for thermoelectricity and also
the laws of motion for biology.} 
\index{Maxwell equation}
\index{equation!Maxwell}
We show that a semipermeable membrane
introduces a new interface (a layer of atomic width) between the electrolyte
and the membrane that can be described only using microscopic and
macroscopic terms simultaneously. We also discuss the operation of
gated ion channels in semipermeable membrane, furthermore, that ions
change their speed from \emph{drift }speed to \emph{potential-assisted}
speed or even \emph{potential-accelerated} speed. The case studies
also include how and why action potential is evoked due to finite
size of the neurons and finite speed of their charge processing. The
time-aware computing procedure (considering the finite speed of charge
carriers) naturally describes the biophysically plausible abstract
model of electric charge processing in biological neurons. 
\textit{\hyperlink{ElectrodiffusionProcesses}{The mentioned signals are electrochemical ones} with temporal behavior}
instead of purely electrical and purely chemical ones
happening in a timeless world, with numerous significant consequences.




Biology uses the idea of "instant interaction" borrowed from classic
physics and describes its "slow" currents using equations developed
for instant interaction; i.e., as if they were "fast" currents.
The lack of idea about slow currents leads
to the introduction of fake currents (which have no source, change
the ion type, introduce non-physical delays between the two secondary
electric entities voltage and current, and so on). \textit{Experimental
biology sees the time, but its theory does not want to admit it.}
Instead of introducing that biological processes need time, biology
suggts that physics has a wrong notion about speed, space, time,
and electricity. Unfortunately, biophysics participates in this business:
in some cases simply translates notions, terms and equations from
classic physics (mainly theory of electricity), without validating
that the original abstraction remains valid in the case of biology.




Nature is infinitely complex, and science must make
abstractions for a particular assembly of phenomena to describe them
with reasonable accuracy using a piece of that language. We create
a dialect of that universal language and validate its usage for our
assembly of phenomena. Different science fields may use different
abstractions and dialects, developing new ones or attempting to inherit
them from other science fields. However, in the latter case, \textit{we
must not borrow their validation without scrutinizing whether in the case of our
abstractions they remain valid}. To describe its electric operations,
biology borrowed laws and equations from physics, where it was assumed
that the speed of interaction is the speed of 
%\gls{EM} 
EM
interaction,
the interactions have the same speed and they are strictly pair-wise.
Although technical electricity knows and describes that macroscopic
currents are a million times slower and different laws describe them,
biology did not borrow the idea or where it did (the telegrapher's
equation); it did it incorrectly.

Believing that the mathematical equations developed for the abstraction
of "instant interaction" of classic physics are valid for biology,
leads to describing a surrealistic virtual nature, where charges are
not present but creating potential
%(see the @link MEASURING_ELECTROLITES_CLAMPING clamping measurement@endlink
%and
%describing \gls{AP} with an @link sec:{sec:PHYSICS_MEASURINGOSCILLATOR} oscillator@endlink);
the  \href{https://doi.org/10.1038/9173}{conductance depends
on the applied voltage}~\cite{KochVoltageDependentConductance:1999},
and that conductance governs biological operation; the mystic reason
of creating 
%\gls{AP}
AP is "based on an indirect inference, that excitatory
inputs interact sublinearly when located somewhere in the distal dendritic
tree" (for a review see section 5.2 in \cite{KochBiophysics:1999});
signals propagate with delays without assuming that it happens due
to the finite speed of current that delivers the signal; applying
telegrapher's equations, which assume that current flows out of the
cable, to the case of axons where the current remains the same while
it travels; modeling and showing dependencies of infinite-speed electric
circuits for finite-speed biological circuits; applying Kirchoff's
Conservation Law to space segments where ions are created inside (diffuse
into the segment); that the temporal operation due to many-synaptic
inputs is governed by some cross-correlation of independent inputs; that the information
processing of time-dependent analog signals is described by a formalism
developed for time-independent digital signals.

Since Euclid, we know "there is no royal road to geometry". We
add that there is no biologist's road to nature. The true nature is
different from the one they assume when they apply the mathematical formalism
developed for the approximations and abstraction of classic science.
One must derive the proper approximations relevant
to the studied phenomena, usually comprising several interactions
with speeds differing by orders of magnitude, develop the appropriate
abstractions and approximations, then develop the corresponding mathematics.
This is also valid for the
case of electricity in metals, from  the speed of 
%\gls{EM}
EM
interaction down to the drift speed of electrons. Even we know that
%@link PHYSICS_SPEEDS_FINITE
the speed of a macroscopic current
%@endlink
is between those two extreme values.
Extrapolating our notions about the macroscopic world to the microscopic
one and assuming the classic interaction speed (the "instant
interaction"), abstracted from the vast interaction
speed of 
%\gls{EM}
EM 
waves to at least ten orders of magnitude lower
interaction speed, is misleading.



