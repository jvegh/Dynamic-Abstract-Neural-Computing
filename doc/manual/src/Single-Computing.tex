
% The concept of action potential
%\subsection{The actions\label{sec:Single-Actions}}


\section[Computing\&Communication]{Implications for computing and communication\label{sec:Single-ImplicationsComputing}}
 
Fig.~\ref{fig:AP_Conceptual} also reveals some secrets of the effective
biological computing.
\begin{itemize}
\item biology makes the "weighted summing" of neuron's synaptic inputs 
simultaneously, in one single operation making multiplication and integration, furthermore selecting the time window and its effective synaptic inputs
\item the heavily used neuronal information "is stored, as it should be, in every circuit"~\cite{SterlingPrinciples:2017}
\item "information stored
directly at a synapse can be retrieved directly"~\cite{SterlingPrinciples:2017}
\item part of the information (in 'volatile' memory) is stored only
for the period when it might be needed (a real temporary cache)
\item "Computing" is much shorter than "Delivering"
\item asynchronous; the operation time varies (no pipelining)
\item "Send only information that is needed, and send it as slowly as possible"~\cite{SterlingPrinciples:2017}
\item using voltage temporal gradients enables transferring more information and functionality;
for example, synchronization~\cite{LosonczyIntegrative:2006}
\item for simulation: there is no need to send and integrate spike shapes,
only \textit{the time of arrival/sending} 

\end{itemize} 


