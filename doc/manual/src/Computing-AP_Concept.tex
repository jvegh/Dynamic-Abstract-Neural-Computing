% The conceptual discusion of abstract Action Potential

\section{Action Potential\label{sec:Computing-ActionPotential}}

In this section we discuss the  \textit{concept} of
%\hypertarget{SINGLE_AP_CONCEPTUAL}
{action potential}
in an abstract sense that enables to define its notions, features and stages.
Our approach here is hybrid: we know that the events are connected 
to ion movements and that the components' cooperation forms the action potential.
The %\hyperlink{PHYSIOLOGY_AP}
{physiological}
and 
%\hyperlink {PHYSICS_AP}
{physical} details
are discussed in the respective chapters \ref{ch:Physiology} and \ref{ch:Physics}, where we provide citations describing the physiological details.
We consider the neuron as an \textit{abstract computing element}
and show how a neuron implements the generalized computing we discuss in chapter\ref{ch:Computing}.

We assume that the neuron is in state "Relaxing", so the membrane's voltage is at the resting value. The membrane and the 
\gls{AIS}
are at the same potential, so no current is present (no "leaking current" exists).
When input charge (through the synapses or directly through the membrane)
 arrives to the membrane, its potential increases. The increased membrane potential means a potential difference between the membrane and the axon, so it drives a current through the AIS.
% \gls{AIS}
The current
 (not identical with the leaking current) 
 \index{leaking current}
 \index{current!leaking}
 decreases the membrane' potential between adjacent synaptic inputs.
   For simplicity, we assume that the axonal inputs cause a step-like change in the membrane's voltage. Between the inputs the current through the
   \gls{AIS}
decreases the membrane's potential. As we discuss, the neuronal computation actually measures the time between the arrival of the first synaptic input and exceeding the threshold; it is in the order of tenths of a millissecond.

 When the resulting potential exceeds a threshold value, for a very short time
 (in the order of tenth of picoseconds) the ion channels in the membrane's wall open
 and a large amount of ions suddenly (in a step-like way) increase the membrane's potential
 in its \hyperlink{atomic_layer}
 {surface layer}; see section~\ref{sec:Physics-TwoSegments}.
The current due to the rushed-in ions creates a local potential gradient and the ions
(with a potential-assisted speed) saturate the layer
(the mechanism is described in section~\ref{sec:Physiology-Electrodiffusion}),
open all ion channels. The rushed-in ions feel the potential gradient toward the \gls{AIS},
but they can move in the layer on the surface
with a finite speed, so the current from the different points of the membrane need different times to reach the \gls{AIS}.
Correspondingly, the current (due to the current "created" by the nearby ion channels) reaches the \gls{AIS}
instantly, while the current from the farthest point needs tenths of a millisecond to get to the \gls{AIS}.
%
After that, given that the current cannot flow out "instantly",
the current produces a kind of "damped oscillation": drops below 
the resting potential and then asymptotically approaches it, without exceeding it.
It is the effect of the neuronal oscillator, marked as "RC-effect".
In some sense, the ionic current "disappears": it get stored in the neuronal condenser
while it travels on the surface of the membrane.

During this process, the membrane's voltage controls the synaptic inputs.
Given that the ions can reach the membrane using a "downhill" method,
the current stops  when the membrane's potential rises
above that of the axonal arbor,
and will not flow until the membrane's potential drops again below the threshold:
the synapses will be disabled and re-enabled.

Biology observed the "absolute refractory" period, which is interpreted 
that the synapses are disabled for a period, and it is a correct observation.
Different is the case for the "relative refractory" period.
Actually, the synaptic inputs arrive at the junction of the axon, and the current must travel to the \gls{AIS}.
that needs time (in the order of tenth of a millisecond). Given that 
the \gls{AP} is measured at the \gls{AIS}
the main contributor' current in the meantime proceeds toward the "hyperpolarized" state,
and so the synaptic inputs apparently contribute
outside of the "absolute refractory" period,
so this extension is called "relative refractory".
Actually, the origin of both periods is the same,
only the effect's time scale is shifted by the ionic current's travel time.

As discussed, after that the membrane's voltage drops below the threshold potential,
the neuron can start a new computing.
The signal "ComputingBegin" is defined as the signal arriving first after then
the neuron membrane's potential crossed the threshold value from the higher voltage direction.
At that point, the membrane's voltage can be above or below the resting potential,
and, correspondingly, the charge integration starts from a value higher or lower than the resting potential.
Effectively, the value of the potential (more precisely, \textit{when} the first
synaptic input arrives at the \gls{AIS}) represents a memory with initially a negative, later positive, time-dependent content.
%

Our stages slightly deviate from the ones commonly used in physiology.
We define the stage "Computing" as the period between the arrival from the first synaptic input
to exceeding the threshold.
The stage "Delivering" is defined as the period while the membrane's voltage stays above the threshold.
The stage "Relaxing" begins when the "Delivering" ends, and may be interrupted by a synaptic input.
Notice that because of the spatiotemporal nature, the time values have a definite meaning only
if the place of the measurent is also provided: it is one coordinate of a space-time point.







The computer representation (actually a state machine) is shown in Figure~\ref{fig:NeuronStateMachine}.                          


Will be based on \cite{VeghComputingModel:2021} \cite{VeghRevisingClassicComputing:2021}
