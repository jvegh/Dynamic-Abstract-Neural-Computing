% The action potential in an abstract approach

\section{Action potential\label{sec:Physiology-DerivingActionPotential}}


The \gls{AP}
is one of science's big mysteries. We put together an isolating membrane,
isolating membrane tubes connected to the membrane, voltage-controlled ion channels
in their walls, and an electrolyte solution around them.
We apply currents/voltage levels or pulses to the tubes,
and at some point (at some appropriate combination of parameters; too small or too large currents result in stopping 
%\gls{AP}s),
APs, the system starts to issue
%\gls{AP}s:
APs:
 \hypertarget{LivingMatter}{\textit{the non-living matter turns into living matter}}. In contrast with the expectations of
 %\gls{HH}
 HH \cite{HodgkinHuxley:1952},
\textit{the 
%\gls{AP}
AP can be described from the first principles of physics} when using the right physical
{approximations and abstractions}.

Neurons interface the living and the non-living components of nature. To understand
the details of their respective operation, an exact  interpretation of notions and laws
of non-living matter is also needed. Applying the laws derived for an approximation
abstracted for the conditions of classic science is misleading and prevents us from understanding
that, \textit{at different abstraction levels, neurons are living components and simultaneously,
still, they can be described by the laws of non-living science}; provided
that we use the right abstractions and approximations, as their case requires.
They are studied by research methods and tools
of fellow sciences and are described by the universal language of nature: mathematics.
However, not necessarily by the mathematical procedures developed for other goals
and used in classic science.

\gls{HH}  attempted to find a mathematical formalism
for their very precise measurements~\cite{HodgkinHuxley:1952} and find out \textit{empirically} what kind of mathematics (invented for the approximations used in classic science) can -- more or less -- describe 
their experiences,
despite that "a number of points were noted on which the calculated behaviour of our model did not agree with the experimental results to provide a correct picture of the
membrane." 
Their followers forgot the doubts and question marks HH described and took their unproven hypotheses as facts.
"These equations and the methods that arose from this combination of modeling and
experiments have since formed the basis for every subsequent model for active cells.
The %\gls{HH}
HH model and a host of simplified equations  derived from
them have inspired the development of \textit{new and beautiful mathematics}." \cite{MathNeuroscience:2010}.
\textit{That mathematics is beautiful but describes 
some alternative nature instead of the real one}, see also section~\ref{sec:Physiology-Fallacies}.


Despite the impressive advances in neuroscience during the past decays,
there are still 'white spots'. '\textit{Why} action potentials are initiated in the axon is still
unclear' \cite{ActionPotentialGenerationNatrium:2008} and
"we should not seek a special organ for 'information storage' -- it is stored,
as it should be, in every circuit" \cite{SterlingPrinciples:2017}.
This latter source points to the important point
'Communication consumes 35 times more energy than computation'. \cite{EnergyNeuralCommunication:2021}
One more point why \textit{computation and communication must not be handled separately}~\cite{EDVACreport1945}.
It also asks the questions \textit{what is information, how it is stored, processed and transmitted}.




We can model a neuron as an oscillator where the membrane changes its potential above a resting potential, receives
(gated) synaptic currents through its axons and through its ion channels, furthermore
external currents/voltages provided by the experimenter.
Those currents are slow, so we must consider their speed
to produce the membrane's correct behavior, either as a time delay
or a time course of a current intensity.



\subsection{The classic picture\label{sec:Physiology-APClassic}}

\begin{advanced}
"there is little hope of calculating the
time course of the sodium and potassium conductances from first principles"\\
(surely, if one uses empirical functions)
\cite{HodgkinHuxley:1952}
\end{advanced}
In the classic picture, we assume that the membrane is equipotential.
When the axonal input current charges it up to some threshold value,
an intense charge-up process starts due to ion inflow. After the
membrane's voltage exceeds some other threshold, a spike begins. After
some time, and for some reason, az outward ``delayed rectifying current''
starts from some hidden source and hyperpolarizes the membrane. Somewhat
later, both currents stop, in a concerted way, for some reason. In
the classic approach, a spike is sent and received instantly (an incoming
spike ``makes a hole''~\cite{KochElectricalPropertiesSpike:1983}),
the charge it delivers is added to the membrane in a snap, and
the neuron fires in a snap somewhat later. The process details are
known (although some processes are only hypothesized, and others are
misunderstood). However, the control mechanism of the process
is unknown or mystic: the classic model answers the question "what"
but leaves the questions of type "why\"
and "how" open. ``\emph{Why}
action potentials are initiated in the axon is still unclear''~\cite{ActionPotentialGenerationNatrium:2008}. 




\begin{figure}
	\includegraphics[width=0.55\textwidth]{fig/ActionPotential_Kandel7-10.png}
	\caption{\textcolor{blue}{Errata in blue.}  (Fig. 7-1 in~\cite{PrinciplesNeuralScience:2013}).
	The sequential opening of voltage-gated $Na^+$
and $K^+$ channels generates the action potential.\textcolor{blue}{Wrong. The $Na^+$ channels are voltage controlled, and they open 
when AP
%\gls{AP}
is initiated. There are no $K^+$ channels in the game. Some \textit{resting} $K^+$ channels exist and they contribute to the total current. Some $K^+$ flows out through the AIS,
%\gls{AIS},
but it is by two order of magnitudes less then shown. 
Actually, these two currents are the condeser currents, 
before and after potential reversal on the condenser. 
The '$Na^+$ current' is the current generated by the 
sharp 'switch-on' potential gradient, with a distortion 
caused by approaxing the current's time course by a polynomial.}
One of
Hodgkin and Huxley’s great achievements was to dissect the
change in conductance during an action potential into separate
components attributable to the opening of $Na^+$ and $K^+$ channels.
\textcolor{blue}{Mostly wrong. There is no change in conductance, it was a wong concept. They were tight in assuming the existence
of ion channels and their opeing/closing, but it is not the case
for $Na^+$.}
The shape of the action potential and the underlying conduct-
ance changes can be calculated from the properties of the
voltage-gated $Na^+$ and $K^+$. \textcolor{blue}{Wrong.
They used empirical measured distributions, instead of propertios of ion channels.}(Adapted, with permission,
from~\cite{HodgkinHuxley:1952}.)
		\label{fig:HHAP_GeneratesActionPotential }
	}
\end{figure}

\subsection{The modern picture\label{sec:Physiology-APModern}}

We know that ion current flows in
through the axons, and the delivered charge produces local transient
voltages~\cite{TransientResponses:2008,KochElectricalPropertiesSpike:1983}
on the membrane around the arbor of the synaptic connection. Furthermore, the ions form "packets"
when immediately before issuing an
% \gls{AP}
AP they  arrive at the 
%\gls{AIS}
AIS~\cite{ActionPotentialGenerationNatrium:2008}. \emph{We hypothesize
	that for the period of generating an 
	%\gls{AP}
	AP (furthermore, for the period of receiving synaptic inputs),
	the membrane does not remain equipotential.} The membrane is
a two-dimensional, very thin, elastic, semipermeable, insulator surface
(a long and narrow rectangle) with
a high concentration of ions on the extracellular side. Axon tubes
are connected to the intracellular side at some points of the long rectangle
and 
%\gls{AIS}
AIS at the other end. The membrane
attempts to remain equipotential and forwards the charge toward regions
at lower potential. This way, the ionic charge 
forms a ``slow'' current and gets distributed over the membrane's
surface along some potential gradient for about a few hundred microseconds.

Due to that charge, a new epoch begins when the membrane potential reaches its threshold value. That voltage opens the valves (ion channels)
in the membrane's wall, ions rush into the intracellular space
(a positive current). The intense ``slow'' current quickly increases
the membrane's potential, so the axonal inflow stops. The membrane
has a single flow-out point, the 
\gls{AIS},
where a less intense ``slow''
current leaves the cell with a delay and arrives at the very beginning
of the axon. In the first phase, the axon pumps the received current
out to the extracellular space (a negative current), causing a measurable
macroscopic current. Later, the pumped-in and pumped-out ions along
the axon are balanced and begin transferring the spike along the
axon. 

%\subsection{Side effects of \gls{AP}}

"Action potentials (APs) have been measured using electrophysiological
methods and understood as electrical signals generated and propagating
along the axonal membrane"~\cite{MechanicalWaves:2015}, and "the
\gls{AP} is accompanied by fast and temporary mechanical changes" (such
as \indexit{axonal radius}, \indexit{pressure}, \indexit{optical properties},
the release and subsequent
\index{heat absorption}
\index{shortening of axon}
absorption of a small amount of heat, and shortening of the axon at
its terminus).

Interestingly, even the paper~\cite{ElectrochemistryMembrane:2022}
that attempts to describe non-ideal membranes, considering also mechanical
deformations, uses only ``fast'' waves. Similarly, the model in~\cite{MechanicalWaves:2015}
predicts a ``traveling wave of voltage'' without seeing that it
also means a traveling wave of current, i.e., a finite speed (``slow'')
current (``we emphasize that the driven waves we consider will travel
at the speed of the electrical \gls{AP}
that drives them''). The
electrostatic repulsion leads to mechanical stress on the membrane.

An interesting parallel with science is that 'classic physics' 
is based on the abstraction that position-related phenomena do not depend on time
(the time derivative of the position)
and that the 'classic physiology' is based on the abstraction that the charge-related
phenomena do not depend on time (the time derivative of the current).
In 'modern physics', although mathematics, based on the Newtonian abstraction, perfectly describes a wide range of phenomena, near to the limiting speed the Einsteinian abstraction
entirely different mathematics must be used to describe nature perfectly.
Similarly, \textit{in the 'modern physiology' the time derivative of charge movement 
must also be taken into account} when describing the dynamic physiological 
behavior of neurons that needs different approximations.
Neglecting the time derivative of the position may result in
calculating speed above the limiting speed (the speed of light),
and neglecting considering the time derivative of charge movement may result 
in a wrong understanding of the neuronal electric operation.



\subsection{Role of AIS\label{sec:Physiology-AP}}


As we also discussed~\cite{RoleOfInformationTransferSpeed:2022},
more interactions are involved in the living matter, and the interactions
need a spatiotemporal description. We use the notion of time dependence
in the Einsteinian sense: the basic entities such as \emph{location
and time are connected through their interaction speed} and they are
not independent parameters in the Newtonian sense. So, we expected
that neurons, as dynamical thermoelectricity-based systems, are described
by thermoelectric time-dependent equations. However, it is not
so.

One of the fundamental reasons is that the way of \emph{providing
time derivatives of the thermoelectricity process was not known} (see
our section~\ref{sec:Nernst-time-derivatives}). The other is
that, conceptually, \emph{the neuron is considered to be a purely
electric system that connects to thermodynamics only through the time-independent
Nernst-Planck equation}. The third is that even the description of
the purely electric operation is wrong: biology separated the primary
abstraction of electric 'charge' from its secondary manifestations
of abstractions 'potential' and 'current'; furthermore, it assumes
that a mysteric power changes biological 'conductance' against all
physics laws (applying laws of electricity to their 'non-ohmic' systems).
The fourth is modeling problems we discuss below. Theory of biology,
among others, stayed at its century-old ideas about equipotential
neuron surface, time-unaware information processing~\cite{VeghNeuralShannon:2022},
although the experimental physiology delivers a vast amount of evidence
for the opposite.

\begin{figure}
\includegraphics[width=\textwidth]{fig/AIS}

\caption{The structure of the Axon Initial Segment. \cite{ActionPotentialGenerationNatrium:2008}
Ann. N.Y. Acad. Sci. 1420 (2018) 46-{}-61, Figure 1 \copyright 2018
New York Academy of Sciences. \label{fig:StructureOfAIS} }

\end{figure}

By assuming that biological operation can be described by well-known
electric terms, Hodgkin and Huxley~\cite{HodgkinHuxley:1952} advanced
neuroscience enormously. However, their seven-decades-old hypotheses
must be updated from several points of view. Among others, they provided
a static empirical description (their differential equations rely
on derivatives of an empirical function fitted to empirical measurement
data, and even in a wrong way). They excluded interpreting the physical
background of their empirical description using an empirical conductance
function; in this way, really, there is ``little hope of calculating
.. from first principles''. Their suggestion about equivalent circuits
introduced the idea that the membrane's potential remains unchanged
during operation despite the ion traffic, the ions in the current
do not affect concentration, and the components of the circuit operate
with the speed of the %\gls{EM}
EM
 waves. By introducing the delayed
current and that some mystic power controls the operation of neurons
by changing their conductance, they gave way to introducing the fallacy
that science and life sciences are almost exclusive fields. Furthermore,
their (unintended) model provokes questions (for a review see \cite{HH_Potential_Controversies_2017})
whether it is model at al and what controversies it delivers.

They meticulously wrote that ``the success of the equations is no
evidence in favour of the mechanism ... that we tentatively had in
mind when formulating them''. Although ``certain features of our
equations were capable of a physical interpretation'', ``the interpretation
given is unlikely to provide a correct picture of the membrane''.
Despite their doubts, biophysics produced fictitious mechanisms to
underpin their equations, describing an admittedly wrong physical
picture instead of setting up a correct physical operation and describing
the processes by deriving physically plausible approximations and
using correct mathematical expressions. Although they warned that
``the agreement {[}between our theory and experiments{]} must not
be taken as evidence that our equations are anything more than an
\emph{empirical description}'', their followers forgot their doubts
and question marks and took their unproven hypotheses as facts. "These
equations and the methods that arose from this combination of modeling
and experiments have since formed the basis for every subsequent model
for active cells. The model and a host of simplified equations derived
from them have inspired the development of \emph{new and beautiful
mathematics}."\cite{MathNeuroscience:2010}. However,
there was no model, and the beautiful mathematics describes a fictitious
neuron.

One of the most influencing bad ideas was expressed by their Eq.(1).
In their time, at that limited microscope resolution, they did not
see any structure within the neuron's membrane, so logically, they
assumed that the measured capacitance and resistance were distributed.
Correspondingly, they introduced an electric equivalent circuit assuming
that the neuronal $RC$ circuit comprised parallelly connected discrete
$R$ and $C$ elements. They described the neuronal operation as based
on an integrator-type equivalent electric circuit with the corresponding
equations.

They assumed that a fixed voltage drives a constant current through
the circuit, and the discrete $R$ and $C$ elements share that current.
Correspondingly, a leaking current must exist, and the resting brain
must dissipate power (as later estimated, around $20\ W$). However,
the operation of the neuronal circuit resulted in well-measurable
hyperpolarization (the output voltage changes its sign), which the
equivalent \emph{parallel} electric circuit cannot produce, so they
assumed that in addition to a $Na^{+}$ current, a delayed $K^{+}$
current to flow through the neuron membrane in \emph{the opposite
direction}, against the flow of $Na^{+}$ ions. In their milestone
work, their goal was to derive equations for practical application,
so they introduced equations describing the measured electric observations.
Unfortunately, they attempted to determine which processes were going
on inside the biological neuron.

In the past years, instrumental advances have enabled us to discover
the ``white spots'' of their time. Around 2018, the 
%\gls{AIS}
AIS
 was
discovered and understood~\cite{AIS_Updated_Viewpoint:2018,ActionPotentialGenerationNatrium:2008}.
From an electric point of view, the 
%\gls{AIS}
AIS
 is an array of ion
channels with well-measurable resistance; it can be abstracted as
a discrete resistance. As the anatomical evidence shows, see Fig.~\ref{fig:StructureOfAIS},
the currents flow into the membrane and flows out through the
%\gls{AIS}.
AIS.
The currents are not shared, and even, the output current cannot directly
be concluded from the sum of the input currents: the charge is temporarily
stored by the membrane (as a distributed condenser). The correct equivalent
circuit is one in which the condenser and resistor are switched in
serial (although the resemblance has limitations), that is a differentiator-type
electric circuit. This circuit is sensitive to voltage gradient, so
\index{voltage gradient}
the rising and falling edges of an input signal (such as a 
%\gls{PSP})
PSP)
can natively produce an opposite voltage on its output, making the
need (and the existence) of the assumed delayed $K^{+}$ current at
least questionable. Also, a recent measurement~\cite{EnergyNeuralCommunication:2021}
concluded that the neuronal computation (contrasted with the resting
state) needs only $0.1\ W$ and neuronal communication needs only
$3.5\ W$; that is, the leaking current, at least due to the parallel
$RC$ circuit, does not exist (see our Figure~\ref{fig:Physical-processes-membrane}:
a current flows only if membrane's potential is above the resting
potential).

We know from the recent discoveries and understanding of the correct
model that currents flow into the condenser (the membrane) and are
taken out through the resistor (the 
\gls{AIS}).
 Our theoretical discussion
solidly underpins that the physical picture behind the commonly accepted
neuronal electric model must be fixed. As we discussed, unlike in
the classic model, \emph{the driving voltage and the membrane current
have a time course}. \emph{No current is shared by the resistor and
condenser, there is no input resistance, resting current of the parallel
oscillator, delayed $K^{+}$ current, and changing conductance.}

The correct equivalent circuit is a \emph{differentiator-type oscillator},
where the output voltage is given by the equation
\begin{equation}
V_{out}(t)=RC\sum\frac{dV_{in}}{dt}\label{eq:NeuronVoltageOut}
\end{equation}

\index{voltage gradient}
 The sum of all \textit{input voltage gradients} generates the \textit{output voltage}, which drives a current through the 
 %\gls{AIS}.
 AIS. The fundamental difference between the two circuit types, that in the correct circuit there is no shared current and there is no direct correction between the input and the output currents. Instead, in the dynamic picture, the changes in the input charge (the temporal course of the current)  generates a resulting voltage gradient
and their sum drives the $RC$ circuit, which (under its laws) generates the output voltage which 
drives a current (pulse) though  the 
%\gls{AIS}.
AIS
Here comes to light \textit{the biggest mistake in deriving HH's equations: the temporal course of the charge
is identical with the current only if the current is constant} such as in the case of \hyperlink{voltage-clamping}{clamping}.

As we derived, the (the measured 
\gls{AP})
 output voltage can be
described by the equation describing the serial $RC$ circuit. Our
equations enable us to calculate the ion current's time course from
the potential's time derivative. We need to sum the time derivatives
of the voltages that drive the neuronal oscillator through its membrane
(of course, considering that the current needs time to travel from
its entry point to the membrane's body) and solve the differential
equation by integrating it in time. The resulting output voltage time
derivative can be measured in front of and after the
\gls{AIS}.
Interestingly,
the time derivative was measured as early as 1939~\cite{COLE_CURTIS_IMPEDANCE:1939},
but its role has not been understood, mainly due to the wrong electric
model. \emph{The causality is reversed. The voltage gradient is the
primary entity produced by the cellular circuit, and that leads to
the production of an \gls{AP} by the neuronal oscillator.}
\index{voltage gradient}
 

\begin{figure}
\iflatexml
\includegraphics[width=\textwidth]{fig/AP_Artificial.svg}
\else
\includegraphics[width=\textwidth]{fig/AP_Artificial.pdf}
\fi

\caption{How the physical processes describe membrane's operation\label{fig:Physical-processes-membrane}. The rushed-in $Na^+$ ions instantly increase the charge on the membrane, then the membrane's capacity discharges (produces an exponentially decaying voltage derivative). The ions are created at different positions on the membrane, so they need different times to reach the 
\gls{AIS},
 where the current produces a peaking voltage derivative. The resulting voltage derivative, the sum of the two derivatives, drives the oscillator. Its integration produces the membrane potential. When the membrane potential crosses the voltage threshold value, it switches the synaptic currents off/on.
}
\end{figure}

Figure~\ref{fig:Physical-processes-membrane} shows how the described
physical processes control neuron's operation. In the middle inset,
when the membrane's surface potential increases above its threshold
potential due to three step-like excitations opens the ion channels,
$Na^{+}$ ions rush in instantly and create an exponentially decreasing,
step-like voltage derivative that charges up the membrane. The step-like
imitated synaptic inputs are resemblant to the real ones: the incoming
\gls{PSP}s produce smaller, rush-in-resemblant, voltage gradient
contributions. The charge creates a thin surface layer current that
can flow out through the \gls{AIS}.
\index{voltage gradient}
 This outward current is negative,
and proportional to the membrane potential above its resting potential.
At the beginning, the rushed-in current (and correspondingly, its
potential gradient contribution) is much higher than the current flowing
out through the 
\gls{AIS},
 so for a while the membrane's potential
(and so: the \gls{AIS}
 current) grows. When they get equal, the
 \gls{AP}
reaches its top potential value. Later the rush-in current gets exhousted
and its potential-generating power drops below that of the 
\gls{AIS}
current, the resulting potential gradient changes its sign and the
membrane potential starts to decrease.


In the previous period, the rush-in charge was stored on the membrane.
Now, when the potential gradient reverses, the driving force starts
to decrease the charge in the layer on the membrane, which per definitionem
means a reversed current; without foreign ionic stream and current
through the
\gls{AIS}.
 This is the \emph{basic difference between
the static picture }that Hodgkin and Huxley hypothesized and \emph{the dynamic one that really describes its behavior}.
The correct equivalent electric circuit of a neuron is a serial, instead of
a parallel, oscillator, and its output voltage is defined dynamically
by its voltage gradients (see Eq.(\ref{eq:NeuronVoltageOut})) instead
of static currents (as physiology erroneously assumes). In the static picture the
oscillator is only an epizodist, while in the time-aware (dynamic)
picture it is a star.

Notice also that \emph{only the resulting $\frac{dV}{dt}$ 
%(\gls{APTD})
(APTD)
disappears} with the passing time. Its two terms are connected through
the membrane potential. As long as the membrane's potential is above
the resting value, a current of variable size and sign will flow, and the output and input currents are not necessarily equal: the capacitive current changes the rules of the game.

The top inset shows how the membrane potential controls the synaptic
inputs. Given the ions from the neuronal arbor~\cite{NeuronalArborisation:2021,NeuritsArbor:2022}
can pass to the membrane using 'downhill' method, they cannot do so
if the membrane's potential is above the threshold. The upper diagram
line shows how this gating changes in the function of time.

Fig.~\ref{fig:VoltageTimeDerivative} shows how the resulting 
%\gls{APTD}
APTD
controls the output 
%\gls{AP}'s
APs
 shape: the derivative changes its
polarity by $\approx\ 500\ mV/ms$ in $\approx 0.5\ ms$, which means
across a $50\ \mu m$ 
\gls{AIS} a $20,000\ V/m$ gradient change
on the 
\gls{AIS}.
 This voltage gradient is sufficient to accelerate
the ions in the ion channels and decelerate them again; this is how
to reverse the current direction. We see the effect of 'ram current'
as \gls{AP}.
\index{voltage gradient}
Notice the broadening effect of the gradient measuring
technology. A voltage difference is measured at a distance difference,
and -- due to the signal's speed -- the time
difference is comparable in size to the period of polarity change
of the signal. 



\subsection{Neural currents\label{sec:Physiology-NeuralCurrents}}
We can subdivide currents within the neuron based on their origin, physical path and temporal behavior.
%

\subsubsection{Patching current\label{sec:Physiology-NeuralCurrentsPATCHING}}
When \hyperlink{voltage-clamping}{patching}, a current is directly introduced to the neuron's body. 
In the case of a constant current where $I=\frac{dQ}{dt}$, the voltage increase $dV$
on the capacity $C$ of the membrane is $\frac{dQ}{C}=\frac{I*dt}{C}$,
so 
\[\frac{d}{dt} V=\frac{I}{C}\]
The direct \textit{constant} current input $\frac{d}{dt} V_{PATCH}$
to the neuron cell body is a simple constant current that causes a constant membrane's voltage derivative contribution.
However, the currents are not necessarily constant. 
If the artificial current follows a math function,
\textit{the time derivative} of that function should be used.
In the case of a native current (i.e., receiving a spike form a presynaptic neuron), the received input has the form of 
%\gls{PSP},
PSP, where the time derivate can be well approximated by a steep exponential function.
One must be careful that (step-like) \textit{sudden  changes
may produce very steep spikes} (see the wave forms in Table \ref{tab:Electric_RCOscillator_Circuits} on differentiating a square wave function); furthermore, as we discuss in section \ref{sec:Physiology-ConcentrationDerivative}, the step-like
concentration change causes exactly the same change in the output voltage, only the time scale differs in a factor of $10^6$.

\subsubsection{Clamping current\label{sec:Physiology-NeuralCurrents_CLAMPING}} 
When clamping, the current is injected through an axon, by switching a clamping
voltage to the axon. Given that the current is delivered through the axon, the mechanisms described in section \ref{sec:Axonal-charge-delivery} must be considered.
%
The current at the switch ON/OFF events behaves as a step function;
that is,  it produces a saturating and a discharging current, respectively.
The switch-on effect is known also in technical electricity; in biology
its time constant is in the order of $1\ ms$,
that is drastically influences the measured biological processes, see Figure \ref{fig:ClampingOnOff_HH}.
Recall that in the case of clamping, the \textit{derivative} contains an exponential function.
In the case of patching, the  \textit{derivative} is the derivative of a (nearly) square-wave function. For a discussion of the measured result, see section~\ref{sec:Physiology-HHOscillator}.


\subsubsection{AIS current\label{sec:Physiology-NeuralCurrents_AIS}}
The \gls{AIS}  represents a non-distributed resistance $R_M$, and the current
flowing through it is

\begin{equation}
	I_{AIS} =  - \frac{V_M-V_{rest}}{R_M} \label{eq:I_AIS}
\end{equation}
(it is an outward current, so it is negative),
and its voltage time derivative is
\begin{equation}
	\frac{d}{dt} V_{AIS}(t) = - \frac{V_M(t)-V_{rest}}{C_M R_M}  \label{eq:dVdt_AIS}
	\end{equation}

Notice that this current depends on $C_M R_M$, all others on $C_M$.
(This current was mis-identified by HH as 'leaking current':
if no other current/voltage derivative is present, the membrane discharges. In resting state the derivative is zero: 
the condenser is charged up and no leaking current flows.
%See  also section \ref{subsec:EnergyConsumption} on the
%energy consumption of neurons and the brain.)



\begin{equation}
	V_{a}(t)=V_{o}*(1-exp(-a*t))*exp(-b*t)\label{eq:PSPVoltage}
\end{equation}


\subsubsection{Synaptic and rushed-in current\label{sec:Physiology-NeuralCurrents_PSP}}
In the case of those currents, as we discussed in the cases of
membrane
and axon,  a saturation-type function
multiplied by a decay-type function describes the current, so the voltage derivative is
%
\begin{equation}
	\frac{dV_{a}}{dt}=\frac{1}{\alpha}*exp(-\frac{1}{\alpha}*t-\frac{1}{\beta}*t)-\frac{1}{\beta}*exp(-\frac{1}{\beta}*t)*exp(1-exp(-\frac{1}{\alpha}*t))\label{eq:PSPderivative}
\end{equation}

%\begin{equation}
%	\frac{d}{dt}V_{SYN}(t)=a*exp(-a*t-b*t)-b*exp(-bt)*exp(1-exp(-a*t))\label{eq:PSPderivative}
%	\end{equation}
The same voltage derivate (with different parameters $a$ and  $b$) is valid
for $\frac{d}{dt}V_{M}(t)$ due to the membrane rush-in current (as discussed above, the voltage derivate is proportional
to the current through a factor $1/C_{M}$). See also Figure~\ref{fig:ArtificialCurrent_AP}.




\subsubsection{Native case\label{sec:Physiology-NeuralCurrents_NATIVE}}
In the native case (the membrane's voltage created instantly and then no external invasion happens),
the resulting voltage derivative is
\begin{equation}
	\frac{d}{dt}V_{OUT}(t) = \frac{d}{dt}V_{M}(t)+\frac{d}{dt} V_{AIS}(t) \label{eq:Simple}
	\end{equation}
Figure \ref{fig:PSP_Derivative} shows the functional forms of $V_M(t)$ and $\frac{d}{dt}V_{M}(t)$ (\textit{
%\gls{PSP}
PSP current} and its \textit{voltage derivative})
at some reasonable parameter values $a$ and  $b$).
(Notice that the front of an arriving spike, as well as at the beginning of \hyperlink{voltage-clamping}{clamping},
the front is almost clearly exponential.)
Notice the sudden change of the derivative after the output (spike delivery) begins: the exponential increase of $V_M(t)$
really causes a steep change in its derivative at low time values.
For different values of parameters $a$ and $b$,
a  variety of function shapes
describing %\gls{AP}s
APs can be created, see Figure~\ref{fig:AP_Variety} and also Figure~\ref{fig:ArtificialCurrent_AP}.


\begin{figure*}
\iflatexml
\includegraphics[width=\textwidth]{fig/AP_Variety.svg}
\else
\includegraphics[width=\textwidth]{fig/AP_Variety.pdf}
\fi
	\caption{The shape of the %\gls{AP}
        AP as the result of integrating the differential
		Equation~(\ref{eq:Simple}), at different input and output
		currents and timing constant\label{fig:AP_Variety}}
\end{figure*}



\subsubsection{Complex case\label{Physiology-DIFFERENTIATOR_COMPLEX}}
In the most complex case, the time derivative of voltage we need to work with is
\begin{equation}
	\frac{d}{dt}V_{OUT}(t) = \frac{d}{dt} V_{AIS}(t) + \frac{d}{dt}V_{M}(t)+ \sum_i \frac{d}{dt} V_{SYN,i}(t)+\frac{d}{dt} V_{ARTIF}(t) \label{eq:Complex}
\end{equation}

The first term is always present. The second term only if previously exceeding
the threshold value caused by membrane's charge-up (an instant effect).
The third term changes during the stages of operation, as we describe below.
The last term is an "artificial" contribution (and so: it depends on experimental settings),
but it is frequently used in experimental research.
Notice that whether \hyperlink{voltage-clamping}{voltage or current clamping} is applied,
it only means what the experimenter keeps constant;
it acts with its \textit{voltage derivative}. The same
holds for the mathematical form of the used current/voltage.

The equation enables us to understand the experience that the shape of the 
%\gls{AP}
AP is always the same. More precisely, the integrals of the contributing $\frac{d}{dt}V(t)$ terms remain the same. Furthermore, if the contributors remain the same, the resulting shape also remains the same. Of course, only in steady state. It changes, if the next spike arrives before the resting potential restored, or synaptic input arrives when synaptic inputs are enabled, or the artificial current changes.



\subsubsection{Currents in different stages\label{MODELING_SINGLE_ELECTRIC_STAGES}}

The neuron's electric operation comprises several stages, and the different physical phenomena
produce different currents in those stages.
The stages of neuronal
operation, and the presence of slow and fast currents, furthermore
the gating mechanisms significantly shade the picture.

As we introduced, the ion currents are 'slow' if they arrive through the axon
(as~\cite{HodgkinHuxley:1952} measured, an apparent 'delay'
can be observed between the voltage and the current).

The 'artificial' contributions $\frac{d}{dt} V_{CLAMP}$ and $\frac{d}{dt} V_{PATCH}$,
of course, depend only on the investigators and no additional (stage-dependent) rule is followed (although the delay may apply).

The contribution $\frac{d}{dt} V_{AIS}(t)$ is always on; the neuron all the time,
independently from its history, operating stage and its inputs, attempts to restore its resting potential.
The $I_{AIS}$ is active all the time, active all the time. However, it is \textit{not} a "leaking current". It is proportional to the difference
of the \textit{membrane's potential above the resting potential}. In resting state, its value is zero, see  The mechanism in \textit{resting state} is different.



The contribution $\frac{d}{dt} V_{M}(t)$, once 'DeliveringBegin' issued, will not be stopped
(except 'Synchronize') until the membrane voltage drops below the threshold value.
If the artificial currents are too high (see \ref{fig:ArtificialCurrent_AP}), the stage 'Delivering' may last forever.

The contributions $\frac{d}{dt} V_{SYN,i}(t)$ are only enabled
when the membrane's voltage is below the threshold level.
The amplitude of the current/voltage derivative depends
on the membrane's voltage.
The synaptic inputs $I_{SYN,i}$  are  active only in the charge-up
and 'relative refractory' period. Actually, when \textit{the membrane potential
is kept above the threshold value}: the ions cannot enter the intracellular space
against the higher membrane potential: the 'normal' inputs can be blocked \cite{DepolarizationBianchi:2012}. See also Figure~\ref{fig:AP_Conceptual}.


\subsection{Charge conservation\label{sec:Physiology-ChargeConservation}}
In our model, an intense ion current generator with step-like behavior
represents the membrane, and a less intense negative current generator (drain)
represents 
%\gls{AIS}.
AIS. We return to the case we describe in connection
with 
%\gls{PSP},
PSP, see Eq.~(\ref{eq:PSPVoltage}), with a crucial difference.
The flow-in and the flow-out points are at a distance and a ``slow''
ion current must flow between them. If the current travels to a fixed
distance with a fixed speed, as we discuss in connection with Equ.~(\ref{eq:CoulombTimeDependent}),
we expect that the output current appears with a delay compared to
the input current. We assume that the charge conserves, i.e., the
input current equals the output current. That means, for a one-dimensional membrane,
\emph{we shall write Kirchoff's Junction Law in the form}
\begin{equation}
	I_{out}(t)=-I_{in}(t\mathbf{-\Delta t})\label{eq:Kirchoff_biological}
\end{equation}

\noindent instead of the usual form, without delay. In Fig.~1 in~\cite{ActionPotentialGenerationNatrium:2008} one can see that the
current travels with speed less than 1~cm/s toward 
%\gls{AIS},
AIS, and
that Kirchoff's Law is valid only in the form given by Eq.~(\ref{eq:Kirchoff_biological}).
It is, essentially, what the telegraph equation expresses for technical
computing: \emph{the macroscopic current has finite speed}. 

We assume that the input current due to the rushed-in ions is similar
to the one we derived in connection with 
%\gls{PSP},
PSP see Equ.~(\ref{eq:PSPVoltage}).
That is, we expect that the resulting (net) current is a ``ghost''
image shown in Fig.~\ref{fig:The-ghost-image_AP}, which can be interpreted
as a kind of interference (a difference between a positive current and
shifted negative current) between the input and output currents, and
expresses \emph{Kirchoff's Junction Law for ``slow'' current in a neuron}.
For the figure, we assumed $\Delta t=0.49\ ms$, and the parameters
used to generate the function displayed in Fig.~\ref{fig:Post-synaptic_potential_lin}.
The negative output current has been observed and measured by~\cite{HodgkinHuxley:1952},
see their Fig.~18, but \textendash{} due to the lack of the idea
of ``slow current'', furthermore using the mistakenly measured empirical
dependencies of ``conductivity'' \textendash{} it has been identified
as outward $K^+$ current. 
A high sodium channel density is present in the 
%\gls{AIS}
AIS~\cite{ActionPotentialGenerationNatrium:2008}
to form
% \gls{AP},
AP, and imaging ions show (see their Fig.~3f in~\cite{ActionPotentialGenerationNatrium:2008})
that $Na^+$ ions arrive at it. Hypothesizing $K^+$ in all cases leads to some discrepancy see for example, 'It is counter-
intuitive that removing a potassium conductance would
decrease the excitability of a neuron' ~\cite{BeanActionPotential:2007}, and 

In our model, the membrane acts as a \emph{voltage generator} with
the time course described by Eq.~(\ref{eq:PSPderivative}), 
with the appropriate coefficients. This change is a drastic departure from the classic picture using a \emph{current generator},
where a ``fast'' current flows through the resistor, generating a voltage 
that charges the capacitor. Initially, the capacitor is empty, so it
will temporarily store the charge; that charge produces the 'damping' contribution later but cannot explain a negative contribution. To explain the experienced hyperpolarization, a $K^+$ current in the opposite direction must be assumed, although it has no source charge in the membrane and no experimental proof underpins its existence \textit{during evoking an 
%\gls{AP}};
AP}; see~\cite{BeanActionPotential:2007}. 

In our
modern picture, the initial ion inflow saturates, and the relatively
low-intensity slow current removes the charges from the membrane's surface.
The membrane attempts to remain equipotential despite the experienced
current drain, but the slow current needs time to reach the 
%\gls{AIS}.
AIS.
The interplay of the finite-speed current flowing on the finite-size
surface and the voltage-dependent exponential outflow shape the
% \gls{AP}. 
AP.
\emph{There is no need to assume the inflow or outflow of specific
	currents and the change of ion type. The extended size of the membrane,
	accompanied by slow ion propagation, entirely explains why the
	spikes are issued and provides its parameters. Similarly, no control
	mechanism is needed: biology takes advantage of the slow ion propagation
	speed.} (The function displayed should be convolved with a function
considering the distribution of distances between the input
and output points; i.e., consider an actual membrane shape).




To describe how the neuron's membrane forms an 
%\gls{AP},
AP, we consider
that the membrane becomes highly charged (i.e., will have a considerable
potential) after opening its ion channels. That potential difference
will drive a macroscopic current toward the 
%\gls{AIS},
AIS, where a macroscopic
current flows out, as described in~\cite{ActionPotentialGenerationNatrium:2008}.
The mathematical formalism is the same as in the case of 
%\gls{PSP},
PSP,
see Eq.~(\ref{eq:PSPVoltage}), except that the current inflow is
more intense, given that the membrane's surface is much larger. Although
the 
%\gls{AIS}
AIS is much smaller, its much higher ion channel density~\cite{ActionPotentialGenerationNatrium:2008}
enables it to forward that intense "longitudinal"
current toward the axon (where it is transmitted as a "transversal
current"; see good textbooks and our discussion).
As we discussed, an 
%\gls{AP}
AP can be described by three parameters: how the rush-in current rises (a function of the area and ion channel density), how the rushed-in charge
can flow out (including how long the current path is), and
the parameters of the neuronal $RC$ circuit.


That current on the condenser with capacity $C_{m}$, alone, would
produce a voltage change $\frac{dV_{chargeup}}{dt}$: it is the input side of the circuit. The
membrane (in cooperation with the 
%\gls{AIS})
AIS) behaves as a differentiator $RC$ circuit. It will significantly change
the form of the voltage's time course on its output side (as discussed in section~\ref{sec:Physics-SlowCurrent},
one can imitate the effects of a ``slow'' current flowing on a system
with distributed parameters using equations created for discrete parameter
case). The membrane potential produces a current that discharges the
condenser, decreasing the potential generated by the membrane's current. 
%at the point in front of the %\gls{AIS} the differential form of Kirchoff's equation describes the resulting
%change of voltage
%simple deliver
%%
%%\begin{equation}
%%	\frac{dV_{AIS}}{dt}=\frac{dV_{chargeup}}{dt}-\frac{V_{AIS}-V_{rest}}{RC}\label{eq:Kirchoff's membrane}
%%\end{equation}
%and the action potential is of form
%
%\begin{equation}
%	V_{AP}(t)=\int_{0}^{t}dt\Big(\frac{dV_{chargeup}}{dt}-\frac{V_{AIS}-V_{rest}}{RC}\Big)\label{eq:Membrane_PDE}
%\end{equation}

Our model hypothesizes that the current due to the rushed-in ions maintains the time course of the \textit{voltage derivative} in the input side, see Eq.(\ref{eq:Simple}).
We shall solve the equation numerically to receive the \textit{output voltage}, the %\gls{AP}.
AP.
The shape of the voltage due to the
slow current on the membrane, described by Eq.~(\ref{eq:PSPVoltage})
and its derivative, described by Eq.~(\ref{eq:PSPderivative}), are
depicted in Fig.~\ref{fig:PotentialDerivative}; with the parameters
concluded from Fig. \ref{fig:MasonFig4}. The formalism and model
are the same also in the case of a membrane; only the coefficients
are different.

 
\begin{figure}
\iflatexml
\includegraphics[width=\textwidth]{fig/PSP.svg}
\else
\includegraphics[width=\textwidth]{fig/PSP.pdf}
\fi
        \caption{The rush-in (and post-synaptic potential) and its derivative, as provided by Eqs.
		(\ref{eq:PSPVoltage}) and (\ref{eq:PSPderivative})\label{fig:PotentialDerivative}.
                The %\gls{PSP}
                PSP diagram line was fitted to data measured by \cite{SynapticTransmissionMason:1991}
	\label{fig:PSP_Derivative}}
\end{figure}


By varying those parameters, a variety of %\gls{AP}
AP shapes can be described 
using the same model, see Fig.~\ref{fig:AP_Variety}.
The various colors and line types demonstrate the influence of parameter
values on the calculated shape of the %\gls{AP}.
AP. We based our calculations
on a resting potential of -65 mV and a threshold offset potential
of +20 mV. The red lines represent 
%\gls{AP}
AP for current intensities similar
to those used to generate the diagram line in Fig.~\ref{fig:Post-synaptic_potential_lin}.
The green and blue lines depict the 
%\gls{AP}
AP for higher and lower
intensity currents, respectively. The continuous lines show the 
%\gls{AP}
AP
for a neuronal oscillator with capacity $C$ we used to generate Fig.~\ref{fig:Post-synaptic_potential_lin}. The dotted and dashed lines represent circuits with higher and lower time constant
values, respectively.  Our research suggests that assuming a \textit{differentiator-type} $RC$
circuit for the neuronal membrane can imitate the effects of the ``slow''
current's temporal behavior, see Fig.~\ref{fig:The-ghost-image_AP}.
As we discussed in section~\ref{sec:Physics-SlowCurrent},
the time constant $RC$ drastically influences the resemblance of the
%(\gls{PSP}-like)
(PSP-like)
 input function shape
and the
%(\gls{AP}-like)
(AP-like) output shape. Furthermore, the higher the charge-up current compared to the fixed-value output current (defined by $R$), the more
resemblant the output voltage shape and the empirical 
%\gls{AP}.
AP.


In the picture, we suggest here, the voltage of the membrane increases
enormously (described in a physically plausible way and with a mathematically
and physically correct time-dependent function), observed as large
transient local voltages~\cite{TransientResponses:2008,KochElectricalPropertiesSpike:1983}.
The membrane, acting as a semipermeable insulator surface, hosts charge
carriers that distribute on it, capable of reaching other areas with
finite surface speed. Consequently, the 
%\gls{AIS}
AIS will experience
only a marginal increase upon receiving an axonal input, and only with a delay. A temporally
distributed charge packet is the sole factor that evokes
the observed voltage increase, without any other assumed in- and outflow
of ions. This novel approach to understanding the initiation of 
%\gls{AP}s
APs sets our model apart.

\subsection{Voltage time derivative\label{sec:Physiology-VoltageTimeDerivative}}

%
The time derivative was measured already in 1939~\cite{COLE_CURTIS_IMPEDANCE:1939}, see Fig. \ref{fig:MeasuringAPTimeDerivative}. However, its role has not yet been recognized. Interestingly, Cole and Curtis derived the 
%\gsl{AP}
AP by integrating the experimentally derived $\frac{d}{dt} V$, essentially in the same way as we do. (they also discussed the widening/smearing effect of the measuring technology)

\begin{figure}
\includegraphics[width=\textwidth]{fig/ColeFirstDerivative.jpg}
	\caption{Measuring the time derivative in~\cite{COLE_CURTIS_IMPEDANCE:1939}, from 1939\label{fig:MeasuringAPTimeDerivative}}
\end{figure}

The shape and different parameters of 
%\gls{AP}
AP has been the subject of numerous studies. For example, \cite{BeanActionPotential:2007} measured 
%\gls{AP},
AP, simultaneously with its time derivative of the 
%\gls{APTD},
APTD for a wealth of neuron types. Those measurements provide a direct proof for the existence of the 
%\gls{APTD}
APTD our theoretical approach introduced, see equations (\ref{eq:PSPderivative}), (\ref{eq:Simple}), (\ref{eq:PSPVoltage}) and Figs.~\ref{fig:VoltageTimeDerivative} and~\ref{fig:PotentialDerivative}. However, notice that the causality is reversed. The current inflow through the neuron's membrane generates 
%\gls{APTD}
APTD,  and its time course generates 
%\gls{AP}
AP through the $RC$ circuit.
Notice that the theoretical APTD is much sharper than
the experimental one. Actually, the latter value is a differentia ratio (instead of differential quotient) from 
measured values. The measuring electrodes's size defines
the position (and time) difference. In the case of measuring a signal with very sharp form the two quotients differ significantly.

\begin{figure}
\iflatexml
\includegraphics[width=\textwidth]{fig/Bean_Fig1d.svg}
\else
\includegraphics[width=\textwidth]{fig/Bean_Fig1d.pdf}
\fi
	\caption{The simultaneously measured %\gls{AP}
	AP and %\gls{APTD}.
	APTD Our theoretically derived 
	%\gls{AP}
	AP and 
	%\gls{APTD}
	APTD are overlayed to Fig.~2d in\cite{BeanActionPotential:2007}. See also our Figure \ref{fig:ArtificialCurrent_AP}.	 "\copyright{2007 Nature Publishing Group}. Bean, B. P. (2007): The action potential in mammalian central neurons. Nature Reviews Neuroscience, 8(6), 451-465. doi:10.1038/nrn2148".
\label{fig:VoltageTimeDerivative}
 }
\end{figure}


Fig.~\ref{fig:AP_dV_V_loop} shows how the rushed-in ions produce the 
\gls{AP}. The blue diagram line describes how the AP depends on the voltage gradient.
The life begins at coordinates (0,0), in the tiny orage circle, with
exciting the neuron. 
\index{voltage gradient}
The synaptic excitation is pulse like, so the
voltage simply rises without a gradient being generated.
When the membrane's voltage threshold reached, the potential does not change,
but the gradient jumps due to the (instant) appearance of the runshed-in ions,
see the rightmost point. 
From this point on, the potential increases while the gradient decreases;
the diagram line proceeds from right to left.
The potential increases as the ions entering the surface layer at some
point farther from the \gls{AIS}: to travel such a distance, needs time.
The highest point reached when the ions from the largest distance could
reach the AIS. (The influx was instant, the charge will continuously decrease
due to the current through the AIS.
The change in the layer behaves as a high-viscosity charged fluid.
The electric and thermodinamic driving forces propagate with enormously
different speed, and the fluid must be contiguous; so the voltage temporally increases before the AIS
(ram current or impact current), and the fluid turns back which means that the ion current changes its direction (condenser effect).
Due to the decreasing current, the voltage starts to decrease, although 
the voltage gradient is still positive (it is the contribution of the
rush-in current, only). The negative current continues and turns first
the gradient to negative whihc turns the potential to negative
(a state known as 'hyperpolarization'), and finally the relaxed current
takes but the potential and the gradient back to zero (the resting potential).

The green dashed diagram line shows the AP in the function of the voltage
measurable on the AIS (the difference of the gradients of the rush-in
and the outflow gradients). 
This diagram line also starts from the orange circle, but since the excitation
increases the potential that generates current (and so potential gradient) 
on the AIS, the gradient goes to negative as the potential increases.
When the potential reaches the threshold value, the gradient jumps to
its maximal positive value and the potential increase at a decreasing pace.
It reaches its maximum value when the resulting gradient reaches zero. 
The negative gradient turns the potential even to negative (hyperpolarization)
while approaching its zero driving force. 
(the outflow current decreases the gradient, so, the parabola gets asymmetric). 
The oscillator comprises one turning point; it is perfectly damped (to provide the fastest operating speed).

\begin{figure}
\iflatexml
\includegraphics[width=\textwidth]{fig/AP_dV_V_loop.svg}
\else
\includegraphics[width=\textwidth]{fig/AP_dV_V_loop.pdf}
\fi
	\caption{How the voltage gradient created by the rush-in current
	drives the Action Potential.
\index{voltage gradient}
\label{fig:AP_dV_V_loop}
 }
\end{figure}

\subsection{Synaptic control\label{sec:Physiology-SynapticControl}}

When researching the electric operation,
'foreign' currents and voltages (as opposed to the 'native' ones) are applied. Our model can describe the effects
of such artificial invasions. The synaptic control can be best understood
on the example of modeling constant external current.
As we  discussed, a constant current (in 'steady-state', after the transients relaxed)
can be modelled as adding a constant $\frac{d}{dt}V$ term to the sum of the voltage time derivatives
directing the neuron's \gls{AP}.
Figure \ref{fig:ArtificialCurrent_AP} calls attention
to some important consequences of applying artificial currents and shows how our model handles them.
(Notice that the subfigures share the logarithmic time scale
the begins at the arrival of the first synaptic input.
For the sake of simplicity we use an arbitrary voltage scale,
and imitate synaptic inputs with an instant membrane voltage step).
For understanding the terms and notions, see also Figures \ref{fig:AP_Conceptual}
and \ref{fig:ArtificialCurrent_GTKWave}.


%
The bottom subfigure displays the action potential observable on the \gls{AIS}.
The '0' case is a simple delivering (see section \ref{sec:Physiology-NeuralCurrents_NATIVE}),
when no external invasion is present. The 
\gls{AP} is as experimentally observed:
in the 'Relaxing' stage the neuron receives three synaptic inputs.
When the first input arrives, the neuron passes to stage 'Computing'.
For the effect of the third input, the neuron membrane exceeds the threshold voltage
and the neuron passes to stage 'Delivering'. First the rushed-in ions increase
the membrane's potential, then the 
\gls{AIS}
 decreases it to its resting value.
As we discussed, the synaptic inputs are disabled in the 'Delivering' stage.
The top subfigure shows how the synaptic inputs are enabled/disabled
during generating the 
\gls{AP} when the threshold level crossed. The synapses
are OFF only during the 'Delivering' stage (conventionally considered as the
'absolute refractory' period). For the sake of simplicity, for this figure we assumed an
instant re-enabling, that is, that after crossing the threshold potential value,
the re-enabled synaptic inputs appear at the 
\gls{AIS} without delay.
The background also displays the voltage caused by the $Na^+$ ions.
The voltage scale is arbitrary, but the time scale is true:
the neuronal condenser "stores" the ions and the stored current
will reverse its direction when the rush-in current reaches its peak value;
the reverse current appears as a negative current (decreases membrane's potential
below the value of its resting potential). 

If the external invasion is relatively small (codename '2'), the stages are reached at
different 'local time' values compared to the case without external invasion. The stage 'Delivering' begins practically at the same time, but the polarization and hyperpolarization peak voltages
are remarkable higher for the '2' case). Notice that the synaptic inhibition time
is considerably longer: the extra charge extends the time until the
membrane's potential can decrease below the threshold level.
If the invasion is stronger (codename '4'), the hyperpolarization still exists,
but the membrane's voltage decreases only for a short period below the
threshold: the external input will increase the voltage above the
threshold again. Notice that (on the top subfigure) the synaptic inputs
are re-enabled at a much later time, and they remain enabled only
until the membrane's current exceeds the threshold again
(actually, the synaptic inputs can approach an ill-defined state).
At a slightly higher invasion current, the membrane's voltage cannot
decrease below the threshold: due to that 'foreign' current,
the synaptic inputs get 'forever' disabled ('blocked'). The experimental evidence
was published in~\cite{DepolarizationBianchi:2012}; also displaying that some
protection exists in neurons against 'overloading'.


\begin{figure}
\iflatexml
\includegraphics[width=\textwidth]{fig/AP_Artificial.svg}
\else
\includegraphics[width=\textwidth]{fig/AP_Artificial.pdf}
\fi
	\caption{The summary of  
	\gls{AP} AP generation.
	The voltage time derivatives (the midle subfigure), resulting by summing the $\frac{d}{dt}V_{AIS}$, $\frac{d}{dt}V_{M}$ and a constant corresponding to the clamping current, that define the membrane's output voltage (the 
	\gls{AP},
	the bottom subfigure), that control synaptic contributions (top subfigure). See also Fig.\ref{fig:ArtificialCurrent_GTKWave}}
	\label{fig:ArtificialCurrent_AP}
\end{figure}

