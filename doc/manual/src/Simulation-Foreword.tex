% The foreword for simulation

\begin{advanced}
"I don't understand the things that I cannot create" R.P.Feynman
\end{advanced}


	The many enormous differences between technical and biological computing make simulation of neural operation  a real challenge.
A fundamental issue is that the biological 
time of the events is not directly proportional to the
computer processing time. Given that the processing time
comprises computing time plus transfer time, time-stamping
cannot provide a solution for time handling. The time stamp records when an event happens in the computing process instead
of its biological time. Given that several neurons
share the computing resources and the computer's execution is sequential,
the events happening at a biologically identical time will generate time stamps at technically different times.
Furthermore, from the same reason, the the generated event
will be considered again at different times with a (technically random) delay compared to the time of generating those events and to each other.
Furthermore, the propagation time through the axons
cannot be included. 
These effects are late consequences of omitting the transfer
time in computing science. 

The goal is to "create" neurons and their systems; in spirit of Feynman. In some sense, it is a "duck test". We simulate a duck and test it it looks like, swims and quacks
We demonstrate that our abstract model passes the \href{https://en.wikipedia.org/wiki/Duck_test}{duck test}
"If it looks like a duck, swims like a duck, and quacks like a duck,
then it probably \textit{is} a duck".
See also Feynman's opinion on understanding. 
\index{duck model}
