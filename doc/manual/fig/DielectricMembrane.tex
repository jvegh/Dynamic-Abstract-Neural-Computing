\documentclass[border= 5mm]{standalone}
\usepackage{tikz}
\usepackage{pgfplots}
\pgfplotsset{compat=1.15}
\usepackage{animate}
\usepackage{adjustbox}
\usetikzlibrary{
	shapes,
	positioning,
	intersections,
	fillbetween,
	arrows,
	fit,
	backgrounds,
	calc,
	tikzmark,
	arrows.meta,
	shadows,
	automata}
\pgfplotsset{compat=1.10} 
\usepackage{pgfplotstable}
   \definecolor{webgreen}{rgb}{0,.5,0}
\definecolor{webblue}{rgb}{0,0,.8}
\definecolor{webred}{rgb}{0.8, 0, 0}   
\definecolor{webbrown}{rgb}{.6,0,0}
\definecolor{webyellow}{rgb}{0.98,0.92,0.73}
\definecolor{webgray}{rgb}{.753,.753,.753}



\begin{document}
	\def\EGapTotal{0.44}
	\def\MembranePos{-5.}
	\def\DeltaZ{0.5}
	\def\z{(0.2*(\MembranePos-\x))}
	\def\zz{(0.2*\x)}
	\def\DielectricField{-0.25*\EGapTotal*(
		ln(\z/(\z+1)))
	}
	\def\MinusDielectricField{-0.25*\EGapTotal*(
	ln(\zz/(\zz+1)))
}
	{\maxsizebox{.5\textwidth}{!}
		{
			
\begin{tikzpicture}%[scale=1.5]
	\begin{axis}[
		%  axis y line*=left,
		%			title={\huge Membrane potential dependence on distance},
		width=\textwidth,
		xlabel=$\large Distance\ from\ membrane\ (10^{-9}\ m)$,
		ylabel=$\large Electric\ field\ (10^7\ V/m)$,
		ymin=-0.45, ymax=0.45,
		xmin=-9, xmax = 7,
		%			ymode=log,
		%			log basis y=10,
		legend style={
			cells={anchor=west},
			legend pos={north east},
		},
		]

% Charged layer
\addplot[
webred,
no marks, mark=*, mark size=2, very thick, forget plot,
] plot coordinates {
	(\MembranePos-\DeltaZ,.6*\EGapTotal)
	(\MembranePos,\EGapTotal)
};


\addplot[
webred,
no marks, mark=*, mark size=2, very thick,
] plot coordinates {
	(\DeltaZ,-.6*\EGapTotal)
	(0,-\EGapTotal)
};


\addplot[
webred, dashed, very thick,
no marks, mark=*, mark size=2, ultra thick, forget plot,
] plot coordinates {
	(\MembranePos+0.0,\EGapTotal)
	(\MembranePos+0.05,-\EGapTotal)
};

\addlegendentry{Ion layer field}

\addplot[samples=100,domain=-15:\MembranePos-\DeltaZ,webred, dotted, ultra thick, forget plot]
{\DielectricField} ;

\addplot[samples=100,domain=\DeltaZ:10,webred, dotted, ultra thick]
{\DielectricField} ;
\addlegendentry{Dielectric field}

%\addlegendentry{Charged layers}

\addplot[
webred, dashed, thick,
no marks, mark=*, mark size=2, very thick,
] plot coordinates {
	(\MembranePos,-\EGapTotal)
	(0,-\EGapTotal)
};
\addlegendentry{Membrane gap field}

%Draw membrane
\addplot[
webgreen,
no marks, mark=*, mark size=2, thin, forget plot
] plot coordinates {
	(\MembranePos,-0.5)
	(\MembranePos,-0.1)
};
\addplot[
webgreen,
no marks, mark=*, mark size=2, thin, forget plot
] plot coordinates {
	(\MembranePos,0.5)
	(\MembranePos,0.1)
};

%\addplot[green!25] fill between[of=ChargedA and MembraneA];
\addplot[
webgreen,
no marks, mark=*, mark size=2, thin, forget plot
] plot coordinates {
	(0,-0.5)
	(0,-0.1)
};
\addplot[
webgreen,
no marks, mark=*, mark size=2, thin%very thick%, forget plot
] plot coordinates {
	(0,0.1)
	(0,0.5)
};
\addlegendentry{Membrane surface}

\addplot[
webblue, dotted, very thick,
no marks, mark=*, mark size=2, ultra thick, forget plot
] plot coordinates {
	(-9,0)
	(\MembranePos,0)
};

\addplot[
webblue, dashed, very thick,
no marks, mark=*, mark size=2, ultra thick, forget plot,
] plot coordinates {
	(\MembranePos-0.1,0)
	(\MembranePos,+\EGapTotal/4)
};

\addplot[
webblue, dashed, very thick,
no marks, mark=*, mark size=2, ultra thick, forget plot,
] plot coordinates {
	(\MembranePos,\EGapTotal/4)
	(\MembranePos+0.15,-\EGapTotal/4)
};

\addplot[
webblue, dashed, thick,
no marks, mark=*, mark size=2, ultra thick, forget plot,
] plot coordinates {
	(\MembranePos,-\EGapTotal/4)
	(0,-\EGapTotal/4)
};

\addplot[
webblue, dotted, very thick,
no marks, mark=*, mark size=2, ultra thick, forget plot,
] plot coordinates {
	(0,0)
	(-0.05,-\EGapTotal/4)
};

\addplot[
webblue, dotted, very thick,
no marks, mark=*, mark size=2, ultra thick,
] plot coordinates {
	(0,0)
	(7,0)
};

\addlegendentry{Classic field}

% Now draw the ion channel
\addplot[
orange, thick,
no marks, mark=*, mark size=2, thin
] plot coordinates {
	(-5,0.1)
	(0,0.1)
};
\addplot[
orange, thick,
no marks, mark=*, mark size=2, thin, forget plot
] plot coordinates {
	(-5,-0.1)
	(0,-0.1)
};
\addlegendentry{Ion channel}
% Add a cap
\addplot[
webblue, %thick,
no marks, mark=*, mark size=2, very thick%, forget plot
] plot coordinates {
	(0,0.1)
	(1.8,-0.06)
};
\addlegendentry{Cap}
	% Draw Delta z			
\addplot[
webgreen,dashed,
no marks, mark=*, mark size=2, very thick, forget plot
] plot coordinates {
	(\MembranePos-\DeltaZ,0.3)
	(\MembranePos-\DeltaZ,0.5)
};

% Draw Delta z			
\addplot[
webgreen,dashed,
no marks, mark=*, mark size=2, very thick%, forget plot
] plot coordinates {
	(+\DeltaZ,-0.3)
	(+\DeltaZ,-0.5)
};
\addlegendentry{Ion layer thickness}


% Add a 'fast' ion 			
\addplot[
webred, 
 mark=*, mark size=3, 
] plot coordinates {
		(-5.15,0.40)
	};

\addplot[
webred, 
mark=*, mark size=3, 
] plot coordinates {
	(0.15,-0.40)
};
\addplot[
webred, 
mark=o, mark size=3, 
] plot coordinates {
	(-5.15,0)
};

% Add a 'slow' ion 			
\addplot[
webgreen, 
mark=*, mark size=3, 
] plot coordinates {
	(-8,0.11)
};


\addplot[
webred, 
mark=o, mark size=3, 
] plot coordinates {
	(0.15,0)
};
\addplot[
webgreen, 
mark=o, mark size=3, 
] plot coordinates {
	(-8,00)
};
\addplot[
webgreen, 
mark=*, mark size=3, 
] plot coordinates {
	(-7,.135)
};
\addplot[
webgreen, 
mark=o, mark size=3, 
] plot coordinates {
	(-7,00)
};

      \node[anchor=west, rectangle, rounded corners,fill=webred!20] (after) at (axis cs:-3.4,0.2){$E_{accelerate}$};
\node (VoltageAA) at (axis cs:-5.15,0.40){};
\node (VoltageBB) at (axis cs:.15,-0.4){};
\draw[->,dashed,  webgreen](after)--(VoltageAA);
\draw[->,dashed,  webgreen](after)--(VoltageBB);
\draw[->,dashed,  webgreen](after)--(axis cs:-5.15,0);
\draw[->,dashed,  webgreen](after)--(axis cs:0.15,0);

      \node[anchor=west, rectangle, rounded corners,fill=webred!20] (after) at (axis cs:-8.5,-.1){$E_{assist}$};
\node (VoltageAB) at (axis cs:-8,0.115){};
\node (VoltageAC) at (axis cs:-7,0.14){};
\draw[->,dashed,  webgreen](after)--(VoltageAB);
\draw[->,dashed,  webgreen](after)--(VoltageAC);
\draw[->,dashed,  webgreen](after)--(axis cs:-8,0);
\draw[->,dashed,  webgreen](after)--(axis cs:-7,0);
% Move arrows of ions
\draw[->,  webgreen,  ultra thick](axis cs:-8,0)--(axis cs:-7,0.);
\draw[->,  webred,  ultra thick](axis cs:-5.15,0)--(axis cs:0.15,0.);

			\end{axis}
			\end{tikzpicture}
		}
	}
	
		
\end{document}

%			\newcommand\ConcentrationmM[1]{\pgfmathparse{#1}}
%			\newcommand\Deltaznm[1]{\pgfmathparse{#1}}
%			% Concentration, in mM
%			% Avogadro_constant (N_A) 6.02214076e23 [1/Mol]
%			% Avogadro_constant (N_A) 6.02214076e20 [1/mMol]
%			% Electron_charge, 1.602176634e−19 [Cb]
%			% FaradayConstant F 9.64853321233100184e4
%			% Assumed width of the physical layer, for calculating \sigma, nanometer
%			% Permittivity
%			\def\Epsilon_0{8.854187813e-12}
%			% Volumetric concentration
%			% c * Avogadro_Constant * Electron_charge
%			% Volumetric charge density in \frac{C * mM }{ m^3}
%			\newcommand\SigmaVmMperCubic[1]{
	%				#1*6.02214076*1.602176634*10}
%%\pgfmathparse{\SigmaVmMperCubic{1}}\pgfmathresult
%		% Return surface charge density [mM,nm] in \frac{C * mM }{ m^2}
%			\newcommand\SigmaAmMPerSquare[1]{#1*6.02214076*1.602176634*10}
%\pgfmathparse{\SigmaAmMPerSquare{1,}\pgfmathresult					\def\MembraneBulkPotential{5}
	%			\def\MembraneFactor{1}
	%			\def\MembraneDistance{x}
	%			\def\MembraneWidth{5}
	%			\def\AtomWidth{0.1}
	%			\def\MembraneOffset{0}
	%			\def\MembraneGradient{\MembraneBulkPotential*(ln(\MembraneDistance/(\MembraneDistance+\MembraneWidth)))}
	%			\def\MembranePotential{0.1}
	%			%*\MembraneBulkPotential}
%				%*(ln(\MembraneDistance/(\MembraneDistance+\MembraneWidth)))/(abs(\MembraneDistance+\MembraneWidth))}


\begin{tikzpicture}%[scale=1.5]
\begin{axis}[
	%  axis y line*=left,
	%			title={\huge Membrane potential dependence on distance},
	width=\textwidth,
	xlabel=\large Distance from membrane $(nm)$,
	ylabel=\large Voltage and gradient $(arb.\ u)$,
	ymin=0, ymax=+100,
	xmin=-6, xmax = -5,
	%			ymode=log,
	%			log basis y=10,
	legend style={
		cells={anchor=west},
		legend pos={north west},
	},
	]
	\def\MembraneBulkPotential{5}
	
	%			\addplot[samples=100,domain=-10:-\MembraneWidth-1,webblue, forget plot]
	%{\MembranePotential} ;
	%\addplot[samples=100,domain=-\MembraneWidth-1:-\MembraneWidth-.05,webblue, forget plot]
	%{\MembranePotential} ;
	%\addplot[samples=100,domain=-\MembraneWidth-1:-\MembraneWidth-.01,webblue]
	%{\MembranePotential} ;
	%\addplot[samples=10,domain=-\MembraneWidth-.05:-\MembraneWidth-.04,webblue
	%{\MembranePotential} ;
	%\addplot[samples=100,domain=-\MembraneWidth-.01:-\MembraneWidth-.001,webblue]
	%\addlegendentry{Membrane potential}
	
	
	%			\addplot[samples=100,domain=-10:-\MembraneWidth-1,webred]
	%{\MembraneGradient} ;
	%			\addplot[samples=100,domain=-\MembraneWidth-1:-\MembraneWidth-.1,webred]
	%{\MembraneGradient} ;
	%			\addplot[samples=100,domain=-\MembraneWidth-.1:-\MembraneWidth-.01,webred]
	%{\MembraneGradient} ;
	%			\addplot[samples=100,domain=-\MembraneWidth-.01:-\MembraneWidth-.001,webred]
	%{\MembraneGradient} ;
	%
	%%			\addplot[samples=100,domain=1:6,webred]
	%{\MembraneGradient} ;
	%			\addplot[samples=100,domain=.1:1,webred]
	%{\MembraneGradient} ;
	%			\addplot[samples=100,domain=.01:.1,webred]
	%{\MembraneGradient} ;
	%			\addplot[samples=100,domain=.001:.01,webred]
	%{\MembraneGradient} ;
	%\addlegendentry{Membrane gradient}
	%
	%
	%%% Draw the second potential
	%			\def\MembraneBulkPotential{10}
	%
	%\addplot[samples=100,domain=-10:-\MembraneWidth-1,webred,very thick]
	%{\MembraneGradient} ;
	%\addplot[samples=100,domain=-\MembraneWidth-1:-\MembraneWidth-.1,webred,very thick]
	%{\MembraneGradient} ;
	%\addplot[samples=100,domain=-\MembraneWidth-.1:-\MembraneWidth-.01,webred,very thick]
	%{\MembraneGradient} ;
	%\addplot[samples=100,domain=-\MembraneWidth-.01:-\MembraneWidth-.001,webred,very thick]
	%{\MembraneGradient} ;
	%
	%
	%			\addplot[samples=100,domain=1:6,webred,very thick]
	%{\MembraneGradient} ;
	%\addplot[samples=100,domain=.1:1,webred,very thick]
	%{\MembraneGradient} ;
	%\addplot[samples=100,domain=.01:.1,webred,very thick]
	%{\MembraneGradient} ;
	%\addplot[samples=100,domain=.001:.01,webred,very thick]
	%{\MembraneGradient} ;
	%
	%\addlegendentry{Bulk potential 10}
	%
	%%% Draw the third potential
	%\def\MembraneBulkPotential{20}
	%
	%
	%\addplot[samples=100,domain=-10:-\MembraneWidth-1,webred]
	%{\MembraneGradient} ;
	%\addplot[samples=100,domain=-\MembraneWidth-1:-\MembraneWidth-.1,webred]
	%{\MembraneGradient} ;
	%\addplot[samples=100,domain=-\MembraneWidth-.1:-\MembraneWidth-.01,webred]
	%{\MembraneGradient} ;
	%\addplot[samples=100,domain=-\MembraneWidth-.01:-\MembraneWidth-.001,webred]
	%{\MembraneGradient} ;
	%
	%			\addplot[samples=100,domain=1:6,webred]
	%{\MembraneGradient} ;
	%\addplot[samples=100,domain=.1:1,webred]
	%{\MembraneGradient} ;
	%\addplot[samples=100,domain=.01:.1,webred]
	%{\MembraneGradient} ;
	%\addplot[samples=100,domain=.001:.01,webred]
	%{\MembraneGradient} ;
	%
	%\addlegendentry{Bulk potential 20}
	%%
	%
