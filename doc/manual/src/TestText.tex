% This is the Foreword to the 'Abstract Dynamic Neuron' book

\chapter*{Foreword\label{ch:Foreword}}

\begin{advanced}
“The Human Brain Project should lay the technical foundation for a new model of ICT-based brain research, driving \textit{integration between data and knowledge from different disciplines}, and catalysing a community effort to achieve a \textit{new understanding of the brain}%, new treatments for brain disease 
\dots and \textit{new brain-like computing technologies}.” — the \href{https://www.humanbrainproject.eu/en/}{Human Brain Project},  summarised its goal @2012
\index{Human Brain Project}
\end{advanced}

\begin{advanced}
"\textit{We stand on the verge of a great journey into the unknown—the interior terrain of thinking, feeling, perceiving, learning, deciding, and acting to achieve our goals—that is the special province of the human brain … No single researcher or discovery will solve the brain’s mysteries.}" — from the preamble to “\href{https://www.sciencedirect.com/science/article/pii/S089662732400655X}{BRAIN 2025: A Scientific Vision}”~\cite{BrainInitiative10:2024} @2015
\index{Brain Initiative}
\end{advanced}

\paragraph*{Understanding the dynamic brain}
The \textbf{\textit{dynamic}} operation of individual neurons, their connections, higher-level organizations, connections,
the brain with its information processing capability, and finally, the mind with its conscience and behavior,
are still among the big mysteries of science: \textit{at which point
\hyperlink{LivingMatter}
{the non-living matter becomes a living one}}.
Moreover, \textit{at which point \hyperlink{CognitiveMatter}{the living matter becomes intelligent}} and conscious; whether and how science can handle all this stuff. 
We really need a \hyperlink{NewUnderstanding}
{new understanding}.

\begin{tcolorbox}[colback=blue!10,%gray background
                  colframe=black,% black frame colour
%                  width=5cm,% Use 5cm total width,
                  arc=3mm, auto outer arc,
                 ]
The \textbf{\textit{dynamic}} operation of individual neurons, their connections, higher-level organizations, connections,
the brain with its information processing capability, and finally, the mind with its conscience and behavior,
are still among the big mysteries of science: \textit{at which point
\hyperlink{LivingMatter}
{the non-living matter becomes a living one}}.
Moreover, \textit{at which point \hyperlink{CognitiveMatter}{the living matter becomes intelligent}} and conscious; whether and how science can handle all this stuff. 
We really need a \hyperlink{NewUnderstanding}
{new understanding}.
\end{tcolorbox}

The "great journey into the unknown"~\cite{BrainInitiative10:2024} must begin earlier and at a much lower level: revisiting the fundamental phenomena, disciplines, laws, interactions, abstractions, omissions, and testing methods of science. Research must build on top of classical science, be reinterpreted for living matter, and have a correctly understood physiology.
There is no independent 'life science', only science. They are based on the same 'first principles' but using different abstractions and approximations for living and non-living matter and having the appropriate relations between them.
Without aligning the knowledge elements along the first principles, the "integration between data and knowledge from different disciplines", lacks "integration". "More Is Different"~\cite{MoreIsDifferent1972}. We arrived at the boundaries
of classical science disciplines and are moving now through terra incognita. 



\paragraph*{Many-disciplinarity}
Brain research is one of the fields where "\textit{nature is not interested in our separations, and many of the interesting phenomena bridge the gaps between fields.}" ~\cite{FeynmanThinking:1980}
\index{Feynman, Richard P}
We need a consistent
model that comprises all relevant interactions and components (and only those!) and aligns with the
notions of the related scientific fields. % Neuroscience is not an exception.
%https://blog.hptbydts.com/richard-feynmans-principles-of-scientific-thinking
Different science disciplines consider different details relevant; see Fig.~\ref{fig:Blind_monks}.
Despite the efforts of the project leader, no picture can be derived about the \textit{elephant}, although \textit{the details of the elephant} are accurate.


\begin{figure}
	\includegraphics[width=0.65\textwidth]{fig/Blind_monks_examining_an_elephant.jpg}
	
	\caption{The difficulty of many-disciplinary research on the example of describing the elephant. (a 2,500 years-old Chinese silk painting)\label{fig:Blind_monks} }
	
\end{figure}


\paragraph*{Zigzag reading}
For the same reasons, we follow von Neumann's method when describing principles of technical computing~\cite{EDVACreport1945}. "The ideal procedure would be, to take up the specific parts in some definite order, to treat
each one of them exhaustively, and go on to the next one only after the predecessor is completely
disposed of. However, this seems hardly feasible. The desirable features of the various parts, and
the decisions based on them, emerge only after a somewhat \hypertarget{zigzagdiscussion}
{zigzagging discussion}. It is, therefore,
necessary to take up one part first, pass after an incomplete discussion to a second part, return after
an equally incomplete discussion of the latter with the combined results to the first part, extend
the discussion of the first part without yet concluding it, then possibly go on to a third part, etc.
Furthermore, these discussions of specific parts will be mixed with discussions of general principles, of the elements to be used, etc."
\index{von Neumann!'zigzag' way}
 In our case, this zigzag way of reading 
is made more accessible by using hyperlinks and cross-references throughout the document. 





\paragraph*{Abstract discussion}
Nature uses an infinite variety of implementing neurons. However, in the Central Nervous System (CNS)
%\glossaryref{CNS} 
they can cooperate
with each other. 'Despite the extraordinary diversity and complexity of neuronal morphology and synaptic connectivity,
\textit{the nervous systems adopts \textbf{a number of
 basic principles}}
for all neurons and synapses'.
\cite{JohnstonWuNeurophysiology:1995}
\index{Johnston Daniel}
\index{Wu Samuel Miao-sin }
We base our holistic discussion on those 
general basic principles
and create an 
\hyperlink{NeuronPhysical}
{'abstract physical neuron'},
skipping the 'implementation details' nature uses.
We need \hyperlink{Abstractions}
{different abstractions and approximations} for describing biological processes. However, an abstraction is usable
in practice only when paired with a generalization: the more abstract the assumption, the more general and widely applicable the concept or conclusion.
\textit{By abstractions, we can reduce the unbelievably detailed world into manageable pieces, and by abstraction, we can learn anything general.}
We must show which approximations are oversimplifications were done and which phenomena are misunderstood, measured, or interpreted in the wrong approach.
 Abstraction is, in fact, everywhere, including inside each of us. It is a core element of cognition. 


\paragraph*{Physics}
"The construction [of living matter] is different from anything we have yet tested in the physical laboratory."
"It is working in a manner that cannot be reduced to the \textbf{\textit{ordinary}} laws of physics"~\cite{Schrodinger_1992}.
\index{Schrödinger Erwin}
We attempted to test and describe living matter, including the brain, with methods based on the ordinary laws of science, which we concluded for non-living matter. Also, we must understand that the fundamental differences need an accurate understanding of the
\hyperlink{BiologicalCurrent}{biological currents} and their \hyperlink{voltage-clamping}{measuring processes}.
Those \textbf{\textit{static}} methods did not consider its "special construction"; furthermore, they do not need (and, as a consequence, do not have) \textbf{\textit{dynamic}} "laws of motion" in the sense as science does. 


Their slowness and complexity require explicitly considering the \textit{time-aware handling} of processes, including the ones of biological and technical computing. 
Essentially, we make the first steps in section~\ref{sec:Physics-Thermodynamics} toward answering E.~Schrödinger's question: "How can \textit{the events in space and time} which take place within \textit{the spatial boundary }of a living organism be accounted for by physics and chemistry?"~\cite{Schrodinger_1992} Notice the need \textit{of using events and describing the spatiotemporal behavior (in other words: implementing them by slow currents in a finite volume)} implied in the question; three items which
our text targets.
\textit{No presently available theoretical description and simulator can perform that task}.
 We show that when considering the correct physics, the finite size of neuronal membranes, the finite speed of
ion currents, and the correct  description of discrete to continuous transition, we can solve the mystery that the combination of non-living materials shows
\hyperlink{SignsOfLife}
{signs of life} at an appropriate combination of their parameter values. We derive the required 'non-ordinary' (non-disciplinary) laws for describing life by physics.



\paragraph*{Mathematics}
We can only admire von Neumann's genial prediction that "the language of the brain, not the language of mathematics"~\cite{vonNeumannBrain}, given that most of the cited experimental evidence was unavailable at his age.
Similarly, one can also agree with von Neumann~\cite{EDVACreport1945} and Sejnowski~\cite{SejnowskiNeuralComputation:1999} that "whatever the system [of the brain] is, it cannot fail to differ from what we consciously and explicitly consider mathematics"; adding that \textit{maybe the appropriate mathematical methods are not yet invented}. 
\index{mathematics!brain}
Our procedure still meets
the requirement given by Feynman:~\cite{FeynmanComputation:2018}
"an \emph{effective procedure} is a set of rules
telling you, moment by moment, what to do to achieve a particular
end; it is an algorithm."  Furthermore, it considers
\index{algorithm}
that ``timing of spike matters''  giving way to interpreting Hebb's learning
rule~\cite{HebbBook:1949,HebbianLearningRule:2008}, which usually remains outside of the scope of mathematics. We formulate problems, provide their numerical solutions, and open the way for mathematics to provide analytical solutions.
\index{Hebb, D.O.}
Not surprisingly, the need for applying new approximations for the non-ordinary laws for describing living matter needs 
slightly different (in this sense, non-ordinary) mathematical formulation describing them.
We emphasize again that the first principles are the same, but the
different "construction" of the living matter needs different --  'non-ordinary' -- approximations
and laws.

\paragraph*{Electricity}
Electronics and brain research were born at the same time, 
developed together and fertilized each other. Sometimes,
the too-tight parallels led to discrepancies, from assuming
the same charge carrier and transfer mechanism for conductors and biological structures to using 
equivalent circuits, or misunderstanding the essence of spiking for
electronically implemented circuits. Those wrong parallels hide the need for
introducing electrodiffusion with its \hyperlink{MixingSpeeds}{mixing speeds} instead of separated electrical and diffusion processes, that life is governed by \hyperlink{slow_current}{finite-speed ("slow") currents} and that \hyperlink{finite_size}{finite-size (distributed) biological objects} cannot be
directly and accurately mapped to point-like (ideal) electrical components.
We need to connect the atomic electricity to the macroscopic one; furthermore,
in biology, concentration changes evoke potential changes and create large potential gradients (and vice versa). Those internal gradients start dynamical electrical \textit{ion} currents in biological tissues. 


\paragraph*{Physiology}

Physiology, which serves as the "implementation base" for neural computing, needs a revolution and replacing "classical physiology" with "modern physiology" by introducing a new paradigm.
Our point of view is new and unusual; it conflicts, on many points, with the commonly accepted opinions of the respective science disciplines.  
The facts and observations are the same, but their interpretation may differ due to the underlying 'non-ordinary' laws of physics.
Biology stayed at a static description level, typically appropriate for describing static states.
It sees that the electric charges are locally unbalanced
inside us, they continuously change, obeying only partly
understood laws, but its laws of motion are missing.
\textit{Neural processes} happen at well-observable speeds, which \textit{need a dynamic description instead of an ad-hoc description of state jumps}.
\textit{We introduce for life sciences their dynamic \hyperlink{PhysicsLawsOfMotion}{laws of motion} (in the sense of Newton, Hamilton, and Schrödinger), (based on the
\hyperlink{NernstPlanckTimeDerivatives}
{time derivatives} of the static entities)}. Furthermore, 
we explain why and how a fraction of the charges constitute
living matter change obeying their laws of motion,
creating \textit{the needed 
\hyperlink{DynamicLayer}
{dynamic component} which can implement
the needed dynamic processes}.
Considering them solves the mystery of 
%%\hyperlink{CellToLife}
{how life science builds on top of science}.
We defy that
"the emergence of life cannot be predicted by the laws of physics"~\cite{ConservationOfInformation:2021}, furthermore, that "the existence of life is against the laws of thermodynamics".
\index{emergence of life}
%Describing physiological processes and understanding neural information processing needs a revolution.

\paragraph*{Neuroscience}
We understood early that "\textit{the fundamental task of the nervous system is to \textbf{communicate and process information}}".
The goal was set decades ago:
'\textit{The ultimate aim of computational neuroscience is
to explain \textbf{how electrical and chemical signals are used} in the brain
to \textbf{represent and process information}}'~\cite{SejnowskiComputationalNeuroscience:1988}.
\index{Sejnowski, T J}
Today, computational neuroscience turned into introducing mathematical models, slightly or not related to reality, implementing them using a "technomorph biology"~\cite{VeghTechnomorphBiology:2025}, 
while it keeps wondering why nature behaves differently.
The worst inheritances of neuroscience are the static view from anatomy and classical physiology;
omitting to revisit periodically the primary hypotheses in light of new research results;
applying the abstractions of classical science
(single speed, isolated, pair-wise, instant interactions in a homogeneous and isotropic infinite medium)
to biological materials without revisiting their validity.
Moreover, the tradition of applying ad-hoc mathematical formulas
without correct physical processes in the background (actually creating an alternative nature)
instead of understanding the basics of the underlying processes.
To start with, we introduce \textit{science-based abstract dynamics}
(by introducing the needed 
\hyperlink{PhysicsLawsOfMotion}
{laws of motion}),
as opposed to the \textit{empirical cell-biology's static} description of neuronal operation.
We agree that "the basic structural units of the nervous system are individual neurons"
\cite{JohnstonWuNeurophysiology:1995}, but
we are also aware of that neurons "are linked together
by dynamically changing constellations of synaptic weights" and
"cell assemblies are best understood in light of their output product"~\cite{BuzsakiCellAssemblies:2010,VeghNeuralShannon:2022}
so we also 
\hyperlink{Multiple_neurons}
{model multiple neurons}.



\paragraph*{Computing}"The brain computes!
This is accepted as a truism by the majority of neuroscientists."~\cite{ KochBiophysics:1999} However, even after many years and grandiose projects, 
"Yet for the most part, we still do not understand the brain’s underlying computational logic"~\cite{BrainInitiative10:2024}.
To understand how "computation is done", we 
\hyperlink{GeneralizedComputing}
{generalized computing}~\cite{VeghRevisingClassicComputing:2021},
in close cooperation with 
%\hyperlink{COMPUTING_COMMUNICATION}
{communication}
\cite{RoleOfInformationTransferSpeed:2022}, for biology.
We understand that "a piecemeal approach will not yield the major jumps in understanding for which the BRAIN Initiative was designed"~\cite{NIHBrainStrategy:2020}.
We synthesize the available knowledge with a fresh eye and intend to make a leap
in understanding neural computing,
scrutinizing our knowledge
pieces one by one for credibility, relation to other pieces, other disciplines,
finding contradictions and their resolutions,  \textit{defying fallacies}.
We show how an elementary neuronal operation carries out computing, why biological 
computing is by orders of magnitude more effective than the technical one,
how the biological implementation enables learning,
how and why do the features of the two computing systems differ.


\paragraph*{Information}Although we experience that the brain processes an enormous amount of information, we know impressive details about how the brain uses it
to react appropriately to stimuli from its environment,
we still do not know the details and the underlying general principles
how neuronal networks represent, process, and store the information they use.
What makes the case worse, is, that, due to the lack of knowledge of the abstract way of neuronal operation,
the so-called "neural information science" uses a wrong mathematical background.
We understand how neurons 
%\hyperlink{InformationForBiology}
{represent and process information}~\cite{VeghNeuralShannon:2022}.
We introduce the appropriate interpretation of
%\hyperlink{InformationForBiology}
{information for biology}.


\paragraph*{Intelligence}
Scrutinizing time awareness of biological computing, learning, and intelligence discovers~\cite{WhyMachineLearningDifferent:2021} that they have practically only their name in common with the technical ones. 
 As a consequence, "biological brains are more efficient learners than modern
 % \gls{ML}
 ML algorithms due to extra ‘prior structure’"~\cite{FittingElephants:2021}.
 We must not forget "There is no reason whatever for believing that our brain is the supreme \textit{ne plus ultra} of an organ of thought in which the world is reflected."~\cite{Schrodinger_1992} 
 Furthermore, "it is also possible that non-biological hardware and computational paradigms may permit yet other varieties of machine intelligence we have not yet conceived"~\cite{FittingElephants:2021,FeynmanComputation:2018}. We start to conceive them by analyzing the fundamental
 processes which can be "directly associated with consciousness"~\cite{Schrodinger_1992}. 


\paragraph*{Simulator}The site is not exclusively about theory: we also give a programmed implementation of the ideas we describe.
\href{../../docs/index.html}{Our simulator}
%@link SIMULATION_SYSTEMC Our simulator @endlink 
has a direct science base instead of ad-hoc mathematical formulas;
and the only one which can reproduce 
the true biological time course
%@link SIMULATION_SYSTEMC the true biological time course@endlink 
of neurons,
from the first science principles, without arbitrary ad-hoc assumptions and limited variability formulas.
Our methods enable discussing the major aspects of phenomena of the natural operation of neurons
to analyze the effects of invasive electricity-related investigation methods on neuroscience.
We offer demos, class implementations, performance benchmarks, and test cases
to demonstrate simulating capabilities. We intend to develop full-value
educational, demonstration, and research tools. 


\paragraph*{Forbidden science}  
Hodgkin and Huxley in 1952~\cite{HodgkinHuxley:1952} advanced neurophysiology by contributing a long series of \textit{observations} on neuronal operation. 
However, as they warned, many of the \textit{mechanisms} must be fixed or replaced: "must emphasize
that the interpretation given is unlikely to provide a correct picture of the membrane". We honor their outstanding work and want to supplement and enhance their interpretations and hypotheses instead of defying them.
% It is not possible, however.
However, their work became the "Holy Bible" of physiology. The editors 'do not believe' (see Fig. ~\ref{fig:Galilei}) if science advanced in the past seven decades and they \href{https://dictionary.cambridge.org/dictionary/english/censor}{censor}
publishing new ideas in scientific journals.
Likely, they did not know that science must "have no respect whatsoever for authority; forget who said it and instead look what he starts with, where he ends up, and ask yourself, 'Is it reasonable?' \dots If we suppress all discussion, all criticism, proclaiming 'This is the answer, my friends; man is saved!' we will doom humanity for a long time to the chains of authority, confined to the limits of our present imagination." (Richard P. Feynman)


\begin{advanced}
The editors' comments on the manuscript "The Physics Behind the Hodgkin-Huxley Empirical Description of the Neuron", submitted to "Physics of Life Reviews". 
Oct 11, 2024. Rejected without reading. PLREV-D-24-00173

"The physics behind the Hodgkin-Huxley model of the neuron is certainly within the scope of PLRev. This model has been extensively studied since it was proposed in 1952 and its proponents won the Nobel Prize in 1963. So it is a challenge to say something original and relevant about this model in 2024. Since the author has no previous publications on the topic, \textit{the Editorial Board does not believe} that the review will have any impact on this very well-established research topic."
\end{advanced}



\begin{figure}
	\includegraphics[width=.65\textwidth]{fig/Galilei.png}
	
	\caption{"Believing" in science (in the age of Galillei). \label{fig:Galilei} }
	
\end{figure}



\begin{advanced}
\textbf{Warning:} Please consider that this development is a one-person undertaking.
Moreover, it shall develop theory, evaluate published experiments, implement software, test it, and document it.
Pre-developed code fragments, science publications, and docs exist, so they develop relatively quickly,
but we need time to put them together consistently.  Please return later and see if something is new (see the date and the version).
\end{advanced}

