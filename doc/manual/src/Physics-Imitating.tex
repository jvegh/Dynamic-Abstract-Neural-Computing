% Physics: imitating ion's electricity, with electrons

\subsection[Imitating ion's electricity]{Imitating ion's electricity\label{sec:Physics-Imitating}}
\subsubsection[Time dependence]{Time dependence\label{sec:Physics-TimeDependence}}
\subsubsection[Charge storing]{Charge storing\label{sec:Physics-ChargeStoring}}
We have evidence
that the membrane's charge is proportional to the membrane's voltage: the membrane
has a fixed capacity $C$. 
We know that the arriving axonal currents (as well as the rushed-in ions after exceeding the membrane's threshold voltage) cause massive transient
changes~\cite{TransientResponses:2008,KochElectricalPropertiesSpike:1983}
\index{voltage gradient}
(in other words: gradients) in the membrane's voltage.
We know that the charges on its surface
can flow out only through the ion channels~\cite{ActionPotentialGenerationNatrium:2008} in the neuron's %\gls{AIS},
AIS,
which we represent
with a resistance $R$. We have evidence that the
AIS
%\gls{AIS}
only mediates the membrane's changing voltage to the axon~\cite{ActionPotentialGenerationNatrium:2008}.
\textit{So, we have good reasons to assume
	that the membrane
	is not equipotential when generating an AP}.
	%\gls{AP}}. 
 Our hypothesis about "slow" currents' presence thoroughly explains the phenomena about the \hypertarget{temporal_dependence}{temporal behavior of neuronal processes} mentioned.
However, given that the classic picture also
explains many phenomena, we must establish the connection between
the two models and draw the borders within which the classic description
can be used, and where our time-aware model must be used.



Some resemblance indeed exists in charge handling in
electrical and biological circuits. However, the validity of parallels
is limited. 
Assume an infinitesimally
fast ion current on the surface will keep the membrane's potential
constant all the time.
We can use the formalism developed for electricity,
even in biology, if we want to use the \emph{point representation}
of a neuron.
The price we pay is that we do not have access to the
voltage and current of our finite-size membrane: they are confined
in our fictive discrete elements (point representations within our
point representation of neurons) despite the evidence listed above. 
This abstraction is appropriate
for designing electric circuits and has a suitable formalism that describes their behavior.
\textit{However, neither membrane nor 
%\gls{AIS}
AIS
 has the facilities to generate %\gls{ AP} 
an AP in this approach.}

The "point representation" model results in a wrong parallel with the \textit{"integrator-type"} $RC$ electric circuits: by assuming discrete resistance and capacitance, we set
up a fake hypothesis that the voltage drops within the resistor, and that current flows into the condenser and the charge is stored in it. A late consequence of the century-old
idea of "point representation" (which immedietely follows from the 'instant interaction')  is that we must omit the temporal
dependence of axonal signal arrival; we try finding a correlation
between neuron's inputs; we do not see the role of fellow neurons in
neural operation; we attempt to describe neuronal information with
inappropriate representation and using inappropriate mathematical methods, etc. 


The "extended point" model reveals that all currents flow into the membrane, which is a \textit{distributed} condenser, and the AIS
%\gls{AIS}
with a resistor $R$ is a \textit{discrete} output component of the circuit;  see also Fig. 1 in ~\cite{ActionPotentialGenerationNatrium:2008}. That is, the neuron shall be modeled as a \textit{"differentiator-type"} $RC$ circuit, having entirely different electric behavior from that of the commonly used "point representation" model (with its implicit \textit{"integrator-type"} $RC$ circuit) predicted.
	From biological point of view, the vital difference is that this circuit type can produce an output voltage 
	with opposite sign, enabling to describe to hyperpolarization, without needing any fake extra mechanisms, such as outflow of an intens  $K^+$ current.



In our model, the neuron membrane is simply a two-dimensional elastic
isolator surface (where needed, we imagine it as a thin, long, and narrow
rectangular piece) that has current sources at different positions (axonal arbors), many concerted current sources in its body (the ion channels in the membrane's wall)
and a current drain (AIS)
%(\gls{AIS})
at the other. The input and output
currents increase/decrease the voltage on the membrane. In our time-aware model, we assume that the ions on the membrane's
surface represent a kind of "free ion cloud" (see also section~\ref{sec:Physics-Current}),
so we can interpret the capacitance $C$ (at least for our differential
equation) in a classic way. However, charge carriers are not necessarily present on the surface. In the
case of a neuronal membrane, the stable basic state is that there are
no charges on the surface. If charge carriers (from an external source)
appear, the potential increase that their appearance causes leads
them to be removed. A ``slow'' current on the surface with a speed
$v=\frac{dx}{dt}$ represents a current $I_{slow}=A*n*q*v$.

In its steady state, the ions (from the  rushed-in ions, axonal currents, or artificial currents) create a uniform potential
over the membrane. In our simplified discussion, 
we omit the less intense input currents (which also cause transient voltage changes, which should be summed with that from the effect of the rushed-in ions) and discuss only the one-time contribution due to the rushed-in ions.
On the one hand, in its non-steady state, the neuronal $RC$ circuit uses the time derivative of the potential due to the rushed-in ions as input, see Equ.~(\ref{eq:PSPderivative})). On the other, the potential drops due to  the current drain (the AIS
%\gls{AIS}
 at the end), where the
current is
\[
I_{AIS}=\frac{V_{AIS}-V_{rest}}{R}
\]
%
According to Kirchoff's Law, the current (and consequently the voltage derivative) through the 
\index{Kirchoff's Laws}
%\gls{AIS}
AIS must be equal to that of the 
membrane due to the rushed-in ions. We can solve the 
differential equation numerically; see section~\ref{sec:Physiology-AP} and Figure~\ref{fig:AP_Variety}. We can also derive
%
\begin{equation}
	v_{AIS}=\frac{V_{AIS}-V_{rest}}{R*A*n*q}\label{eq:SlowCurrentSpeed}
\end{equation}
%
\noindent that is, the speed of the ``slow'' current is proportional
to the voltage $V_{AIS}$. The current $I_{slow}$ will change
the membrane's voltage: 
\begin{equation}
\frac{dV}{dt}=\frac{A*n*q*v}{C};\ \frac{dV}{dx}=\frac{dV}{dt}\frac{dt}{dx}=\frac{A*n*q}
{C}\label{eq:SlowCurrentVoltage}
\end{equation}

That is, the potential in the function of the distance will drop in
the same way as if the membrane had a distributed resistance $R$.
However, the resistance is located to the AIS,
%\gls{AIS},
 as if it were a
discrete element. In electronics, the capacity $C$ is interpreted
as opposite charges on the condenser's plates. In biology, no similar
stored charge exists. The \emph{charges spend some time on the surface},
inversely proportional to the current's speed (the inward positive
current due to rushed-in ions and the outward positive current of
the pumped-out ions has been observed, but not the corresponding negative
currents). The distributed resistance and the specific capacitance
are constant in the function of position over the membrane's surface so that those values can
be used in differential equations based on Kirchoff's Laws. Notice,
\index{Kirchoff's Laws}
however, that currents joining the membrane at different points may
spend different times on the surface (meaning different capacitance
values), so the capacity changes in the function of the time, in this way distorting the time constant $RC$ and so the shape of AP.
%	\gls{AP}.

The potential in the function of the time and the speed of the slow
current mutually generate each other, as described by Eq.~(\ref{eq:SlowCurrentSpeed}).
In a steady state, no current flows. When some current arrives through
the axons, or flows out through the 
%\gls{AIS},
AIS, a slow current
starts to balance the potential difference created by the current.
Changing the amount of charge on the surface transiently leads to
a non-equipotential membrane. Notice the difference: if we assume
instant interaction, we assume a constant membrane potential using
discrete elements $R$ and $C$. The voltage drops on the discrete
element $R$, and the charge is stored in the discrete element $C$.
\emph{The voltage outside the discrete elements is constant, except for the voltage step, due to some incoming current (including the AIS,
%	the \gls{AIS}),
	 and there is no way to interpret how and why the AP
	%\gls{AP}
	is created.} On the contrary, if the current is slow, it
needs time to reach another position (we can change the membrane's
local "charge storing" ability), and
it can either increase or decrease the local voltage. When using a voltage
generator with appropriate temporal behavior, \emph{the ``slow''
	current explains why and how an AP
	%\gls{AP}
	in a biological neuron is
	evoked.}

We can hypothesize that 
\begin{itemize}
	\item "making a hole" in the membrane~\cite{KochElectricalPropertiesSpike:1983}
	means that "slow" ions are pressed into the membrane through the
	axon.
	\item the inflection point is the turning point where the outward
	current exceeds the inward current, and it can be considered to be
	the time of the arrival of a spike (in the case of the first spike,
	it can be the signal 'Begin Computing'~\cite{VeghComputingModel:2021}).
	\item the inflow and outflow happen in parallel (the slopes of the %\gls{PSP}
	PSP voltage course differ from those of the current pulse; see also their
	numeric time constants in Fig.~\ref{fig:MasonFig4}); that is, we
	will see the difference between a ``slow'' and a ``fast'' current,
	with a particular temporal behavior.
\end{itemize}


\subsubsection[Time delay]{Time delay\label{sec:Physics-TimeDelay}}

The time needed to move a charge to a distance comprises two contributions.
To move a charged particle in a piece of material (``the wire''),
first, we must produce a force to accelerate the particle inside the
wire at the position of the particle (the needed time is the distance
to its location divided by the \emph{speed of propagation of the electric
	potential field}). The second contribution is that the charged particle needs time
to reach the other end of the wire (the distance to the external world
divided by the \emph{particle's drift speed}). The object must be
accelerated to that speed by that electric field; for the sake of
simplicity, we consider the needed time negligible. To\emph{ calculate
	the "\hypertarget{time_delay}{time delay}", we need to sum the field and charge propagation
	times.} Let us suppose that the electric
field's propagation speed is infinite and the charge is in the immediate
vicinity of the end of the wire. Fortunately, different physical mechanisms
(such as ``free electron cloud'') can produce the illusion of a
much faster macroscopic \emph{current speed}. In that case, the travel
time of the charge is negligible. However, we can expect only in that
case that the charge promptly contributes to the current, i.e., the
current follows the voltage without delay. 

We consider the cases of galvanic wire and electrolytic wire. There
is no essential difference in the field propagation time: for our
human senses (and even slower electronic tools), it is a good approximation
that the electric field appears promptly along the wire, including
the position of the charged particle. In galvanic wires, the electrons
behave like an electron cloud: uniformly distributed in the wire.
When the electric field appears (in the sense above: promptly), there
are electrons in the infinitely small vicinity to the end of the wire.
The field speeds them up immediately, so they exit the wire, and some
other electrons enter the wire at the other end simultaneously. The charge
carriers enter and exit immediately after an external potential is
applied. 

The phase change of voltage and current follow each other without
a (noticeable) delay. Ohm's Law is valid for this case: the derived
entity connecting them (resistance or conductivity) is constant. The
Law expresses the charge conservation: the same number of
carriers passes the cross-section at any time. Remember that \emph{the
	essential conditions were that free charge carriers were present and
	uniformly distributed in the wire. Furthermore, they were moved by
	only one microscopic force} (not considering the forces implementing
an average macroscopic \textquotedblleft resistance\textquotedblright ).
Even in metallic components, the derived material characteristics
depend on many factors when we apply a step-like change in the voltage
or the current. The so-called on-resistance is also known outside
electrolytes and is influenced by various parameters such as temperature
and supply voltage.

In an electrolytic wire, the ions in the electrolyte may be uniformly
distributed (they form a kind of ``free ion cloud'' inside the electrolyte),
i.e., after the electric field is applied, the ions can immediately
exit the electrolyte and produce an electric current. In summary,
the ions are very slowly moving charged objects (compared to the free
electrons; BTW: the electrons move only slightly faster than ions;
only the cloud provides the illusion of their high speed). However,
they can create a prompt ionic current, \emph{provided that they are
	present in the corresponding volume and their concentration is isotropic}.
The living cell with its semipermeable membranes can produce
situations where isolated structures do not fulfill that condition,
and the less careful observer identifies the situation as non-ohmic
behavior. As  we discussed, the axonal tubes are empty 
(no charge carrier) at the beginning of a \hyperlink{voltage-clamping}{clamping experiment}
(see the measurement results in Fig.~\ref{fig:The-time-course_Clamping}), and they are filled in their steady state (at beginning discharging), producing
entirely different temporal behavior ("changing conductance").


