% Spatiotemporal behavior

\section{Spatiotemporal\label{sec:Physics-Spatiotemporal}}

%\subsection{Time, space, and matter\label{sec:Physics-TimeSpaceMatter}}

Classical physics is based on the Newtonian idea that  space and time are absolute so everything happens simultaneously. Moreover, their observationis also instantaneous. Consequently, when their objects
interact, it must be instantaneous; in other words,
all interactions have the same (and, logically: infinitely high) speed. Furthermore, electromagnetic
waves with the same high (logically, infinitely high) speed inform the observer. This
self-consistent abstraction enables us to provide a "nice"
mathematical description of nature in various phenomena: the classic
science. In the first year of college, we learned that the idea resulted
in "nice" reciprocal square dependencies, Kepler's and Coulomb's
\index{Kepler's Law}
\index{Coulomb's law}
laws. We discussed that the macroscopic phenomenon "current" is
implemented at the microscopic level by transferring (in different
forms) discrete charge, and that movement of charges has no effect on the environment. Furthermore, that \textit{without charge (and, without atomic charge carriers),
neither potential nor current exists}. In the next semester, we learned
that the speed of light is finite and that solids show a macroscopic
behavior "resistance" against forwarding microscopic charges.
One semester later, we learned that nature behaves differently
at high speeds, at low sizes and energies;
furthermore, that the transition between the continuous and discrete 
views needs new concepts.

In biology, it was evident that
the transfer (conduction) time must be considered together with the computing (synaptic) time (in this sense, presynaptic to postsynaptic transmission time).  The name "\hypertarget{spatiotemporal}{spatiotemporal}"
and a (separated) time dependence is commonly used~\cite{PerturbationNeuralComputation:2002},
in the sense that  Precise Firing Sequence (PFS) "\textit{tended to be correlated with the animal's behavior}"; furthermore, that "\textit{the results suggest that relevant information is carried by the fine temporal structure of cortical activity}"~\cite{SpatiotemporalPrut:1998}.
\index{time-space}
The "\textit{neural dynamics}" was studied 
and "\textit{spatiotemporal spreading of population activity was mapped}"~\cite{SpatiotemporalPlenz:1996} by methods used to describe the \textit{static computing methods}: 
interspike intervals histograms, auto-correlation and cross-correlation.
Because of the peculiarities of this information handling, \textit{there are severe doubts whether 
the notions of the classic neural information theory are valid
for biological computing systems}~\cite{VeghNeuralShannon:2022}.
The correct method of describing biological computation is still missing, given that the significant item of the computing
is missed: \textit{the time and position are connected through the 
	information transfer speed} (called conduction velocity).


Will be based on~\cite{RoleOfInformationTransferSpeed:2022,VeghIntroducingTemporalBehaviorScience:2020,BuzsakiVeghSpaceTime:2023}

 
\begin{figure}
\iflatexml
\includegraphics[width=\textwidth]{fig/RelativisticComputation.png}
\else
\includegraphics[width=.65\textwidth]{fig/RelativisticComputation.pdf}
\fi

	
	\caption{The temporal diagram, i.e., the way of calculation to combine the spatial distance (transmission time, blue arrows) and computing time (green arrows) illustrated in the time-space coordinate system. The orange-green vertical arrow shows that the second computing unit must idly wait until the transmitted result reaches its position, because of the finite transmission time. The axes $x$ and $y$ refer to space coordinates (transformed to time using the conduction velocity), the axis $t$ refers to the time itself. The arrows starting from points 0, 1 and 2
		on the $x$ axis illustrate timing for three different propagation speeds.
		The red vector points from the 
		beginning to the end of the process. 
		Its length may serve as a statistical
		entity to describe temporal distance of the units. \label{fig:RelativisticComputation}}
\end{figure}

In Fig.~\ref{fig:RelativisticComputation}, for visibility, two spatial
and one temporal coordinate are shown. In the following figures,
some illustration enable to omit one more spatial dimension, i.e., 
effectively to draw the events as a two-dimensional diagram.

