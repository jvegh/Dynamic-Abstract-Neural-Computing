% The foreword for single-neuron modeling

The existence of life is still a mystery for science. As E.~Schr\"odinger formulated, "the construction [of living matter] is different from anything we have yet tested in the physical laboratory \dots it is working in a manner that cannot be reduced to the \textit{ordinary} laws of physics"~\cite{Schrodinger:1992}. Why he desperately inserted the italicised word was his firm conviction that  "\textit{not on the ground that there is any 'new force' or whatnot}, directing the behaviour of the single atoms within a living organism, but because \textit{the construction is different from anything we have yet tested in the physical laboratory}." He failed to find those laws in a disciplinary way. R.P.~Feynman was right in saying that
"the separation of fields \dots %, as we have emphasised, 
is merely a human convenience \dots \textit{Nature is not interested in our separations.}"~\cite{FeynmanThinking:1980} 

As Schr\"odinger and Feynman implicitly suggested, we revisit the approximations that led to classical physics (where we derived the well-known 'ordinary' laws).
% We see that the construction of non-living matter is one such approximation.
We scrutinize the "construction" and its "working" in a cross-disciplinary way, using non-ordinary approximations and abstractions.
Maybe the "different construction" the living matter represents only needs different (cross-disciplinary) approximations, and they result in non-ordinary laws, which, "once they have been revealed,
will form just as integral a part of this science as the former".~\cite{Schrodinger:1992}
Maybe, then, laws based on the same first principles in a different approximation can describe living matter?

We are as general as possible when discussing the
physics of ions and electrolytes; furthermore, the physics for biology.
However, our specific goal is to establish a cross-disciplinary 
model of neuronal operation.


"Despite the extraordinary diversity and complexity of neuronal morphology and synaptic connectivity,
\textit{the nervous systems adopts a number of
 basic principles}"~\cite{JohnstonWuNeurophysiology:1995}.
 In this chapter, we derive those principles in an explicite form from the first principles of science and we prepare that discussion in an abstract level.
 We prepare the concepts and principles which enable us to
answer E. Schrödinger's question concerning physics.
In the spirit of Johnston and Wu \cite{JohnstonWuNeurophysiology:1995},
we introduce the relevant major components of a single abstract neuron 
that still follow the
{basic principles} when processing neuronal information,
moreover, the abstract principles of how they cooperate, conceptually.
Our motto is similar to the frequently cited saying \href{https://en.wikipedia.org/wiki/Cherchez_la_femme"}{'Cherchez la femme'}:
\textit{Look for the charge},
meaning "\textit{no matter what the problem, some charge is often the root cause}" (however, we consider also effects of the thermodynamics 
on charge, given that the charges are ions instead of electrons. Furthermore, we shortly consider the mechanical, optical, etc. consequences.).
The expression comes from the novel \textit{The Mohicans of Paris} by Alexandre Dumas and is frequently used
in detective novels in the sense that there are always complications and unusual
situations, but "no matter what the problem, a woman is often the root cause". 
Although, in most cases, the cause is seemingly different. However, the primary entity when speaking about neural operations, is the charge.
Exactly as formulated: "Transient electrical signals are particularly
important for carrying \textit{time-sensitive information} rapidly and over long distances. These transient electrical
signals—receptor potentials, synaptic potentials, and
action potentials—are all produced by \textit{temporary
changes} in the electric current into and out of the
cell, changes that drive the electrical potential across
the cell membrane away from its resting value." "we consider
how transient electrical signals are generated in the neuron." ~\cite{PrinciplesNeuralScience:2013}, page 126. We note, however, that the electric phenomena are closely related to other disciplines, mainly thermodynamics.

In this chapter we consider that \href{https://neuronaldynamics.epfl.ch/online/Ch1.S1.html}{neurotransmitters, receptors and specialized membrane proteins}
\textit{only implement a kind of (time and energy-consuming)
	chemical/enzymatic decoupling
	of the signal transmission mechanism.} The idea resembles opto-coupling in electronics:
makes \textit{the signal transmission independent from the local potential value}.
Suppose neurons use galvanic coupling
when the resting potential of one of the neurons equals the extracellular space. In that case,
all connected neurons' resting potential is
equal to that of the extracellular space. Without this decoupling, the death
of one neuron would immediately lead to the death of the entire neural network.
Furthermore, the neurons could not make independent signal processing.


%We generalized the notions of computing~\cite{VeghRevisingClassicComputing:2021}
%and information~\cite{VeghNeuralShannon:2022} for biology and use the notions introduced there throughout the document.
