
\iflatexml
\else
\minitoc
\fi
\chapter[Intelligence]{Intelligence\label{ch:Intelligence}}
will be based mainly on \cite{WhyMachineLearningDifferent:2021}


It is at least hard, if possible at all, to define the meaning of the word "intelligence"
and really dozens of meanings are used around~\cite{MachineVersusLivingIntelligence:2024}. 
The fundamental questions to reply are 
“Whether it is implemented by molecules, cells, liquid crystals, silicon or digital code, the essential operations of understanding are the same. Can the system acquire information external to itself? Can it generate an internal model of the external world by encoding information about it such that it can make predictions and inferences?" 
\cite{SeekingClarityAI:2024}.	


The confusion starts at a much lower level. "At one extreme, the 'cognitive' in \hypertarget{CognitiveMatter}{\textit{cognitive neurosciencen}} has replaced the older term information processing. At the other extreme the term 'cognition' refers to those higher level processes fundamental to the formation of conscious experience. In common parlance, the term 'cognition' means thinking and reasoning."~\cite{PrinciplesNeuralScience:2013}. At the lowest level, both implementations do information processing. However, even at their lowest level, they process differently interpreted information on different structures using different methods. The notion of 'cognition' (similarly to 'conscience', and other notions) are not transferable between those implementations.

The major operating difference is the sequential operation of technical systems that directly affects imitating biological operations.
One of the primary motivations for using neural networks was the demand for processing actions on the correct biological time scale:
“Many theoretical neurobiologists have turned to different types of models that include parallel processing, which they call neural networks.”\cite{PrinciplesNeuralScience:2013}, page 37.
"The branch of computer science known as artificial intelligence originally used serial processing to simulate the brain’s cognitive processes—pattern recognition, learning, memory, and motor performance.
These serial models performed many tasks rather well, including playing chess. However, they performed poorly with other computations that the brain does almost instantaneously"~\cite{PrinciplesNeuralScience:2013}, page 37. 
More differences, steming from their different 'technology', are discussed in~\cite{WhyMachineLearningDifferent:2021}.

\section[Cognition]{Information vs intelligence vs cognition \label{sec:Information-Cognition}}


\begin{advanced}
"At one extreme the
'cognitive' in cognitive neuroscience has replaced
the older term \textit{information processing}.
\\
At the other extreme the term 'cognition' refers
to those higher level processes fundamental to the for-
mation of conscious experience.\\
In common parlance the term 'cognition' means
thinking and reasoning."~\cite{PrinciplesNeuralScience:2013}
When we must use those words, we use it exclusively in the 
first meaning: the collective activity of neurons for processing information at the 'technically' lowest level.
\end{advanced}

Although it is challenging both to list the crucial differences between technological components attempting to mimic a biological component and to separate the HW
%\gls{HW}
and SW
%\gls{SW}
issues, some of the most important ones are mentioned below. 

The confusion starts at a much lower level. "At one extreme the 'cognitive' in cognitive neuroscience has replaced the older term 'information processing'.
At the other extreme the term 'cognition' refers to those higher level processes fundamental to the formation of conscious experience.
In common parlance the term 'cognition' means thinking and reasoning."~\cite{PrinciplesNeuralScience:2013}.
At the lowest level, both implementations do information processing.
However, even at their lowest level, they process differently interpreted information on different structures using different methods.
The notion of 'cognition' (similarly to 'conscience', and other notions) are not transferable between those implementations.

Fundamentally, we are in line with the standpoint of E.~Schrödinger~\cite{Schrodinger:1992} (The physical basis of conscieousness):
"What kind of material process is directly associated with consciousness? \dots consciousness is linked up with certain kinds of events in organized, living matter, namely, with certain nervous functions \dots It is still more gratuitous to indulge in thoughts about whether perhaps other events as well, events in inorganic matter, let alone all material events, are in some way or other associated with consciousness."

