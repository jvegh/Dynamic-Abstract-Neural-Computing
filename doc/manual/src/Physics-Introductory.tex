% The introduction for physics

\section{Introduction\label{sec:Physics-Introduction}}


The roles of time, space and matter are subjects of endless debates in science.
Considering finite interaction speeds is against using a "nice and classical physics"
with its nice mathematical formulas, but omitting the different speeds
misled and may mislead research in several fields.
Biology produces situations where the complexity of phenomena
and the needed carefulness meets the ones needed in cosmology.
The difference is that, in biology, the consequences of phenomena are immediate
and they can be studied experimentally.


 To describe the related
phenomena, we must scrutinize, case by case, which interactions are significant and
which interaction(s) can be omitted; instead of setting up ad-hoc models for living matter that contradict
each other and fundamental scientific principles when used outside their narrow range of validity. To provide their
correct physics-based description, we must understand the corresponding behavior of living material, including that it works with slow ion currents,
electrically active, non-isotropic, structured materials,
and consequently, its temporal behavior (the speeds of interactions)
matters. We must consider macroscopic and microscopic phenomena (continuous and discrete features) at the same time, different
science fields, and their interplay (or mixing). "\href{https://www.pnas.org/doi/10.1073/pnas.1705704114}{Living complex systems in
particular create high-ordered functionalities by pairing up low-ordered
complementary processes, e.g., one process to build and the other
to correct}".~\cite{BiologicalConservationLaw:2017}
\index{law of conservation in biology~\cite{BiologicalConservationLaw:2017}}
We need to
check the validity of our abstractions. 


Galilei said, "Mathematics is the language in which God has written
the universe". However, it is not sure that when we attempt to read
\index{Galilei, Galileo}
a piece of the universe written in that language, we use the right
piece and dialect of the language, and even that humans already invented the
needed piece. 
For example,  \href{https://en.wikipedia.org/wiki/Calculus} {mathematical calculus}
(integral and differential) was invented mainly for the practical needs of analyzing spatial motion of celestial bodies.
Similarly, Minkowski's mathematics theory proliferated widely~ \cite{EinsteinMinkowskiRelativity:1977}
\index{Minkowski, Herman}
\index{mathematics of finite speeds ~\cite{EinsteinMinkowskiRelativity:1977}}
only after inventing the special theory of relativity.
Although the mathematical description was developed earlier, there was no practical
need to apply it.
The classical laws of motion were valid only until more meticulous observations required
to consider speed and acceleration (time derivatives of position) dependence in addition to position dependence.
Newton's \textit{static} laws remained valid, but for the \textit{dynamic} description we must revisit the second law of motion. 

Also, we must not forget that "mathematics is not just a language. Mathematics is a language plus reasoning. It's like a language plus logic. Mathematics is a tool for reasoning." (Richard P. Feynman)
Mathematical formulas work with numbers but math theorems and statements begin with "If ... then".
They have their range of validity, even when they describe nature.
Using mathematics to describe the classical equations of motion,
to calculate forces and times that speed up bodies
above the speed of light is possible, but in that case mathematics is applied to
an inappropriate approximation to nature. When approaching the speed of light,
different physical approximations (that calls for different mathematical handling) are to be used.
\textit{A mathematical formula, without naming which interactions it describes
 and naming
under which conditions and approximations the formula can be applied, are just numbers without meaning.
It surely describes something but only eventually describes what we studied.}
Galilei made measurements with objects having friction, but his careful analysis
extrapolated his results to the abstraction that no friction was present.
\textit{We know his name because of making meticulous abstractions and omissions (and mainly:
recognizing the need to do so!)} instead of publishing a vast amount of
half-understood measured data.

In this chapter we underpin that "living matter, while not eluding the 'laws of physics' as established up to date, is likely to involve 'other laws of physics' hitherto unknown, which however, once they have been revealed, will form just as integral a part of science as the former"~\cite{Schrodinger:1992}. We scrutinize the oversimplifications, wrong abstractions and derive the laws
which are not unknown but need more careful formulation of the first principles when applied to living matter. Eight decades later, E. Schrödinger's question is still open "The origins of life stands among the great open scientific questions of our time~\cite{OriginsOfLifeWalker:2017}. We hope the analyzis in this chapter takes us one step closer to the reply.