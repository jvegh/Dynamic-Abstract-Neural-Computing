\section{State-of-the-art\label{Physiology-StateOfTheArt}}

Today, two major branches of disciplinary theories compete 
for being applied to describe the neuronal operation \cite{CriticalElectricity:2018,ThermodynamicAPDrukarch:2022,PerspectivesNerveSignalPropagation:2024,ComparisonHHandSoliton:2010,HH_Potential_Controversies_2017,PiezoelectricNeuron:2025}. Those cited references also compare those theories.
The electrical view essentially stems from the (Nobel Prize-winning) work of Hodgkin and Huxley \cite{HodgkinHuxley:1952}. They could not make a perfect job, mainly due to the lack of
discoveries made several decades later, including ours, about handling
the finite speed of the biological ionic currents (and, due to that, the dynamic features), which drastically change
their conclusions. In contrast with their \textit{empirical description}, which delivered \textit{mathematically formulated measured observations}, our discussion, although it follows essentially
the same principles, reinterprets and pinpoints the used fundamental terms, sets up a physical model, and \textit{explains} the physically underpinned processes. ``However, the theory could not explain the physical
phenomena such as reversible heat changes, density changes,
and geometrical changes observed in the experiments'' \cite{ComparisonHHandSoliton:2010}.
The thermodynamic view roots in the (possibly Nobel-prize-winning)
idea of Heimburg and Jackson \cite{SolitonPropagation:2005}. They proposed that the action potential is essentially a sound wave (a soliton). ``However, there are several other questions that this has to answer like ion flow involvement in nerve signal propagation as stated by the \gls{HH} model and also the faster propagation in myelinated nerves than in unmyelinated'' \cite{ComparisonHHandSoliton:2010}.
If we consider that the ion flow means charge propagation in a membrane tube where the specific capacity depends on the thickness of the wall (the myelin sheath) \cite{VeghMembranePotential:2025}, the faster propagation is not mysterious anymore.
Given that
our (non-disciplinary) model natively connects charge and mass of ions~\cite{VeghNon-ordinaryLaws:2025}, and it can explain the missing phenomena.


The excellent textbook~\cite{PrinciplesNeuralScience:2013} separates the neuron membrane's states into resting and transient states. The statement that "the resting membrane potential results
from the separation of charge across the
cell membrane" is only half the truth; we tell the second half in section~\ref{sec:Single-RestingPotential}. 
We refine their statement "by discussing how resting ion channels
establish and maintain the resting potential"~\cite{PrinciplesNeuralScience:2013}, page 126.
As we show in section~\ref{sec:Single-RestingPotential}, two different physical mechanisms
operate in the resting and the transient states. A primary reason for misguiding neurophysiology was projecting the mechanism of the resting state to the transient state.


The book~\cite{PrinciplesNeuralScience:2013} asks the central questions, "How do ionic [i.e., electrical and chemical] gradients contribute to the resting membrane potential? What prevents
the ionic gradients from dissipating by diffusion of
ions across the membrane through the resting channels?"
However, it leaves them essentially open by giving only a qualitative answer, discussing only membrane permeability without explaining how the resting potential is created and regulated through charging and polarization.
%
We demonstrate in section~\ref{sec:Single-RestingPotential} that classical methods of electricity enable us to calculate the membrane's potential resulting from charge separation and polarization.
%One can make some physically plausible assumptions to provide numerical figures.
Our results underpin that "the crucial system in biology isn't a molecule or a molecular class whatsoever, but the interface created by biomolecules in water"~\cite{LivingSystemPhysics:2021}. However, we add that mainly
physical processes and speed gradients control the processes.
More precisely, inseparable thermodynamic and electrical processes keep the potential set by the inseparable electrical and thermodynamic backbone defined by the lipid/protein structure of the cell. It is one of the frequent cases when 
one effect implements a functionality (the backbone closely defines the frames of the operation), and the other (in this case, combined thermodynamics and electricity) corrects it when it implements a complex operational (dynamical) functionality:
"stimulated phase transitions enable the phase-dependent processes to replace each other ... one process to build and the other
to correct"~\cite{BiologicalConservationLaw:2017}.


