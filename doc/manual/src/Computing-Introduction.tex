% The introduction to computing

\section{Introduction\label{sec:Computing-Introduction}}

Here \textit{computing} is handled in a broader sense: information processing \textit{in any implementation}.
It covers conventional computing, biomorphic computing,  biological (neural) computing, and computing relating, among others, (the technology of) artificial intelligence.
The computing objects use both their inputs and their internal state to calculate
their output. The time-aware computing means to consider that \textit{computing means both processing the available data
and delivering data to and from the computing object}.
Furthermore, that those operations must be synchronized
(and in this way they block each other); and that not only that those processes need time,
but \textit{the inputs, the output and the internal states all have their temporal behavior}.
We show that taking into account that temporal dependence explicitly,
leads to considerable differences in their behavior as opposed with the behavior
expected based on the time-unaware description. Please take care when reading.
The text is, of course, computing-oriented, so it uses words processor, core, thread,
hardware thread, memory, etc. However, it uses them in a slightly different way,
in a different meaning. So, please read the corresponding manual, or skim it at least,
before going into details. The approach we take seems to be overly complicated,
but it is needed to build a more effective and capable computing. It majorly simplifies
modern many-thread computing, but its real advantage manifest in large-scale computing.

Technical sciences (mainly electronics and computing science)
have developed to the level where elementary electronic components in number comparable
to the elementary components of the 
CNS
%\gls{CNS}
can be assembled.
Those large systems attempt to resemble each other. On the one side,
biology inspires huge electronical systems (from 
%\gls{HPC}
HPC to  %\gls{ANN}
ANN
).
On the other side, electronic systems (mainly large-scale computers,
but also special-purpose electronic simulators)
attemp to imitate brain-like biological systems, with goals ranging from simulating the dynamics
of molecular processes to creating artificial intelligence.
Furthermore, there are attemps to combine and interface them.
