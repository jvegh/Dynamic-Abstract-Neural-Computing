
\begin{schemamodule}{LaTeXML}
\item[\textit{Included}]\moduleref{LaTeXML-common}, \moduleref{LaTeXML-inline}, \moduleref{LaTeXML-block}, \moduleref{LaTeXML-misc}, \moduleref{LaTeXML-meta}, \moduleref{LaTeXML-para}, \moduleref{LaTeXML-math}, \moduleref{LaTeXML-tabular}, \moduleref{LaTeXML-picture}, \moduleref{LaTeXML-structure}, \moduleref{LaTeXML-bib}
\patterndef{Inline.model}{Combined model for inline content.
}{\item[\textit{Content}:] (\typename{text} ~\textbar~ \patternref{Inline.class} ~\textbar~ \patternref{Misc.class} ~\textbar~ \patternref{Meta.class})\textsuperscript{*}\item[\textit{Used by}:] \elementref{acknowledgements}, \elementref{anchor}, \elementref{bib-data}, \elementref{bib-date}, \elementref{bib-edition}, \elementref{bib-extract}, \elementref{bib-identifier}, \elementref{bib-key}, \elementref{bib-language}, \elementref{bib-links}, \elementref{bib-note}, \elementref{bib-organization}, \elementref{bib-part}, \elementref{bib-place}, \elementref{bib-publisher}, \elementref{bib-review}, \elementref{bib-status}, \elementref{bib-subtitle}, \elementref{bib-title}, \elementref{bib-type}, \elementref{bib-url}, \elementref{bibrefphrase}, \elementref{cite}, \elementref{classification}, \elementref{constraint}, \elementref{contact}, \elementref{date}, \elementref{del}, \elementref{emph}, \elementref{givenname}, \elementref{glossaryphrase}, \elementref{glossaryref}, \elementref{indexphrase}, \elementref{indexrefs}, \elementref{indexsee}, \elementref{keywords}, \elementref{lineage}, \elementref{p}, \elementref{personname}, \elementref{ref}, \elementref{sub}, \elementref{subtitle}, \elementref{sup}, \elementref{surname}, \elementref{tag}, \elementref{td}, \elementref{text}, \elementref{verbatim}}

\patterndef{Block.model}{Combined model for physical block-level content.
}{\item[\textit{Content}:] (\patternref{Block.class} ~\textbar~ \patternref{Misc.class} ~\textbar~ \patternref{Meta.class})\textsuperscript{*}\item[\textit{Used by}:] \elementref{abstract}, \elementref{block}, \elementref{figure}, \elementref{float}, \elementref{inline-block}, \elementref{para}, \elementref{quote}, \elementref{table}}

\patterndef{Flow.model}{Combined model for general flow containing both inline and block level content.
}{\item[\textit{Content}:] (\typename{text} ~\textbar~ \patternref{Inline.class} ~\textbar~ \patternref{Block.class} ~\textbar~ \patternref{Misc.class} ~\textbar~ \patternref{Meta.class})\textsuperscript{*}\item[\textit{Used by}:] \elementref{bibblock}, \elementref{note}, \elementref{rdf}}

\patterndef{Para.model}{Combined model for logical block-level context.
}{\item[\textit{Content}:] (\patternref{Para.class} ~\textbar~ \patternref{Meta.class})\textsuperscript{*}\item[\textit{Used by}:] \patternref{appendix.body.class}, \patternref{bibliography.body.class}, \patternref{chapter.body.class}, \patternref{document.body.class}, \patternref{glossary.body.class}, \patternref{index.body.class}, \patternref{paragraph.body.class}, \patternref{part.body.class}, \patternref{section.body.class}, \patternref{sidebar.body.class}, \patternref{slide.body.class}, \patternref{subparagraph.body.class}, \patternref{subsection.body.class}, \patternref{subsubsection.body.class}, \elementref{inline-logical-block}, \elementref{item}, \elementref{logical-block}, \elementref{proof}, \elementref{theorem}}

\item[\textit{Start}]\textbf{==}\ \elementref{document}
\end{schemamodule}
\begin{schemamodule}{LaTeXML-common}
\patterndef{Inline.class}{All strictly inline elements.
}{\item[\textit{Expansion}:] (\elementref{text} ~\textbar~ \elementref{emph} ~\textbar~ \elementref{del} ~\textbar~ \elementref{sub} ~\textbar~ \elementref{sup} ~\textbar~ \elementref{glossaryref} ~\textbar~ \elementref{rule} ~\textbar~ \elementref{anchor} ~\textbar~ \elementref{ref} ~\textbar~ \elementref{cite} ~\textbar~ \elementref{bibref} ~\textbar~ \elementref{Math})\item[\textit{Used by}:] \patternref{Flow.model}, \patternref{Inline.model}, \elementref{XMText}, \elementref{caption}, \elementref{clippath}, \elementref{g}, \elementref{inline-item}, \elementref{listingline}, \elementref{picture}, \elementref{title}, \elementref{toccaption}, \elementref{toctitle}}

\patterndef{Block.class}{All `physical' block elements. 
A physical block is typically displayed as a block, but
may not constitute a complete logical unit.
}{\item[\textit{Expansion}:] (\elementref{p} ~\textbar~ \elementref{equation} ~\textbar~ \elementref{equationgroup} ~\textbar~ \elementref{quote} ~\textbar~ \elementref{block} ~\textbar~ \elementref{listing} ~\textbar~ \elementref{itemize} ~\textbar~ \elementref{enumerate} ~\textbar~ \elementref{description} ~\textbar~ \elementref{pagination})\item[\textit{Used by}:] \patternref{Block.model}, \patternref{Flow.model}, \elementref{titlepage}}

\patterndef{Misc.class}{Additional miscellaneous elements that can appear in
both inline and block contexts.
}{\item[\textit{Expansion}:] (\elementref{inline-itemize} ~\textbar~ \elementref{inline-enumerate} ~\textbar~ \elementref{inline-description} ~\textbar~ \elementref{inline-block} ~\textbar~ \elementref{verbatim} ~\textbar~ \elementref{break} ~\textbar~ \elementref{graphics} ~\textbar~ \texttt{svg:svg} ~\textbar~ \elementref{rawhtml} ~\textbar~ \elementref{rawliteral} ~\textbar~ \elementref{ERROR} ~\textbar~ \elementref{inline-logical-block} ~\textbar~ \elementref{tabular} ~\textbar~ \elementref{picture} ~\textbar~ \elementref{inline-sectional-block})\item[\textit{Used by}:] \patternref{Block.model}, \patternref{Flow.model}, \patternref{Inline.model}, \elementref{XMText}, \elementref{caption}, \elementref{clippath}, \elementref{creator}, \elementref{equation}, \elementref{g}, \elementref{inline-item}, \elementref{listingline}, \elementref{picture}, \elementref{title}, \elementref{toccaption}, \elementref{toctitle}}

\patterndef{Para.class}{All logical block level elements.
A logical block typically contains one or more physical block elements.
For example, a common situation might be \elementref{p},\elementref{equation},\elementref{p},
where the entire sequence comprises a single sentence.
}{\item[\textit{Expansion}:] (\elementref{para} ~\textbar~ \elementref{logical-block} ~\textbar~ \elementref{theorem} ~\textbar~ \elementref{proof} ~\textbar~ \elementref{figure} ~\textbar~ \elementref{table} ~\textbar~ \elementref{float} ~\textbar~ \elementref{pagination} ~\textbar~ \elementref{TOC})\item[\textit{Used by}:] \patternref{BackMatter.class}, \patternref{Para.model}}

\patterndef{Meta.class}{All metadata elements, typically representing hidden data.
}{\item[\textit{Expansion}:] (\elementref{note} ~\textbar~ \elementref{declare} ~\textbar~ \elementref{indexmark} ~\textbar~ \elementref{glossarydefinition} ~\textbar~ \elementref{rdf} ~\textbar~ \elementref{resource} ~\textbar~ \elementref{navigation})\item[\textit{Used by}:] \patternref{BackMatter.class}, \patternref{Block.model}, \patternref{Flow.model}, \patternref{Inline.model}, \patternref{Para.model}, \elementref{caption}, \elementref{clippath}, \elementref{document}, \elementref{equation}, \elementref{equationgroup}, \elementref{g}, \elementref{inline-item}, \elementref{listingline}, \elementref{picture}, \elementref{title}, \elementref{toccaption}, \elementref{toctitle}}

\patterndef{Length.type}{The type for attributes specifying a length.
Should be a number followed by a length, typically px.
NOTE: To be narrowed later.
}{\item[\textit{Content}:] \typename{text}\item[\textit{Used by}:] \patternref{Fontable.attributes}, \patternref{Positionable.attributes}, \patternref{Transformable.attributes}, \elementref{XMArray}, \elementref{equationgroup}, \elementref{item}, \elementref{tabular}, \elementref{td}}

\patterndef{Color.type}{The type for attributes specifying a color.
NOTE: To be narrowed later.
}{\item[\textit{Content}:] \typename{text}}

\patterndef{Foreign.attributes}{Attributes in foreign namespaces
}{\attrdef{*:*}{}{\typename{text}}\item[\textit{Exluding attribute }]\texttt{xml:*}\item[\textit{Used by}:] \patternref{Common.attributes}}

\patterndef{Common.attributes}{Attributes shared by ALL elements.
}{\item[\textit{Attributes}:] \patternref{Foreign.attributes}, \patternref{RDF.attributes}\attrdef{class}{a space separated list of tokens, as in CSS.
The \attr{class} can be used to add differentiate different instances of elements
without introducing new element declarations.
However, this generally shouldn't be used for deep semantic distinctions.
This attribute is carried over to HTML and can be used for CSS selection.
[Note that the default XSLT stylesheets for html and xhtml
add the latexml element names to the class of html elements
for more convenience in using CSS.]
}{\typename{NMTOKENS}}\attrdef{cssstyle}{CSS styling rules.
These will only be effective when the target system supports CSS.
}{\typename{text}}\attrdef{xml:lang}{Language attribute
}{\typename{text}}\item[\textit{Used by}:] \patternref{Sectional.attributes}, \elementref{ERROR}, \elementref{MathBranch}, \elementref{MathFork}, \elementref{Math}, \elementref{TOC}, \elementref{XMApp}, \elementref{XMArg}, \elementref{XMArray}, \elementref{XMCell}, \elementref{XMDual}, \elementref{XMHint}, \elementref{XMRef}, \elementref{XMRow}, \elementref{XMText}, \elementref{XMTok}, \elementref{XMWrap}, \elementref{XMath}, \elementref{abstract}, \elementref{acknowledgements}, \elementref{anchor}, \elementref{arc}, \elementref{bezier}, \elementref{bib-data}, \elementref{bib-date}, \elementref{bib-edition}, \elementref{bib-extract}, \elementref{bib-identifier}, \elementref{bib-key}, \elementref{bib-language}, \elementref{bib-links}, \elementref{bib-name}, \elementref{bib-note}, \elementref{bib-organization}, \elementref{bib-part}, \elementref{bib-place}, \elementref{bib-publisher}, \elementref{bib-related}, \elementref{bib-review}, \elementref{bib-status}, \elementref{bib-subtitle}, \elementref{bib-title}, \elementref{bib-type}, \elementref{bib-url}, \elementref{bibentry}, \elementref{bibitem}, \elementref{biblist}, \elementref{bibref}, \elementref{bibrefphrase}, \elementref{block}, \elementref{break}, \elementref{caption}, \elementref{circle}, \elementref{cite}, \elementref{classification}, \elementref{clip}, \elementref{clippath}, \elementref{contact}, \elementref{creator}, \elementref{curve}, \elementref{date}, \elementref{del}, \elementref{description}, \elementref{dots}, \elementref{ellipse}, \elementref{emph}, \elementref{enumerate}, \elementref{equation}, \elementref{equationgroup}, \elementref{figure}, \elementref{float}, \elementref{g}, \elementref{glossarydefinition}, \elementref{glossaryentry}, \elementref{glossarylist}, \elementref{glossaryphrase}, \elementref{glossaryref}, \elementref{graphics}, \elementref{grid}, \elementref{indexentry}, \elementref{indexlist}, \elementref{indexmark}, \elementref{indexphrase}, \elementref{indexrefs}, \elementref{indexsee}, \elementref{inline-block}, \elementref{inline-description}, \elementref{inline-enumerate}, \elementref{inline-item}, \elementref{inline-itemize}, \elementref{inline-logical-block}, \elementref{item}, \elementref{itemize}, \elementref{keywords}, \elementref{line}, \elementref{listing}, \elementref{listingline}, \elementref{logical-block}, \elementref{navigation}, \elementref{note}, \elementref{p}, \elementref{pagination}, \elementref{para}, \elementref{parabola}, \elementref{path}, \elementref{personname}, \elementref{picture}, \elementref{polygon}, \elementref{proof}, \elementref{quote}, \elementref{rdf}, \elementref{rect}, \elementref{ref}, \elementref{resource}, \elementref{rule}, \elementref{sub}, \elementref{subtitle}, \elementref{sup}, \elementref{table}, \elementref{tabular}, \elementref{tag}, \elementref{tbody}, \elementref{td}, \elementref{text}, \elementref{tfoot}, \elementref{thead}, \elementref{theorem}, \elementref{title}, \elementref{toccaption}, \elementref{tocentry}, \elementref{toclist}, \elementref{toctitle}, \elementref{tr}, \elementref{verbatim}, \elementref{wedge}}

\patterndef{ID.attributes}{Attributes for elements that can be cross-referenced
from inside or outside the document.
}{\attrdef{xml:id}{the unique identifier of the element, 
usually generated automatically by the latexml.
}{\typename{ID}}\attrdef{fragid}{a "fragment identifier" derived from the \attr{xml:id} 
relative to a page split from the complete document.
This is used internally and may go away some day.
}{\typename{text}}\item[\textit{Used by}:] \patternref{Labelled.attributes}, \elementref{ERROR}, \elementref{Math}, \elementref{XMApp}, \elementref{XMArg}, \elementref{XMArray}, \elementref{XMCell}, \elementref{XMDual}, \elementref{XMHint}, \elementref{XMRef}, \elementref{XMRow}, \elementref{XMText}, \elementref{XMTok}, \elementref{XMWrap}, \elementref{XMath}, \elementref{anchor}, \elementref{bibentry}, \elementref{bibitem}, \elementref{block}, \elementref{declare}, \elementref{del}, \elementref{description}, \elementref{emph}, \elementref{enumerate}, \elementref{glossaryentry}, \elementref{glossarylist}, \elementref{graphics}, \elementref{indexentry}, \elementref{indexlist}, \elementref{inline-block}, \elementref{inline-description}, \elementref{inline-enumerate}, \elementref{inline-itemize}, \elementref{inline-logical-block}, \elementref{itemize}, \elementref{logical-block}, \elementref{p}, \elementref{para}, \elementref{picture}, \elementref{quote}, \elementref{sub}, \elementref{sup}, \elementref{tabular}, \elementref{td}, \elementref{text}, \elementref{tr}, \elementref{verbatim}}

\patterndef{IDREF.attributes}{Attributes for elements that can cross-reference other elements.
}{\attrdef{idref}{the identifier of the referred-to element.
}{\typename{IDREF}}\item[\textit{Used by}:] \patternref{Refable.attributes}, \elementref{XMRef}, \elementref{bibref}, \elementref{glossaryphrase}}

\patterndef{Listable.attributes}{Attributes for items that can be put into lists, like index, table of contents.
}{\attrdef{inlist}{Records which lists, such as toc=table of contents,..., this object could be listed in.
Space separated set of toc, lof, lot, etc.
}{\typename{text}}\item[\textit{Used by}:] \patternref{Labelled.attributes}, \elementref{bibref}, \elementref{cite}, \elementref{glossarydefinition}, \elementref{glossaryref}, \elementref{indexmark}}

\patterndef{Listing.attributes}{Attributes for items that create lists, like index, table of contents.
}{\attrdef{lists}{Records which lists, such as toc(=table of contents), this object should create a list of.
Space separated set of toc, lof, lot, etc.
}{\typename{text}}\item[\textit{Used by}:] \elementref{bibliography}, \elementref{glossary}, \elementref{index}}

\patterndef{Labelled.attributes}{Attributes for elements that can be labelled from within LaTeX.
These attributes deal with assigning a label  and generating cross references.
The label migrates to an \attr{xml:id} and \attr{href} and the element
can serve as a hypertext target.
}{\item[\textit{Attributes}:] \patternref{ID.attributes}, \patternref{Listable.attributes}\attrdef{labels}{Records the various labels that LaTeX uses for crossreferencing.
(note that \cs{label} can associate more than one label with an object!)
It consists of space separated labels for the element.
The \cs{label} macro provides the label prefixed by \texttt{LABEL:};
Spaces in a label are replaced by underscore.
Other mechanisms (like acro?) might use other prefixes (but \texttt{ID:} is reserved!)
}{\typename{text}}\item[\textit{Used by}:] \patternref{Sectional.attributes}, \elementref{equation}, \elementref{equationgroup}, \elementref{figure}, \elementref{float}, \elementref{inline-item}, \elementref{item}, \elementref{listing}, \elementref{listingline}, \elementref{note}, \elementref{proof}, \elementref{table}, \elementref{theorem}}

\patterndef{Refable.attributes}{Attributes for elements that can be referred to from within LaTeX.
Such elements may serve as the starting point of a hypertext link.
The reference can be made using \attr{label}, \attr{xml:id} or \attr{href};
these attributes will be converted, as needed, from the former to the latter.
}{\item[\textit{Attributes}:] \patternref{IDREF.attributes}\attrdef{labelref}{reference to a LaTeX labelled object;
See the \attr{labels} attribute of \patternref{Labelled.attributes}.
}{\typename{text}}\attrdef{href}{reference to an arbitrary url.
}{\typename{text}}\item[\textit{Used by}:] \elementref{bib-identifier}, \elementref{bib-review}, \elementref{bib-url}, \elementref{contact}, \elementref{glossaryref}, \elementref{personname}, \elementref{ref}}

\patterndef{Fontable.attributes}{Attributes for elements that contain (indirectly) text whose font can be specified.
}{\attrdef{font}{Indicates the font to use. It consists of a space separated sequence
of values representing the
family (\texttt{serif}, \texttt{sansserif}, \texttt{math}, \texttt{typewriter},
    \texttt{caligraphic}, \texttt{fraktur}, \texttt{script}, \ldots),
series (\texttt{medium}, \texttt{bold}, \ldots),
and shape (\texttt{upright}, \texttt{italic}, \texttt{slanted}, \texttt{smallcaps}, \ldots).
Only the values differing from the current context are given.
Each component is open-ended, for extensibility; it is thus unclear
whether unknown values specify family, series or shape.
In postprocessing, these values are carried to the \attr{class} attribute,
and can thus be effected by CSS.
}{\typename{text}}\attrdef{fontsize}{Indicates the text size to use, as a length, as in CSS.
Normally, this should be a percentage value relative to the containing element.
}{\patternref{Length.type}}\item[\textit{Used by}:] \elementref{XMTok}, \elementref{caption}, \elementref{del}, \elementref{emph}, \elementref{glossaryref}, \elementref{ref}, \elementref{text}, \elementref{title}, \elementref{verbatim}}

\patterndef{Colorable.attributes}{Attributes for elements that draw something, text or otherwise, that can be colored.
}{\attrdef{color}{the color to use (for foreground material); any CSS compatible color specification.
In postprocessing, these values are carried to the \attr{class} attribute,
and can thus be effected by CSS.
}{\typename{text}}\attrdef{opacity}{the opacity of foreground material; a number between 0 and 1.
}{\typename{float}}\item[\textit{Used by}:] \elementref{XMApp}, \elementref{XMTok}, \elementref{caption}, \elementref{del}, \elementref{emph}, \elementref{glossaryref}, \elementref{ref}, \elementref{rule}, \elementref{text}, \elementref{title}, \elementref{verbatim}}

\patterndef{Backgroundable.attributes}{Attributes for elements that take up space and make sense to have a background color.
This is independent of the colors of any things that it may draw.
}{\attrdef{backgroundcolor}{the color to use for the background of the element; any CSS compatible color specification.
In postprocessing, these values are carried to the \attr{class} attribute,
and can thus be effected by CSS; the background will presumably
correspond to a bounding rectangle, but is determined by the CSS rendering engine.
}{\typename{text}}\attrdef{framed}{the kind of frame or outline for the box.
}{(\attrval{rectangle} ~\textbar~ \attrval{underline} ~\textbar~ \typename{text})}\attrdef{framecolor}{the color of the frame or outlie for the box.
}{\typename{text}}\item[\textit{Used by}:] \patternref{Sectional.attributes}, \elementref{Math}, \elementref{XMApp}, \elementref{XMCell}, \elementref{XMRow}, \elementref{XMText}, \elementref{XMTok}, \elementref{XMWrap}, \elementref{block}, \elementref{caption}, \elementref{constraint}, \elementref{del}, \elementref{description}, \elementref{emph}, \elementref{enumerate}, \elementref{equation}, \elementref{equationgroup}, \elementref{figure}, \elementref{float}, \elementref{glossaryref}, \elementref{inline-block}, \elementref{inline-description}, \elementref{inline-enumerate}, \elementref{inline-item}, \elementref{inline-itemize}, \elementref{inline-logical-block}, \elementref{item}, \elementref{itemize}, \elementref{listing}, \elementref{logical-block}, \elementref{p}, \elementref{para}, \elementref{proof}, \elementref{quote}, \elementref{ref}, \elementref{rule}, \elementref{table}, \elementref{tabular}, \elementref{tag}, \elementref{tbody}, \elementref{td}, \elementref{text}, \elementref{tfoot}, \elementref{thead}, \elementref{theorem}, \elementref{title}, \elementref{tr}, \elementref{verbatim}}

\patterndef{Positionable.attributes}{Attributes shared by low-level, generic inline and block elements
that can be sized or shifted.
}{\attrdef{width}{the desired width of the box
}{\patternref{Length.type}}\attrdef{height}{the desired height of the box
}{\patternref{Length.type}}\attrdef{depth}{the desired depth of the box
}{\patternref{Length.type}}\attrdef{xoffset}{horizontal shift the position of the box.
}{\patternref{Length.type}}\attrdef{yoffset}{vertical shift the position of the box.
}{\patternref{Length.type}}\attrdef{align}{alignment of material within the box.
}{(\attrval{left} ~\textbar~ \attrval{center} ~\textbar~ \attrval{right} ~\textbar~ \attrval{justified})}\attrdef{vattach}{specifies which line of the box is aligned to the baseline of the containing object.
The default is baseline. 
}{(\attrval{top} ~\textbar~ \attrval{middle} ~\textbar~ \attrval{bottom} ~\textbar~ \attrval{baseline})}\attrdef{float}{the horizontal floating placement parameter that determines where the object is displayed.
}{(\attrval{right} ~\textbar~ \attrval{left} ~\textbar~ \typename{text})}\item[\textit{Used by}:] \patternref{XMath.attributes}, \elementref{block}, \elementref{figure}, \elementref{float}, \elementref{inline-block}, \elementref{inline-logical-block}, \elementref{listing}, \elementref{logical-block}, \elementref{p}, \elementref{para}, \elementref{rule}, \elementref{table}, \elementref{text}}

\patterndef{Transformable.attributes}{Attributes shared by (hopefully few) elements that can be transformed.
Such elements should also have Positionable.attributes.
Transformation order of an individual element is assumed to be
translate, scale, rotate; wrap elements to achieve different orders.
Attributes \attr{innerwidth}, \attr{innerheight} and \attr{innerdepth} describe
the size of the contents of the element before transformation;
The result size would be encoded in Positional.attributes.
}{\attrdef{xtranslate}{horizontal shift the position of the inner element.
}{\patternref{Length.type}}\attrdef{ytranslate}{vertical shift the position of the inner element.
}{\patternref{Length.type}}\attrdef{xscale}{horizontal scaling of the inner element.
}{\typename{text}}\attrdef{yscale}{vertical scalign of the inner element.
}{\typename{text}}\attrdef{angle}{the rotation angle, counter-clockwise, in degrees.
}{\typename{text}}\attrdef{innerwidth}{the expected width of the contents of the inner element
}{\patternref{Length.type}}\attrdef{innerheight}{the expected height of the contents of the inner element
}{\patternref{Length.type}}\attrdef{innerdepth}{the expected depth of the contents of the inner element
}{\patternref{Length.type}}\item[\textit{Used by}:] \elementref{figure}, \elementref{float}, \elementref{g}, \elementref{inline-block}, \elementref{table}}

\patterndef{Imageable.attributes}{Attributes for elements that may be converted to image form
during postprocessing, such as math, graphics, pictures, etc.
Note that these attributes are generally not filled in until postprocessing,
but that they could be init
}{\attrdef{imagesrc}{the file, possibly generated from other data.
}{\typename{anyURI}}\attrdef{imagewidth}{the width in pixels of \attr{imagesrc}.
}{\typename{nonNegativeInteger}}\attrdef{imageheight}{the height in pixels of \attr{imagesrc}.
Note that, unlike \TeX, this is the total height, including the depth (if any).
}{\typename{nonNegativeInteger}}\attrdef{imagedepth}{the depth in pixels of \attr{imagesrc}, being the location of the
baseline of the content shown in the image.
When displayed inilne, an image with a positive \attr{depth}
should be shifted down relative to the baseline of neighboring material.
}{\typename{integer}}\attrdef{description}{a description of the image
}{\typename{text}}\item[\textit{Used by}:] \elementref{Math}, \elementref{graphics}, \elementref{picture}}

\patterndef{RDF.attributes}{Attributes for RDFa (Resource Description Framework),
following RDFa Core 1.1 \url{http://www.w3.org/TR/rdfa-syntax/}.
The following descriptions give a short overview of the usage of the attributes,
but see the specification for the complete details, which are sometimes complex.
}{\attrdef{vocab}{indicates the default vocabulary
(generally should be managed by LaTeXML and only appear on root node)
}{\typename{text}}\attrdef{prefix}{specifies a mapping between CURIE prefixes and IRI (URI).
(generally should be managed by LaTeXML and only appear on root node)
}{\typename{text}}\attrdef{about}{indicates the subject for predicates appearing on the same or descendant nodes.
}{\typename{text}}\attrdef{aboutlabelref}{gives the label for the document element that serves as the subject;
it will be converted to \attr{aboutidref} and \attr{about} during post-processing.
}{\typename{text}}\attrdef{aboutidref}{gives the id for the document element that serves as the subject;
it will be converted to \attr{about} during post-processing.
}{\typename{text}}\attrdef{resource}{indicates the subject for predicates appearing on descendant nodes,
and also indicates the object for predicates
when \attr{property} appears on the same node,
or when \attr{rel} or \attr{rev} appears on an ancestor.
}{\typename{text}}\attrdef{resourcelabelref}{gives the label for the document element that serves as the resource object;
it will be converted to \attr{resourceidref} and \attr{resource} during post-processing.
}{\typename{text}}\attrdef{resourceidref}{gives the id for the document element that serves as the resource object;
it will be converted to \attr{resource} during post-processing.
}{\typename{text}}\attrdef{property}{indicates the predicate and
asserts that the subject is related by that predicate to the object.
The subject is determined by one of \attr{about} on same node,
\attr{resource} or \attr{typeof} on an ancestor node, or by the document root.
The object is determined by one of \attr{resource}, \attr{href}, \attr{content}
or \attr{typeof} on the same node, or by the text content of the node.
}{\typename{text}}\attrdef{rel}{indicates the predicate exactly as \attr{property} except that it can
assert multiple RDF triples where the objects are the nearest descendent \attr{resource}s.
}{\typename{text}}\attrdef{rev}{indicates the predicate exactly as \attr{rel} except that it indicates
the reverse relationship (with subject and object swapped).
}{\typename{text}}\attrdef{typeof}{indicates the type of the \attr{resource} and thus implicitly asserts a relation
that the object has the given type.
Additionally, if no \attr{resource} was given on the same node,
indicates an anonymous subject and or object exactly as \attr{resource}
}{\typename{text}}\attrdef{datatype}{indicates the datatype of the target resource;
}{\typename{text}}\attrdef{content}{indicates the content of the property to be used as the object,
to be used instead of the content of the element itself;
}{\typename{text}}\item[\textit{Used by}:] \patternref{Common.attributes}}

\patterndef{Data.attributes}{Attributes for raw data storage
}{\attrdef{data}{the data itself
}{\typename{text}}\attrdef{datamimetype}{the MIME type of the data
}{\typename{text}}\attrdef{dataencoding}{the encoding of the data (either empty, base64, or )
}{\typename{text}}\attrdef{dataname}{the suggested file name of the data
}{\typename{text}}\item[\textit{Used by}:] \elementref{figure}, \elementref{float}, \elementref{listing}, \elementref{proof}, \elementref{table}}

\end{schemamodule}
\begin{schemamodule}{LaTeXML-inline}
\patternadd{Inline.class}{The inline module defines basic inline elements used throughout.
}{\item[\textbar=] (\elementref{text} ~\textbar~ \elementref{emph} ~\textbar~ \elementref{del} ~\textbar~ \elementref{sub} ~\textbar~ \elementref{sup} ~\textbar~ \elementref{glossaryref} ~\textbar~ \elementref{rule} ~\textbar~ \elementref{anchor} ~\textbar~ \elementref{ref} ~\textbar~ \elementref{cite} ~\textbar~ \elementref{bibref})}

\elementdef{text}{General container for styled text.
Attributes cover a variety of styling and position shifting properties.
}{\item[\textit{Attributes}:] \patternref{Common.attributes}, \patternref{ID.attributes}, \patternref{Positionable.attributes}, \patternref{Fontable.attributes}, \patternref{Colorable.attributes}, \patternref{Backgroundable.attributes}\item[\textit{Content}:] \patternref{Inline.model}\item[\textit{Used by}:] \patternref{Inline.class}, \elementref{MathFork}, \elementref{declare}, \elementref{equation}}



\elementdef{emph}{Emphasized text.
}{\item[\textit{Attributes}:] \patternref{Common.attributes}, \patternref{ID.attributes}, \patternref{Fontable.attributes}, \patternref{Colorable.attributes}, \patternref{Backgroundable.attributes}\item[\textit{Content}:] \patternref{Inline.model}\item[\textit{Used by}:] \patternref{Inline.class}}



\elementdef{del}{Deleted text.
}{\item[\textit{Attributes}:] \patternref{Common.attributes}, \patternref{ID.attributes}, \patternref{Fontable.attributes}, \patternref{Colorable.attributes}, \patternref{Backgroundable.attributes}\item[\textit{Content}:] \patternref{Inline.model}\item[\textit{Used by}:] \patternref{Inline.class}}



\elementdef{sub}{Textual subscript text.
}{\item[\textit{Attributes}:] \patternref{Common.attributes}, \patternref{ID.attributes}\item[\textit{Content}:] \patternref{Inline.model}\item[\textit{Used by}:] \patternref{Inline.class}}



\elementdef{sup}{Textual superscript text.
}{\item[\textit{Attributes}:] \patternref{Common.attributes}, \patternref{ID.attributes}\item[\textit{Content}:] \patternref{Inline.model}\item[\textit{Used by}:] \patternref{Inline.class}}



\elementdef{glossaryref}{Represents the usage of a term from a glossary.
}{\item[\textit{Attributes}:] \patternref{Common.attributes}, \patternref{Refable.attributes}, \patternref{Listable.attributes}, \patternref{Fontable.attributes}, \patternref{Colorable.attributes}, \patternref{Backgroundable.attributes}\attrdef{key}{should be used to identifier used for the glossaryref.
}{\typename{text}}\attrdef{title}{gives a expanded form of the glossaryref (unused?),
}{\typename{text}}\attrdef{show}{a pattern encoding how the text content should be filled in during
postprocessing, if it is empty.
It consists of the words
  \texttt{type} (standing for the object type, eg. Ch.),
  \texttt{refnum}, \texttt{typerefnum} and \texttt{title}
or \texttt{toctitle} (for the shortform of the title)
mixed with arbitrary characters.
}{\typename{text}}\item[\textit{Content}:] \patternref{Inline.model}\item[\textit{Used by}:] \patternref{Inline.class}}



\elementdef{rule}{A Rule.
}{\item[\textit{Attributes}:] \patternref{Common.attributes}, \patternref{Positionable.attributes}, \patternref{Colorable.attributes}, \patternref{Backgroundable.attributes}\item[\textit{Content}:] \typename{empty}\item[\textit{Used by}:] \patternref{Inline.class}}



\elementdef{ref}{A hyperlink reference to some other object. 
When converted to HTML, the content would be the content of the anchor.
The destination can be specified by one of the 
attributes \attr{labelref}, \attr{idref} or \attr{href};
Missing fields will usually be filled in during postprocessing,
based on data extracted from the document(s).
}{\item[\textit{Attributes}:] \patternref{Common.attributes}, \patternref{Refable.attributes}, \patternref{Fontable.attributes}, \patternref{Colorable.attributes}, \patternref{Backgroundable.attributes}\attrdef{show}{a pattern encoding how the text content should be filled in during
postprocessing, if it is empty.
It consists of the words
  \texttt{type} (standing for the object type, eg. Ch.),
  \texttt{refnum} and \texttt{title} (including type and refnum)
or \texttt{toctitle} (for the shortform of the title)
mixed with arbitrary characters.
}{\typename{text}}\attrdef{title}{gives a description of the target, not repeating the content,
used for accessibility or a tooltip in HTML.
Typically filled in by postprocessor.
}{\typename{text}}\attrdef{fulltitle}{gives a longer form description of the target,
useful when the link appears outside its original context, eg in navigation.
Typically filled in by postprocessor.
}{\typename{text}}\item[\textit{Content}:] \patternref{Inline.model}\item[\textit{Used by}:] \patternref{Inline.class}, \elementref{navigation}, \elementref{tocentry}}



\elementdef{anchor}{Inline anchor.
}{\item[\textit{Attributes}:] \patternref{Common.attributes}, \patternref{ID.attributes}\item[\textit{Content}:] \patternref{Inline.model}\item[\textit{Used by}:] \patternref{Inline.class}}



\elementdef{cite}{A container for a bibliographic citation. The model is inline to
allow arbitrary comments before and after the expected \elementref{bibref}(s)
which are the specific citation.
}{\item[\textit{Attributes}:] \patternref{Common.attributes}, \patternref{Listable.attributes}\item[\textit{Content}:] \patternref{Inline.model}\item[\textit{Used by}:] \patternref{Inline.class}}



\elementdef{bibref}{A bibliographic citation refering to a specific bibliographic item.
Postprocessing will turn this into an \elementref{ref} for the actual link.
}{\item[\textit{Attributes}:] \patternref{Common.attributes}, \patternref{IDREF.attributes}, \patternref{Listable.attributes}\attrdef{bibrefs}{a comma separated list of bibliographic keys.
(See the \attr{key} attribute of \elementref{bibitem} and \elementref{bibentry})
}{\typename{text}}\attrdef{show}{a pattern encoding how the text content (of an empty bibref) will be filled in.
Consists of strings \texttt{author}, \texttt{fullauthor}, \texttt{year},
\texttt{number} and \texttt{title}
(to be replaced by data from the bibliographic item)
mixed with arbitrary characters.
}{\typename{text}}\attrdef{separator}{separator between formatted references
}{\typename{text}}\attrdef{yyseparator}{separator between formatted years when duplicate authors are combined.
}{\typename{text}}\item[\textit{Content}:] \elementref{bibrefphrase}\textsuperscript{*}\item[\textit{Used by}:] \patternref{Inline.class}}



\elementdef{bibrefphrase}{A preceding or following phrase used in composing a bibliographic reference,
such as listing pages or chapter.
}{\item[\textit{Attributes}:] \patternref{Common.attributes}\item[\textit{Content}:] \patternref{Inline.model}\item[\textit{Used by}:] \elementref{bibref}}



\end{schemamodule}
\begin{schemamodule}{LaTeXML-block}
\patternadd{Block.class}{The block module defines the following `physical' block elements.
}{\item[\textbar=] (\elementref{p} ~\textbar~ \elementref{equation} ~\textbar~ \elementref{equationgroup} ~\textbar~ \elementref{quote} ~\textbar~ \elementref{block} ~\textbar~ \elementref{listing} ~\textbar~ \elementref{itemize} ~\textbar~ \elementref{enumerate} ~\textbar~ \elementref{description} ~\textbar~ \elementref{pagination})}

\patternadd{Misc.class}{These are inline forms of logical lists
(they are included in Misc since that has been the general strategy)
}{\item[\textbar=] (\elementref{inline-itemize} ~\textbar~ \elementref{inline-enumerate} ~\textbar~ \elementref{inline-description})}

\patterndef{EquationMeta.class}{Additional Metadata that can be present in equations.
}{\item[\textit{Content}:] \elementref{constraint}\item[\textit{Used by}:] \elementref{equation}, \elementref{equationgroup}}

\elementdef{p}{A physical paragraph.
}{\item[\textit{Attributes}:] \patternref{Common.attributes}, \patternref{ID.attributes}, \patternref{Positionable.attributes}, \patternref{Backgroundable.attributes}\item[\textit{Content}:] \patternref{Inline.model}\item[\textit{Used by}:] \patternref{Block.class}, \elementref{equationgroup}}



\elementdef{constraint}{A constraint upon an equation.
}{\item[\textit{Attributes}:] \patternref{Backgroundable.attributes}\attrdef{hidden}{}{\typename{boolean}}\item[\textit{Content}:] \patternref{Inline.model}\item[\textit{Used by}:] \patternref{EquationMeta.class}}



\elementdef{equation}{An Equation.  The model is just Inline which includes \elementref{Math},
the main expected ingredient.
However, other things can end up in display math, too, so we use Inline.
Note that tabular is here only because it's a common, if misguided, idiom;
the processor will lift such elements out of math, when possible
}{\item[\textit{Attributes}:] \patternref{Common.attributes}, \patternref{Labelled.attributes}, \patternref{Backgroundable.attributes}\item[\textit{Content}:] (\elementref{tags} ~\textbar~ \elementref{Math} ~\textbar~ \elementref{MathFork} ~\textbar~ \elementref{text} ~\textbar~ \patternref{Misc.class} ~\textbar~ \patternref{Meta.class} ~\textbar~ \patternref{EquationMeta.class})\textsuperscript{*}\item[\textit{Used by}:] \patternref{Block.class}, \elementref{equationgroup}, \elementref{listingline}}



\elementdef{equationgroup}{A group of equations, perhaps aligned (Though this is nowhere recorded).
}{\item[\textit{Attributes}:] \patternref{Common.attributes}, \patternref{Labelled.attributes}, \patternref{Backgroundable.attributes}\attrdef{rowsep}{the spacing between rows (equations, intertext,...)
}{\patternref{Length.type}}\item[\textit{Content}:] (\elementref{tags} ~\textbar~ \elementref{equationgroup} ~\textbar~ \elementref{equation} ~\textbar~ \elementref{p} ~\textbar~ \patternref{Meta.class} ~\textbar~ \patternref{EquationMeta.class})\textsuperscript{*}\item[\textit{Used by}:] \patternref{Block.class}, \elementref{equationgroup}, \elementref{listingline}}



\elementdef{MathFork}{A wrapper for Math that provides alternative,
but typically less semantically meaningful,
formatted representations.
The first child is the meaningful form,
the extra children provide formatted forms,
for example being table rows or cells arising from an eqnarray.
}{\item[\textit{Attributes}:] \patternref{Common.attributes}\item[\textit{Content}:] ((\elementref{Math} ~\textbar~ \elementref{text}), \elementref{MathBranch}\textsuperscript{*})\item[\textit{Used by}:] \elementref{equation}}



\elementdef{MathBranch}{A container for an alternatively formatted math representation.
}{\item[\textit{Attributes}:] \patternref{Common.attributes}\attrdef{format}{}{\typename{text}}\item[\textit{Content}:] (\elementref{Math} ~\textbar~ \elementref{tr} ~\textbar~ \elementref{td})\textsuperscript{*}\item[\textit{Used by}:] \elementref{MathFork}}



\elementdef{quote}{A quotation.
}{\item[\textit{Attributes}:] \patternref{Common.attributes}, \patternref{ID.attributes}, \patternref{Backgroundable.attributes}\attrdef{role}{The kind of quotation; could be something like verse, or translation
}{\typename{text}}\item[\textit{Content}:] \patternref{Block.model}\item[\textit{Used by}:] \patternref{Block.class}}



\elementdef{block}{A generic block (fallback).
}{\item[\textit{Attributes}:] \patternref{Common.attributes}, \patternref{ID.attributes}, \patternref{Positionable.attributes}, \patternref{Backgroundable.attributes}\item[\textit{Content}:] \patternref{Block.model}\item[\textit{Used by}:] \patternref{Block.class}}



\elementdef{listing}{An Listing, (without caption: see \elementref{float})
}{\item[\textit{Attributes}:] \patternref{Common.attributes}, \patternref{Labelled.attributes}, \patternref{Positionable.attributes}, \patternref{Backgroundable.attributes}, \patternref{Data.attributes}\item[\textit{Content}:] \elementref{listingline}\textsuperscript{*}\item[\textit{Used by}:] \patternref{Block.class}}



\elementdef{listingline}{a line in a listing
}{\item[\textit{Attributes}:] \patternref{Common.attributes}, \patternref{Labelled.attributes}\item[\textit{Content}:] (\elementref{tags}\textsuperscript{?}, (\typename{text} ~\textbar~ \patternref{Inline.class} ~\textbar~ \patternref{Misc.class} ~\textbar~ \patternref{Meta.class} ~\textbar~ \elementref{equation} ~\textbar~ \elementref{equationgroup})\textsuperscript{*})\item[\textit{Used by}:] \elementref{listing}}



\elementdef{itemize}{An itemized list.
}{\item[\textit{Attributes}:] \patternref{Common.attributes}, \patternref{ID.attributes}, \patternref{Backgroundable.attributes}\item[\textit{Content}:] \elementref{item}\textsuperscript{*}\item[\textit{Used by}:] \patternref{Block.class}}



\elementdef{enumerate}{An enumerated list.
}{\item[\textit{Attributes}:] \patternref{Common.attributes}, \patternref{ID.attributes}, \patternref{Backgroundable.attributes}\item[\textit{Content}:] \elementref{item}\textsuperscript{*}\item[\textit{Used by}:] \patternref{Block.class}}



\elementdef{description}{A description list. The \elementref{item}s within are expected to have a \elementref{tag}
which represents the term being described in each \elementref{item}.
}{\item[\textit{Attributes}:] \patternref{Common.attributes}, \patternref{ID.attributes}, \patternref{Backgroundable.attributes}\item[\textit{Content}:] \elementref{item}\textsuperscript{*}\item[\textit{Used by}:] \patternref{Block.class}}



\elementdef{item}{An item within a list (\elementref{itemize},\elementref{enumerate} or \elementref{description}).
}{\item[\textit{Attributes}:] \patternref{Common.attributes}, \patternref{Labelled.attributes}, \patternref{Backgroundable.attributes}\attrdef{itemsep}{the vertical spacing between items
}{\patternref{Length.type}}\item[\textit{Content}:] (\elementref{tags}\textsuperscript{?}, \patternref{Para.model})\item[\textit{Used by}:] \elementref{description}, \elementref{enumerate}, \elementref{itemize}}



\elementdef{inline-itemize}{An inline form of itemized list.
}{\item[\textit{Attributes}:] \patternref{Common.attributes}, \patternref{ID.attributes}, \patternref{Backgroundable.attributes}\item[\textit{Content}:] \elementref{inline-item}\textsuperscript{*}\item[\textit{Used by}:] \patternref{Misc.class}}



\elementdef{inline-enumerate}{An inline form of enumerated list.
}{\item[\textit{Attributes}:] \patternref{Common.attributes}, \patternref{ID.attributes}, \patternref{Backgroundable.attributes}\item[\textit{Content}:] \elementref{inline-item}\textsuperscript{*}\item[\textit{Used by}:] \patternref{Misc.class}}



\elementdef{inline-description}{An inline form of description list.
The \elementref{inline-item}s within are expected to have a \elementref{tags}
which represents the term being described in each \elementref{inline-item}.
}{\item[\textit{Attributes}:] \patternref{Common.attributes}, \patternref{ID.attributes}, \patternref{Backgroundable.attributes}\item[\textit{Content}:] \elementref{inline-item}\textsuperscript{*}\item[\textit{Used by}:] \patternref{Misc.class}}



\elementdef{inline-item}{An item within an inline list (\elementref{inline-itemize},\elementref{inline-enumerate}
or \elementref{inline-description}).
}{\item[\textit{Attributes}:] \patternref{Common.attributes}, \patternref{Labelled.attributes}, \patternref{Backgroundable.attributes}\item[\textit{Content}:] (\elementref{tags}\textsuperscript{?}, (\patternref{Inline.class} ~\textbar~ \patternref{Misc.class} ~\textbar~ \patternref{Meta.class})\textsuperscript{*})\item[\textit{Used by}:] \elementref{inline-description}, \elementref{inline-enumerate}, \elementref{inline-itemize}}



\elementdef{tags}{A container for one or more \elementref{tag}s.
At most one will have no \attr{role}, which would be the default display.
Other \elementref{tag} will have the role attribute for use in
special forms of referencing.
}{\item[\textit{Content}:] \elementref{tag}\textsuperscript{*}\item[\textit{Used by}:] \patternref{SectionalFrontMatter.class}, \elementref{bibitem}, \elementref{declare}, \elementref{equation}, \elementref{equationgroup}, \elementref{figure}, \elementref{float}, \elementref{inline-item}, \elementref{item}, \elementref{listingline}, \elementref{note}, \elementref{proof}, \elementref{table}, \elementref{theorem}}


\elementdef{tag}{A tag within an item indicating the term or bullet for a given item.
}{\item[\textit{Attributes}:] \patternref{Common.attributes}, \patternref{Backgroundable.attributes}\attrdef{role}{specifies the purpose this tag is used for: no value is default display
}{\typename{text}}\attrdef{open}{specifies an open delimiters used to display the tag.
}{\typename{text}}\attrdef{close}{specifies an close delimiters used to display the tag.
}{\typename{text}}\item[\textit{Content}:] \patternref{Inline.model}\item[\textit{Used by}:] \elementref{caption}, \elementref{tags}, \elementref{title}, \elementref{toccaption}, \elementref{toctitle}}



\elementdef{pagination}{A page break or related pagination information.
}{\item[\textit{Attributes}:] \patternref{Common.attributes}\attrdef{role}{what kind of pagination
}{\typename{text}}\item[\textit{Content}:] \typename{empty}\item[\textit{Used by}:] \patternref{Block.class}, \patternref{Para.class}}



\end{schemamodule}
\begin{schemamodule}{LaTeXML-misc}
\patternadd{Misc.class}{ Miscellaneous (Misc) elements are (typically) visible
elements which don't have clear inline or block character;
they can appear in both inline and block contexts.
}{\item[\textbar=] (\elementref{inline-block} ~\textbar~ \elementref{verbatim} ~\textbar~ \elementref{break} ~\textbar~ \elementref{graphics} ~\textbar~ \texttt{svg:svg} ~\textbar~ \elementref{rawhtml} ~\textbar~ \elementref{rawliteral} ~\textbar~ \elementref{ERROR})}

\elementdef{inline-block}{An inline block. Actually, can appear in inline or block mode, but
typesets its contents as a block.
}{\item[\textit{Attributes}:] \patternref{Common.attributes}, \patternref{ID.attributes}, \patternref{Positionable.attributes}, \patternref{Transformable.attributes}, \patternref{Backgroundable.attributes}\item[\textit{Content}:] \patternref{Block.model}\item[\textit{Used by}:] \patternref{Misc.class}}



\elementdef{verbatim}{Verbatim content
}{\item[\textit{Attributes}:] \patternref{Common.attributes}, \patternref{ID.attributes}, \patternref{Fontable.attributes}, \patternref{Colorable.attributes}, \patternref{Backgroundable.attributes}\item[\textit{Content}:] \patternref{Inline.model}\item[\textit{Used by}:] \patternref{Misc.class}}



\elementdef{break}{A forced line break.
}{\item[\textit{Attributes}:] \patternref{Common.attributes}\item[\textit{Content}:] \typename{empty}\item[\textit{Used by}:] \patternref{Misc.class}}



\elementdef{graphics}{A graphical insertion of an external file.
}{\item[\textit{Attributes}:] \patternref{Common.attributes}, \patternref{ID.attributes}, \patternref{Imageable.attributes}\attrdef{graphic}{the path to the graphics file. This is the (often minimally specified) path
to a graphics file omitting the type extension.  Once resolved to a specific
image file, the \attr{imagesrc} (from Imageable.attributes) is used.
}{\typename{text}}\attrdef{candidates}{a comma separated list of candidate graphics files that could be used to
for \attr{graphic}.  A post-processor or application may choose from these,
or may make its own selection or synthesis to implement the graphic for a given target.
}{\typename{text}}\attrdef{options}{an encoding of the scaling and positioning options
to be used in processing the graphic.
}{\typename{text}}\item[\textit{Content}:] \typename{empty}\item[\textit{Used by}:] \patternref{Misc.class}}






\elementdef{rawhtml}{A container for arbitrary markup in the xhtml namespace
(not currently validated against any particular html schema)
}{\item[\textit{Content}:] \texttt{xhtml:*}\textsuperscript{*}\item[\textit{Used by}:] \patternref{Misc.class}}

\elementdef{rawliteral}{A container for even more arbitrary directives like jsp, php, etc
Doesn't create an element, but an open angle bracket followed by \attr{open}
then the text content, followed by a close angle bracket followed by \attr{close}.
}{\attrdef{open}{}{\typename{text}}\attrdef{close}{}{\typename{text}}\item[\textit{Content}:] \typename{text}\item[\textit{Used by}:] \patternref{Misc.class}}

\elementdef{ERROR}{error object for undefined control sequences, or whatever
}{\item[\textit{Attributes}:] \patternref{Common.attributes}, \patternref{ID.attributes}\item[\textit{Content}:] \typename{text}\textsuperscript{*}\item[\textit{Used by}:] \patternref{Misc.class}, \patternref{XMath.class}}



\end{schemamodule}
\begin{schemamodule}{LaTeXML-meta}
\patternadd{Meta.class}{Meta elements are generally hidden;
they can appear in both inline and block contexts.
}{\item[\textbar=] (\elementref{note} ~\textbar~ \elementref{declare} ~\textbar~ \elementref{indexmark} ~\textbar~ \elementref{glossarydefinition} ~\textbar~ \elementref{rdf} ~\textbar~ \elementref{resource} ~\textbar~ \elementref{navigation})}

\elementdef{note}{Metadata that covers several `out of band' annotations.
It's content allows both inline and block-level content.
}{\item[\textit{Attributes}:] \patternref{Common.attributes}, \patternref{Labelled.attributes}\attrdef{mark}{indicates the desired visible marker to be linked to the note.
}{\typename{text}}\attrdef{role}{indicates the kind of note
}{(\attrval{footnote} ~\textbar~ \typename{text})}\item[\textit{Content}:] (\elementref{tags}\textsuperscript{?}, \patternref{Flow.model})\item[\textit{Used by}:] \patternref{Meta.class}}



\elementdef{declare}{declare records declarative mathematical information.
}{\item[\textit{Attributes}:] \patternref{ID.attributes}\attrdef{type}{the type of declaration
}{\typename{text}}\attrdef{definiens}{the thing being defined (if global), else must have xml:id
}{\typename{text}}\attrdef{sortkey}{the sort key for use creating notation indices
}{\typename{text}}\item[\textit{Content}:] (\elementref{tags}\textsuperscript{?}, \elementref{text}\textsuperscript{?})\item[\textit{Used by}:] \patternref{Meta.class}}



\elementdef{indexmark}{Metadata to record an indexing position. The content is
a sequence of \elementref{indexphrase}, each representing a level in
a multilevel indexing entry.
}{\item[\textit{Attributes}:] \patternref{Common.attributes}, \patternref{Listable.attributes}\attrdef{see\_also}{a flattened form (like \attr{key}) of another \elementref{indexmark},
used to crossreference.
}{\typename{text}}\attrdef{style}{NOTE: describe this.
}{\typename{text}}\item[\textit{Content}:] (\elementref{indexphrase}\textsuperscript{*}, \elementref{indexsee}\textsuperscript{*})\item[\textit{Used by}:] \patternref{Meta.class}}



\elementdef{indexphrase}{A phrase within an \elementref{indexmark}
}{\item[\textit{Attributes}:] \patternref{Common.attributes}\attrdef{key}{a flattened form of the phrase for generating an \attr{ID}.
}{\typename{text}}\item[\textit{Content}:] \patternref{Inline.model}\item[\textit{Used by}:] \elementref{indexentry}, \elementref{indexmark}}



\elementdef{indexsee}{A see-also phrase within an \elementref{indexmark}
}{\item[\textit{Attributes}:] \patternref{Common.attributes}\attrdef{key}{a flattened form of the phrase for generating an \attr{ID}.
}{\typename{text}}\attrdef{name}{a name for the see phrase, such as "see also".
}{\typename{text}}\item[\textit{Content}:] \patternref{Inline.model}\item[\textit{Used by}:] \elementref{indexmark}}



\elementdef{glossarydefinition}{A definition within an \elementref{glossaryentry}
}{\item[\textit{Attributes}:] \patternref{Common.attributes}, \patternref{Listable.attributes}\attrdef{key}{a flattened form of the definition for generating an \attr{ID}.
}{\typename{text}}\item[\textit{Content}:] \elementref{glossaryphrase}\textsuperscript{*}\item[\textit{Used by}:] \patternref{Meta.class}}



\elementdef{glossaryphrase}{A phrase being clarified within an \elementref{glossaryentry}
}{\item[\textit{Attributes}:] \patternref{Common.attributes}, \patternref{IDREF.attributes}\attrdef{key}{a flattened form of the phrase for generating an \attr{ID}.
}{\typename{text}}\attrdef{role}{a keyword naming the format of this phrase (to match \attr{show} in \elementref{glossaryref}).
}{\typename{text}}\item[\textit{Content}:] \patternref{Inline.model}\item[\textit{Used by}:] \elementref{glossarydefinition}, \elementref{glossaryentry}}



\elementdef{rdf}{A container for RDF annotations.
(See document structure for rdf-prefixes attribute)
}{\item[\textit{Attributes}:] \patternref{Common.attributes}\item[\textit{Content}:] \patternref{Flow.model}\item[\textit{Used by}:] \patternref{Meta.class}}



\elementdef{resource}{a resource for use in further processing such as javascript or CSS
}{\item[\textit{Attributes}:] \patternref{Common.attributes}\attrdef{src}{the source url to the resource
}{\typename{text}}\attrdef{type}{the mime type of the resource
}{\typename{text}}\attrdef{media}{the media for which this resource is applicable
(in the sense of CSS).
}{\typename{text}}\item[\textit{Content}:] \typename{text}\textsuperscript{*}\item[\textit{Used by}:] \patternref{Meta.class}}



\elementdef{navigation}{Records navigation cross-referencing information,
or serves as a container for page navigational blocks.
An \elementref{inline-logical-block} child should have attribute \attr{class}
being one of \texttt{ltx\_page\_navbar}, \texttt{ltx\_page\_header}
or \texttt{ltx\_page\_footer} and its contents will be used to create those components of webpages.
Lacking those, a \elementref{TOC} requests a table of contents
in the navigation bar. Page headers and footers will be synthesized from
Links from the current page or document to related ones;
these are represented by \elementref{ref} elements with \attr{rel}
being up, down, previous, next, and so forth.
}{\item[\textit{Attributes}:] \patternref{Common.attributes}\item[\textit{Content}:] (\elementref{ref} ~\textbar~ \elementref{TOC} ~\textbar~ \elementref{inline-logical-block})\textsuperscript{*}\item[\textit{Used by}:] \patternref{Meta.class}}



\end{schemamodule}
\begin{schemamodule}{LaTeXML-para}
\patternadd{Para.class}{This module defines the following `logical' block elements.
}{\item[\textbar=] (\elementref{para} ~\textbar~ \elementref{logical-block} ~\textbar~ \elementref{theorem} ~\textbar~ \elementref{proof} ~\textbar~ \elementref{figure} ~\textbar~ \elementref{table} ~\textbar~ \elementref{float} ~\textbar~ \elementref{pagination})}

\patternadd{Misc.class}{Additionally, it defines these miscellaneous elements that can appear
in both inline and block contexts.
}{\item[\textbar=] \elementref{inline-logical-block}}

\elementdef{para}{A Logical paragraph. It has an \attr{id}, but not a \attr{label}.
}{\item[\textit{Attributes}:] \patternref{Common.attributes}, \patternref{ID.attributes}, \patternref{Positionable.attributes}, \patternref{Backgroundable.attributes}\item[\textit{Content}:] \patternref{Block.model}\item[\textit{Used by}:] \patternref{Para.class}}



\elementdef{logical-block}{A logical-block. Actually, like block can appear in inline or block mode, but
typesets its contents as para.
}{\item[\textit{Attributes}:] \patternref{Common.attributes}, \patternref{ID.attributes}, \patternref{Positionable.attributes}, \patternref{Backgroundable.attributes}\item[\textit{Content}:] \patternref{Para.model}\item[\textit{Used by}:] \patternref{Para.class}}



\elementdef{inline-logical-block}{An inline logical-block. Actually, can appear in inline or block mode, but
typesets its contents as para.
}{\item[\textit{Attributes}:] \patternref{Common.attributes}, \patternref{ID.attributes}, \patternref{Positionable.attributes}, \patternref{Backgroundable.attributes}\item[\textit{Content}:] \patternref{Para.model}\item[\textit{Used by}:] \patternref{Misc.class}, \elementref{navigation}}



\elementdef{theorem}{A theorem or similar object. The \attr{class} attribute can be used to distinguish
different kinds of theorem.
}{\item[\textit{Attributes}:] \patternref{Common.attributes}, \patternref{Labelled.attributes}, \patternref{Backgroundable.attributes}\item[\textit{Content}:] (\elementref{tags}\textsuperscript{?}, \elementref{title}\textsuperscript{?}, \patternref{Para.model})\item[\textit{Used by}:] \patternref{Para.class}}



\elementdef{proof}{A proof or similar object. The \attr{class} attribute can be used to distinguish
different kinds of proof.
}{\item[\textit{Attributes}:] \patternref{Common.attributes}, \patternref{Labelled.attributes}, \patternref{Backgroundable.attributes}, \patternref{Data.attributes}\item[\textit{Content}:] (\elementref{tags}\textsuperscript{?}, \elementref{title}\textsuperscript{?}, \patternref{Para.model})\item[\textit{Used by}:] \patternref{Para.class}}



\patterndef{Caption.class}{These are the additional elements representing figure and
table captions.
NOTE: Could title sensibly be reused here, instead?
Or, should caption be used for theorem and proof?
}{\item[\textit{Content}:] (\elementref{caption} ~\textbar~ \elementref{toccaption})\item[\textit{Used by}:] \elementref{figure}, \elementref{float}, \elementref{table}}

\elementdef{figure}{A  figure, possibly captioned.
}{\item[\textit{Attributes}:] \patternref{Common.attributes}, \patternref{Labelled.attributes}, \patternref{Positionable.attributes}, \patternref{Transformable.attributes}, \patternref{Backgroundable.attributes}, \patternref{Data.attributes}\attrdef{placement}{the vertical floating placement parameter that determines where the object is displayed.
}{\typename{text}}\item[\textit{Content}:] (\elementref{tags}\textsuperscript{?} ~\textbar~ \elementref{figure} ~\textbar~ \elementref{table} ~\textbar~ \elementref{float} ~\textbar~ \patternref{Block.model} ~\textbar~ \patternref{Caption.class})\textsuperscript{*}\item[\textit{Used by}:] \patternref{Para.class}, \elementref{figure}, \elementref{float}, \elementref{table}}



\elementdef{table}{A  Table, possibly captioned. This is not necessarily a \elementref{tabular}.
}{\item[\textit{Attributes}:] \patternref{Common.attributes}, \patternref{Labelled.attributes}, \patternref{Positionable.attributes}, \patternref{Transformable.attributes}, \patternref{Backgroundable.attributes}, \patternref{Data.attributes}\attrdef{placement}{the vertical floating placement parameter that determines where the object is displayed.
}{\typename{text}}\item[\textit{Content}:] (\elementref{tags}\textsuperscript{?} ~\textbar~ \elementref{table} ~\textbar~ \elementref{figure} ~\textbar~ \elementref{float} ~\textbar~ \patternref{Block.model} ~\textbar~ \patternref{Caption.class})\textsuperscript{*}\item[\textit{Used by}:] \patternref{Para.class}, \elementref{figure}, \elementref{float}, \elementref{table}}



\elementdef{float}{A generic float, possibly captioned, something other than a table or figure
}{\item[\textit{Attributes}:] \patternref{Common.attributes}, \patternref{Labelled.attributes}, \patternref{Positionable.attributes}, \patternref{Transformable.attributes}, \patternref{Backgroundable.attributes}, \patternref{Data.attributes}\attrdef{role}{The kind of float; could be something like a listing, or some other thing
}{\typename{text}}\attrdef{placement}{the vertical floating placement parameter that determines where the object is displayed.
}{\typename{text}}\item[\textit{Content}:] (\elementref{tags}\textsuperscript{?} ~\textbar~ \elementref{float} ~\textbar~ \elementref{figure} ~\textbar~ \elementref{table} ~\textbar~ \patternref{Block.model} ~\textbar~ \patternref{Caption.class})\textsuperscript{*}\item[\textit{Used by}:] \patternref{Para.class}, \elementref{figure}, \elementref{float}, \elementref{table}}



\elementdef{caption}{A caption for a \elementref{table} or \elementref{figure}.
}{\item[\textit{Attributes}:] \patternref{Common.attributes}, \patternref{Fontable.attributes}, \patternref{Colorable.attributes}, \patternref{Backgroundable.attributes}\item[\textit{Content}:] (\elementref{tag} ~\textbar~ \typename{text} ~\textbar~ \patternref{Inline.class} ~\textbar~ \patternref{Misc.class} ~\textbar~ \patternref{Meta.class})\textsuperscript{*}\item[\textit{Used by}:] \patternref{Caption.class}}



\elementdef{toccaption}{A short form of \elementref{table} or \elementref{figure} caption,
used for lists of figures or similar.
}{\item[\textit{Attributes}:] \patternref{Common.attributes}\item[\textit{Content}:] (\elementref{tag} ~\textbar~ \typename{text} ~\textbar~ \patternref{Inline.class} ~\textbar~ \patternref{Misc.class} ~\textbar~ \patternref{Meta.class})\textsuperscript{*}\item[\textit{Used by}:] \patternref{Caption.class}}



\end{schemamodule}
\begin{schemamodule}{LaTeXML-math}
\patternadd{Inline.class}{The math module defines LaTeXML's internal representation of mathematical
content, including the basic math container \elementref{Math}.  This element is
considered inline, as it will be contained within some other block-level
element, eg. \elementref{equation} for display-math.
}{\item[\textbar=] \elementref{Math}}

\patterndef{Math.class}{This class defines the content of the \elementref{Math} element.
Additionally, it could contain MathML or OpenMath, after postprocessing.
}{\item[\textit{Content}:] \elementref{XMath}\item[\textit{Used by}:] \elementref{Math}}

\patterndef{XMath.class}{These elements comprise the internal math representation, being
the content of the \elementref{XMath} element.
}{\item[\textit{Content}:] (\elementref{XMApp} ~\textbar~ \elementref{XMTok} ~\textbar~ \elementref{XMRef} ~\textbar~ \elementref{XMHint} ~\textbar~ \elementref{XMArg} ~\textbar~ \elementref{XMWrap} ~\textbar~ \elementref{XMDual} ~\textbar~ \elementref{XMText} ~\textbar~ \elementref{XMArray} ~\textbar~ \elementref{ERROR})\item[\textit{Used by}:] \elementref{XMApp}, \elementref{XMArg}, \elementref{XMCell}, \elementref{XMDual}, \elementref{XMWrap}, \elementref{XMath}}

\elementdef{Math}{Outer container for all math. This holds the internal
\elementref{XMath} representation, as well as image data and other representations.
}{\item[\textit{Attributes}:] \patternref{Common.attributes}, \patternref{Imageable.attributes}, \patternref{ID.attributes}, \patternref{Backgroundable.attributes}\attrdef{mode}{display or inline mode.
}{(\attrval{display} ~\textbar~ \attrval{inline})}\attrdef{tex}{reconstruction of the \TeX\ that generated the math.
}{\typename{text}}\attrdef{content-tex}{more semantic version of \attr{tex}.
}{\typename{text}}\attrdef{text}{a textified representation of the math.
}{\typename{text}}\attrdef{lexemes}{preserved grammar-near lexemes for export to external apps
}{\typename{text}}\item[\textit{Content}:] \patternref{Math.class}\textsuperscript{*}\item[\textit{Used by}:] \patternref{Inline.class}, \elementref{MathBranch}, \elementref{MathFork}, \elementref{equation}}



\patterndef{XMath.attributes}{Common attributes for the various XMath elements.
}{\item[\textit{Attributes}:] \patternref{Positionable.attributes}\attrdef{role}{The role that this item plays in the Grammar.
}{\typename{text}}\attrdef{enclose}{an enclose style to enclose the object
with legitimate values being those of MathML's menclose notations;
}{\typename{text}}\attrdef{lpadding}{left, or leading, (presumably non-semantic) padding space.
}{\typename{text}}\attrdef{rpadding}{right, or trailing, (presumably non-semantic) padding space.
}{\typename{text}}\attrdef{name}{The name of the token, typically the control sequence that created it.
}{\typename{text}}\attrdef{meaning}{A more semantic name corresponding to the intended meaning,
such as the OpenMath name.
}{\typename{text}}\attrdef{omcd}{The OpenMath CD for which \attr{meaning} is a symbol.
}{\typename{text}}\attrdef{scriptpos}{An encoding of the position of sub/superscripts
Before parsing, it takes two forms. On a base token or element,
it is one of (pre|mid|post), indicating where any script can be placed.
On a script token, it is an integer level.
After parsing, the concatenation is moved to the sub|super-script "operator".
}{\typename{text}}\attrdef{possibleFunction}{an annotation placed by the parser when it suspects this token may be used as a function.
}{\typename{text}}\attrdef{decl\_id}{an id to where the declaration of this object is given,
preferably the xml:id of an ltx:declare
}{\typename{text}}\attrdef{href}{reference to an arbitrary url.
}{\typename{text}}\item[\textit{Used by}:] \elementref{XMApp}, \elementref{XMArg}, \elementref{XMArray}, \elementref{XMDual}, \elementref{XMHint}, \elementref{XMRef}, \elementref{XMText}, \elementref{XMTok}, \elementref{XMWrap}}

\elementdef{XMath}{Internal representation of mathematics.
}{\item[\textit{Attributes}:] \patternref{Common.attributes}, \patternref{ID.attributes}\item[\textit{Content}:] \patternref{XMath.class}\textsuperscript{*}\item[\textit{Used by}:] \patternref{Math.class}}



\elementdef{XMTok}{General mathematical token.
}{\item[\textit{Attributes}:] \patternref{Common.attributes}, \patternref{XMath.attributes}, \patternref{ID.attributes}, \patternref{Fontable.attributes}, \patternref{Colorable.attributes}, \patternref{Backgroundable.attributes}\attrdef{thickness}{A thickness used for drawing any lines which are part of presenting the token,
such as the fraction line for the fraction operator.
}{\typename{text}}\attrdef{stretchy}{Whether or not the symbol should be stretchy.
This shares MathML's ambiguity about horizontal versus vertical stretchiness.
When not set, defaults to whatever MathML's operator dictionary says.
}{\typename{boolean}}\attrdef{mathstyle}{The math style used for displaying the application of this token
when it represents some sort of fraction, variable-sized operator or stack of expressions
(note that this applies to binomials or other stacks of expressions as well as fractions).
Values of \texttt{display} or \texttt{text} correspond to \TeX's
displaystyle or textstyle, while \texttt{inline} indicates the
stack should be arranged horizontally (the layout may depend on the operator).
}{(\attrval{display} ~\textbar~ \attrval{text} ~\textbar~ \attrval{script} ~\textbar~ \attrval{scriptscript})}\item[\textit{Content}:] \typename{text}\textsuperscript{*}\item[\textit{Used by}:] \patternref{XMath.class}}



\elementdef{XMApp}{Generalized application of a function, operator, whatever (the first child)
to arguments (the remaining children).
The attributes are a subset of those for \elementref{XMTok}.
}{\item[\textit{Attributes}:] \patternref{Common.attributes}, \patternref{XMath.attributes}, \patternref{ID.attributes}, \patternref{Colorable.attributes}, \patternref{Backgroundable.attributes}\item[\textit{Content}:] \patternref{XMath.class}\textsuperscript{*}\item[\textit{Used by}:] \patternref{XMath.class}}



\elementdef{XMDual}{Parallel markup of content (first child) and presentation (second child)
of a mathematical object.
Typically, the arguments are shared between the two branches:
they appear in the content branch, with \attr{id}'s,
and \elementref{XMRef} is used in the presentation branch
}{\item[\textit{Attributes}:] \patternref{Common.attributes}, \patternref{XMath.attributes}, \patternref{ID.attributes}\item[\textit{Content}:] (\patternref{XMath.class}, \patternref{XMath.class})\item[\textit{Used by}:] \patternref{XMath.class}}



\elementdef{XMHint}{Various layout hints, usually spacing, generally ignored in parsing.
The attributes are a subset of those for \elementref{XMTok}.
}{\item[\textit{Attributes}:] \patternref{Common.attributes}, \patternref{XMath.attributes}, \patternref{ID.attributes}\item[\textit{Content}:] \typename{empty}\item[\textit{Used by}:] \patternref{XMath.class}}



\elementdef{XMText}{Text appearing within math.
}{\item[\textit{Attributes}:] \patternref{Common.attributes}, \patternref{XMath.attributes}, \patternref{Backgroundable.attributes}, \patternref{ID.attributes}\item[\textit{Content}:] (\typename{text} ~\textbar~ \patternref{Inline.class} ~\textbar~ \patternref{Misc.class})\textsuperscript{*}\item[\textit{Used by}:] \patternref{XMath.class}}



\elementdef{XMWrap}{Wrapper for a sequence of tokens used to assert the role of the
contents in its parent. This element generally disappears after parsing.
The attributes are a subset of those for \elementref{XMTok}.
}{\item[\textit{Attributes}:] \patternref{Common.attributes}, \patternref{XMath.attributes}, \patternref{Backgroundable.attributes}, \patternref{ID.attributes}\attrdef{rule}{The grammatical rule that should apply to the contained sequence
}{\typename{text}}\attrdef{style}{}{\typename{text}}\item[\textit{Content}:] \patternref{XMath.class}\textsuperscript{*}\item[\textit{Used by}:] \patternref{XMath.class}}



\elementdef{XMArg}{Wrapper for an argument to a structured macro.
It implies that its content can be parsed independently of its parent,
and thus generally disappears after parsing.
}{\item[\textit{Attributes}:] \patternref{Common.attributes}, \patternref{XMath.attributes}, \patternref{ID.attributes}\attrdef{rule}{The grammatical rule that should apply to the contained sequence
}{\typename{text}}\item[\textit{Content}:] \patternref{XMath.class}\textsuperscript{*}\item[\textit{Used by}:] \patternref{XMath.class}}



\elementdef{XMRef}{Structure sharing element typically used in the presentation
branch of an \elementref{XMDual} to refer to the arguments present in the content branch.
}{\item[\textit{Attributes}:] \patternref{Common.attributes}, \patternref{XMath.attributes}, \patternref{ID.attributes}, \patternref{IDREF.attributes}\item[\textit{Content}:] \typename{empty}\item[\textit{Used by}:] \patternref{XMath.class}}



\elementdef{XMArray}{Math Array/Alignment structure.
}{\item[\textit{Attributes}:] \patternref{Common.attributes}, \patternref{XMath.attributes}, \patternref{ID.attributes}\attrdef{rowsep}{the spacing between rows
}{\patternref{Length.type}}\attrdef{colsep}{the spacing between columns
}{\patternref{Length.type}}\item[\textit{Content}:] \elementref{XMRow}\textsuperscript{*}\item[\textit{Used by}:] \patternref{XMath.class}}



\elementdef{XMRow}{A row in a math alignment.
}{\item[\textit{Attributes}:] \patternref{Common.attributes}, \patternref{Backgroundable.attributes}, \patternref{ID.attributes}\item[\textit{Content}:] \elementref{XMCell}\textsuperscript{*}\item[\textit{Used by}:] \elementref{XMArray}}



\elementdef{XMCell}{A cell in a row of a math alignment.
}{\item[\textit{Attributes}:] \patternref{Common.attributes}, \patternref{Backgroundable.attributes}, \patternref{ID.attributes}\attrdef{colspan}{indicates how many columns this cell spans or covers.
}{\typename{nonNegativeInteger}}\attrdef{rowspan}{indicates how many rows this cell spans or covers.
}{\typename{nonNegativeInteger}}\attrdef{align}{ specifies the alignment of the content.
}{\typename{text}}\attrdef{width}{specifies the desired width for the column.
}{\typename{text}}\attrdef{border}{records a sequence of t or tt, r or rr, b or bb and l or ll
for borders or doubled borders on any side of the cell.
}{\typename{text}}\attrdef{thead}{whether this cell corresponds to a table row or column heading or both
}{(\attrval{column} ~\textbar~ \attrval{column row} ~\textbar~ \attrval{row})}\item[\textit{Content}:] \patternref{XMath.class}\textsuperscript{*}\item[\textit{Used by}:] \elementref{XMRow}}



\end{schemamodule}
\begin{schemamodule}{LaTeXML-tabular}
\patternadd{Misc.class}{This module defines the basic tabular, or alignment, structure.
It is roughly parallel to the HTML model.
}{\item[\textbar=] \elementref{tabular}}

\elementdef{tabular}{An alignment structure corresponding to tabular  or various similar forms.
The model is basically a copy of HTML4's table.
}{\item[\textit{Attributes}:] \patternref{Common.attributes}, \patternref{Backgroundable.attributes}, \patternref{ID.attributes}\attrdef{vattach}{which row's baseline aligns with the container's baseline.
}{(\attrval{top} ~\textbar~ \attrval{middle} ~\textbar~ \attrval{bottom})}\attrdef{width}{the desired width of the tabular.
}{\patternref{Length.type}}\attrdef{rowsep}{the spacing between rows
}{\patternref{Length.type}}\attrdef{colsep}{the spacing between columns
}{\patternref{Length.type}}\item[\textit{Content}:] (\elementref{thead} ~\textbar~ \elementref{tfoot} ~\textbar~ \elementref{tbody} ~\textbar~ \elementref{tr})\textsuperscript{*}\item[\textit{Used by}:] \patternref{Misc.class}}



\elementdef{thead}{A container for a set of rows that correspond to the header of the tabular.
}{\item[\textit{Attributes}:] \patternref{Common.attributes}, \patternref{Backgroundable.attributes}\item[\textit{Content}:] \elementref{tr}\textsuperscript{*}\item[\textit{Used by}:] \elementref{tabular}}



\elementdef{tfoot}{A container for a set of rows that correspond to the footer of the tabular.
}{\item[\textit{Attributes}:] \patternref{Common.attributes}, \patternref{Backgroundable.attributes}\item[\textit{Content}:] \elementref{tr}\textsuperscript{*}\item[\textit{Used by}:] \elementref{tabular}}



\elementdef{tbody}{A container for a set of rows corresponding to the body of the tabular.
}{\item[\textit{Attributes}:] \patternref{Common.attributes}, \patternref{Backgroundable.attributes}\item[\textit{Content}:] \elementref{tr}\textsuperscript{*}\item[\textit{Used by}:] \elementref{tabular}}



\elementdef{tr}{A row of a tabular.
}{\item[\textit{Attributes}:] \patternref{Common.attributes}, \patternref{Backgroundable.attributes}, \patternref{ID.attributes}\item[\textit{Content}:] \elementref{td}\textsuperscript{*}\item[\textit{Used by}:] \elementref{MathBranch}, \elementref{tabular}, \elementref{tbody}, \elementref{tfoot}, \elementref{thead}}



\elementdef{td}{A cell in a row of a tabular.
}{\item[\textit{Attributes}:] \patternref{Common.attributes}, \patternref{Backgroundable.attributes}, \patternref{ID.attributes}\attrdef{colspan}{indicates how many columns this cell spans or covers.
}{\typename{nonNegativeInteger}}\attrdef{rowspan}{indicates how many rows this cell spans or covers.
}{\typename{nonNegativeInteger}}\attrdef{align}{ specifies the horizontal alignment of the content.
The allowed values are open-ended to accomodate \texttt{char:.} type alignments.
}{(\attrval{left} ~\textbar~ \attrval{right} ~\textbar~ \attrval{center} ~\textbar~ \attrval{justify} ~\textbar~ \typename{text})}\attrdef{width}{specifies the desired width for the column.
}{\patternref{Length.type}}\attrdef{vattach}{how the cell contents aligns with the row's baseline.
}{(\attrval{top} ~\textbar~ \attrval{middle} ~\textbar~ \attrval{bottom})}\attrdef{border}{records a sequence of t or tt, r or rr, b or bb and l or ll
for borders or doubled borders on any side of the cell.
}{\typename{text}}\attrdef{thead}{whether this cell corresponds to a table row or column heading or both
(whether in head or foot).
}{(\attrval{column} ~\textbar~ \attrval{column row} ~\textbar~ \attrval{row})}\item[\textit{Content}:] \patternref{Inline.model}\item[\textit{Used by}:] \elementref{MathBranch}, \elementref{tr}}



\end{schemamodule}
\begin{schemamodule}{LaTeXML-picture}
\patternadd{Misc.class}{This module defines a picture environment, roughly a subset of SVG.
NOTE: Eventually we will drop these subset elements and incorporate SVG itself.
}{\item[\textbar=] \elementref{picture}}

\patterndef{Picture.class}{}{\item[\textit{Content}:] (\elementref{g} ~\textbar~ \elementref{rect} ~\textbar~ \elementref{line} ~\textbar~ \elementref{circle} ~\textbar~ \elementref{path} ~\textbar~ \elementref{arc} ~\textbar~ \elementref{wedge} ~\textbar~ \elementref{ellipse} ~\textbar~ \elementref{polygon} ~\textbar~ \elementref{bezier} ~\textbar~ \elementref{parabola} ~\textbar~ \elementref{curve} ~\textbar~ \elementref{dots} ~\textbar~ \elementref{grid} ~\textbar~ \elementref{clip} ~\textbar~ \texttt{svg:svg})\item[\textit{Used by}:] \elementref{clippath}, \elementref{g}, \elementref{picture}}

\patterndef{Picture.attributes}{These attributes correspond roughly to SVG, but need documentation.
}{\attrdef{x}{}{\typename{text}}\attrdef{y}{}{\typename{text}}\attrdef{r}{}{\typename{text}}\attrdef{rx}{}{\typename{text}}\attrdef{ry}{}{\typename{text}}\attrdef{width}{}{\typename{text}}\attrdef{height}{}{\typename{text}}\attrdef{fill}{}{\typename{text}}\attrdef{stroke}{}{\typename{text}}\attrdef{stroke-width}{}{\typename{text}}\attrdef{stroke-dasharray}{}{\typename{text}}\attrdef{transform}{}{\typename{text}}\attrdef{terminators}{}{\typename{text}}\attrdef{arrowlength}{}{\typename{text}}\attrdef{points}{}{\typename{text}}\attrdef{showpoints}{}{\typename{text}}\attrdef{displayedpoints}{}{\typename{text}}\attrdef{arc}{}{\typename{text}}\attrdef{angle1}{}{\typename{text}}\attrdef{angle2}{}{\typename{text}}\attrdef{arcsepA}{}{\typename{text}}\attrdef{arcsepB}{}{\typename{text}}\attrdef{curvature}{}{\typename{text}}\item[\textit{Used by}:] \elementref{arc}, \elementref{bezier}, \elementref{circle}, \elementref{clip}, \elementref{clippath}, \elementref{curve}, \elementref{dots}, \elementref{ellipse}, \elementref{g}, \elementref{grid}, \elementref{line}, \elementref{parabola}, \elementref{path}, \elementref{picture}, \elementref{polygon}, \elementref{rect}, \elementref{wedge}}

\patterndef{PictureGroup.attributes}{These attributes correspond roughly to SVG, but need documentation.
}{\attrdef{pos}{}{\typename{text}}\attrdef{framed}{}{\typename{boolean}}\attrdef{frametype}{}{(\attrval{rect} ~\textbar~ \attrval{circle} ~\textbar~ \attrval{oval})}\attrdef{fillframe}{}{\typename{boolean}}\attrdef{boxsep}{}{\typename{text}}\attrdef{shadowbox}{}{\typename{boolean}}\attrdef{doubleline}{}{\typename{boolean}}\item[\textit{Used by}:] \elementref{g}}

\elementdef{picture}{A picture environment.
}{\item[\textit{Attributes}:] \patternref{Common.attributes}, \patternref{ID.attributes}, \patternref{Picture.attributes}, \patternref{Imageable.attributes}\attrdef{clip}{}{\typename{boolean}}\attrdef{baseline}{}{\typename{text}}\attrdef{unitlength}{}{\typename{text}}\attrdef{xunitlength}{}{\typename{text}}\attrdef{yunitlength}{}{\typename{text}}\attrdef{origin-x}{}{\typename{text}}\attrdef{origin-y}{}{\typename{text}}\attrdef{tex}{}{\typename{text}}\attrdef{content-tex}{}{\typename{text}}\item[\textit{Content}:] (\patternref{Picture.class} ~\textbar~ \patternref{Inline.class} ~\textbar~ \patternref{Misc.class} ~\textbar~ \patternref{Meta.class})\textsuperscript{*}\item[\textit{Used by}:] \patternref{Misc.class}}



\elementdef{g}{A graphical grouping; the content is inherits by the transformations, 
positioning and other properties.
}{\item[\textit{Attributes}:] \patternref{Common.attributes}, \patternref{Transformable.attributes}, \patternref{Picture.attributes}, \patternref{PictureGroup.attributes}\item[\textit{Content}:] (\patternref{Picture.class} ~\textbar~ \patternref{Inline.class} ~\textbar~ \patternref{Misc.class} ~\textbar~ \patternref{Meta.class})\textsuperscript{*}\item[\textit{Used by}:] \patternref{Picture.class}}



\elementdef{rect}{A rectangle within a \elementref{picture}.
}{\item[\textit{Attributes}:] \patternref{Common.attributes}, \patternref{Picture.attributes}\item[\textit{Content}:] \typename{empty}\item[\textit{Used by}:] \patternref{Picture.class}}



\elementdef{line}{A line within a \elementref{picture}.
}{\item[\textit{Attributes}:] \patternref{Common.attributes}, \patternref{Picture.attributes}\item[\textit{Content}:] \typename{empty}\item[\textit{Used by}:] \patternref{Picture.class}}



\elementdef{polygon}{A polygon within a \elementref{picture}.
}{\item[\textit{Attributes}:] \patternref{Common.attributes}, \patternref{Picture.attributes}\item[\textit{Content}:] \typename{empty}\item[\textit{Used by}:] \patternref{Picture.class}}



\elementdef{wedge}{A wedge within a \elementref{picture}.
}{\item[\textit{Attributes}:] \patternref{Common.attributes}, \patternref{Picture.attributes}\item[\textit{Content}:] \typename{empty}\item[\textit{Used by}:] \patternref{Picture.class}}



\elementdef{arc}{An arc within a \elementref{picture}.
}{\item[\textit{Attributes}:] \patternref{Common.attributes}, \patternref{Picture.attributes}\item[\textit{Content}:] \typename{empty}\item[\textit{Used by}:] \patternref{Picture.class}}



\elementdef{circle}{A circle within a \elementref{picture}.
}{\item[\textit{Attributes}:] \patternref{Common.attributes}, \patternref{Picture.attributes}\item[\textit{Content}:] \typename{empty}\item[\textit{Used by}:] \patternref{Picture.class}}



\elementdef{ellipse}{An ellipse within a \elementref{picture}.
}{\item[\textit{Attributes}:] \patternref{Common.attributes}, \patternref{Picture.attributes}\item[\textit{Content}:] \typename{empty}\item[\textit{Used by}:] \patternref{Picture.class}}



\elementdef{path}{A path within a \elementref{picture}.
}{\item[\textit{Attributes}:] \patternref{Common.attributes}, \patternref{Picture.attributes}\item[\textit{Content}:] \typename{empty}\item[\textit{Used by}:] \patternref{Picture.class}}



\elementdef{bezier}{A bezier curve within a \elementref{picture}.
}{\item[\textit{Attributes}:] \patternref{Common.attributes}, \patternref{Picture.attributes}\item[\textit{Content}:] \typename{empty}\item[\textit{Used by}:] \patternref{Picture.class}}



\elementdef{curve}{A curve within a \elementref{picture}.
}{\item[\textit{Attributes}:] \patternref{Common.attributes}, \patternref{Picture.attributes}\item[\textit{Content}:] \typename{empty}\item[\textit{Used by}:] \patternref{Picture.class}}



\elementdef{parabola}{A parabola curve within a \elementref{picture}.
}{\item[\textit{Attributes}:] \patternref{Common.attributes}, \patternref{Picture.attributes}\item[\textit{Content}:] \typename{empty}\item[\textit{Used by}:] \patternref{Picture.class}}



\elementdef{dots}{A sequence of dots (?) within a \elementref{picture}.
}{\item[\textit{Attributes}:] \patternref{Common.attributes}, \patternref{Picture.attributes}\attrdef{dotstyle}{}{\typename{text}}\attrdef{dotsize}{}{\typename{text}}\attrdef{dotscale}{}{\typename{text}}\item[\textit{Content}:] \typename{empty}\item[\textit{Used by}:] \patternref{Picture.class}}



\elementdef{grid}{A grid within a \elementref{picture}.
}{\item[\textit{Attributes}:] \patternref{Common.attributes}, \patternref{Picture.attributes}\item[\textit{Content}:] \typename{empty}\item[\textit{Used by}:] \patternref{Picture.class}}



\elementdef{clip}{Establishes a clipping region within a \elementref{picture}.
}{\item[\textit{Attributes}:] \patternref{Common.attributes}, \patternref{Picture.attributes}\item[\textit{Content}:] \elementref{clippath}\textsuperscript{*}\item[\textit{Used by}:] \patternref{Picture.class}}



\elementdef{clippath}{Establishes a clipping region within a \elementref{picture}.
}{\item[\textit{Attributes}:] \patternref{Common.attributes}, \patternref{Picture.attributes}\item[\textit{Content}:] (\patternref{Picture.class} ~\textbar~ \patternref{Inline.class} ~\textbar~ \patternref{Misc.class} ~\textbar~ \patternref{Meta.class})\textsuperscript{*}\item[\textit{Used by}:] \elementref{clip}}



\end{schemamodule}
\begin{schemamodule}{LaTeXML-structure}
\elementdef{document}{The document root.
}{\item[\textit{Attributes}:] \patternref{Sectional.attributes}\item[\textit{Content}:] ((\patternref{FrontMatter.class} ~\textbar~ \patternref{SectionalFrontMatter.class} ~\textbar~ \patternref{Meta.class} ~\textbar~ \elementref{titlepage})\textsuperscript{*}, (\patternref{document.body.class} ~\textbar~ \patternref{BackMatter.class})\textsuperscript{*})}

\patterndef{document.body.class}{The content allowable as the main body of the document.
}{\item[\textit{Content}:] (\patternref{Para.model} ~\textbar~ \elementref{paragraph} ~\textbar~ \elementref{subsubsection} ~\textbar~ \elementref{subsection} ~\textbar~ \elementref{section} ~\textbar~ \elementref{chapter} ~\textbar~ \elementref{part} ~\textbar~ \elementref{slide} ~\textbar~ \elementref{slidesequence} ~\textbar~ \elementref{sidebar} ~\textbar~ \elementref{sectional-block})\item[\textit{Used by}:] \elementref{document}, \elementref{inline-sectional-block}, \elementref{sectional-block}}

\patternadd{Misc.class}{}{\item[\textbar=] \elementref{inline-sectional-block}}



\elementdef{sectional-block}{A block containing sectional material. Note that this may tend to break the overall heirarchy.
}{\item[\textit{Attributes}:] \patternref{Sectional.attributes}\item[\textit{Content}:] \patternref{document.body.class}\textsuperscript{*}\item[\textit{Used by}:] \patternref{document.body.class}}



\elementdef{inline-sectional-block}{A block containing sectional material. Note that this may tend to break the overall heirarchy.
}{\item[\textit{Attributes}:] \patternref{Sectional.attributes}\item[\textit{Content}:] \patternref{document.body.class}\textsuperscript{*}\item[\textit{Used by}:] \patternref{Misc.class}}



\elementdef{part}{A part within a document.
}{\item[\textit{Attributes}:] \patternref{Sectional.attributes}\item[\textit{Content}:] (\patternref{SectionalFrontMatter.class}\textsuperscript{*}, (\patternref{part.body.class} ~\textbar~ \patternref{BackMatter.class})\textsuperscript{*})\item[\textit{Used by}:] \patternref{document.body.class}}

\patterndef{part.body.class}{The content allowable as the main body of a part.
}{\item[\textit{Content}:] (\patternref{Para.model} ~\textbar~ \elementref{subparagraph} ~\textbar~ \elementref{paragraph} ~\textbar~ \elementref{subsubsection} ~\textbar~ \elementref{subsection} ~\textbar~ \elementref{section} ~\textbar~ \elementref{chapter} ~\textbar~ \elementref{slide} ~\textbar~ \elementref{slidesequence} ~\textbar~ \elementref{sidebar})\item[\textit{Used by}:] \elementref{part}}



\elementdef{chapter}{A Chapter within a document.
}{\item[\textit{Attributes}:] \patternref{Sectional.attributes}\item[\textit{Content}:] (\patternref{SectionalFrontMatter.class}\textsuperscript{*}, (\patternref{chapter.body.class} ~\textbar~ \patternref{BackMatter.class})\textsuperscript{*})\item[\textit{Used by}:] \patternref{document.body.class}, \patternref{part.body.class}}

\patterndef{chapter.body.class}{The content allowable as the main body of a chapter.
}{\item[\textit{Content}:] (\patternref{Para.model} ~\textbar~ \elementref{subparagraph} ~\textbar~ \elementref{paragraph} ~\textbar~ \elementref{subsubsection} ~\textbar~ \elementref{subsection} ~\textbar~ \elementref{section} ~\textbar~ \elementref{slide} ~\textbar~ \elementref{slidesequence} ~\textbar~ \elementref{sidebar})\item[\textit{Used by}:] \elementref{chapter}}



\elementdef{section}{A Section within a document.
}{\item[\textit{Attributes}:] \patternref{Sectional.attributes}\item[\textit{Content}:] (\patternref{SectionalFrontMatter.class}\textsuperscript{*}, (\patternref{section.body.class} ~\textbar~ \patternref{BackMatter.class})\textsuperscript{*})\item[\textit{Used by}:] \patternref{appendix.body.class}, \patternref{chapter.body.class}, \patternref{document.body.class}, \patternref{part.body.class}}

\patterndef{section.body.class}{The content allowable as the main body of a section.
}{\item[\textit{Content}:] (\patternref{Para.model} ~\textbar~ \elementref{subparagraph} ~\textbar~ \elementref{paragraph} ~\textbar~ \elementref{subsubsection} ~\textbar~ \elementref{subsection} ~\textbar~ \elementref{slide} ~\textbar~ \elementref{slidesequence} ~\textbar~ \elementref{sidebar})\item[\textit{Used by}:] \elementref{section}}



\elementdef{subsection}{A Subsection within a document.
}{\item[\textit{Attributes}:] \patternref{Sectional.attributes}\item[\textit{Content}:] (\patternref{SectionalFrontMatter.class}\textsuperscript{*}, (\patternref{subsection.body.class} ~\textbar~ \patternref{BackMatter.class})\textsuperscript{*})\item[\textit{Used by}:] \patternref{appendix.body.class}, \patternref{chapter.body.class}, \patternref{document.body.class}, \patternref{part.body.class}, \patternref{section.body.class}}

\patterndef{subsection.body.class}{The content allowable as the main body of a subsection.
}{\item[\textit{Content}:] (\patternref{Para.model} ~\textbar~ \elementref{subparagraph} ~\textbar~ \elementref{paragraph} ~\textbar~ \elementref{subsubsection} ~\textbar~ \elementref{slide} ~\textbar~ \elementref{slidesequence} ~\textbar~ \elementref{sidebar})\item[\textit{Used by}:] \elementref{subsection}}



\elementdef{subsubsection}{A Subsubsection within a document.
}{\item[\textit{Attributes}:] \patternref{Sectional.attributes}\item[\textit{Content}:] (\patternref{SectionalFrontMatter.class}\textsuperscript{*}, (\patternref{subsubsection.body.class} ~\textbar~ \patternref{BackMatter.class})\textsuperscript{*})\item[\textit{Used by}:] \patternref{appendix.body.class}, \patternref{chapter.body.class}, \patternref{document.body.class}, \patternref{part.body.class}, \patternref{section.body.class}, \patternref{subsection.body.class}}

\patterndef{subsubsection.body.class}{The content allowable as the main body of a subsubsection.
}{\item[\textit{Content}:] (\patternref{Para.model} ~\textbar~ \elementref{subparagraph} ~\textbar~ \elementref{paragraph} ~\textbar~ \elementref{slide} ~\textbar~ \elementref{slidesequence} ~\textbar~ \elementref{sidebar})\item[\textit{Used by}:] \elementref{subsubsection}}



\elementdef{paragraph}{A Paragraph within a document. This corresponds to a `formal' marked, possibly labelled
LaTeX Paragraph,  in distinction from an unlabelled logical paragraph.
}{\item[\textit{Attributes}:] \patternref{Sectional.attributes}\item[\textit{Content}:] (\patternref{SectionalFrontMatter.class}\textsuperscript{*}, (\patternref{paragraph.body.class} ~\textbar~ \patternref{BackMatter.class})\textsuperscript{*})\item[\textit{Used by}:] \patternref{appendix.body.class}, \patternref{chapter.body.class}, \patternref{document.body.class}, \patternref{part.body.class}, \patternref{section.body.class}, \patternref{subsection.body.class}, \patternref{subsubsection.body.class}}

\patterndef{paragraph.body.class}{The content allowable as the main body of a paragraph.
}{\item[\textit{Content}:] (\patternref{Para.model} ~\textbar~ \elementref{subparagraph} ~\textbar~ \elementref{slide} ~\textbar~ \elementref{slidesequence} ~\textbar~ \elementref{sidebar})\item[\textit{Used by}:] \elementref{paragraph}}



\elementdef{subparagraph}{A Subparagraph within a document.
}{\item[\textit{Attributes}:] \patternref{Sectional.attributes}\item[\textit{Content}:] (\patternref{SectionalFrontMatter.class}\textsuperscript{*}, (\patternref{subparagraph.body.class} ~\textbar~ \patternref{BackMatter.class})\textsuperscript{*})\item[\textit{Used by}:] \patternref{appendix.body.class}, \patternref{chapter.body.class}, \patternref{paragraph.body.class}, \patternref{part.body.class}, \patternref{section.body.class}, \patternref{subsection.body.class}, \patternref{subsubsection.body.class}}

\patterndef{subparagraph.body.class}{The content allowable as the main body of a subparagraph.
}{\item[\textit{Content}:] (\patternref{Para.model} ~\textbar~ \elementref{slide} ~\textbar~ \elementref{slidesequence} ~\textbar~ \elementref{sidebar})\item[\textit{Used by}:] \elementref{subparagraph}}



\elementdef{slidesequence}{A slidesequence within a slideshow.
Each slide contains a set slides, typically those that are revealed constructively. 
}{\item[\textit{Attributes}:] \patternref{Sectional.attributes}\item[\textit{Content}:] \elementref{slide}\textsuperscript{*}\item[\textit{Used by}:] \patternref{chapter.body.class}, \patternref{document.body.class}, \patternref{paragraph.body.class}, \patternref{part.body.class}, \patternref{section.body.class}, \patternref{subparagraph.body.class}, \patternref{subsection.body.class}, \patternref{subsubsection.body.class}}



\elementdef{slide}{A Slide within a slideshow, that may or may not be contained within a slidesequence.
}{\item[\textit{Attributes}:] \patternref{Sectional.attributes}\attrdef{overlay}{\attr{overlay} is the number of the current overlay.
This must be specified when part of a slidesequence, else it may be omitted.
Should be unique and rising within a slidesequence.
}{\typename{text}}\item[\textit{Content}:] ((\patternref{SectionalFrontMatter.class} ~\textbar~ \elementref{subtitle})\textsuperscript{*}, (\patternref{slide.body.class} ~\textbar~ \patternref{BackMatter.class})\textsuperscript{*})\item[\textit{Used by}:] \patternref{appendix.body.class}, \patternref{chapter.body.class}, \patternref{document.body.class}, \patternref{paragraph.body.class}, \patternref{part.body.class}, \patternref{section.body.class}, \patternref{subparagraph.body.class}, \patternref{subsection.body.class}, \patternref{subsubsection.body.class}, \elementref{slidesequence}}

\patterndef{slide.body.class}{The content allowable as the main body of a \elementref{slide}.
}{\item[\textit{Content}:] \patternref{Para.model}\item[\textit{Used by}:] \elementref{slide}}



\elementdef{sidebar}{A Sidebar; a short section-like object that floats outside the main flow.
}{\item[\textit{Attributes}:] \patternref{Sectional.attributes}\item[\textit{Content}:] ((\patternref{FrontMatter.class} ~\textbar~ \patternref{SectionalFrontMatter.class})\textsuperscript{*}, (\patternref{sidebar.body.class} ~\textbar~ \patternref{BackMatter.class})\textsuperscript{*})\item[\textit{Used by}:] \patternref{appendix.body.class}, \patternref{chapter.body.class}, \patternref{document.body.class}, \patternref{paragraph.body.class}, \patternref{part.body.class}, \patternref{section.body.class}, \patternref{subparagraph.body.class}, \patternref{subsection.body.class}, \patternref{subsubsection.body.class}}

\patterndef{sidebar.body.class}{The content allowable as the main body of a \elementref{sidebar}.
}{\item[\textit{Content}:] \patternref{Para.model}\item[\textit{Used by}:] \elementref{sidebar}}



\elementdef{appendix}{An Appendix within a document.
}{\item[\textit{Attributes}:] \patternref{Sectional.attributes}\item[\textit{Content}:] (\patternref{SectionalFrontMatter.class}\textsuperscript{*}, \patternref{appendix.body.class}\textsuperscript{*})\item[\textit{Used by}:] \patternref{BackMatter.class}}

\patterndef{appendix.body.class}{The content allowable as the main body of a chapter.
}{\item[\textit{Content}:] (\patternref{Para.model} ~\textbar~ \elementref{subparagraph} ~\textbar~ \elementref{paragraph} ~\textbar~ \elementref{subsubsection} ~\textbar~ \elementref{subsection} ~\textbar~ \elementref{section} ~\textbar~ \elementref{slide} ~\textbar~ \elementref{sidebar})\item[\textit{Used by}:] \elementref{appendix}}



\elementdef{bibliography}{A Bibliography within a document.
}{\item[\textit{Attributes}:] \patternref{Sectional.attributes}, \patternref{Listing.attributes}\attrdef{files}{the list of bib files used to create the bibliography.
}{\typename{text}}\attrdef{bibstyle}{the bibliographic style to be used to format the bibliography
(presumably a BibTeX bst file name)
}{\typename{text}}\attrdef{citestyle}{the citation style to be used when citing items from the bibliography
}{\typename{text}}\attrdef{sort}{whether the bibliographic items should be sorted or in order of citation.
}{\typename{boolean}}\item[\textit{Content}:] (\patternref{FrontMatter.class}\textsuperscript{*}, \patternref{SectionalFrontMatter.class}\textsuperscript{*}, \patternref{bibliography.body.class}\textsuperscript{*})\item[\textit{Used by}:] \patternref{BackMatter.class}}

\patterndef{bibliography.body.class}{The content allowable as the main body of a chapter.
}{\item[\textit{Content}:] (\patternref{Para.model} ~\textbar~ \elementref{biblist})\item[\textit{Used by}:] \elementref{bibliography}}



\elementdef{index}{An Index within a document.
}{\item[\textit{Attributes}:] \patternref{Sectional.attributes}, \patternref{Listing.attributes}\attrdef{role}{The kind of index (obsolete?)
}{\typename{text}}\item[\textit{Content}:] (\patternref{SectionalFrontMatter.class}\textsuperscript{*}, \patternref{index.body.class}\textsuperscript{*})\item[\textit{Used by}:] \patternref{BackMatter.class}}

\patterndef{index.body.class}{The content allowable as the main body of a chapter.
}{\item[\textit{Content}:] (\patternref{Para.model} ~\textbar~ \elementref{indexlist})\item[\textit{Used by}:] \elementref{index}}



\elementdef{indexlist}{A heirarchical index structure typically generated during postprocessing
from the collection of \elementref{indexmark} in the document
(or document collection).
}{\item[\textit{Attributes}:] \patternref{Common.attributes}, \patternref{ID.attributes}\item[\textit{Content}:] \elementref{indexentry}\textsuperscript{*}\item[\textit{Used by}:] \patternref{index.body.class}, \elementref{indexentry}}



\elementdef{indexentry}{An entry in an \elementref{indexlist} consisting of a phrase, references to
points in the document where the phrase was found, and possibly
a nested \elementref{indexlist} represented index levels below this one.
}{\item[\textit{Attributes}:] \patternref{Common.attributes}, \patternref{ID.attributes}\item[\textit{Content}:] (\elementref{indexphrase}, \elementref{indexrefs}\textsuperscript{?}, \elementref{indexlist}\textsuperscript{?})\item[\textit{Used by}:] \elementref{indexlist}}



\elementdef{indexrefs}{A container for the references (\elementref{ref}) to where an \elementref{indexphrase} was
encountered in the document. The model is Inline to allow
arbitrary text, in addition to the expected \elementref{ref}'s.
}{\item[\textit{Attributes}:] \patternref{Common.attributes}\item[\textit{Content}:] \patternref{Inline.model}\item[\textit{Used by}:] \elementref{glossaryentry}, \elementref{indexentry}}



\elementdef{glossary}{An Glossary within a document.
}{\item[\textit{Attributes}:] \patternref{Sectional.attributes}, \patternref{Listing.attributes}\attrdef{role}{The kind of glossary
}{\typename{text}}\item[\textit{Content}:] (\patternref{SectionalFrontMatter.class}\textsuperscript{*}, \patternref{glossary.body.class}\textsuperscript{*})\item[\textit{Used by}:] \patternref{BackMatter.class}}

\patterndef{glossary.body.class}{The content allowable as the main body of a chapter.
}{\item[\textit{Content}:] (\patternref{Para.model} ~\textbar~ \elementref{glossarylist})\item[\textit{Used by}:] \elementref{glossary}}



\elementdef{glossarylist}{A glossary list typically generated during postprocessing
from the collection of \elementref{glossaryphrase}'s in the document
(or document collection).
}{\item[\textit{Attributes}:] \patternref{Common.attributes}, \patternref{ID.attributes}\item[\textit{Content}:] \elementref{glossaryentry}\textsuperscript{*}\item[\textit{Used by}:] \patternref{glossary.body.class}}



\elementdef{glossaryentry}{An entry in an \elementref{glossarylist} consisting of a phrase,
(one or more, presumably in increasing detail?), possibly a definition,
and references to points in the document where the phrase was found.
}{\item[\textit{Attributes}:] \patternref{Common.attributes}, \patternref{ID.attributes}\attrdef{role}{The kind of glossary
}{\typename{text}}\attrdef{key}{a flattened form of the phrase for generating an \attr{ID}.
}{\typename{text}}\item[\textit{Content}:] (\elementref{glossaryphrase}\textsuperscript{*}, \elementref{indexrefs}\textsuperscript{?})\item[\textit{Used by}:] \elementref{glossarylist}}



\elementdef{title}{The title of a document, section or similar document structure container.
}{\item[\textit{Attributes}:] \patternref{Common.attributes}, \patternref{Fontable.attributes}, \patternref{Colorable.attributes}, \patternref{Backgroundable.attributes}\item[\textit{Content}:] (\elementref{tag} ~\textbar~ \typename{text} ~\textbar~ \patternref{Inline.class} ~\textbar~ \patternref{Misc.class} ~\textbar~ \patternref{Meta.class})\textsuperscript{*}\item[\textit{Used by}:] \patternref{SectionalFrontMatter.class}, \elementref{TOC}, \elementref{proof}, \elementref{theorem}}



\elementdef{toctitle}{The short form of a title, for use in tables of contents or similar.
}{\item[\textit{Attributes}:] \patternref{Common.attributes}\item[\textit{Content}:] (\elementref{tag} ~\textbar~ \typename{text} ~\textbar~ \patternref{Inline.class} ~\textbar~ \patternref{Misc.class} ~\textbar~ \patternref{Meta.class})\textsuperscript{*}\item[\textit{Used by}:] \patternref{SectionalFrontMatter.class}}



\elementdef{subtitle}{A subtitle, or secondary title.
}{\item[\textit{Attributes}:] \patternref{Common.attributes}\item[\textit{Content}:] \patternref{Inline.model}\item[\textit{Used by}:] \patternref{FrontMatter.class}, \elementref{slide}}



\elementdef{creator}{Generalized document creator.
}{\item[\textit{Attributes}:] \patternref{Common.attributes}, \patternref{FrontMatter.attributes}\attrdef{role}{indicates the role of the person in creating the docment.
Commonly useful values are specified, but is open-ended to support extension.
}{(\attrval{author} ~\textbar~ \attrval{editor} ~\textbar~ \attrval{translator} ~\textbar~ \attrval{contributor} ~\textbar~ \attrval{translator} ~\textbar~ \typename{text})}\attrdef{before}{specifies opening text to display before this creator in a formatted titlepage.
This would be typically appear outside the author information, like "and".
}{\typename{text}}\attrdef{after}{specifies closing text, punctuation or conjunction to display after
this creator in a formatted titlepage.
}{\typename{text}}\item[\textit{Content}:] (\patternref{Person.class} ~\textbar~ \patternref{Misc.class})\textsuperscript{*}\item[\textit{Used by}:] \patternref{SectionalFrontMatter.class}}

\patterndef{Person.class}{The content allowed in authors, editors, etc.
}{\item[\textit{Content}:] (\elementref{personname} ~\textbar~ \elementref{contact})\item[\textit{Used by}:] \elementref{creator}}



\elementdef{personname}{A person's name.
}{\item[\textit{Attributes}:] \patternref{Common.attributes}, \patternref{Refable.attributes}\item[\textit{Content}:] \patternref{Inline.model}\item[\textit{Used by}:] \patternref{Person.class}}



\elementdef{contact}{Generalized contact information for a document creator.
Note that this element can be repeated to give different types 
of contact information (using \attr{role}) for the same creator.
}{\item[\textit{Attributes}:] \patternref{Common.attributes}, \patternref{FrontMatter.attributes}, \patternref{Refable.attributes}\attrdef{role}{indicates the type of contact information contained.
Commonly useful values are specified, but is open-ended to support extension.
}{(\attrval{affiliation} ~\textbar~ \attrval{address} ~\textbar~ \attrval{current\_address} ~\textbar~ \attrval{email} ~\textbar~ \attrval{url} ~\textbar~ \attrval{thanks} ~\textbar~ \attrval{dedicatory} ~\textbar~ \attrval{orcid} ~\textbar~ \typename{text})}\item[\textit{Content}:] \patternref{Inline.model}\item[\textit{Used by}:] \patternref{Person.class}}



\elementdef{date}{Generalized document date.
Note that this element can be repeated to give the dates
of different events (using \attr{role}) for the same document.
}{\item[\textit{Attributes}:] \patternref{Common.attributes}, \patternref{FrontMatter.attributes}\attrdef{role}{indicates the relevance of the date to the document.
Commonly useful values are specified, but is open-ended to support extension.
}{(\attrval{creation} ~\textbar~ \attrval{conversion} ~\textbar~ \attrval{posted} ~\textbar~ \attrval{received} ~\textbar~ \attrval{revised} ~\textbar~ \attrval{accepted} ~\textbar~ \typename{text})}\item[\textit{Content}:] \patternref{Inline.model}\item[\textit{Used by}:] \patternref{FrontMatter.class}}



\elementdef{abstract}{A document abstract.
}{\item[\textit{Attributes}:] \patternref{Common.attributes}, \patternref{FrontMatter.attributes}\item[\textit{Content}:] \patternref{Block.model}\item[\textit{Used by}:] \patternref{FrontMatter.class}}



\elementdef{acknowledgements}{Acknowledgements for the document.
}{\item[\textit{Attributes}:] \patternref{Common.attributes}, \patternref{FrontMatter.attributes}\item[\textit{Content}:] \patternref{Inline.model}\item[\textit{Used by}:] \patternref{BackMatter.class}, \patternref{FrontMatter.class}}



\elementdef{keywords}{Keywords for the document. The content is freeform.
}{\item[\textit{Attributes}:] \patternref{Common.attributes}, \patternref{FrontMatter.attributes}\item[\textit{Content}:] \patternref{Inline.model}\item[\textit{Used by}:] \patternref{FrontMatter.class}}



\elementdef{classification}{A classification of the document.
}{\item[\textit{Attributes}:] \patternref{Common.attributes}, \patternref{FrontMatter.attributes}\attrdef{scheme}{indicates what classification scheme was used.
}{\typename{text}}\item[\textit{Content}:] \patternref{Inline.model}\item[\textit{Used by}:] \patternref{FrontMatter.class}}



\elementdef{titlepage}{block of random stuff marked as a titlepage
}{\item[\textit{Attributes}:] \patternref{Sectional.attributes}\item[\textit{Content}:] (\patternref{FrontMatter.class} ~\textbar~ \patternref{SectionalFrontMatter.class} ~\textbar~ \patternref{Block.class})\textsuperscript{*}\item[\textit{Used by}:] \elementref{document}}



\elementdef{TOC}{(Generalized) Table Of Contents, represents table of contents
as well as list of figures, tables, and other such things.
This will generally be placed by a \cs{tableofcontents} command
and filled in by postprocessing.
}{\item[\textit{Attributes}:] \patternref{Common.attributes}, \patternref{FrontMatter.attributes}\attrdef{lists}{indicates the kind of lists; space separated names of lists like "toc","lof", etc.
}{\typename{text}}\attrdef{select}{indicates what kind of document elements to list, in the form of
one or more tags such as \texttt{ltx:chapter} separated by \texttt{|}
(suggestive of an xpath expression).
}{\typename{text}}\attrdef{scope}{indicates the scope set of elements to include:
\texttt{current} (default) is all in current document;
\texttt{global} indicates all in the document set;
otherwise an xml:id
}{(\attrval{current} ~\textbar~ \attrval{global} ~\textbar~ \typename{text})}\attrdef{show}{indicates what things to show in each entry
}{\typename{text}}\attrdef{format}{indicates how to format the listing
}{(\attrval{normal} ~\textbar~ \attrval{short} ~\textbar~ \attrval{veryshort} ~\textbar~ \typename{text})}\item[\textit{Content}:] (\elementref{title}\textsuperscript{?}, \elementref{toclist}\textsuperscript{?})\item[\textit{Used by}:] \patternref{Para.class}, \elementref{navigation}}



\elementdef{toclist}{The actual table of contents list, filled in.
}{\item[\textit{Attributes}:] \patternref{Common.attributes}\item[\textit{Content}:] \elementref{tocentry}\textsuperscript{*}\item[\textit{Used by}:] \elementref{TOC}, \elementref{tocentry}}



\elementdef{tocentry}{An entry in a \elementref{toclist}.
}{\item[\textit{Attributes}:] \patternref{Common.attributes}\item[\textit{Content}:] (\elementref{ref} ~\textbar~ \elementref{toclist})\textsuperscript{*}\item[\textit{Used by}:] \elementref{toclist}}



\patterndef{Sectional.attributes}{Attributes shared by all sectional elements
}{\item[\textit{Attributes}:] \patternref{Common.attributes}, \patternref{Labelled.attributes}, \patternref{Backgroundable.attributes}\attrdef{rdf-prefixes}{Stores RDFa prefixes as space separated pairs,
with the pairs being prefix and url separated by a colon;
this should only appear at the root element.
}{\typename{text}}\item[\textit{Used by}:] \elementref{appendix}, \elementref{bibliography}, \elementref{chapter}, \elementref{document}, \elementref{glossary}, \elementref{index}, \elementref{inline-sectional-block}, \elementref{paragraph}, \elementref{part}, \elementref{section}, \elementref{sectional-block}, \elementref{sidebar}, \elementref{slide}, \elementref{slidesequence}, \elementref{subparagraph}, \elementref{subsection}, \elementref{subsubsection}, \elementref{titlepage}}

\patterndef{FrontMatter.attributes}{Attributes for other elements that can be used in frontmatter.
}{\attrdef{name}{Records the name of the type of object this is to be used when composing the
presentation.  The value of this attribute is often set by language localization packages.
}{\typename{text}}\item[\textit{Used by}:] \elementref{TOC}, \elementref{abstract}, \elementref{acknowledgements}, \elementref{classification}, \elementref{contact}, \elementref{creator}, \elementref{date}, \elementref{keywords}}

\patterndef{SectionalFrontMatter.class}{The content allowed for the front matter of each sectional unit,
and the document.
}{\item[\textit{Content}:] (\elementref{tags}\textsuperscript{?} ~\textbar~ \elementref{title} ~\textbar~ \elementref{toctitle} ~\textbar~ \elementref{creator})\item[\textit{Used by}:] \elementref{appendix}, \elementref{bibliography}, \elementref{chapter}, \elementref{document}, \elementref{glossary}, \elementref{index}, \elementref{paragraph}, \elementref{part}, \elementref{section}, \elementref{sidebar}, \elementref{slide}, \elementref{subparagraph}, \elementref{subsection}, \elementref{subsubsection}, \elementref{titlepage}}

\patterndef{FrontMatter.class}{The content allowed (in addition to \patternref{SectionalFrontMatter.class})
for the front matter of a document.
}{\item[\textit{Content}:] (\elementref{subtitle} ~\textbar~ \elementref{date} ~\textbar~ \elementref{abstract} ~\textbar~ \elementref{acknowledgements} ~\textbar~ \elementref{keywords} ~\textbar~ \elementref{classification})\item[\textit{Used by}:] \elementref{bibliography}, \elementref{document}, \elementref{sidebar}, \elementref{titlepage}}

\patterndef{BackMatter.class}{The content allowed a the end of a document.
Note that this includes random trailing Block and Para material,
to support articles with figures and similar data appearing `at end'.
}{\item[\textit{Content}:] (\elementref{bibliography} ~\textbar~ \elementref{appendix} ~\textbar~ \elementref{index} ~\textbar~ \elementref{glossary} ~\textbar~ \elementref{acknowledgements} ~\textbar~ \patternref{Para.class} ~\textbar~ \patternref{Meta.class})\item[\textit{Used by}:] \elementref{chapter}, \elementref{document}, \elementref{paragraph}, \elementref{part}, \elementref{section}, \elementref{sidebar}, \elementref{slide}, \elementref{subparagraph}, \elementref{subsection}, \elementref{subsubsection}}

\patternadd{Para.class}{}{\item[\textbar=] \elementref{TOC}}

\end{schemamodule}
\begin{schemamodule}{LaTeXML-bib}
\elementdef{biblist}{A list of bibliographic \elementref{bibentry} or \elementref{bibitem}.
}{\item[\textit{Attributes}:] \patternref{Common.attributes}\item[\textit{Content}:] (\elementref{bibentry} ~\textbar~ \elementref{bibitem})\textsuperscript{*}\item[\textit{Used by}:] \patternref{bibliography.body.class}}



\elementdef{bibitem}{A formatted bibliographic item, typically as written explicit
in a LaTeX article. This has generally lost most of the semantics
present in the BibTeX data.
}{\item[\textit{Attributes}:] \patternref{Common.attributes}, \patternref{ID.attributes}\attrdef{key}{The unique key for this object; this key is referenced by the
\attr{bibrefs} attribute of \elementref{bibref}.
}{\typename{text}}\item[\textit{Content}:] (\elementref{tags}\textsuperscript{?}, \elementref{bibblock}\textsuperscript{*})\item[\textit{Used by}:] \elementref{biblist}}



\elementdef{bibblock}{A block of data appearing within a \elementref{bibitem}.
}{\item[\textit{Content}:] \patternref{Flow.model}\item[\textit{Used by}:] \elementref{bibitem}}



\elementdef{bibentry}{Semantic representation of a bibliography entry, 
typically resulting from parsing BibTeX
}{\item[\textit{Attributes}:] \patternref{Common.attributes}, \patternref{ID.attributes}\attrdef{key}{The unique key for this object; this key is referenced by the
\attr{bibrefs} attribute of \elementref{bibref}.
}{\typename{text}}\attrdef{type}{The type of the referenced object. The values are a superset of
those types recognized by BibTeX, but is also open-ended for extensibility.
}{\patternref{bibentry.type}}\item[\textit{Content}:] \patternref{Bibentry.class}\textsuperscript{*}\item[\textit{Used by}:] \elementref{biblist}}

\patterndef{bibentry.type}{}{\item[\textit{Content}:] (\attrval{article} ~\textbar~ \attrval{book} ~\textbar~ \attrval{booklet} ~\textbar~ \attrval{conference} ~\textbar~ \attrval{inbook} ~\textbar~ \attrval{incollection} ~\textbar~ \attrval{inproceedings} ~\textbar~ \attrval{manual} ~\textbar~ \attrval{mastersthesis} ~\textbar~ \attrval{misc} ~\textbar~ \attrval{phdthesis} ~\textbar~ \attrval{proceedings} ~\textbar~ \attrval{techreport} ~\textbar~ \attrval{unpublished} ~\textbar~ \attrval{report} ~\textbar~ \attrval{thesis} ~\textbar~ \attrval{website} ~\textbar~ \attrval{software} ~\textbar~ \attrval{periodical} ~\textbar~ \attrval{collection} ~\textbar~ \attrval{collection.article} ~\textbar~ \attrval{proceedings.article} ~\textbar~ \typename{text})\item[\textit{Used by}:] \elementref{bib-related}, \elementref{bibentry}}



\elementdef{bib-name}{Name of some participant in creating a bibliographic entry.
}{\item[\textit{Attributes}:] \patternref{Common.attributes}\attrdef{role}{The role that this participant played in creating the entry.
}{(\attrval{author} ~\textbar~ \attrval{editor} ~\textbar~ \attrval{translator} ~\textbar~ \typename{text})}\item[\textit{Content}:] \patternref{Bibname.model}\item[\textit{Used by}:] \patternref{Bibentry.class}}



\patterndef{Bibname.model}{The content model of the bibliographic name fields (\elementref{bib-name})
}{\item[\textit{Content}:] \elementref{surname}, \elementref{givenname}\textsuperscript{?}, \elementref{lineage}\textsuperscript{?}\item[\textit{Expansion}:] (\elementref{surname}, \elementref{givenname}\textsuperscript{?}, \elementref{lineage}\textsuperscript{?})\item[\textit{Used by}:] \elementref{bib-name}}

\elementdef{surname}{Surname of a participant (\elementref{bib-name}).
}{\item[\textit{Content}:] \patternref{Inline.model}\item[\textit{Used by}:] \patternref{Bibname.model}}



\elementdef{givenname}{Given name of a participant (\elementref{bib-name}).
}{\item[\textit{Content}:] \patternref{Inline.model}\item[\textit{Used by}:] \patternref{Bibname.model}}



\elementdef{lineage}{Lineage of a participant (\elementref{bib-name}), eg. Jr. or similar.
}{\item[\textit{Content}:] \patternref{Inline.model}\item[\textit{Used by}:] \patternref{Bibname.model}}



\elementdef{bib-title}{Title of a bibliographic entry.
}{\item[\textit{Attributes}:] \patternref{Common.attributes}\item[\textit{Content}:] \patternref{Inline.model}\item[\textit{Used by}:] \patternref{Bibentry.class}}



\elementdef{bib-subtitle}{Subtitle of a bibliographic entry.
}{\item[\textit{Attributes}:] \patternref{Common.attributes}\item[\textit{Content}:] \patternref{Inline.model}\item[\textit{Used by}:] \patternref{Bibentry.class}}



\elementdef{bib-key}{Unique key of a bibliographic entry.
}{\item[\textit{Attributes}:] \patternref{Common.attributes}\item[\textit{Content}:] \patternref{Inline.model}\item[\textit{Used by}:] \patternref{Bibentry.class}}



\elementdef{bib-type}{Type of a bibliographic entry.
}{\item[\textit{Attributes}:] \patternref{Common.attributes}\item[\textit{Content}:] \patternref{Inline.model}\item[\textit{Used by}:] \patternref{Bibentry.class}}



\elementdef{bib-date}{Date of a bibliographic entry.
}{\item[\textit{Attributes}:] \patternref{Common.attributes}\attrdef{role}{characterizes what happened on the given date
}{(\attrval{publication} ~\textbar~ \attrval{copyright} ~\textbar~ \typename{text})}\item[\textit{Content}:] \patternref{Inline.model}\item[\textit{Used by}:] \patternref{Bibentry.class}}



\elementdef{bib-publisher}{Publisher of a bibliographic entry.
}{\item[\textit{Attributes}:] \patternref{Common.attributes}\item[\textit{Content}:] \patternref{Inline.model}\item[\textit{Used by}:] \patternref{Bibentry.class}}



\elementdef{bib-organization}{Organization responsible for a bibliographic entry.
}{\item[\textit{Attributes}:] \patternref{Common.attributes}\item[\textit{Content}:] \patternref{Inline.model}\item[\textit{Used by}:] \patternref{Bibentry.class}}



\elementdef{bib-place}{Location of publisher or event
}{\item[\textit{Attributes}:] \patternref{Common.attributes}\item[\textit{Content}:] \patternref{Inline.model}\item[\textit{Used by}:] \patternref{Bibentry.class}}



\elementdef{bib-related}{A Related bibliographic object, such as the book or journal
that the current item is related to.
}{\item[\textit{Attributes}:] \patternref{Common.attributes}\attrdef{type}{The type of this related entry.
}{\patternref{bibentry.type}}\attrdef{role}{How this object relates to the containing object.
Particularly important is \attrval{host} which indicates that
the outer object is a part of this object.
}{(\attrval{host} ~\textbar~ \attrval{event} ~\textbar~ \attrval{original} ~\textbar~ \typename{text})}\attrdef{bibrefs}{If the bibrefs attribute is given, it is the key of another object in the bibliography,
and this element should be empty; otherwise the object should be described by
the content of the element.
}{\typename{text}}\item[\textit{Content}:] \patternref{Bibentry.class}\textsuperscript{*}\item[\textit{Used by}:] \patternref{Bibentry.class}}



\elementdef{bib-part}{Describes how the current object is related to a related (\elementref{bib-related})
object, in particular page, part, volume numbers and similar.
}{\item[\textit{Attributes}:] \patternref{Common.attributes}\attrdef{role}{indicates how the value partitions the containing object.
}{(\attrval{pages} ~\textbar~ \attrval{part} ~\textbar~ \attrval{volume} ~\textbar~ \attrval{issue} ~\textbar~ \attrval{number} ~\textbar~ \attrval{chapter} ~\textbar~ \attrval{section} ~\textbar~ \attrval{paragraph} ~\textbar~ \typename{text})}\item[\textit{Content}:] \patternref{Inline.model}\item[\textit{Used by}:] \patternref{Bibentry.class}}



\elementdef{bib-edition}{Edition of a bibliographic entry.
}{\item[\textit{Attributes}:] \patternref{Common.attributes}\item[\textit{Content}:] \patternref{Inline.model}\item[\textit{Used by}:] \patternref{Bibentry.class}}



\elementdef{bib-status}{Status of a bibliographic entry.
}{\item[\textit{Attributes}:] \patternref{Common.attributes}\item[\textit{Content}:] \patternref{Inline.model}\item[\textit{Used by}:] \patternref{Bibentry.class}}



\elementdef{bib-identifier}{Some form of document identfier. The content is descriptive.
}{\item[\textit{Attributes}:] \patternref{Common.attributes}, \patternref{Refable.attributes}\attrdef{scheme}{indicates what sort of identifier it is; it is open-ended for extensibility.
}{(\attrval{doi} ~\textbar~ \attrval{issn} ~\textbar~ \attrval{isbn} ~\textbar~ \attrval{mr} ~\textbar~ \typename{text})}\attrdef{id}{the identifier.
}{\typename{text}}\item[\textit{Content}:] \patternref{Inline.model}\item[\textit{Used by}:] \patternref{Bibentry.class}}



\elementdef{bib-review}{Review of a bibliographic entry. The content is descriptive.
}{\item[\textit{Attributes}:] \patternref{Common.attributes}, \patternref{Refable.attributes}\attrdef{scheme}{indicates what sort of identifier it is; it is open-ended for extensibility.
}{(\attrval{doi} ~\textbar~ \attrval{issn} ~\textbar~ \attrval{isbn} ~\textbar~ \attrval{mr} ~\textbar~ \typename{text})}\attrdef{id}{the identifier.
}{\typename{text}}\item[\textit{Content}:] \patternref{Inline.model}\item[\textit{Used by}:] \patternref{Bibentry.class}}



\elementdef{bib-links}{Links to other things like preprints, source code, etc.
}{\item[\textit{Attributes}:] \patternref{Common.attributes}\item[\textit{Content}:] \patternref{Inline.model}\item[\textit{Used by}:] \patternref{Bibentry.class}}



\elementdef{bib-language}{Language of a bibliographic entry.
}{\item[\textit{Attributes}:] \patternref{Common.attributes}\item[\textit{Content}:] \patternref{Inline.model}\item[\textit{Used by}:] \patternref{Bibentry.class}}



\elementdef{bib-url}{A URL for a bibliographic entry. The content is descriptive
}{\item[\textit{Attributes}:] \patternref{Common.attributes}, \patternref{Refable.attributes}\item[\textit{Content}:] \patternref{Inline.model}\item[\textit{Used by}:] \patternref{Bibentry.class}}



\elementdef{bib-extract}{An extract from the referenced object.
}{\item[\textit{Attributes}:] \patternref{Common.attributes}\attrdef{role}{Classify what kind of extract
}{(\attrval{keywords} ~\textbar~ \attrval{abstract} ~\textbar~ \attrval{contents} ~\textbar~ \typename{text})}\item[\textit{Content}:] \patternref{Inline.model}\item[\textit{Used by}:] \patternref{Bibentry.class}}



\elementdef{bib-note}{Notes about a bibliographic entry.
}{\item[\textit{Attributes}:] \patternref{Common.attributes}\attrdef{role}{Classify the kind of note
}{(\attrval{annotation} ~\textbar~ \attrval{publication} ~\textbar~ \typename{text})}\item[\textit{Content}:] \patternref{Inline.model}\item[\textit{Used by}:] \patternref{Bibentry.class}}



\elementdef{bib-data}{Random data, not necessarily even text.
(future questions: should model be text or ANY? maybe should have encoding attribute?).
}{\item[\textit{Attributes}:] \patternref{Common.attributes}\attrdef{role}{Classify the relationship of the data to the entry.
}{\typename{text}}\attrdef{type}{Classify the type of the data.
}{\typename{text}}\item[\textit{Content}:] \patternref{Inline.model}\item[\textit{Used by}:] \patternref{Bibentry.class}}



\patterndef{Bibentry.class}{}{\item[\textit{Content}:] (\elementref{bib-name} ~\textbar~ \elementref{bib-title} ~\textbar~ \elementref{bib-subtitle} ~\textbar~ \elementref{bib-key} ~\textbar~ \elementref{bib-type} ~\textbar~ \elementref{bib-date} ~\textbar~ \elementref{bib-publisher} ~\textbar~ \elementref{bib-organization} ~\textbar~ \elementref{bib-place} ~\textbar~ \elementref{bib-part} ~\textbar~ \elementref{bib-related} ~\textbar~ \elementref{bib-edition} ~\textbar~ \elementref{bib-status} ~\textbar~ \elementref{bib-language} ~\textbar~ \elementref{bib-url} ~\textbar~ \elementref{bib-note} ~\textbar~ \elementref{bib-extract} ~\textbar~ \elementref{bib-identifier} ~\textbar~ \elementref{bib-review} ~\textbar~ \elementref{bib-links} ~\textbar~ \elementref{bib-data})\item[\textit{Used by}:] \elementref{bib-related}, \elementref{bibentry}}

\end{schemamodule}