% Physics of electric dynamics, from viewpoint of biology


\subsection[Dynamics]{Ions' dynamics\label{sec:Physics-ElectrodiffusionDynamics}\label{sec:Physics-Speculations}}
As a kind of consistency check, for estimating the orders of magnitude of movement's parameters of charged heavy balls.


\subsubsection{Rush-in speed\label{sec:Physics-RushInSpeed}}
If we assume that a $50\ mV$ potential difference exists across the 
$5\ nm$ thick membrane, it translates to a $10^7\ V/m$ gradient~\cite{MolecularBiology:2002}.
The value of the force acting on a unit charge is
\begin{equation}
F_{Na^+}= 10^7 *1.60217663* 10^{-19} = 1.6*10^{-12}\ N
\end{equation}
\noindent
Given that the mass of a $Na^+$ ion is $m_{Na} = 3.81915* 10^{-26}\ kg$, it causes an acceleration
\begin{equation}
a=\frac{F_{Na^+}}{m_{Na^+}} = 0.42 * 10^{14}\ \frac{m}{s^{2}}
\label{eq:Physics-RushInAceleration}
\end{equation}
\noindent
The distance $d$ to travel is the thickness of the neuronal membrane $5*10^{-9}\ nm$
and we assume a continuosly accelerating ion, and using $d=\frac{1}{2}*a*t^2$
\begin{equation}
t_{traverse} = \sqrt{\frac{2*5*10^{-9}}{0.42*10^{14}}} = 1.54*10^{-11}\ s
\end{equation}
\noindent
In this period the ion is accelerated to the speed  $v=a*t$
\begin{equation}
v_{Na^+} = 0.42 * 10^{14}\ *\ 1.54*10^{-11} = 600\ m/s
\end{equation}
The current we can attribute to an ion channel,
provided that $10^3$ ions pass through an single ion channel, it means
\begin{equation}
I_{channel} = \frac{10^3 * 1.60217663* 10^{-19}}{1.54*10^{-11}} = 10^{-5} A
\end{equation}
\noindent
(The thermal speed is about an order of magnitude higher.
It is in the range of several $\mu A$s, for a short period. It is distributed in the 
spatiotemporal surface of a neuron, and contributes to a current of dozen of $pA$s.)

Notice an important difference. The acceleration of an ion is unbelievably large. It is sufficiently large to keep the potential at the same value,
provided that the ions must follow a small change, such as due to the
"leaking" current through the 
%\gls{AIS}
AIS.
Similarly, the ions can 'instantly' follow quick changes such as
a square wave gradient.
However, if many similar ions are ahead, their repulsion decreases
the acceleration, and the ion travels only at a few $m/s$ speed.
The huge forces and accelerations means that a potential change
acts immediately. However, the force decays quickly. The ions start to move
'instantly', but the charge carriers can move only with a limited speed,
much below the interaction speed of 
EM
%\gls{EM}
interactions. The effect can propagate only with that lower speed.

See the case of axon: there exist a mechanical constraint that 
the ions cannot spread through the wall (they must keep the direction), and 
the ion package propagates as Equs.\ref{eq:Nernst-dVdt} and~\ref{eq:Nernst-dCdt} describe it.

\begin{advanced}
As shown above, the ions' passage time, is much below $1\ ns$. 
To switch that very intense and very short current pulse, one needs
an at least as fast switch.
The mechanical handling, using mechanical caps,
(see the value of acceleration) is not possible.
It needs an electronic control, see the layers of the two sides of the membrane.

Furthermore, the time is only slightly higher than the magnitude of
molecular transitions. It is highly unprobable, that it can happen
$10^3$ times 
reconfiguration of long molecule chains can happen in such short periods. 

\end{advanced}

\subsubsection{Rush-in concentration\label{sec:Physics-RushInConcentration}}

As we described in section~\ref{sec:Fick-Electrodiffusion}, this sudden
concentration change provokes a voltage gradient as described by Eq.~(\ref{eq:PhysicsGradientRatio}). From that point on, gradients
$\frac{dV}{dt}$ and $\frac{dC}{dt}$ excite each other, as described
by Eqs.~(\ref{eq:Nernst-1V}) and~(\ref{eq:Nernst-1C}).
\index{voltage gradient}
In simple physical picture, the rush-in ions appearing on the neuron's membrane
are confined in the volume formed in an atomic layer on its surface.
They attempt to be uniformly distributed. The electric repulsion 
enables a hyper-viscous behavior for the electric fluid, so the surface -- without invasion -- would be equipotential. The membrane is an excellent isolator,
except at the 
\gls{AIS}. Here the ion channels represent a resistance (also limits the current), so a current starts, see
section~\ref{sec:Physiology-NeuralCurrents_AIS},
creating a potential gradient.

Here comes into play the limiting effect of the interaction speed.
Given that the electric repulsion is mediated by the concentration, 
the decrease of the potential can happen with the speed of the slow current
in the proximity of the 
%\gls{AIS}
AIS and the ions from the distant region can increase it with their potential-assisted speed. 
The potential gradient creates a speed gradient in proximity of the
%\gls{AIS}
AIS, while it remains zero at larger distances. The speed gradient
propagates with the speed given by Eq.~(\ref{eq:StokesEinsteinSpeed2})
and its delayed effect creates a "ram current" effect, which 
-- in a good approximation -- is resemblant to the effect of an $RC$ circuit.
The charge is "stored" (cannot flow out because of the limiting resistor)
for some time and at approprite parameters the neuron behaves as a damped $RC$ circuit.




\subsubsection{Rush-in charge\label{sec:Physics-RushinCharge}}

Let us suppose we have a simple electrodiffusion, say, positive ions rush-in into the intracellular space of a neuron through the membrane's ion channels, in one packet, about $10^5$ ions per channel. For all channels, in the order of $10^2$ channels on the membrane, the number of ions appearing (suddenly, in $psec$ time per channel (see section~\ref{sec:Physics-RushInSpeed}) and in $nsec$ time per membrane) on the surface is only $10^7$. This charge appears as current at the beginning of the
%\gls{AIS}
AIS and moves (the neuron's membrane discharges) without an external potential.
The huge difference in the charge density (and concentration) on the membrane's surface and the no-charge at the beginning of the axon can explain why a current flows. (The other way round: ions escape into the volume and they repulse each other; so they will move in the direction of the drain).

The Nernst-Planck equation could explain that the concentration gradient causes a potential gradient, i.e., explain why do we see a current without voltage. However, the number of ions is not surely sufficient to apply thermodynamics.
At the same time, the ions exist in a volume having thickness about tenths of nanometer, i.e., the density of the ions can be sufficient to behave as a macroscopic charged fluid in that limited volume.
This case is neither "net" macroscopic nor microscopic. On the one side, the Nernst-Planck
 equation refers to large (on this scale) volume. On the other, evidence shows that dozens of $pA$ current flows, and behaves in a macroscopic way.  

When we assume $10^7$ rushed-in ions for evoking a single action potential, it means $1.6*10^{-12}\ Cb$ charge. We take the typical resistance and capacitance values from~\cite{KochBiophysics:1999}:
we assume $100\ pF$ capacitance and $100\ M\Omega$ resistance (i.e., $\tau=10^{-2}\ s$). 
If we assume that the charge flows out in a period of $10\ ms$, we should measure a $160\ pA$ current on the axon, furthermore, that current causes a $160\ mV$ voltage drop (aka Action Potential) on the resistance.
See also section~\ref{sec:Physiology-VoltageSquareVawe}: $300\ pA$ flows
in a $5\ ms$ period.
Similar value can be derived from~\cite{BeanActionPotential:2007}.
All those values are in the order of the measured values.


 
So, the charge from the micro-world can be excellently mapped to the measured macroscopic current. On the other way round, due to the Nernst-Planck equation~(\ref{eq:NernstPlanck}), 
\index{Nernst-Planck equation}
it could be expressed as a consequence of the sudden increase in the concentration (a concentration gradient $\frac{dC}{dx}$) of the chemical ions on the surface.
Provided that thermodynamics can be applied to a volume of area of $10^{-2}\ mm^2$ and thickness say up to $10*10^{-9}\ m$.
With those numbers we arrive at that in that layer the ion density during generating an action potential is  $10^{23}\ m^{-3}$, which is not far from the numbers where thermodynamics can be applied;
so we assume that the Nernst-Planck equation (and the ones we described as "laws of motion of electrodiffusion") can be used to describe why an
%\gls{AP}
AP evokes in a neuron.



\subsubsection{Ion selectivity\label{sec:Physics-IonSelectivity}}
We assume that in a balanced state, a  $Na^{+}$ and a  $K^{+}$
ion are placed at the entrance of the ion channel.
According to the calculations above, $Na^{+}$ ions pass the ion channel in $15\ ps$.
%
At the same time, since the mass of a $K^+$ ion is $m_{K} = 6.492* 10^{-26}\ kg$, 
and the same force accelerates it, its acceleration
\begin{equation}
a=\frac{F_{K^+}}{m_{K^+}} = 0.246 * 10^{14}\ \frac{m}{s^{2}}
\end{equation}
%==F_{Na^+}
With this acceleration, the $K^+$ ions traverse in the the ion channel a distance only
$d=\frac{1}{2}*0.246 * 10^{14}*(1.54*10^{-11})^2 = 2.92 * 10^{-9}\ nm$.
That means that when the $Na^+$ ions already arrived at the low-concentration layer,
at that time the $K^+$ ion passed only $\frac{2}{3}*d$ distance. If the potential on the 
arrival side increases due to the already arrived ions, they will experience 
a repulsive potential, and they will not be able to arrive: they cannot pass 
the uphill potential barrier and they turn back, with the downhill gradient.

Given the current intensity above, that the ion channels' size is not
much larger than an ion, furthermore that the time available for interaction
is shorter than the chemical transition time for the molecules, 



\begin{advanced}
As shown above, the ions' passage time is uncomparably fast compared to the neuronal processes' characteristic time.
Furthermore, the time is about two orders of magnitude shorter
than the atomic transitions, so it is highly unprobable, that
reconfiguration of long molecule chains can be realistic.
To switch that very intense and very short current, one needs
an at least as \textit{fast switch}.
The mechanical handling, using mechanical caps,
(see the value of acceleration) is not possible.
It needs an electronic control, see the layers on the two sides of the membrane,
see section~\ref{sec:Physics-ElectrolyteLayers}.
\end{advanced}


%
%\subsubsection{Layer speed\label{sec:Physics-MembraneCurrent}}
%We assume that two $Na^+$ ions are at a distance of $10\ nm$.
%The force between them is 
%\index{Coulomb's law}
%\begin{equation}
%F_{Na^+}= \frac{1}{4*\pi*\epsilon_o}\frac{q_1*q_2}{r^2} = 9*10^9\biggl( \frac{1.60217663* 10^{-19}}{10^{-8}}\biggr)^2 = 2.311027 * 10^{-14}\ N
%\end{equation}
%\noindent

